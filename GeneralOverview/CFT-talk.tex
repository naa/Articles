\documentclass[a4paper,12pt]{article}
\usepackage[unicode,verbose]{hyperref}
\usepackage{amsmath,amssymb,amsthm} \usepackage{pb-diagram}
\usepackage{ucs}
\usepackage[utf8x]{inputenc}
\usepackage[russian]{babel}
\usepackage{cmap}
\usepackage{graphicx}
\pagestyle{plain}
 \newtheorem{theorem}{Теорема}
\theoremstyle{definition} 
\newtheorem{Def}{Definition}
\newtheorem{axiom}{Аксиома}

\topmargin=-3.5cm
\oddsidemargin=-1.5cm
\evensidemargin=-1.5cm
\textwidth=19cm
\textheight=28cm

\title{Что такое конформная теория поля?}
\author{Антон Назаров\\ antonnaz@gmail.com}
\begin{document}
\maketitle

Здравствуйте!

Мой доклад называется ``Что такое конформная теория поля?'' и я сразу
отвечу на этот вопрос. Конформная теория поля -- это квантовая теория
поля, обладающая конформной инвариантностью.  Я объясню, что значат
эти слова, но сперва мотивация. 

Рассмотрим нашу любимую модель Изинга. Это простая модель магнетика.
У нас есть квадратная решетка, в вершинах которой находятся спины
$\sigma$, принимающие значения $\pm 1$. Энергия конфигурации дается
суммой произведений ближайших соседей
$E[\sigma]=\sum_{\left<ij\right>} \sigma_i \sigma_j$, а вероятность
конфигурации равна экспоненте от минус бета на энергию
$P[\sigma]=\frac{1}{Z}e^{-\beta E[\sigma]}$. Нас интересуют локальные 
наблюдаемые -- средние значения функций от спинов в заданных узлах
$f(z_1,\dots,z_n)=\sum_{[\sigma]} f(\sigma(z_1),\dots,\sigma(z_n))
P[\sigma]$.

Так как в настоящем магнетике очень много узлов, а расстояние между
ними очень маленькое, возникает естественное желание перейти к
непрерывному описанию. Мы можем переписать энергию через разностные производные
$E[\sigma]=\sum_x \sum_y \left( \frac{(\sigma(x,y)-\sigma(x+\Delta
    x,y))^2}{\Delta x^2} + \frac{(\sigma(x,y)-\sigma(x,y+\Delta
    y))^2}{\Delta y^2}\right) +\mathrm{const}$
и перейти к пределу $\Delta x, \Delta y\to 0$. Тогда для энергии
получится выражение типа 
$E[\varphi]=\int dx dy \left( (\partial_x \varphi)^2+(\partial_y
  \varphi)^2\right)$

У нас получилась теория поля. Нам надо еще накладывать дополнительные
условия на поля, которые были бы аналогом того, что спины $\sigma=\pm
1$.  Наши наблюдаемые -- это корреляционные функции. Они должны быть
средними значениями по всем конфигурациям поля, поэтому теория поля
должна быть квантовой.  

Мы знаем, что в критической точке корреляционные функции обладают
определенными свойствами по отношению к конформным преобразованиям. 

Что же такое теория поля?

Напомню про классическую механику. Она появилась из наблюдений за
планетами, поэтому в ней есть точечные объекты с координатами
$p_i, q_i$  в $n$-мерном пространстве. Координаты удовлетворяют
дифференциальным уравнениям 
$\dot p_i = -\partial_{q_i} H(p,q,t)$, $\dot q_i = \partial_{p_i}
H(p,q,t)$, 

Решения ищутся в классе каких-то хороших функций, например, гладких.

Теория поля -- это естественное обобщение классической механики, когда
речь идет не об отдельных точечных объектах, а о среде. Физики
смотрели на то, как железные опилки выстраиваются вокруг магнита и
видели картинки силовых линий, поэтому предположили, что существует
некоторое поле, которое их выстраивает. 

Само поле должно удовлетворять каким-то уравнениям, например, $\vec
E=-\nabla \varphi$, $\Delta \varphi=4\pi \rho$. 
Так как мы хотим, чтобы у нас могли быть точечные заряды
$\rho=\delta(\vec x - \vec x_0)$, то мы не можем обойтись функциями,
приходится переходить к распределениям.
Пространство тест-функций Шварца мы будем обозначать через
$S(\mathbb{R}^n)=\{f:\mathbb{R}^n\to \mathbb{C}\}$, распределения --
это непрерывные функционалы $T:S(\mathbb{R}^n)\to \mathbb{C}$ на
пространстве тест-функций. Понятно, что для каждой измеримой функции
$g$ есть распределение $T_g (f) = \int_{\mathbb{R}^n} g(x) f(x) dx$.
Распределения можно дифференцировать $(\partial^{\alpha} T) (f) =
(-1)^{|\alpha|} T(\partial^{\alpha} f)$. Кроме того, любое
интересующее нас распределение можно представить в виде $T=\sum_{0\leq
  |\alpha| \leq n} \partial^{\alpha} T_{g_a}$.

Таким образом, классическая теория поля -- это наука о
распределениях, удовлетворяющих некоторым дифференциальным уравнениям
в частных производных.  

Теперь слово ``квантовая''. В начале XX века обнаружилось, что у
микроскопических частиц не удается одновременно точно измерить
координату и импульс. Была предложена следующая конструкция --
состояние системы описывается некоторым вектором в гильбертовом
пространстве ${\cal H}$, а измерению координаты или импульса
соответствует действие операторами $\hat q_i, \hat p_i$. Эти операторы
не коммутируют $[\hat q, \hat p]=i \hbar$, поэтому собственное
состояние оператора координаты не может быть собственным состоянием
оператора импульса. 

Подробнее мы обсуждали квантовую механику на учебном семинаре в
прошлом году, сейчас нам важно, что у нас появилось гильбертово
пространство состояний ${\cal H}$ и пространство операторов на нем
${\cal O}({\cal H})$.

Как совместить эту картину с теорией поля?

Естественно, заменить поля на операторнозначные распределения
$\varphi:S(\mathbb{C}^n)\to {\cal O}({\cal H})$. При этом нам нужно
потребовать выполнения следующих условий. В гильбертовом пространстве
состояний ${\cal H}$ должно быть плотное множество $D\subset {\cal
  H}$, такое что
\begin{itemize}
\item $\forall f\in S(\mathbb{C}^n), D\subset D_{\varphi(f)}$, где
  $D_{\varphi(f)}$ -- область определения $\varphi(f)$.
\item $f\to \varphi(f)|_D$ -- линейное отображение из
  $S(\mathbb{C}^n)$ в $\mathrm{End}(D)$
\item $\forall \nu\in D, \omega\in {\cal H}$ отображение $f\to
  \left<\omega, \varphi(f)(\nu)\right>$ -- непрерывное распределение
\end{itemize}
Кроме того, мы потребуем в ${\cal H}$ наличия выделенного состояния
$\Omega\in {\cal H}$ -- вакуума. 

Путь $B_0$ -- счетное множество индексов, которые будут нумеровать
поля в теории. Мы будем пользоваться мультииндексами
$i=(i_1,\dots,i_n)\in B_0^n$, и множество мультииндексов обозначим
через $\{i\}=B$. 

Мы хотим определить наблюдаемые -- корреляционные функции
$G_{i_1,\dots, i_n}(z_1,\dots,z_n)$, так что $$G_{i_1,\dots,
  i_n}(z_1,\dots,z_n)=\left<\Omega,\varphi_{i_1}(z_1)\dots
  \varphi_{i_n}(z_n)\Omega\right>.$$

Они будут жить на $M_n=\{(z_1,\dots,z_n): z_i\neq z_j\}$, еще мы будем
использовать $M_n^+: \mathrm{Re}z_i>0$. Введем последовательность
пространств тест-функций с ограниченным носителем $\{S_n^+: S_0^+=\mathbb{C}, S_n^+=\{f\in
  S(\mathbb{C}^n: \mathrm{supp}(f)\subset M_n^+\}$.

Теперь мы можем сформулировать нашу теорию в виде системы аксиом на
непрерывные полиномиально ограниченные функции $G_{i_1,\dots,
  i_n}:M_n\to \mathbb{C}$, $G_{\emptyset}=1$.
\begin{axiom}
  Локальность:   для всех $(i_{1},\dots,i_{n})\in B_{0}^{n}$,
  $(z_{1},\dots, z_{n})\in M_{n}$ и $\pi\in S_{n}$ --- перестановок
  множества из $n$ элементов верно равенство 
  $G_{i_{1},\dots,i_{n}}(z_{1},\dots,z_{n})=G_{i_{\pi(1)},\dots i_{\pi(n)}}(z_{\pi(1)},\dots, z_{\pi(n)})$
\end{axiom}
\begin{axiom}
  Преобразования при движениях комплексной плоскости: 
рассмотрим группу движений двумерного пространства $E_{2}$,
генераторами которой являются повороты
$r_{\alpha}:\mathbb{C}\to\mathbb{C}, \; z\to e^{i\alpha}z,\; \alpha\in
\mathbb{R}$ и трансляции $t_{a}:\mathbb{C}\to\mathbb{C},\; z\to z+a,\;
a\in\mathbb{C}$, 
Тогда для любого индекса $i\in B_{0}$  существуют независимые конформные веса $h_{i},\bar h_{i}\in \mathbb{R}$, такие, что для всех преобразований $w\in E_{2}$
  $G_{i_{1},\dots,i_{n}}(z_{1},\dots,z_{n},\bar z_{1},\dots, \bar z_{n})=
\prod_{j=1}^{n}\left(\frac{dw(z_{j})}{dz}\right)^{h_{i_{j}}}\left(\overline{\frac{dw(z_{j})}{dz}}\right)^{\bar{h}_{i_{j}}}
G_{i_1,\dots i_n}(w_{1},\dots, w_{n},\bar w_{1},\dots,\bar w_{n})$,
где $w_{i}=w(z_{i})$, а $s_{i}=h_{i}-\bar h_{i}, d_{i}=h_{i}+\bar
h_{i}$  -- конформный спин и скейлинговая размерность. 
\end{axiom}
\begin{axiom}
  Положительность по отношению к отражениям:
  Обозначим через $\Theta:\mathbb{C}\to\mathbb{C}$  отображение $z=t+i y\to \Theta(z)= -t+i y$. Тогда аксиома утверждает, что существует инволюция $\star:B_{0}\to B_{0}$, $\star^{2}=\mathrm{id}_{B_{0}}$, которая продолжается на $B$ ($\star:i\to i^{*}$), и выполняются свойства
  \begin{enumerate}
  \item Верно равенство $G_{i}(z)=G_{i^{*}}(\Theta(z))=G_{i^{*}}(-z^{*})$
  \item Обозначим через $\underline{S}^{+}$ пространство
    последовательностей тест-функций $\underline{f}=(f_{i})_{i\in B},
    f_{i}\in S^{+}_{n}$. Тогда неравенство
    $$\left<\underline{f},\underline{f}\right>=
      \sum_{i,j\in B}\sum_{n,m}\int_{M_{n+m}}G_{i^{*} j}(\Theta(z_{1}),\dots ,\Theta(z_{n}),w_{1},\dots,w_{m}) f_{i}(z)^{*}f_{j}(w) d^{n}z d^{m}w \geq 0,$$
      верно  $\forall \underline{f}\in \underline{S}^{+}$. 
  \end{enumerate}
\end{axiom}

Если эти аксиомы выполнены для системы функций $\{G_i\}$, то мы можем
восстановить гильбертово пространство ${\cal H}$. У нас есть
положительная полуопределенная форма на $\underline{S}^+$, тогда
${\cal H}$ -- это пополнение $\underline{S}^+$, факторизованное по
ядру этой формы. 

Можно построить и поля $\varphi_j$ для всех $j\in B_0$. Для $f\in S^+,
\underline g\in \underline S^+, [\underline{g}]\in {\cal H}$ определим
$\varphi_j(f)([\underline{g}])=\left((\underline{g}\times
  f)_{i_1,\dots,i_{n+1}}\right)_{i\in B}$, где $(\underline{g}\times
f)_{i_1,\dots,i_{n+1}}(z_1,\dots, z_{n+1})=g_{i_1,\dots, i_m}
(z_1,\dots, z_n) f(z_{n+1}) \delta_{j,i_{n+1}}$.

Еще у нас есть вакуум $\Omega=[f]: f_{\emptyset}=1, f_i=0\; \forall
i\neq \emptyset$. Кроме того, на ${\cal H}$ у нас действует унитарное
представление $E_2$, а инвариантное относительно него подпространство
$D$ плотно. 

Теорема определяет структуру двумерной евклидовой теории поля:
\begin{theorem}
  \begin{enumerate}
  \item 
    Для всех $j\in B_{0}$ отображения $\varphi_{j}:S^{+}\to
    \mathrm{End}(D)$ линейны, $\varphi_{j}$ -- полевые операторы,
    $\varphi_{j}(D)\subset D, \Omega\in D$ и вакуум $\Omega$
    инвариантен относительно унитарных представлений $U$ группы
    $E_{2}$. 
  \item Поля $\varphi_{j}$ преобразуются ковариантно по отношению к представлению $U$, для $w\in E_{2}$:
    $$U(w)\varphi_{j}(z)U(w)^{*}=\left(\frac{\partial w}{\partial z}\right)^{h_{j}}\varphi_{j}(w(z))$$
  \item Матричные коэффициенты $\left<\Omega|\varphi_{i_{1}}(z_{1})\dots \varphi_{i_{n}}(z_{n})|\Omega\right>$ представляются аналитическими функциями, которые при $\mathrm{Re}z_{n}>\dots>\mathrm{Re}z_{1}>0$ совпадают с корреляционными функциями
    $$\left<\Omega|\varphi_{i_{1}}(z_{1})\dots \varphi_{i_{n}}(z_{n})|\Omega\right>=G_{i_{1},\dots,i_{n}}(z_{1},\dots,z_{n})$$
  \end{enumerate}
\end{theorem}
Кроме того, если рассмотреть корреляционные функции, зависящие от
двойного набора переменных $G_{i_1,\dots,i_n}(z_1,\dots, z_n,\bar
z_1,\dots,\bar z_n)$, они оказываются голоморфными на $M_n^>\times
M_n^>$, где $M_n^>=\{z\in M_n^+:\mathrm{Re}
z_n>\dots>\mathrm{Re}z_1>0\}$.

Чтобы получилась конформная теория поля нам осталось добавить
требование масштабной инвариантности. 
\begin{axiom}
  Корреляционная функция $G_{i}, i\in B$ преобразуется ковариантно при масштабных преобразованиях $w(z)=e^{\tau}z$, то есть
    $$G_{i_{1},\dots,i_{n}}(z_{1},\dots,z_{n})=\left(e^{\tau}\right)^{h_{1}+\dots+h_{n}+\bar{h}_{1}+\dots+\bar{h}_{n}} G_{i_{1},\dots,i_{n}}(e^{\tau} z_{1},\dots,e^{\tau} z_{n}),$$
  где $(z_{1},\dots,z_{n})\in M_{n},\quad h_{j}=h_{i_{j}}$
\end{axiom}

Заметим, что двух- и трех-точечные функции теперь полностью
определены:
$G_{ij}(z_1,z_2,\bar z_1, \bar z_2)=C_{ij} (z_1-z_2)^{-(h_i+h_j)}(\bar
z_1-\bar z_2)^{-(\bar h_i+\bar h_j)}$ и
$G_{ijk}=C_{ijk}z_{12}^{-h_1-h_2+h_3}
z_{23}^{-h_2-h_3+h_1}z_{13}^{-h_1-h_3+h_2} 
\bar z_{12}^{-\bar h_1-\bar h_2+\bar h_3} \bar z_{23}^{-\bar h_2-\bar
  h_3+\bar h_1}\bar z_{13}^{-\bar h_1-\bar h_3+\bar h_2}$, где
$C_{ij}, C_{ijk}\in \mathbb{C}$.

При таких предположениях теория не очень практичная -- мы ничего не
знаем про структуру гильбертова пространства ${\cal H}$ и поля
$\varphi_j$. Поэтому из физических предположений мы предполагаем
также, что среди полей есть четыре особенных поля -- компоненты
тензора энергии-импульса. 

\begin{axiom}
  (Существование тензора энергии-импульса)
  Среди полей $\varphi_{i},\; i\in B_{0}$ есть четыре поля
  $T_{\mu\nu},\; \mu,\nu=0,1$, такие, что $T_{\mu\nu}=T_{\nu\mu},\quad
  T_{\mu\nu}^{*}=T_{\nu\mu}(\Theta(z)),\quad \partial_{0} T_{\mu
    0}+\partial_{1}T_{\mu 1}=0$, скейлинговая размерность поля
  $d(T_{\mu\nu})=h_{\mu\nu}+\bar{h}_{\mu\nu}=2$,  конформный спин $s(T_{00}-T_{11}\pm 2i T_{01})=\pm 2$. 
\end{axiom}
Можно показать, что $\mathrm{tr} T_{\mu\nu}=0$ и $T=T_{00}-i T_{01}$ не зависит от $\bar z$, то есть $\bar \partial T=0$. Операторы $L_{n}=\oint \frac{dz}{2\pi i} z^{n+1} T(z)$
удовлетворяют коммутационным соотношениям алгебры Вирасоро 
$$ [L_m,L_n]=(m-n)L_{m+n}+\frac{c}{12}(m^3-m)\delta_{m+n,0}.$$

Примарными будем называть поля $\varphi_{i}, i\in B_{0}$, такие, что
$[L_{n}, \varphi_{i}(z)]=z^{n+1}\partial
\varphi_{i}(z)+h_{i}(n+1)z^{n}\varphi_{i}(z)$,  $\forall n\in\mathbb{Z}$.

Для каждого примарного поля $\varphi_{i}$ можно определить конформное
семейство $[\varphi_{i}]$, состоящее из вторичных полей
$\varphi_{i}^{\alpha}(z)=L_{-\alpha_{1}}(z)\dots
L_{-\alpha_{n}}(z)\varphi_{i}(z)$, где $L_{-n}(z)=\frac{1}{2\pi
  i}\oint\frac{T(\xi)}{(\xi-z)^{n+1}} d\xi$. Заметим, что
$L_{n}(0)=L_{n}$ и корреляционные функции вторичных полей могут быть
выражены через корреляционные функции примарных.  Последняя аксиома
определяет операторное разложение:
\begin{axiom}
Корреляционные функции примарных полей  при $z_{i}\to z_{j}$ удовлетворяют уравнению:
$$\left<\Omega|\varphi_{i_{1}}(z_{1})\dots\varphi_{i}(z_{i})\dots \varphi_{j}(z_{j})\dots \varphi_{i_{n}}(z_{n})|\Omega\right>=$$
$$  \sum_{k\in B_{0}}C_{ijk} (z_{i}-z_{j})^{h_{k}-h_{i}-h_{j}} \left<\Omega|\varphi_{i_{1}}(z_{1})\dots\varphi_{k}(z_{k})\dots \varphi_{i_{n}}(z_{n})|\Omega\right>$$
$$  +\mbox{регулярные члены}$$
\end{axiom}

Если дополнительно предположить, что операторное разложение
ассоциативно (так называемый ``конформный бутстрап''),  то вся теория
определяется набором примарных полей $\varphi_{j}, j\in B_{0}$,  их
размерностями и коэффициентами операторного разложения $C_{ijk}$. 

Поля разделились на модули алгебры Вирасоро. Каждому полю $\varphi_i$
соответствует состояние $\lim_{z\to 0}\varphi_i(z)\Omega$. Можно
считать, что гильбертово пространство состояний тоже раскладывается на
модули алгебры Вирасоро.

В модуле Верма может оказаться такое состояние $\left|\omega\right>$,
что $L_{n}\left|\omega\right>=0, \quad \forall  n>0$. Действие
понижающих операторов на такие состояния порождает  инвариантное
подпространство, то есть модуль Верма оказывается  приводимым. Такие
состояния обладают нулевой нормой и  исключаются. Их иногда называют
нулевыми состояниями (null-state).  Как искать  такие состояния?

На уровне один
$L_{1}\left(L_{-1}\left|h\right>\right)=0\Longrightarrow h=0$, то есть
все состояния, полученные из  вакуума понижающим оператором являются
нефизическими состояниями с нулевой нормой, как и должно быть. 

На уровне два будем искать нулевое состояние в виде
$\left|\omega\right>=(L_{-2}+a L_{-1}^{2})\left| h \right> $. Из
$L_{1}\left|\omega\right>=0$ получаем
$$ 0=([L_{1},L_{-2}] +a [L_{1},L_{-1}^{2}])\left|h\right> = (3+2a(2h+1))L_{-1}\left|h\right>$$
То есть  $ a=-\frac{3}{2(2h+1)}$
Аналогично, действуя $L_{2}$ получаем условие $ c=2h\frac{5-8h}{2h+1}$

Так как состояние $(L_{-2}-\frac{3}{2(2h+1)}
L_{-1}^{2})\left|h\right>$ имеет нулевую норму, для корреляторов
получаем
 $$\left<(L_{-2}+aL_{1}^{2})\phi_{1}(w_{1}) \dots \phi_{n}(w_{n}) \right>=0.$$ 
Из определения $L_{n}$ можно переписать это условие в виде дифференциального уравнения:
$$\left[  a \frac{\partial^{2}}{\partial w_{1}^{2}} +\sum_{j\neq 1} \left(\frac{h_{j}}{(w_{1}-w_{j})^{2}}+\frac{1}{w_{1}-w_{j}} \frac{\partial}{\partial w_{j}}\right)\right] \left< \phi_{1}(w_{1}) \dots \phi_{n}(w_{n}) \right> =0$$

То есть мы получили дифференциальные уравнения на корреляционные функции. 

Рассмотрим в качестве примера модель Изинга. Ей соответствует
конформная теория поля с тремя примарными полями
$\varphi_{1}: (0,0),\quad \varphi_{2}:
\left(\frac{1}{2},\frac{1}{2}\right),\quad
\varphi_{3}:\left(\frac{1}{16},\frac{1}{16}\right)$ и центральным
зарядом $1/2$. Заметим, что для поля спина $\sigma=\varphi_{3}$ с конформной
размерностью $h=1/16$ выполнено условие $  c=2h\frac{5-8h}{2h+1}$. То есть мы можем записать уравнение для корреляционных функций спинов. Рассмотрим четырехточечную функцию
   $$G^{(4)} =\langle \sigma(z_1,\bar z_1)\dots \sigma(z_4,\bar z_4)\rangle$$

Из требования конформной ковариантности для корреляционных функций можно переписать $G^{(4)}$ в виде
  $$G^{(4)} = \mathrm{const}\cdot \left(\frac{z_{13}z_{24}}{z_{12} z_{23}z_{34}z_{41}}\right)^{\frac{1}{8}} \overline{ \left(\frac{z_{13}z_{24}}{z_{12} z_{23}z_{34}z_{41}}\right)}^{\frac{1}{8}}F(x,\bar x),$$
где $x=\frac{z_{12}z_{34}}{z_{13}z_{24}}, z_{ij}=\left| z_{i}-z_{j}\right|$. Тогда дифференциальное уравнение на $F$ примет вид
 $$ \left( x(1-x) \frac{\partial^{2}}{\partial x^{2}} +(\frac{1}{2}-x)\frac{\partial}{\partial x}+\frac{1}{16}\right) F(x,\bar x)=0$$
и аналогично для $\bar x$.
Решение этого уравнения дается гипергеометрической функцией. В данном случае она упрощается до $f_{1,2}(x)=(1\pm \sqrt{1-x})^{\frac{1}{2}}$. Тогда  четырехточечную функцию можно записать в виде комбинации решений
$$  G^{(4)} =  \left|\frac{z_{13}z_{24}}{z_{12} z_{23}z_{34}z_{41}}\right|^{\frac{1}{4}}\sum_{i,j=1}^{2}a_{ij}f_{i}(x)f_{j}(\bar x)$$

Возвращаясь к физической системе мы должны положить $\bar x= x^{*}$, тогда из однозначности четырехточечной функции комбинация может быть только $a(|f_{1}(x)|^{2}+|f_{2}(x)|^{2})$. Из сравнения с операторным разложением для $\sigma(z)\sigma(w)$ нетрудно получить, что $a=1/2$.

В итоге для четырехточечной функции получается следующее выражение:
  $$G^{(4)} =\langle \sigma(z_1,\bar z_1)\dots \sigma(z_4,\bar z_4)\rangle = \frac{1}{2} \left|\frac{z_{13}z_{24}}{z_{12} z_{23}z_{34}z_{41}}\right|^{\frac{1}{4}}\left(\left|1+\sqrt{1-x}\right| + \left|1-\sqrt{1-x}\right|\right)$$

Таким образом конформная теория поля позволяет не только находить
критические индексы, но и получать явные выражения  для корреляционных
функций статистических моделей в критической  точке. 




\end{document}
