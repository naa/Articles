\documentclass[12pt]{article}
\usepackage{amsfonts}

%%%%%%%%%%%%%%%%%%%%%%%%%%%%%%%%%%%%%%%%%%%%%%%%%%%%%%%%%%%%%%%%%%%%%%%%%%%%%%%%%%%%%%%%%%%%%%%%%%%
\usepackage{amsmath,amssymb,amsthm}
\usepackage{multicol}
\usepackage[pdftex]{color}
\usepackage{graphicx}
\usepackage[russian]{babel}
\usepackage[utf8]{inputenc}
\usepackage{cmap}

%\usepackage[makeindex]{imakeidx}
\usepackage{makeidx}
\usepackage[plainpages=false,pdfpagelabels,pagebackref]{hyperref}
\usepackage{showidx}
\makeindex


\newtheorem{Def}{Определение}[section]
\newtheorem{theorem}{Теорема}
\newtheorem{statement}{Утверждение}
\newtheorem{Cnj}[Def]{Гипотеза}
\newtheorem{Prop}[Def]{Свойство}
\newtheorem{example}{Пример}[section]
\newtheorem{axiom}{Аксиома}

\newcommand{\co}[1]{\stackrel{\circ }{#1}}
\newcommand{\gf}{\mathfrak{g}}
\newcommand{\nfp}{\mathfrak{n}^{+}}
\newcommand{\nfm}{\mathfrak{n}^{-}}
\newcommand{\af}{\mathfrak{a}}
\newcommand{\uf}{\mathfrak{u}}
\newcommand{\sfr}{\mathfrak{s}}
\newcommand{\aft}{\widetilde{\mathfrak{a}}}
\newcommand{\afb}{\mathfrak{a}_{\bot}}
\newcommand{\hf}{\mathfrak{h}}
\newcommand{\hfb}{\mathfrak{h}_{\bot}}
\newcommand{\pf}{\mathfrak{p}}

\newcommand{\gfh}{\hat{\mathfrak{g}}}
\newcommand{\afh}{\hat{\mathfrak{a}}}
\newcommand{\sfh}{\hat{\mathfrak{s}}}
\newcommand{\bff}{\mathfrak{b}}
\newcommand{\hfg}{\hf_{\gf}}

\begin{document}
\title{Заметки по теории групп Ли, алгебр Ли и их представлений}

\maketitle

\begin{abstract}
  Здесь мы обсуждаем основные проблемы теории групп Ли, указываем на
  аналогию с теорией Галуа и приводим некоторые нерешенные задачи.
\end{abstract}

\section{Мотивация}
\begin{itemize}
\item Как появилась появилась теория групп Ли?
\item В чем состоит аналогия с теорией Галуа?
\item Какие задачи должна была решать теория непрерывных групп?
\item Какие задачи не смогла решить? Почему?
\item Какие задачи еще не решены?
\end{itemize}

\section{Теория групп Ли и теория Галуа}
\label{sec:galois}

Группы Галуа появились для решения полиномиальных (алгебраических)
уравнений, группы Ли -- для решения обыкновенных дифференциальных
уравнений. 

Галуа ввел конечные группы для решения уравнений второй, третьей и
четвертой степени и доказательства неразрешимости уравнений пятой
степени. Речь идет о группах инвариантности уравнений.




Ли аналогично ввел непрерывные группы инвариантности дифференциальных
уравнений.

Группы делятся на простые и разрешимые. Разрешимые связаны с
интегрируемостью \index{интегрируемость} и упрощением обыкновенных дифференциальных уравнений. 

Дифференциальные уравнения могут быть разрешены, если их группа
инвариантности разрешима. 

Классические функции математической физики возникают как матричные
элементы представлений простых групп Ли. 


\printindex

\bibliography{bibliography}{}
\bibliographystyle{utphys}

\end{document}
