\documentclass[a4paper,12pt]{article}
\usepackage[unicode,verbose]{hyperref}
\usepackage{amsmath,amssymb,amsthm} \usepackage{pb-diagram}
\usepackage{ucs}
\usepackage[utf8x]{inputenc}
\usepackage[russian]{babel}
\usepackage{cmap}
\usepackage{graphicx}
\pagestyle{plain}
\theoremstyle{definition} \newtheorem{Def}{Definition}
\title{WZW-модели и теория Черна-Саймонса}
\author{Антон Назаров\\ antonnaz@gmail.com}
\begin{document}
\maketitle
\begin{abstract}
WZW-модель - это сигма-модель с дополнительным членом Весса-Зумино, связанным с топологией. Существует связь таких моделей с пределом бесконечного числа цветов в калибровочных теориях. WZW-модели также связаны с моделями топологической теории поля - моделями Черна-Саймонса-Виттена в трех измерениях. Квантовые состояния в теории Черна-Саймонса с уровнем $k$ ($SU(2)_k$) эквивалентны квантовым состояниям WZW-модели, связанной с представлениями уровня $k$ аффинной алгебры Ли $\widehat{su(2)}$.
\end{abstract}

\section{Конформная инвариантность}
\label{sec:conformal-invariance}

Откуда берется конформная инвариантность в физике конденсированного состояния и в теории струн?

Конформные преобразования:

    \begin{align} & M_{\mu\nu} \equiv i(x_\mu\partial_\nu-x_\nu\partial_\mu) \,, \\ &P_\mu \equiv-i\partial_\mu \,, \\ &D \equiv-ix_\mu\partial^\mu \,, \\ &K_\mu \equiv i(x^2\partial_\mu-2x_\mu x_\nu\partial^\nu) \,, \end{align} 

Конформная инвариантность тесно связана с масштабной инвариантностью и иногда даже прямо из нее следует. В физике конденсированного состояния масштабная инвариантность имеет место при фазовом переходе (стремление параметра порядка к бесконечности). Флуктуации при фазовом переходе конформно-инвариантны, что позволяет классифицировать фазовые переходы в терминах конформной теории поля \cite{Polyakov:1970xd}, \cite{belavin1984ics}.

В теории струн действие инвариантно относительно репараметризаций мировой поверхности, в том числе конформных преобразований. 
Однако эта инвариантность не может быть сохранена при квантовании, поэтому возникает конформная аномалия и центральный заряд алгебры Вирасоро.

В двумерной теории поля локальная конформная группа бесконечномерна, хотя глобальная конформная группа конечномерна. Например, в конформной теории поля на сфере Римана глобальные конформные преобразования образуют группу Мебиуса $PSL(2,\mathbb{C})$. Алгебра же инфинитезимальных преобразований в классической теории --- это алгебра Витта, при квантовании возникает конформная аномалия, что соответствует центральному расширению алгебры. Так появляется алгебра Вирасоро.

\subsection{Конформная инвариантность в теории струн}
\label{sec:conformal-invariance-strings}

Действие свободной релятивистской частицы --- это длина траектории.
\begin{equation}
  \label{eq:1}
  \begin{split}
  x^{\mu}(0)=x^{\mu}_0,\; x^{\mu}(1)=x^{\mu}_1\\
  S=\int_{0}^{1} \sqrt{\left(\frac{dx^{\mu}}{d\tau}^2\right)}d\tau
\end{split}
\end{equation}
Оно Пуанкаре-инвариантно $x^{\mu}\to A^{\mu}_{\nu}x^{\nu}+ B^{\mu}$ и не зависит от репараметризаций $\tau\to f(\tau),\; x^{\mu}(\tau)\to x^{\mu}(f(\tau))$.

Аналогично, струнное действие дается площадью мировой поверхности
\begin{equation}
  \label{eq:2}
  \begin{split}
    X^{\mu}(x^1,x^2),\; \mu=1\dots d\\
    S_N [X^{\mu}(x^1,x^2)]=\int \sqrt{det \frac{\partial X^{\mu}}{\partial x^a}\frac{\partial X^{\mu}}{\partial x^b}} dx^1 dx^2
  \end{split}
\end{equation}
Такое действие называется действием Намбу. Оно инвариантно относительно преобразований Лоренца и репараметризаций.

Амплитуда перехода для струны дается следующей формулой:
\begin{equation}
  \label{eq:3}
  A=\int {\cal D}X^{\mu}(x^1,x^2)e^{-S_N[X^{\mu}(x^1,x^2)]}
\end{equation}
Сложность, как всегда, состоит в определении меры интегрирования. Она плохо определена как и в большинстве случаев, так что теория струн здесь не выделяется.

Мы хотим сделать замену ``переменных интегрирования'', чтобы избавиться от произвола в выборе параметризации мировой поверхности.

Теория должна быть инвариантна относительно репараметризаций:
\begin{equation}
  \label{eq:4}
  \begin{split}
    x^a \to f^a (x^1,x^2)\\
    X^{\mu}(x^1,x^2)\to \tilde{X}^{\mu}(x^1,x^2)=X^{\mu}(f^1(x^1,x^2),f^2(x^1,x^2))
  \end{split}
\end{equation}
Но в действии Намбу эта инвариантность не видна, поэтому мы перейдем к эквивалентному действию Полякова:
\begin{equation}
  \label{eq:5}
  \begin{split}
    S_P [g^{ab}(x^1,x^2), X^{\mu}(x^1,x^2)]=\int g^{ab}(x)\partial_a X^{\mu} \partial_b X^{\mu}\sqrt{g} \;d^2 x\\
    g=\det g_{ab}
  \end{split}
\end{equation}

Действия Полякова и Намбу эквивалентны в том смысле, что уравнения движения совпадают после исключения вспомогательного поля $g_{ab}$.
\begin{equation}
  \label{eq:6}
  \delta_{X}S_N=0 \Longleftrightarrow \left\{
      \begin{aligned}
        \delta_X S_P=0\\
        \delta_g S_P=0
      \end{aligned}
      \right.
\end{equation}
Действие Полякова инвариантно относительно преобразований Лоренца и репараметризаций мировой поверхности:
\begin{equation}
  \label{eq:7}
  \begin{split}
    x^a \to f^a (x^1,x^2)\\
    X^{\mu}(x^1,x^2)\to \tilde{X}^{\mu}(x^1,x^2)=X^{\mu}(f^1(x^1,x^2),f^2(x^1,x^2))\\
    g_{ab}(x)\to \tilde{g}_{ab}(x)=\frac{\partial f^{a_1}}{\partial x^{a}}\frac{\partial f^{b_1}}{\partial x^b} g_{a_1 b_1}(x)
  \end{split}  
\end{equation}
Кроме того, есть дополнительная симметрия, которая называется вейлевской или конформной инвариантностью:
\begin{equation}
  \label{eq:8}
  \begin{split}
    g_{ab}(x)\to e^{\sigma(x)}g_{ab}(x)\\
    g\to e^{\sigma} g\\
    g^{ab}\to e^{-\sigma} g^{ab}\\
    S_P[e^{\sigma(x)} g_{ab}(x),X^{\mu}] = S_P[g_{ab}(x),X^{\mu}] 
  \end{split}
\end{equation}

То есть у нас довольно большая свобода --- мы можем, например, получить произвольную метрику на сфере из постоянной метрики путем репараметризации и преобразования Вейля
\begin{equation}
  \label{eq:9}
  g_{ab}(x)=[e^{\sigma(x)}\hat{g}_{ab}]^{x^a\to f^a (x)}
\end{equation}
Для поверхностей большего рода существуют дискретные семейства эквивалентных метрик.

We could not preserve all the symmetries during quantization. So we choose to save Lorenz invariance and reparametrisation invariance and break conformal invariance. It is so called conformal anomaly.

\begin{equation}
  \label{eq:10}
  A=\int {\cal D}X^{\mu}(x^1,x^2)e^{-S_P[g_{ab},X^{\mu}(x^1,x^2)]}
\end{equation}

It is easy to see that norm is invariant under reparametrisations:
\begin{equation}
  \label{eq:11}
  \begin{split}
    \|\delta X^{\mu}\|^2 = \int \left(\delta X^{\mu}(x)\right)^2 \sqrt{g} d^2x\\
    \|\delta g_{ab}\|^2 = \int \delta g_{a_1 b_1}\delta g_{a_2 b_2} g^{a_1 a_2} g^{b_1 b_2} \sqrt{g}\; d^2x
  \end{split}
\end{equation}

To investigate what happens top Weyl invariance after quantization let's introduce so called effective action. We integrate out fields $X^{\mu}$ and get following result.
\begin{equation}
  \label{eq:12}
  e^{-S_X^{eff}[g_{ab}]} \overset{def}{=} \int {\cal D}X^{\mu} e^{-S_P [g_{ab},X]}
\end{equation}
\begin{equation}
  \label{eq:13}
  \begin{split}
    S^{eff}_X [e^{\sigma(x)} g_{ab}(x) ] = S_X^{eff}[g_{ab}]+\frac{d}{48\pi}W[g,\sigma]\\
    W[g,\sigma]=\int \left[ \frac{1}{2} g^{ab}\partial_a\sigma \partial_b\sigma+R[g]\sigma+\mu e^{\sigma}\right] \sqrt{g}\; d^2x
  \end{split}
\end{equation}
Here $R$ is scalar curvature of $g_{ab}$, and $\mu$ is a parameter depending on regularisation, for example for proper time regularisation $\mu=\frac{1}{4\pi\epsilon}$. The term $W[g,\sigma]$ is called Liouville's action.

{\bf Notice} $\int R[x] \sqrt{g}d^2x = 4\pi \chi_E$, $\chi_E=2-2h$, $\chi_E$ - Euler characteristic of the surface, $h$ - genus of the surface.

$W[g,\sigma]$  is called conformal anomaly.

Variance of action
\begin{equation}
  \label{eq:14}
  \begin{split}
    \delta S[g_{ab},X] = -\frac{1}{4\pi} \int  \delta g^{ab} T_{ab} \sqrt{g} \; d^2x\\
    g_{ab}\to g_{ab}+\delta g_{ab}
  \end{split}
\end{equation}
\begin{equation}
  \label{eq:15}
  \begin{split}
    g_{ab} \to e^{\sigma} g_{ab} \xrightarrow[\sigma \ll 1]{} (1+\delta\sigma) g_{ab}\\
    \delta S [g_{ab},X] = -\frac{1}{4\pi} \int \delta\sigma g^{ab} T_{ab} \sqrt{g}\, d^2 x
  \end{split}
\end{equation}
\begin{equation}
  \label{eq:16}
    \delta S =0 \Rightarrow g^{ab}T_{ab}=0
\end{equation}
We see that Weyl invariance is equivalent to tracelessness of energy-momentum tensor.

After the quantization we have
\begin{equation}
  \label{eq:17}
  g^{ab}\left< T_{ab} \right> = -\frac{d}{12} R[g]
\end{equation}
\begin{Def}
  {\bf Two dimensional conformal field theory} - is the theory that transforms as in (\ref{eq:13}).

  The factor before $W[g,\sigma]$ is called {\it central charge}.
\end{Def}


At this moment we have too much freedom in our Feynman integral (\ref{eq:10}). We have diffeomorphisms (\ref{eq:4}) and Weyl transformations (\ref{eq:8}). We will fix the gauge so that the metric is given by the formula 
\begin{equation}
  \label{eq:20}
  g_{ab}=[e^{\phi} \hat{g}_{ab}]^{f^a}
\end{equation}

Faddeev-Popov ghosts appear as usual after this procedure.

\begin{equation}
  \label{eq:19}
  \begin{split}
    A=\int {\cal D} X^{\mu}(x^1,x^2)e^{-S_P[g_{ab},X^{\mu}(x^1,x^2)]} = \\
    = \int {\cal D}\phi\; {\cal D}X\; \mathcal{D} B_{ab}\; \mathcal{D} C^a e^{-S_P [e^{\phi}\hat{g}_{ab},X]-
      S_{gh}[e^{\phi}\hat{g}_{ab},B,C]}
  \end{split}
\end{equation}
Here $B,C$ are ghost fields, $B_{ab}(X)$ is symmetric and traceless tensor. All the functional integrals use metric $e^{\phi}\hat{g}$. 
\begin{equation}
  \label{eq:21}
  S_{gh} = \int g^{ac} B_{ab} \nabla_c C^b \sqrt{g} \, d^2x
\end{equation}
We can fix Weyl invariance so that measures of integration over matter fields and ghosts fields in (\ref{eq:19}) don't depend on $\phi$. Then we get conformal anomaly in the exponent:
\begin{equation}
  \label{eq:22}
  A = \int \mathcal{D}_{e^{\phi} \hat{g}} \mathcal{D}_{\hat{g}} X \mathcal{D}_{\hat{g}} B_{ab} \mathcal{D}_{\hat{g}} C^a e^{-S_P [e^{\phi}\hat{g}_{ab},X]-S_{gh}[e^{\phi}\hat{g}_{ab},B,C]+\frac{d-26}{48\pi} W[\hat{g}_{ab},\phi]}
\end{equation}
$\frac{d}{48\pi}$ comes from matter fields, and $\frac{-26}{48\pi}$ from ghost fields, since both actions are actions of two-dimensional conformal field theory.

Ordinary bosonic string lives in 26-dimensional space and doesn't interact with Liouville's gravity. Such a string is called ``critical string'' since it can exist only in critical dimension. Other string models are called non-critical strings.

First integration in (\ref{eq:22}) is very inconvenient, since we integrate over $\phi$ with measure depending on $\phi$. This measure $\mathcal{D}_{e^{\phi}\hat{g}}\phi$ corresponds to interval
\begin{equation}
  \label{eq:23}
  \|\delta\phi\|^2_P = \int e^{\phi} (\delta\phi)^2 \sqrt{\hat{g}}\, d^2 x
\end{equation}
It is non-linear and is not invariant under the translations $\phi(x)\to \phi(x)+\eta(x)$.
So for some time this was a problem since nobody knew how to work with such objects.

But David-Distler-Kawai proposed to change action in order to take into account change of measure of integration to normal measure.

They conjectured that the amplitude can be written in the form
\begin{equation}
  \label{eq:24}
  A=\int \mathcal{D}_{\hat{g}}\phi\, \mathcal{D}_{\hat{g}} X\, \mathcal{D}_{\hat{g}} (B,C)\, e^{-S_P[\hat{g},X]-S_{gh}[\hat{g},B,C]-S_L[\hat{g},\phi]}
\end{equation}
where the measure $\mathcal{D}_{\hat{g}} \phi$ corresponds to the interval $\|\delta\phi\|^2_{DDK} = \int (\delta\phi)^2 \sqrt{\hat{g}}\, d^2x$ and it is invariant under the translations $\phi\to \phi+\eta$ ($\delta\phi\to \delta\phi$). 
$S_L$ here is Liouville's action with two undefined for now parameters $Q$ and $b$:
\begin{equation}
  \label{eq:25}
  S_L[\hat{g},\phi] = \frac{1}{8\pi} \int \left[ \frac{1}{2} \hat{g}^{ab} \partial_a \phi \partial_b \phi + QR\phi +\mu e^{2b\phi}\right] \sqrt{\hat{g}} \, d^2x
\end{equation}
We choose following normalization:
\begin{equation}
  \label{eq:26}
  \langle \phi(x) \phi(0) \rangle = \log |x|^2
\end{equation}
Liouville's term should be invariant under dilation $\hat{g}_{ab}\to e^{\sigma}\hat{g}_{ab}$.

After we put
\begin{equation}
  \label{eq:27}
  Q=\frac{1}{b}+b
\end{equation}
Liouville's action becomes conformal, so that
\begin{equation}
  \label{eq:28}
  \begin{split}
    \int \mathcal{D}_{\hat{g}} \phi e^{-S_L[\hat g,\phi]} = e^{-S_L^{eff}[\hat g]}\\
    S_L^{eff} [e^{\sigma}\hat{g}]=S_L^{eff}[\hat g] +\frac{C_L}{12} W[\hat g,\sigma]
  \end{split}
\end{equation}
Where
\begin{equation}
  \label{eq:29}
  C_L=1+6 Q^2
\end{equation}

Then we require total central charge to vanish:
\begin{equation}
  \label{eq:30}
  C_X+C_{gh}+C_L=C_{tot}=0
\end{equation}

Now the amplitude doesn't depend on $\hat{g}$ and parameters $Q$,$b$ are fixed.
\begin{equation}
  \label{eq:18}
  d+1+6\left(b+\frac{1}{b}\right)^2=26
\end{equation}
In some finite area we can assume
\begin{equation}
  \label{eq:31}
  \hat{g}_{ab}=\delta_{ab}
\end{equation}
There also could be additional term connected with the topology.

Let's change variables
\begin{equation}
  \label{eq:32}
  \begin{split}
    x^1+i\, x^2=z\\
    x^1-i\, x^2=\bar{z}\\
    (dx^1)^2+(dx^2)^2=dz\,d\bar{z}
  \end{split}
\end{equation}
For fields and terms of the action we have
\begin{equation}
  \label{eq:33}
  \begin{split}
    C^z=C,\; C^{\bar{z}}=\bar{C}\\
    B_{zz}=B, \; B_{\bar{z}\bar{z}}=\bar{B}\\
    S_{gh}=\int d^2z [B\bar{\partial}C+\bar{B}\partial \bar{C}]\\
    S_P[X]=\int d^2z \bar{\partial}X \partial X\\
    S_L[\phi] \propto \int d^2z \left[\bar{\partial} \phi \partial \phi+\mu e^{2 b \phi}\right] + \mbox{(topological terms)}
  \end{split}
\end{equation}
\begin{equation}
  \label{eq:34}
  e^{-S^{eff}[g]}=\int \mathcal{D} X e^{-\int g^{ab}\partial_a X \partial_b X \sqrt{g}\, d^2x} = \int \mathcal{D}_g X e^{-\int X \Delta X\sqrt{g}\, d^2x}
\end{equation}
Laplace operator depends on metric
\begin{equation}
  \label{eq:35}
  \Delta = \frac{1}{\sqrt{g}}\partial_a \sqrt{g} g^{ab}\partial_b
\end{equation}

It is self-conjugate operator.

We can represent field $X$ as combination of eigenfunctions of Laplace operator $X(x)=\sum c_n X_n,\; \Delta X_n = \lambda_n X_n,\; \lambda_n=\lambda_n[g]$. Then we will integrate over all eigenfunctions:
\begin{equation}
  \label{eq:36}
  e^{-S^{eff}[g]}=\left(\prod_{n} \lambda_n\right)^{-\frac{1}{2}}=(\det \Delta)^{-\frac{1}{2}}
\end{equation}
Obviously we could not calculate $\lambda_n$, but we can get the determinant.
\begin{equation}
  \label{eq:37}
  S^{eff}[g]=\log \det \Delta=tr \log \Delta
\end{equation}
\begin{equation}
  \label{eq:38}
  \log\frac{\lambda}{\lambda_0}=\int_0^{\infty}\frac{e^{-\lambda t}-e^{\lambda_0 t}}{t}dt
\end{equation}
\begin{equation}
  \label{eq:39}
  \delta S^{eff}[g]=tr (\log\Delta -\log\Delta_{g_0})=\delta tr \int_0^{\infty}\frac{e^{-\Delta t}}{t} dt
\end{equation}
The last integral diverges, so we need to regularise it. We will use heat kernel regularisation.
\begin{equation}
  \label{eq:40}
  \delta S^{eff}[g]=\delta tr \int_{\epsilon}^{\infty}\frac{e^{-\Delta t}}{t} dt  
\end{equation}
As we already mentioned, $\hat{g}_{ab}$ could be put equal to $\rho dz d\bar{z}$ in finite area. Then
\begin{equation}
  \label{eq:41}
  \Delta=\frac{1}{\rho}\bar{\partial}\partial
\end{equation}
For variation of effective action we have
  \begin{multline}
  \label{eq:42}
    \delta_{\rho}S^{eff}[g]=\delta_{\rho} tr \int_{\epsilon}^{\infty} e^{-\frac{1}{\rho}\bar{\partial}\partial t}\frac{dt}{t}=\\
    - tr \int_{\epsilon}^{\infty}\delta_{\Delta}e^{-\Delta t} dt=
    -tr \int_{\epsilon}^{\infty}\frac{\delta\rho}{\rho}\Delta e^{-\Delta t} dt=\\
    -tr \int_{\epsilon}^{\infty}\frac{\delta\rho}{\rho}\frac{d}{dt} e^{-\Delta t} dt=-tr \frac{\delta\rho}{\rho}e^{-\Delta \epsilon}
  \end{multline}

Let's introduce Green function for heat equation
\begin{equation}
  \label{eq:43}
  G(t,x,y)=\sum_n e^{-\lambda_n t}X_n(x) X_n(y)
\end{equation}
\begin{equation}
  \label{eq:44}
  \begin{split}
    \frac{\partial G}{\partial t}=\Delta_x G(t,x,y)\\
    G(o,x,y)=\delta^2 (x-y)\\
    G(t,x,x)=\frac{1}{4\pi t}+aR+O(t)
  \end{split}
\end{equation}
Then for variation of action we have
\begin{equation}
  \label{eq:45}
  \delta S^{eff}[g]=\int d^2x \frac{\delta\rho}{\rho}\left(\frac{1}{4\pi\epsilon}+aR\right)
\end{equation}



In Conformal Field Theory 
\begin{equation}
  \label{eq:46}
  g^{ab}\left<T_{ab}\right>=\frac{C}{12}R[g]
\end{equation}
$C$ is a central charge.

For flat metric $g_{ab}=\delta_{ab}$ we have
\begin{equation}
  \label{eq:47}
  g^{ab}\langle T_{ab} \rangle=0
\end{equation}
Then we put $T_{zz}=T,\quad T_{\bar{z}\bar{z}}=\bar{T},\quad T_{z\bar{z}}=T_{\bar{z}z}$. In flat space from continuity equation $\bar{\partial}T_{zz}+\partial T_{\bar{z}\bar{z}}=0$ we have
\begin{equation}
  \label{eq:49}
  \begin{split}
    T_{z\bar{z}}=0\\
    \bar{\partial}T_{zz}=\bar{\partial}T=0,\quad T \mbox{- holomorphic}\\
    \partial T_{\bar{z}\bar{z}}=\partial \bar{T}=0,\quad \bar{T} \mbox{- antiholomorphic}\\
  \end{split}
\end{equation}
In curved space continuity equation reads $\nabla_a T^{ab}=0$ and we can choose coordinate system where $g_{ab}dx^a dx^b = \rho dz\, d\bar{z}=e^{\sigma}dz\,d\bar{z}$. Then we introduce pseudo energy $T$:
\begin{equation}
  \label{eq:50}
  T_{zz}+\frac{C}{12}t_{zz}=T
\end{equation}
where $t_{zz}=(\partial \sigma)^2-2\partial^2 \sigma$. It is easy to check that $\bar{\partial}T=0$ in this case.

For transformations $z\to w(z)$
\begin{equation}
  \label{eq:51}
  \begin{split}
    T_{zz}\to \left(\frac{dw}{dz}\right)^2 T_{ww}(w(z))\\
    t\to \left(\frac{dw}{dz}\right)t(w)+2\{w,z\}\\
    \mbox{where}\; \{w,z\}=\left(\frac{\partial_z^3 w}{\partial_z w}-\frac{3}{2}\left(\frac{\partial_z^2 w}{\partial_z}\right)^2\right)\\
    T\to \left(\frac{dw}{dz}\right)^2 T(w)+\frac{C}{12}\{w,z\}\\
  \end{split}
\end{equation}
Energy-momentum tensor is the generator of coordinate transformations $x^a\to x^a+\epsilon^a(x)$:
\begin{equation}
  \label{eq:52}
  \delta_{\epsilon}\langle A_1(x_1)\dots A_N(x_N)\rangle = \int d^2x\, \partial_a \epsilon_b(x)\langle T_{ab}(x) A_1(x_1)\dots A_N(x_N)\rangle
\end{equation}

\begin{equation}
  \label{eq:53}
  \begin{split}
    T_{aa}=0\\
    \partial_a \epsilon_b+\partial_b \epsilon_a\sim \delta_{ab}\\
    \epsilon^a=\epsilon x^a \;\mbox{or}\; x^a\to \frac{x^a}{(\vec x)^2}
  \end{split}
\end{equation}
Correlation functions don't change under conformal transformations.
\begin{equation}
  \label{eq:54}
  \delta_{\epsilon}A(x)=\int_{\partial D} dy^a\, \epsilon^b \tilde{T}_{ab}(y)A(x)-\int_{D}d^2y\, \partial^a \epsilon^b(y)T_{ab}(y)A(x)
\end{equation}
Where $\tilde{T}_{ab}=\epsilon_{ac}T_{cb}$.

Let
\begin{equation}
  \label{eq:55}
  \begin{split}
    \partial_a \epsilon_b+\partial_b \epsilon_a\sim \delta_{ab}\\
    \bar{\partial}\epsilon=0\\
    z\to z+\epsilon(z,\bar{z})\\
    \bar{z}\to\bar{z}+\bar{\epsilon}(z,\bar{z})
  \end{split}
\end{equation}
\begin{equation}
  \label{eq:56}
  \delta_{\epsilon}A(z,\bar{z})=\oint_z du\, \epsilon(u) T(u) A(z,\bar{z})
\end{equation}
Substituting $T(z)$ in place of arbitrary operator $A$ we get 
\begin{equation}
  \label{eq:57}
  \delta_{\epsilon}T(z)=\oint_z du\, \epsilon(u)T(u)T(z)
\end{equation}

If $w(z)=z+\epsilon(z)$ where $\epsilon$ is small, $T$ transforms as following:
\begin{equation}
  \label{eq:58}
  T(z)\to \epsilon T'(z) + 2\epsilon' T(z)+\frac{C}{12}\epsilon'''
\end{equation}
\begin{equation}
  \label{eq:59}
  \delta_{\epsilon}T(z)=\oint du\, \epsilon(u) T(u) T(z) = \epsilon T'(z) + 2\epsilon' T(z)+\frac{C}{12}\epsilon'''
\end{equation}
We suppose that for full set of operators $\{A_j(x)\}$ there exists operator product expansion:
!!! How can it be proved ????
\begin{equation}
  \label{eq:60}
  A_i(x)A_j(0)=\sum C^k_{ij}(x)A_k(0)
\end{equation}
Then for product $T(u)T(z)$ we have
\begin{equation}
  \label{eq:61}
  T(u)T(z)=\sum C_k(u-z) A_k(z)
\end{equation}
We can recover the poles from formula (\ref{eq:59}):
\begin{equation}
  \label{eq:62}
  T(u)T(z)=\frac{C}{2(u-z)^4}+\frac{2T(z)}{(u-z)^2}+\frac{T'(z)}{u-z}+\mbox{non-singular terms}
\end{equation}
Now introduce the operators $L_n$:
\begin{equation}
  \label{eq:63}
  L_n A(z)\equiv \frac{}{2\pi i}\oint_{C}du\, (u-z)^{n+1}T(u)A(z)
\end{equation}
It is possible to show that for $n,m\in \mathbb{Z}$
\begin{equation}
  \label{eq:64}
  [L_n,L_m]=(n-m) L_{n+m}+\frac{C}{12}(n^3-n)\delta_{n,-m}
\end{equation}
and $\{L_n\}$ constitute Virasoro algebra. 

All the fields in the theory are sums of multiplets of Virasoro algebra:
\begin{equation}
  \label{eq:65}
  \{A_j\}=\sum [\Phi_{\Delta,A}] %%% It is incorrect, fix this formula
\end{equation}

There is a primary field in every multiplet:
\begin{equation}
  \label{eq:66}
  \begin{split}
    \Phi_{\Delta}(z)\underset{z\to w(z)}{\longrightarrow} \left(\frac{dw}{dz}\right)^{\Delta}\Phi_{\Delta}(w(z))\\
    L_n \Phi=0,\quad n>0\\
    L_0 \Phi=\Delta \Phi\\
  \end{split}
\end{equation}
All other fields are built of primary fields and are called secondary:
\begin{equation}
  \label{eq:67}
  L_{-n_1}L_{-n_2}\dots \Phi_{\Delta}
\end{equation}

Correlation functions are expressed as combinations of correlation functions of primary fields.

The only thing that is yet undefined is the set of scaling dimensions of primary fields. So to fix the theory we should define $\Phi_a,\; \Delta_a,\; C^c_{ab}$. Then our theory is solved. If the number of primary fields is finite, such a model is called {\it minimal}.

For example, following fields are Weyl-invariant:
\begin{equation}
  \label{eq:68}
  O_a=\int \Phi_{\Delta}e^{2a\phi}\,d^2x,\quad \mbox{where}\;\Delta+a(Q-a)=1
\end{equation}
$O_a$ - is the main observable of string theory with gravitational dimension $\delta_a=-\frac{a}{b}$.
\begin{equation}
  \label{eq:69}
  \int\langle O_{a_1}\dots O_{a_n}\rangle_{\mu} e^{-\mu A}d\mu=\langle O_{a_1}\dots O_{a_n}\rangle_{A}=A^{\sum \delta_k},\quad \delta_k=\frac{a_k}{b}
\end{equation}
\subsection{Connection with string theory}

We can start from arbitrary conformal field theory, find $b$ from equation $C+6\left(b+\frac{1}{b}\right)^2=25$ and obtain some non-critical string theory. But for $d>1$ $b$ becomes complex and this fact leads to unsolved yet problems.

\subsection{Глобальная конформная инвариантность и уравнения Книжника-Замолодчикова}
\label{sec:knizhnik-zamolodchikov}

\subsection{Модулярные преобразования}
\label{sec:modular}


Конформная инвариантность в квантовой и классической теории поля.
Глобальная конформная инвариантность.
Уравнения Книжника-Замолодчикова.
Локальная конформная инвариантность. Бесследовость тензора энергии-импульса.
Голоморфная факторизация.

\section{Модели Весса-Зумино-Новикова-Виттена}
\label{sec:WZNW}

Связь WZW-моделей с топологией, обоснование с точки зрения предела калибровочной теории.

Стартуем с нелинейной $\sigma$-модели.
\begin{equation}
  \label{eq:48}
  S_0=\frac{1}{4a^2}\int d^2x\; Tr' (\partial^{\mu}g^{-1}\partial_{\mu}g)
\end{equation}
Здесь $a^2>0$ - положительный параметр, $g(x)\in G$ - поле со значениями в группе $G$, которую мы будем считать полупростой, а через $Tr'$ мы обозначили след, не зависящий от выбора представления алгебры Ли $\mathfrak{g} $
\begin{equation}
  \label{eq:70}
  Tr'=\frac{1}{x_{rep}}Tr
\end{equation}
$x_{rep}$ - индекс Дынкина представления.

В нелинейной $\sigma$-модели конформная инвариантность теряется на квантовом уровне.
Голоморфный и антиголоморфный токи не сохраняются по отдельности. Уравнения движения имеют вид    
\begin{equation}
  \label{eq:71}
  \partial^{\mu}(g^{-1}\partial_{\mu}g)=0
\end{equation}
Токи задаются выражением
\begin{equation}
  \label{eq:currents}
  J_{\mu}=g^{-1}\partial_{\mu}g
\end{equation}
или в комплексных координатах
\begin{equation}
  \label{eq:74}
  \begin{matrix}
    & \tilde{J}_z=g^{-1}\partial_z g, & \tilde{J}_{\bar{z}}=g^{-1}\partial_{\bar{z}}g\\
    & \partial_z \tilde{J}_{\bar{z}}+\partial_{\bar{z}}\tilde{J}_z=0 & \\
  \end{matrix}
\end{equation}
Члены в уравнении движения не разделяются, так как  $\partial_{\mu}(\epsilon^{\mu\nu}J_{\nu})\neq 0$.

Поэтому мы добавляем член Весса-Зумино к действию и переопределяем токи:
\begin{equation}
  \label{eq:72}
  J_z=\partial_z g\;g^{-1}, \qquad J_{\bar{z}}=g^{-1}\partial{\bar z}g
\end{equation}

Член Весса-Зумино имеет вид
\begin{equation}
  \label{eq:73}
\Gamma=  - \frac{i }{24\pi} \int_{B}\epsilon_{ijk} Tr'\left(
    \tilde g^{-1}\frac{\partial \tilde g}{\partial y^i}
      \tilde g^{-1}\frac{\partial \tilde g}{\partial y^j}
      \tilde g^{-1}\frac{\partial \tilde g}{\partial y^k}\right) d^3y
\end{equation}

Он определен на трехмерном многообразии $B$, ограниченном исходным двумерным пространством. 
Через $\tilde{g}$ мы обозначили продолжение поля  $g$ на $B$. Такое продолжение не единственно. В компактифицированном трехмерном пространстве компактное двумерное многообразие разделяет два трехмерных многообразия. Разность значений члена Весса-Зумино  $\Delta\Gamma$ на этих многообразиях дается правой частью уравнения (\ref{eq:73}) с интегралом, продолженным на все компактное трехмерное пространство. Так как оно топологически эквивалентно три-сфере, получаем
\begin{equation}
  \label{eq:75}
\Delta\Gamma=  - \frac{i }{24\pi} \int_{S^3}\epsilon_{ijk} Tr'\left(
    \tilde g^{-1}\frac{\partial \tilde g}{\partial y^i}
      \tilde g^{-1}\frac{\partial \tilde g}{\partial y^j}
      \tilde g^{-1}\frac{\partial \tilde g}{\partial y^k}\right) d^3y
\end{equation}
$\Delta\Gamma$ определен по модулю $2\pi i$, поэтому Евклидов функциональный интеграл с весом  $exp(-\Gamma)$ хорошо определен. Значит константа связи, умножаемая на этот член, должна быть целочисленной.

Теперь мы рассматриваем действи
\begin{equation}
  \label{eq:76}
  S=S_0+k\Gamma
\end{equation}
где $k$ - целое.
Уравнение движения для полного действия (\ref{eq:76}):
\begin{equation}
  \label{eq:77}
  \partial^{\mu}(g^{-1}\partial_{\mu}g)+\frac{a^2 ik}{4\pi}\epsilon_{\mu\nu}\partial^{\mu}(g^{-1}\partial^{\nu}g)=0
\end{equation}
В комплексных координатах оно записывается в виде
\begin{equation}
  \label{eq:78}
  (1+\frac{a^2 k}{4\pi})\partial_z(g^{-1}\partial_{\bar z}g)+(1-\frac{a^2 k}{4\pi})\partial_{\bar z}(g^{-1}\partial_z g)=0
\end{equation}
Видно, что при $a^2=\frac{4\pi}{k}$ у нас имеются интересующие нас законы сохранения
\begin{equation}
  \label{eq:79}
  \partial_z(g^{-1}\partial{\bar z}g)=0
\end{equation}
Для токов
\begin{equation}
  \label{eq:4}
  \partial_{\bar z}J=0,\quad \partial_z \bar J=0
\end{equation}

Решение классического уравнения движения имеет вид
\begin{equation}
  \label{eq:80}
  g(z,\bar z)=f(z)\bar f(\bar z)
\end{equation}
при произвольных $f(z)$ и $\bar f (\bar z)$.

Сохранение по отдельности токов $J_z,\; J_{\bar z}$ приводит к инвариантности действия при преобразованиях
\begin{equation}
  \label{eq:81}
   g(z,\bar z)\to \Omega(z)g(z,\bar z)\bar \Omega^{-1}(\bar z)
\end{equation}
где $\Omega,\;\bar \Omega \in G$. То есть мы получили локальную $G(z)\times G(\bar z)$-инвариантность. 

Для перехода к квантовому случаю мы переопределяем токи
\begin{equation}
  \label{eq:82}
  J(z)\equiv -k \partial_zg g^{-1}\quad \bar J(\bar z)=k g^{-1}\partial_{\bar z}g
\end{equation}
Тогда вариация действия при инфинитезимальных преобразованиях $\Omega=1+\omega,\; \bar \Omega =1+\bar \omega$ дается выражением
\begin{equation}
  \label{eq:83}
  \delta_{\omega,\bar\omega}S=\frac{i}{4\pi}\oint dz Tr' (\omega(z)J(z))-\frac{i}{4\pi}\oint d\bar z Tr'(\bar\omega(\bar z)\bar J(\bar z))
\end{equation}
Раскладывая токи
\begin{equation}
  \label{eq:85}
  \begin{aligned}
    J=\sum J^a t^a,\bar J=\sum \bar J^a t^a \\
    \omega=\sum \omega^a t^a\\
  \end{aligned}
\end{equation}
получаем
\begin{equation}
  \label{eq:86}
  \delta_{\omega,\bar \omega}S=-\frac{1}{2\pi i}\oint dz \sum\omega^a J^a+\frac{1}{2\pi i} \oint d\bar z \sum \bar \omega^a \bar J^a
\end{equation}
Мы также получили тождества Уорда $\delta\left< X\right>=\left<(\delta S)X\right>$
\begin{equation}
  \label{eq:87}
  \delta_{\omega,\bar \omega}\left< X \right>=-\frac{1}{2\pi i}\oint dz \sum\omega^a \left< J^a X\right>+
  \frac{1}{2\pi i} \oint d\bar z \sum \bar \omega^a \left< \bar J^a X\right>
\end{equation}
Для токов имеем
\begin{equation}
  \label{eq:88}
  \delta_{\omega}J=[\omega,J]-k\partial_z\omega,\quad \delta_{\omega}J^a=\sum i f_{abc}\omega^b J^c-k\partial_z\omega^a
\end{equation}
Операторное разложение для токов имеет вид 
\begin{equation}
  \label{eq:89}
  J^a(z) J^b(w) \sim \frac{k\delta_{ab}}{(z-w)^2}+\sum i f_{abc}\frac{J^c(w)}{(z-w)}
\end{equation}
Раскладывая токи в ряд, получаем
\begin{equation}
  \label{eq:90}
  \begin{aligned}
    J^a(z)=\sum_{n\in \mathbb Z}z^{n-1}J^a_n\\    
    \left[J^a_n,J^b_m\right]=\sum_c i f^{abc}J^c_{n+m}+kn\delta^{ab}\delta_{n+m,0}  
  \end{aligned}
\end{equation}
Теперь мы видим, что компоненты токов образуют аффинную алгебру Ли $\hat g$.  


Тензор энергии-импульса вводится при помощи конструкции Сугавары как сумма нормально упорядоченных компонент токов
\begin{equation}
  \label{eq:6}
  T(z)=\frac{1}{2(k+h^v)}\sum_a N(J^a J^a)(z)
\end{equation}
Здесь $h^v$ - дуальное число Кокстера.

Тензор энергии-импульса можно разложить на моды $L_n$
\begin{equation}
  \label{eq:91}
  L_n=\frac{1}{2(k+h^v)}\sum_a\sum_m:J^a_m J^a_{n-m}:
\end{equation}
Тогда коммутационные соотношения для мод  $L_n$ имеют вид
\begin{equation}
  \label{eq:92}
  \begin{aligned}
    \left[L_n,L_m\right]=(n-m)L_{n+m}+\frac{c}{12}(n^3-n)\delta_{n+m,0}\\
    \left[L_n,J^a_m\right]=-mJ^a_{n+m}
  \end{aligned}
\end{equation}

Таким образом, конструкция Сугавары --- это способ вложения алгебры Вирасоро в универсальную обертывающую аффинной алгебры Ли $\hat{g}$

Полная киральная алгебра модели Весса-Зумино-Виттена равна полупрямому произведению $Vir\ltimes \hat g$

Примарными называются поля, которые преобразуются ковариантно под действием $G(z)\times G(\bar z)$, как $g(z,\bar z)$. В терминах операторного разложения это свойство переформулируется следующим образом:
\begin{equation}
  \label{eq:84}
  \begin{aligned}
    J^a(z)g(w,\bar w)\sim \frac{-t^a g(w,\bar w)}{(z-w)}\\
    \bar J^a(z)g(w,\bar w)\sim \frac{ g(w,\bar w)t^a}{(z-w)}
  \end{aligned}
\end{equation}
Любое поле $\phi_{\lambda,\mu}$, преобразующееся ковариантно по отношению к некоторому представлению, заданному весом  $\lambda$ в голоморфном секторе и весом $\mu$ в антиголоморфном, является примарным полем WZW-модели.

В модах это свойство записывается в виде
\begin{equation}
  \label{eq:93}
  \begin{aligned}
    & (J_0^a \phi_{\lambda})=-t^a_{\lambda}\phi_{\lambda}\\
    & (J^a_n\phi_{\lambda})=0\quad \mbox{для}\; n>0\\
  \end{aligned}
\end{equation}
Мы можем сопоставить состояние $\left|\phi_{\lambda}\right>$ полю $\phi_{\lambda}$
  \begin{equation}
    \label{eq:94}
    \phi_{\lambda}(0)=\left|\phi_{\lambda}\right>
  \end{equation}
Тогда условия  (\ref{eq:93}) для примарных полей  дают
\begin{equation}
  \label{eq:95}
  \begin{aligned}
    & J_0^a\left|\phi_{\lambda}\right>=-t^a_{\lambda}\left|\phi_{\lambda}\right>\\
    & J^a_n\left|\phi_{\lambda}\right>=0 \quad \mbox{для}\; n>0 \\
  \end{aligned}
\end{equation}
Легко видеть, что действие генераторов алгебры Вирасоро имеет вид
\begin{equation}
  \label{eq:96}
  L_0\left|\phi_{\lambda}\right>=\frac{1}{2(k+h^v)}\sum_aJ^a_0J^a_0\left|\phi_{\lambda}\right>=\frac{(\lambda,\lambda+2\rho)}{2(k+h^v)}\left|\phi_{\lambda}\right>
\end{equation}
Здесь использовано явное выражение для собственных значений квадратичного оператора Казимира.

Примарные поля живут в интегрируемых конечномерных представлениях, так как бесконечномерные и неинтегрируемые поля отщепляются в корреляционных функциях.

Все вторичные состояния имеют вид
\begin{equation}
  \label{eq:97}
  J^{a_1}_{-n_1}J^{a_2}_{n_2}\dots\left|\phi_{\lambda}\right>
\end{equation}
Для корреляционных функций примарных полей можно получить уравнения Книжника-Замолодчикова, которые следуют из глобальной $G\times G$-инвариантности
\begin{equation}
  \label{eq:98}
  \left(\partial_{z_i}+\frac{1}{k+h^v}\sum_{i\neq j}\frac{\sum_a t^a_i\otimes t^a_j}{z_i-z_k}\right)
  \left<\phi_1(z_1)\dots \phi_n(z_n)\right>=0
\end{equation}
Таким образом, теория полностью определяется представлениями аффинной алгебры Ли $\hat{\mathfrak{g}} $.

\subsection{Литература}
\label{sec:wzw-literature}

Эта часть написана по книге \cite{difrancesco1997cft}, так же стоит отметить обзор \cite{Walton:1999xc}.

\subsection{Калибровочная WZW-модель}
\label{sec:gauged-wzw}

Добавить сюда действие G-WZW, обсудить связь с coset construction и branching functions.

(Добавить статью Gawedzki в ссылки и перевести оттуда содержательную часть).

Теперь мы знаем, что такое conformal blocks, так что это надо здесь объяснить при обсуждении G-WZW и coset construction.

\section{Теория Черна-Саймонса}
\label{sec:Chern-Simons}

Теория Черна-Саймонса --- это калибровочная теория, то есть классические полевые конфигурации в теории на $M$ с калибровочной группой $G$ описываются главным $G$-расслоением над $M$. Форму связности главного $G$-расслоения над $M$ обозначим через $A:M\to g$, она принимает значения в $g$. В общем случае связность $A$ определяется на отдельных картах, значения $A$ на разных картах связаны калибровочными преобразованиями. Калибровочные преобразования характеризуются тем, что ковариантная производная $D=d+\frac{1}{2}[A,\cdot]$ преобразуется в присоединенном представлении $G$. 

Тогда действие записывается в виде:
\begin{equation}
  \label{eq:1}
  S=\frac{k}{4\pi}\int_M \mathrm{tr}(A\wedge dA+\frac{2}{3}A\wedge A\wedge A)
\end{equation}


Кривизна связности
\begin{equation}
  \label{eq:2}
  F=DA=dA+A\wedge A\in \Omega^2(M,g)
\end{equation}
Уравнение движения
\begin{equation}
  \label{eq:3}
  \frac{\delta S}{\delta A}=0=\frac{k}{2\pi}F
\end{equation}
Решениями являются плоские связности, которые определяются голономиями вокруг нестягиваемых циклов на $M$. Плоские связности находятся в однозначном соответствии с классами эквивалентности гомоморфизмов из фундаментальной группы $M$ в калибровочную группу $G$. Также говорят о том, что плоские связности определяются своими голономиями и
\begin{equation}
  \label{eq:8}
  \mbox{критические значения}/\mbox{калибровочным преобразованиям}\leftrightarrow \mathrm{Hom}(\pi_1(M),G)/G
\end{equation}

Действие зависит только от топологии $M$, метрика нигде явно не появляется. Корреляторы тоже зависят только от топологии. 
Хотя действие и зависит от калибровки, статсумма в квантовой теории хорошо определена при целом $k$ и напряженность калибровочного поля обнуляется на границах $M$.

Если у $M$ есть граница $N=\partial M$, то есть дополнительные данные, которые описывают выбор тривиализации главного $G$-расслоения на $N$. Такой выбор задает отображение из $N$ в $G$. Динамика этого отображения описывается WZW-моделью на $N$ с уровнем $k$.  

Рассматриваем калибровочное преобразование действия Черна-Саймонса.
При калибровочном преобразовании $g$ $A$ преобразуется как
\begin{equation}
  \label{eq:5}
  A_{\mu}\to g^* A= g^{-1}A_{\mu}g+g^{-1}\partial_{\mu}g
\end{equation}
Для действия Черна-Саймонса имеем
\begin{equation}
  \label{eq:7}
  S(g^* A)=S(A) +\frac{k}{4\pi}\int_{\partial M}\mathrm{Tr} (A\wedge dg \; g^{-1})-2\pi k\int_M g^* \sigma
\end{equation}
Здесь 

\begin{equation}
  \label{eq:4}
  \sigma=\frac{1}{24\pi^2} \mathrm{Tr} (\mu\wedge\mu\wedge\mu)
\end{equation}
где $\mu=X^{-1} dX, X\in G$ - форма Маурера-Картана. 

Получаем добавку в \eqref{eq:7}, определенную на границе. Она выглядит как член Весса-Зумино. Из требования калибровочной инвариантности квантовых корреляторов  получаем квантование $k$, так как функциональный интеграл должен быть однозначно определен. 

\subsection{Квантование}
\label{sec:quantization}

\subsubsection{Задача квантования}
\label{sec:task}

При каноническом квантовании теории Черна-Саймонса состояние определяется на каждой двумерной поверхности $\Sigma\subset M$. Как в любой квантовой теории поля, состояния соответствуют лучам в гильбертовом пространстве. Так как мы имеем дело с топологической теорией поля типа Шварца, то у нас нет предопределенного выделенного времени, поэтому $\Sigma$ - произвольная поверхность Коши.

Коразмерность $\Sigma$ равна 1, поэтому можно разрезать $M$ вдоль $\Sigma$ и получить многообразие с границей, на котором классическая динамика описывается WZW-моделью \ref{sec:Chern-Simons}. Виттен показал, что это соответствие сохраняется и в квантовой механике. То есть гильбертово пространство состояний всегда конечномерно и может быть отождествлено с пространством конформных блоков $G$-WZW-модели с уровнем $k$. Конформные блоки - это локально голоморфные и антиголоморфные множители, произведения которых складываются в корреляционные функции двумерной конформной теории поля. 

Например, если $\Sigma=S^2$, то гильбертово пространство одномерно и существует только одно состояние. При $\Sigma=T^2$ состояния соответствуют интегрируемым представлениям уровня $k$  аффинного расширения алгебры Ли $g$. Рассмотрение поверхностей более высокого рода не требуется для решения теории Черна-Саймонса.

\subsubsection{Наблюдаемые}
\label{sec:observables}

Наблюдаемые в теории Черна-Саймонса - это $n$-точечные функции калибровочно-инвариантных операторов, чаще всего рассматривают петли Вильсона. Петля Вильсона - это голономия вокруг кольца в $M$, вычисленная в некотором представлении $R$ группы $G$. Так как мы будем рассматривать произведения петель Вильсона, то мы можем считать представления неприводимыми.
\begin{equation}
  \label{eq:33}
    \langle W_R(K) \rangle =\text{Tr}_R \, P \, \exp{i \oint_K A}  
\end{equation}
Здесь $A$- 1-форма связности, мы берем главное значение интеграла по Коши, $P \exp$ - экспонента, упорядоченная вдоль пути.

Рассмотрим зацепление $L$ в $M$, которое представляет собой набор из $l$ несвязных циклов. Особенно интересна $l$-точечная корреляционная функция, представляющая собой произведение петель Вильсона в фундаментальном представлении $G$ вокруг этих циклов. Эту корреляционную функцию можно нормировать, разделив ее на 0-точечную функцию (статсумму $Z$).

Если $M$ - сфера, то такие нормированные функции пропорциональны известным полиномам (инвариантам) узлов. Например, при $G=U(N)$ теория Черна-Саймонса с уровнем $k$ дает
\begin{equation}
  \label{eq:34}
      \frac{\sin(\pi/(k+N))}{\sin(\pi N/(k+N))}\times\;\mbox{полином HOMFLY}
\end{equation}
При $N=2$ полином HOMFLY переходит в полином Джонса. В случае $SO(N)$ получается полином Кауффмана.

\subsubsection{Связь с G/G WZW-моделями}
\label{sec:gg-wzw}

Задача квантования сводится к вычислению функциональных интегралов вида
\begin{equation}
  \label{eq:9}
  \left< F \right> = \frac{\int \mathcal{D} A F[A] e^{i S(A)}}{\int \mathcal{D} A e^{iS(A)}}
\end{equation}
В первую очередь нас интересует производящий функционал (статсумма, partition function)
\begin{equation}
  \label{eq:14}
  Z_{\Sigma\times I}=\int \mathcal{D} A e^{iS(A)}
\end{equation}

Если многообразие $M$ имеет вид $M=\Sigma\times S^1$, то функциональный интеграл можно строго определить и явно вычислить. В противном случае можно вычислять его по теории возмущений. Как явное вычисление для многообразий специального вида так и вычисление по теории возмущений показывают связь теории Черна-Саймонса с WZW-моделями. Мы начнем с рассмотрения случая $M=\Sigma\times S^1$, где $\Sigma$ - компактная двумерная поверхность. Рассмотрение такого функционального интеграла можно свести к анализу теории Черна-Саймонса на пространстве $\Sigma\times I$, где $I=[0,1]$, с заданными граничными условиями в точках $\{0\}, \{1\}$.

В случае $M=\Sigma\times S^1$ существует такой выбор калибровки, при котором функциональный интеграл имеет гауссову форму. 
Сначала разделим поле $A$ на компоненты:
\begin{equation}
  \label{eq:10}
  A=A+A_0 dt
\end{equation}
Для вычисления функционального интеграла зафиксируем следующую калибровку
\begin{equation}
  \label{eq:11}
  \partial_0 A_0 =0
\end{equation}
Это условие устраняет не весь произвол, так как остаются ещё не зависящие от времени калибровочные преобразования. Обозначим через $\gamma$ голономию $A_0$ вокруг $S^1$, то есть
\begin{equation}
  \label{eq:12}
  \gamma=P \exp\left(\oint A_0\right)
\end{equation}
При калибровочном преобразовании $h: A_0\to A_0^h\equiv h^{-1} A_0 h+h^{-1}\partial_0 h$ $\gamma$ переходит в сопряженный элемент:
\begin{equation}
  \label{eq:13}
  \gamma\to h^{-1} \gamma h
\end{equation}
Мы можем наложить калибровочное условие, потребовав нахождения $\gamma$ в максимальном торе $T\subset G$.

Покажем, как мы можем получить действие калибровочной WZW-модели в результате вычисления функционального интеграла. 
Действие калибровочной WZW-модели отличается от действия \eqref{eq:76} наличием дополнительного члена, зависящего от калибровочного поля
\begin{equation}
  \label{eq:15}
  S_{/H}(g,A)=-\frac{1}{2\pi}\int_{\Sigma}d^2z (A_z\partial_{\bar z} g g^{-1}-A_{\bar z} g^{-1}\partial_z g+ A_z A_{\bar z} - g^{-1}A_z g A_{\bar z})
\end{equation}
Если исходное действие \eqref{eq:76} обладает глобальной инвариантностью относительно группы $G=G_L\times G_R: g\to a g b^{-1}$, то здесь калибровочная группа $H\subset G$ - это достаточно хорошая ее подгруппа, например, $H\subset G_{adj}(g\to a g a^{-1})$. Итоговое действие калибровочной WZW-модели
\begin{equation}
  \label{eq:16}
  S_{G/H}(g,A)=S_{WZW}(g)+S_{/H}(g,A)
\end{equation}
При $H=G$
\begin{equation}
  \label{eq:17}
  S_{G/G}(g,A)=-\frac{1}{8\pi}\int_{\Sigma} \mathrm{tr} g^{-1} d_A g * g^{-1} d_a g-i\Gamma(g,A)
\end{equation}
\begin{equation}
  \label{eq:18}
  \Gamma(g,A)=\frac{1}{12\pi}\int_{M:\partial M=\Sigma}\mathrm{tr}(g^{-1}dg)^3-\frac{1}{4\pi}\int_{\Sigma}\mathrm{tr}\left(A(dg\; g^{-1}+g^{-1} dg)+Ag^{-1} A g\right)
\end{equation}

Возвращаясь к действию Черна-Саймонса на $\Sigma\times I$, мы фиксируем калибровку \eqref{eq:11} и вводим
\begin{equation}
  \label{eq:19}
  \gamma_t\equiv P\exp\int_0^t A_0=\exp tA_0
\end{equation}
При этом $\gamma_0=1,\; \gamma_1=\gamma$ и $A^{(\gamma^{-1})}_0=0$.

Вводим духи Фаддева-Попова, при этом к действию добавляются члены, фиксирующие калибровку:
\begin{equation}
  \label{eq:20}
  \int_{\Sigma\times S^1}\mathrm{tr}(BA_0+\bar{c}D_0c),\quad \mbox{где} \; D_0=\partial_0+A_0
\end{equation}
и накладываем условия
\begin{equation}
  \label{eq:21}
  \oint B=\oint c=\oint \bar{c}=0
\end{equation}
Мы можем проинтегрировать по $A_t$, тогда $A_0$ останется только как внешнее фиксированное поле, а в интеграле появится множитель $\mathrm{Det}'D_0$. Обозначим оставшиеся компоненты через $B$ и зафиксируем граничные условия:
\begin{equation}
  \label{eq:22}
  B_z|_{\Sigma\times \{0\}}=A_z,\quad B_{\bar{z}}|_{\Sigma\times\{1\}}=A_{\bar z}
\end{equation}
Тогда надо добавить граничные члены к действию:
\begin{equation}
  \label{eq:23}
  S_{CS}(A_0,B)\to S_{CS}(A_0,B)-\frac{1}{4\pi}\int_{\Sigma\times \{0\}}B_z B_{\bar z}-\frac{1}{4\pi}\int_{\Sigma\times \{1\}}B_z B_{\bar z}
\end{equation}
Тогда нам надо вычислить
\begin{equation}
  \label{eq:24}
  Z_{\Sigma\times S^1}=\int \mathcal{D}A_0\mathcal{D}A \mathrm{Det}'D_0 Z_{\Sigma\times I}[A_0; A_z, A_{\bar z}]\exp\left(\frac{ik}{2\pi}\int_{\Sigma}A_z A_{\bar z}\right) 
\end{equation}
Последний член - Кэлеров потенциал, необходимый для голоморфного представления граничных членов \eqref{eq:23}.

На многообразии с границами $\exp iS_{CS}$ не инвариантна относительно произвольных калибровочных преобразований \eqref{eq:7}
\begin{equation}
  \label{eq:25}
  \exp iS_{CS}(A^g)=\Theta(A,g)^k \exp iS_{CS}(A)
\end{equation}
\begin{equation}
  \label{eq:26}
  \Theta(A,g)=\exp \left(-\frac{i}{12\pi}\int_M(g^{-1}dg)^3+\frac{i}{4\pi}\int_{\partial M} A dg g^{-1}\right)
\end{equation}

Тогда для амплитуды $Z_{\Sigma\times I}[A_0;A_z,A_{\bar z}]$ при преобразовании $\gamma_t=\exp tA_0$
\begin{equation}
  \label{eq:27}
  Z_{\Sigma\times I}[A_0;A_z,A_{\bar z}^{\gamma}]=Z_{\Sigma\times I}[A_0^{(\gamma_t^{-1})}=0;A_z,A_{\bar z}] C[A_{\bar z},\gamma]
\end{equation}
где $C[A_{\bar z},\gamma]$ - коцикл Полякова-Вигманна:
\begin{equation}
  \label{eq:28}
  C[A_{\bar z},\gamma]=\exp iS_{G/G}(\gamma^{-1},A_z=0,A_{\bar z})
\end{equation}

Нам надо также перейти от меры $\mathcal{D}A_0$ на алгебре Ли к мере Хаара $\mathcal{D}\gamma$ на группе, при этом получается множитель, который сокращает член $Det'$. Имеем:
\begin{equation}
  \label{eq:29}
   Z_{\Sigma\times S^1}=\int \mathcal{D}\gamma\mathcal{D}A Z_{\Sigma\times I}[A_0=0; A_z, A_{\bar z}] C[A_{\bar z},\gamma] \exp\left(\frac{ik}{2\pi}\int_{\Sigma}A_z A_{\bar z}^{\gamma}\right) 
\end{equation}

Осталось вычислить обычный интеграл $Z_{\Sigma\times I}[A_0;A_z,A_{\bar z}]$, только нужно быть аккуратными с граничными условиями:
\begin{equation}
  \label{eq:30}
  Z_{\Sigma\times I}[A_0;A_z,A_{\bar z}]= N \exp \left( -\frac{ik}{2\pi}\int_{\Sigma\times\{0\}} A_z A_{\bar z}\right)
\end{equation}
Подставляя, получаем для функционального интеграла
\begin{equation}
  \label{eq:31}
    Z_{\Sigma\times S^1}\propto\int \mathcal{D}\gamma\mathcal{D}A \exp ik\left(\frac{1}{k}S_{WZW}(\gamma^{-1})-\frac{1}{2\pi}\int_{\Sigma}d^2z (A_{\bar z}\partial_z \gamma \gamma^{-1}+A_zA_{\bar z}-A_z A_{\bar z}^{\gamma})\right)
\end{equation}
\begin{equation}
  \label{eq:32}
  Z_{\Sigma\times S^1}\propto\int \mathcal{D}\gamma\mathcal{D}A \exp iS_{G/G}(\gamma^{-1},A)
\end{equation}

При рассмотрении специальных видов поверхности $\Sigma$ (диск, тор) удается перейти к чистой WZW-модели.

\subsection{Литература}
\label{sec:cs-literature}

Оригинальные результаты по квантованию теории Черна-Саймонса и связь с WZW впервые опубликованы в работе \cite{witten1989quantum}. Существуют записки с лекций Виттена \cite{hu2001lecture}, но они не очень понятные. С более математической точки зрения вопрос о связи конформной теории поля и теории Черна-Саймонса обсуждается в книге \cite{kohno2002conformal}. Теория Черна-Саймонса допускает также геометрическое квантование, чему посвящена диссертация \cite{axelrod1991geometric}. Материал секции \ref{sec:quantization} написан по работе \cite{blau1993derivation}, а разделы \ref{sec:task}, \ref{sec:observables} представляют собой перевод английской статьи в Википедии (\url{http://en.wikipedia.org/wiki/Chern-Simons_theory}).

\bibliography{program,wzw}{}
\bibliographystyle{utphys}

\end{document}
