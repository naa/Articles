\documentclass{article}
\usepackage[russian]{babel}
\usepackage[utf8]{inputenc}
\title{Конформная теория поля}
\author{Антон Назаров}
\date{13 сентября 2012}
\begin{document}
\maketitle
Конформная теория поля может быть сформулирована достаточно строго.
Этим она отличается от других моделей квантовой теории поля.

Мы кратко опишем переход от дискретных систем к непрерывным и от
классических к квантовым. Обсудим, как при этом возникают понятия
полей, состояний, корреляционных функций, операторного разложения.
Затем мы сформулируем аксиомы, которым должны удовлетворять
корреляционные функции и покажем, что по корреляционным функциям
восстанавливаются поля и гильбертово пространство состояний.

Из требования определенного поведения корреляционных функций при
конформных преобразованиях возникает алгебра Вирасоро. Поля в
конформной теории образуют модули алгебры Вирасоро. Из структуры
модулей следуют уравнения на корреляционные функции, которые, кстати,
совпадают с уравнениями на мартингалы эволюции Шрамма-Лёвнера.
\end{document}
