\documentclass[a4paper,12pt]{article}
\usepackage[unicode,verbose]{hyperref}
\usepackage{amsmath,amssymb,amsthm} \usepackage{pb-diagram}
\usepackage{ucs}
\usepackage[utf8x]{inputenc}
\usepackage[russian]{babel}
\usepackage{cmap}
\usepackage{graphicx}
\pagestyle{plain}
\theoremstyle{definition} \newtheorem{Def}{Definition}
\newenvironment{comment}
{\par\noindent{\bf Comment}\\}
{\\\hfill$\scriptstyle\blacksquare$\par}

\newtheorem{statement}{Statement}
\theoremstyle{definition} 
\newcommand{\go}{\overset{\circ }{\frak{g}}}
\newcommand{\ao}{\overset{\circ }{\frak{a}}}
\newcommand{\co}[1]{\overset{\circ }{#1}}

\title{Интегрируемые спиновые цепочки}
\date{}
\begin{document}

\maketitle

\section{Лекция 1. 13 февраля 2011}

Лекция началась с формулировки двух примеров интегрируемых спиновых цепочек -- XXX-цепочки со спином 1/2 и цепочки со спином, равным -1.

XXX-цепочка -- это одномерная квантовая система, представляющая собой набор из N частиц со спинами. Спины могут быть как полуцелыми, так и в общем положении. Гамильтониан системы раскладывающимся в сумму взаимодействий ближайших соседей. Члены этой суммы представляют собой тензорные произведения спинов в соседних узлах. Пространство состояний представляет собой N-ую тензорную степень пространства представления алгебры sl(2), соответствующего данному значению спина. В случае спина 1/2 пространство представления -- это двумерное комплексное векторное пространство, а для спина -1 -- бесконечномерное пространство. 

Задача состоит в точном решении квантовой системы, то есть нахождении спектра и собственных состояний гамильтониана. 

Из sl(2)-инвариантности гамильтониана XXX-цепочки следует, что пространство состояний разбивается на sl(2)-инвариантные подпространства меньшей размерности, так как в результате действия генераторов алгебры sl(2) на собственные состояния гамильтониана будут опять получаться собственные состояния гамильтониана. Таким образом задача диагонализации гамильтониана упростилась. То есть чтобы диагонализовать гамильтониан, надо разложить N-ую тензорную степень пространства представления алгебры sl(2), соответствующего спину, в прямую сумму инвариантных подпространств неприводимых представлений, и диагонализовать гамильтониан на этих подпространствах меньшей размерности.

\section{Лекция 2. 17 февраля 2011}

Лекция была посвящена обсуждению цепочки со спинами в общем положении. В этом случае в качестве реализации пространства представления, соответствующего спину, можно взять пространство полиномов, тогда все пространство состояний будет пространством полиномов N переменных. Действие генераторов алгебры sl(2) на таком пространстве реализуется дифференциальными операторами. Однако в случае спина в общем положении частные производные в этом операторе не обязательно будут в целых степенях. Чтобы правильно интерпретировать такие операторы, их нужно переписать через гамма-функции и представить в интегральном виде.
Гамильтониан представляет собой сумму членов, отвечающих взаимодействию ближайших соседей. Такие члены даются интегральным оператором.  Была представлена явная реализация представлений алгебры sl(2) в виде дифференциальных операторов на пространстве полиномов. Показано, что в случае полуцелого спина возникает инвариантное конечномерное подпространство, на котором действует неприводимое представление алгебры sl(2).

\section{Лекция 3. 24 февраля 2011}

Мы проверяем sl(2)-инвариантность гамильтониана, представляющего собой интегральный оператор. Для этого мы вычисляем действие экспонент генераторов алгебры sl(2) на функции. Вычисление можно свести к решению уравнения, имеющего тот же вид, что и уравнения ренормгруппы в квантовой теории поля. Из sl(2)-инвариантности гамильтониана следует, что собственные функции можно искать на пространстве однородных полиномов.

Затем мы начинаем рассмотрение общей конструкции интегрируемых спиновых цепочек. Идея состоит в построении набора из N-1 операторов Q, коммутирующих с гамильтонианом и между собой. Трансфер-матрица -- это производящая функция таких операторов. Она дается следом по вспомогательному пространству от произведения операторов Лакса, действующих в тензорном произведении пространств, связанных с узлами и вспомогательного пространства. Оператор Лакса и трансфер-матрица зависят от спектрального параметра u. Коэффициенты разложения трансфер-матрицы по степеням спектрального параметра являются операторами, коммутирующими с гамильтонианом.

Произведение операторов Лакса по всем узлам называется матрицей монодромии. Трансфер-матрица - это ее след.

\section{Лекция 4. 3 марта 2011}

Рассматриваем, какие требования на трансфер-матрицу и оператор Лакса накладывает sl(2)-инвариантность и требования коммутативности операторов Q. Оказывается, что трансфер-матрицы при разных значениях спектрального параметра должны коммутировать. Оператор Лакса должен являться оператором Казимира алгебры sl(2) и удовлетворять RLL-соотношениям, из которых следует коммутативность трансфер- матрицы. 

В RLL-соотношения входит новый объект -- R-матрица, которая удовлетворяет уравнению Янга-Бакстера. Мы используем R-матрицу, представляющую собой сумму спектрального параметра (умноженного на единичный оператор) и оператора перестановки. 
Для такого вида R-матрицы мы получаем выражение для L-оператора.

\section{Лекция 5. 10 марта 2011}

Теперь нам необходимо проверить коммутативность трансфер-матрицы. 
Сначала мы проверяем ее для случая двух узлов. В этом случае условие коммутативности легко следует из RLL-соотношений. Для большего числа узлов можно доказать из коммутативности L-операторов для разных узлов при различных значениях спектрального параметра. 

После того, как мы доказали необходимые свойства трансфер-матрицы, мы можем ввести оставшиеся ингредиенты алгебраического анзаца Бете. Для этого запишем трансфер-матрицу в виде матрицы 2x2 с компонентами A, B, C, D. При этом A,B,C,D - операторы, зависящие от спектрального параметра.

Алгебраический анзац Бете основан на предположении о существовании особого состояния в спектре трансфер-матрицы, называющегося псевдовакуумом.  Это состояние обнуляется оператором B. Все остальные состояния порождаются действием оператора C на псевдовакуум. Таким образом B выполняет роль оператора уничтожения, а C - рождения. 

Далее мы рассматриваем действие трансфер-матрицы на состояния, порожденные из псевдовакуума и получаем уравнения Бете.
\section{Лекция 6. 17 марта 2011}

Лекция посвящена рассмотрению примера уравнений Бете-анзаца для случая цепочки из двух узлов. Явно выписывается вид и действие операторов, составляющих матрицу монодромии. Затем выводятся уравнения Бете. Уравнения Бете - это алгебраические уравнения на функции q(u), через которые выражается спектр трансфер-матрицы, а значит и всех интегралов движения, включая гамильтониан.

\section{Лекция 7. 24 марта 2011}

В этой лекции приводится способ построения гамильтониана из трансфер-матрицы и вычисление его спектра из уравнений Бете-анзаца. Оказывается, что гамильтониан в виде суммы операторов перестановки, который приводился в первых лекциях в качестве примера, можно получить из трансфер-матрицы путем дифференцирования ее логарифма по спектральному параметру в точке 1/2. Зная вид и коммутационные соотношения для трансфер-матрицы мы теперь можем предъявить набор коммутирующих с гамильтонианом операторов и доказать интегрируемость. 

\section{Лекция 8. 31 марта 2011}

Теперь мы демонстрируем возможность получения гамильтониана для спиновой цепочки со спином -1, который был приведен в первой лекции в интегральном представлении, из явного вида R-матрицы и оператора Лакса. Для этого выводится явный вид операторов перестановки, проверяются их свойства и выписывается выражение для действия R-матрицы. Она зависит только от разности спектральных параметров. Двухчастичный гамильтониан дается произведением производной R-матрицы и обратного к R-матрице оператора в точке 0. 
\section{Лекция 9. 7 апреля 2011}

Завершая доказательство, начатое в предыдущей лекции, мы получаем интегральный вид для гамильтониана. После этого, в принципе, можно находить его спектр при помощи анзаца Бете. Но этот метод не всегда работает, так как зависит от наличия особого состояния - псевдовакуума. 

Поэтому мы переходим к описанию альтернативного подхода, связанного с построением Q-оператора Бакстера. В этом подходе нам не требуется предположение о существовании псевдовакуума. Q-оператор должен зависеть от спектрального параметра и коммутировать с трансфер-матрицей и сам с собой. Вид и соотношения для Q-оператора должны следовать из уравнений Янга-Бакстера для R-матрицы. 
\section{Лекция 10. 14 апреля 2011}
Q-оператор состоит из блоков, условия на вид которых находятся из требуемых коммутационных соотношений с трансфер-матрицей. Оказывается, что блоки можно получить из явного вида для R-матрицы. В данной лекции мы используем явный вид, подходящий для бесконечномерных представлений алгебры sl(s) - модулей Верма. После получения выражения для блоков, мы видим, что Q-оператор дается (нормированным) произведением по всем узлам. Затем определяется действие полученного оператора на полиномах. 

\section{Лекция 11. 21 апреля 2011}

В этой лекции мы проверяем, что коммутационные соотношения для Q-оператора с трансфер-матрицей, полученного в прошлой лекции, эквивалентны RLL-соотношениям, связывающим R-матрицу и оператор Лакса или уравнениям Янга-Бакстера. Теперь нам остается доказать коммутативность Q-оператора с собой при разных значениях спектрального параметра. Для этого нам нужно начать с представления оператора в виде произведения блоков, выражающихся через R-матрицу и рассмотреть действие коммутатора Q-оператора на произвольные функции.

\section{Лекция 12. 5 мая 2011}
Завершая вычисление, начатое в прошлой лекции, мы получаем уравнение на собственные функции Q-оператора. Оказывается, что собственная функция эквивалентна функции q(u), возникающей в уравнениях Бете-анзаца. 

Далее мы показываем, что гамильтониан дается разностью производных от логарифма Q-оператора в при противоположных значениях спектрального параметра. В результате из подхода с Q-оператором удается извлечь те же результаты для спектра, что и при использовании Бете-анзаца.

\section{Лекция 13. 12 мая 2011}

В этой лекции мы обсуждаем проблему построения решений методом Q-оператора для случая конечномерных представлений алгебры sl(2). Конечномерные представления связаны с бесконечномерными модулями Верма резольвентой Бернштейна-Гельфанда-Гельфанда.  Оказывается, что перенос конструкции на Q-операторы сопряжен с некоторыми техническими сложностями. 

\bibliography{program}{}
\bibliographystyle{utphys}
\end{document}
