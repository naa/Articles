\documentclass[a4paper,12pt]{article}
\usepackage[unicode,verbose]{hyperref}
\usepackage{amsmath,amssymb,amsthm} \usepackage{pb-diagram}
\usepackage{ucs}
%\usepackage[utf8x]{inputenc}
%\usepackage[russian]{babel}
\usepackage{cmap}
\usepackage{graphicx}
\pagestyle{plain}
\theoremstyle{definition} \newtheorem{Def}{Definition}
\newenvironment{comment}
{\par\noindent{\bf Comment}\\}
{\\\hfill$\scriptstyle\blacksquare$\par}

\newtheorem{statement}{Statement}
\theoremstyle{definition} \newtheorem{Def}{Definition}
\newcommand{\go}{\overset{\circ }{\frak{g}}}
\newcommand{\ao}{\overset{\circ }{\frak{a}}}
\newcommand{\co}[1]{\overset{\circ }{#1}}

\begin{document}

\title{Интегрируемые спиновые цепочки}

\begin{abstract}
\end{abstract}

\section{Лекция 1. 13 февраля 2011}

Лекция началась с формулировки двух примеров интегрируемых спиновых цепочек -- XXX-цепочки со спином 1/2 и цепочки со спином, равным -1.

XXX-цепочка -- это одномерная квантовая система, представляющая собой набор из N частиц со спинами. Спины могут быть как полуцелыми, так и в общем положении. Гамильтониан системы раскладывающимся в сумму взаимодействий ближайших соседей. Члены этой суммы представляют собой тензорные произведения спинов в соседних узлах. Пространство состояний представляет собой N-ую тензорную степень пространства представления алгебры sl(2), соответствующего данному значению спина. В случае спина 1/2 пространство представления -- это двумерное комплексное векторное пространство, а для спина -1 -- бесконечномерное пространство. 

Задача состоит в точном решении квантовой системы, то есть нахождении спектра и собственных состояний гамильтониана. 

Из sl(2)-инвариантности гамильтониана XXX-цепочки следует, что пространство состояний разбивается на sl(2)-инвариантные подпространства меньшей размерности, так как в результате действия генераторов алгебры sl(2) на собственные состояния гамильтониана будут опять получаться собственные состояния гамильтониана. Таким образом задача диагонализации гамильтониана упростилась. То есть чтобы диагонализовать гамильтониан, надо разложить N-ую тензорную степень пространства представления алгебры sl(2), соответствующего спину, в прямую сумму инвариантных подпространств неприводимых представлений, и диагонализовать гамильтониан на этих подпространствах меньшей размерности.

\section{Лекция 2. 17 февраля 2011}

Лекция была посвящена обсуждению цепочки со спинами в общем положении. В этом случае в качестве реализации пространства представления, соответствующего спину, можно взять пространство полиномов, тогда все пространство состояний будет пространством полиномов N переменных. Действие генераторов алгебры sl(2) на таком пространстве реализуется дифференциальными операторами. Однако в случае спина в общем положении частные производные в этом операторе не обязательно будут в целых степенях. Чтобы правильно интерпретировать такие операторы, их нужно переписать через гамма-функции и представить в интегральном виде.
Гамильтониан представляет собой сумму членов, отвечающих взаимодействию ближайших соседей. Такие члены даются интегральным оператором.  Была представлена явная реализация представлений алгебры sl(2) в виде дифференциальных операторов на пространстве полиномов. Показано, что в случае полуцелого спина возникает инвариантное конечномерное подпространство, на котором действует неприводимое представление алгебры sl(2).

\section{Лекция 3. 24 февраля 2011}

Мы проверяем sl(2)-инвариантность гамильтониана, представляющего собой интегральный оператор. Для этого мы вычисляем действие экспонент генераторов алгебры sl(2) на функции. Вычисление можно свести к решению уравнения, имеющего тот же вид, что и уравнения ренормгруппы в квантовой теории поля. Так как в данном случае пространство состояний бесконечномерное, то sl(2)-инвариантность гамильтониана не упрощает задачу так же сильно, как в случае спина 1/2.

Затем мы начинаем рассмотрение общей конструкции интегрируемых спиновых цепочек. 
\section{Лекция 4. 3 марта 2011}

\section{Лекция 5. 10 марта 2011}

\section{Лекция 6. 17 марта 2011}

\section{Лекция 7. 24 марта 2011}

\section{Лекция 8. 31 марта 2011}

\section{Лекция 9. 7 апреля 2011}

\section{Лекция 10. 14 апреля 2011}



\bibliography{program}{}
\bibliographystyle{utphys}
\end{document}
