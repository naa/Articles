\documentclass[a4paper,12pt]{article}
\usepackage[utf8x]{inputenc}
\usepackage[russian]{babel}
\usepackage{ucs}
\usepackage[unicode,verbose]{hyperref}
\usepackage{amsmath,amssymb,amsthm}
\usepackage{pb-diagram}
\usepackage{multicol}
\usepackage{cmap}
\usepackage{color}
\usepackage{graphicx}
\usepackage{epstopdf}
\pagestyle{plain}

%\usepackage{verbatim} 
\newenvironment{comment}
{\par\noindent{\bf TODO}\\}
{\\\hfill$\scriptstyle\blacksquare$\par}

\newtheorem{statement}{Statement}
\newtheorem{lemma}{Lemma}
\newtheorem{theorem}{Theorem}
\theoremstyle{definition}
\newtheorem{corollary}{Corollary}[theorem]
\theoremstyle{definition}
\newtheorem{mynote}{Note}[section]
\theoremstyle{definition}
\newtheorem{definition}{Definition}
\newcommand{\go}{\stackrel{\circ }{\mathfrak{g}}}
\newcommand{\ao}{\stackrel{\circ }{\mathfrak{a}}}
\newcommand{\co}[1]{\stackrel{\circ }{#1}}
\newcommand{\pia}{\pi_{\mathfrak{a}}}
\newcommand{\piab}{\pi_{\mathfrak{a}_{\bot}}}
\newcommand{\af}{\mathfrak{a}}
\newcommand{\afb}{\mathfrak{a}_{\bot}}


\title{Лекции по теории поля в двух измерениях\\
\small{Академический университет, кафедра теоретической физики}
}
\author{Антон Назаров\\
  \small{СПбГУ, физический факультет}\\
  \small{кафедра физики высоких энергий и элементарных частиц}\\
  \texttt{antonnaz@gmail.com}
}
\date{Осенний семестр 2010 года}

\begin{document}
\maketitle
\begin{abstract}
  Текст представляет собой конспект лекций по квантовой теории поля в двух измерениях. Лекции читаются в осеннем семестре 2010 года в Академическом университете для магистрантов 6 курса, группа теории твердого тела. 
\end{abstract}

\section{Программа курса}
\label{sec:program}
\begin{enumerate}
\item Введение.
  \begin{itemize}
  \item Двумерные модели
  \item Фазовые переходы, критические индексы
  \item Универсальность
  \item Методы квантовой теории поля в статистической физике
  \end{itemize}
\item Перенормировки, ренормгруппа
\item Решение модели Изинга
  \begin{itemize}
  \item Интегрируемость
  \item Подход Каданова
  \end{itemize}
\item Конформная инвариантность
  \begin{itemize}
  \item Глобальная конформная инвариантность
  \item Ограничения на корреляционные функции
  \item Локальная конформная инвариантность в двух измерениях
  \item Алгебра Витта и алгебра Вирасоро.
  \end{itemize}
\item Конформная теория поля
  \begin{itemize}
  \item Примарные и вторичные поля
  \item Минимальные модели
  \end{itemize}
\end{enumerate}

\section{Лекция 1}
\label{sec:lecture-1}

\subsection{Введение}
\label{sec:intro}
Чем интересны двумерные системы? Во-первых, они существуют в природе. Например, критическое поведение на поверхности металлов напоминает поведение модели Изинга в двух измерениях \cite{campuzano1985110}. Во-вторых, в двух измерениях даже в простых моделях есть фазовые переходы. Это не так, например,  в одномерной модели Изинга. В простых двумерных моделях можно вычислить критические индексы и изучать поведение системы при фазовых переходах. Гипотеза универсальности утверждает, что все системы в критической точке разбиваются на небольшое число классов. Системы из одного класса ведут себя одинаково, имеют одинаковые критические индексы. Благодаря гипотезе универсальности существует лишь небольшое число типов критического поведения, поэтому значения индексов, вычисленные теоретически в простых моделях соответствуют гораздо более сложным реальным системам.
В критической точке наблюдается масштабная инвариантность, что ведет к конформной инвариантности \cite{Polyakov:1970xd}. В двух измерениях конформная инвариантность дает очень много сведений о системе, так как алгебра конформных преобразований бесконечномерна. В результате поведение системы может быть описано строго математически методами двумерной конформной теории поля. Двумерная конформная теория поля имеет и другое применение, не связанное с описанием фазовых переходов --- это теория струн, которая считается перспективным кандидатом на роль квантовой теории гравитации.

\subsection{Фазовые переходы. Основные понятия}
\label{sec:phase-transitions}
Напомним некоторые основные понятия теории фазовых переходов на примере двумерной модели Изинга.
Двумерная модель Изинга --- это простейшая модель магнетика. Она формулируется на решетке, которую мы будем, для простоты, считать прямоугольной. Вершины решетки нумеруются латинскими индексами $i,j$. В вершинах решетки находятся частицы со спинами $s_j$ равными $\pm 1$ (вверх или вниз). Взаимодействуют только ближайшие соседи, константу взаимодействия обозначим через $J$. Для энергии системы имеем
\begin{equation}
  \label{eq:1}
  E=J\sum_{\left<i,j\right>} s_i\cdot s_j-h\sum_i s_i
\end{equation}
Суммирование в первом члене ведется только по парам ближайших соседей, второй член --- это взаимодействие со внешним полем $h$.

Существует несколько точных решений двумерной модели Изинга в отсутствие внешнего поля. Решений со внешним полем и в большем числе измерений пока не известно. Недавнее достижение Станислава Смирнова, за которое он получил в этом году премию Филдса, непосредственно связано с темой наших лекций. В своих работах Станислав Смирнов впервые строго доказал наличие конформного предела в двумерной модели Изинга \cite{smirnov2007conformal,smirnov2001critical}.

Мы рассматриваем статистические системы. Все макроскопические характеристики таких систем вычисляются из статсуммы, которая связана с микроскопическим описанием системы. Для больцмановского распределения вероятность системы находиться в состоянии с номером $i$ с энергией $E_i$
\begin{equation}
  \label{eq:2}
  P_i=\frac{1}{Z}e^{-\beta E_i}, \quad \beta=\frac{1}{T}
\end{equation}
\begin{equation}
  \label{eq:3}
  Z=\sum_i e^{-\beta E_i}
\end{equation}
Свободная энергия
\begin{equation}
  \label{eq:4}
  F=-T\ln Z
\end{equation}
Внутренняя энергия
\begin{equation}
  \label{eq:5}
  U=-\frac{1}{Z}\frac{\partial Z}{\partial \beta}=-T^2 \frac{\partial}{\partial T}\left(\frac{F}{T}\right)
\end{equation}
Телоемкость равна производной внутренней энергии по температуре при заданном объеме
\begin{equation}
  \label{eq:6}
  C=\left(\frac{\partial U}{\partial T}\right)_V=-T\frac{\partial^2 F}{\partial T^2}
\end{equation}
Таким образом статсумма является производящей функцией всех термодинамических величин. Вычисление статсуммы в реальных системах --- это сложная задача.
Макроскопическое описание имеет смысл только в термодинамическом пределе, когда число частиц стремится к бесконечности $N\to \infty$. Фазовые переходы в модели Изинга наблюдаются только в этом пределе, при конечных размерах системы никаких переходов нет.
\bibliography{2dqft}{}
\bibliographystyle{utphys}

\end{document}
