\documentclass[a4paper,12pt]{article}
\usepackage[utf8x]{inputenc}
\usepackage[russian]{babel}
\usepackage{ucs}
\usepackage[unicode,verbose]{hyperref}
\usepackage{amsmath,amssymb,amsthm}
\usepackage{mathtools}
\usepackage{pb-diagram}
\usepackage{multicol}
\usepackage{cmap}
\usepackage{color}
\usepackage{graphicx}
\pagestyle{plain}


%\usepackage{verbatim} 
\newenvironment{comment}
{\par\noindent{\bf TODO}\\}
{\\\hfill$\scriptstyle\blacksquare$\par}

\newtheorem{statement}{Statement}
\newtheorem{lemma}{Lemma}
\newtheorem{theorem}{Theorem}
\theoremstyle{definition}
\newtheorem{corollary}{Corollary}[theorem]
\theoremstyle{definition}
\newtheorem{mynote}{Замечание}[section]
\theoremstyle{definition}
\newtheorem{definition}{Определение}
\theoremstyle{definition}
\newtheorem{exercise}{Упражнение}
\newtheorem{example}{Пример}

\newcommand{\go}{\stackrel{\circ }{\mathfrak{g}}}
\newcommand{\ao}{\stackrel{\circ }{\mathfrak{a}}}
\newcommand{\co}[1]{\stackrel{\circ }{#1}}
\newcommand{\pia}{\pi_{\mathfrak{a}}}
\newcommand{\piab}{\pi_{\mathfrak{a}_{\bot}}}
\newcommand{\af}{\mathfrak{a}}
\newcommand{\gf}{\mathfrak{g}}
\newcommand{\afb}{\mathfrak{a}_{\bot}}

%% \topmargin=-3.5cm
%% \oddsidemargin=-1.5cm
%% \evensidemargin=-1.5cm
%% \textwidth=19cm
%% \textheight=28cm
%% 
\title{Методы теории групп в квантовой теории поля\\
\small{Кафедра физики высоких энергий и элементарных частиц СПбГУ}
}
\author{Антон Назаров\\
  \small{СПбГУ, физический факультет, кафедра физики высоких энергий и элементарных частиц;}\\
  \texttt{antonnaz@gmail.com}
}
\date{Осенний семестр 2014 года}

\begin{document}
\maketitle
\thispagestyle{empty}
\begin{abstract}
  Текст представляет собой конспект лекций курса ``Методы теории групп в квантовой теории поля'' (Теория представлений-2). Лекции читались в осеннем семестре 2014 года в СПбГУ для магистрантов 5 курса (кафедра физики высоких энергий и элементарных частиц). 
\end{abstract}
\tableofcontents

\section{Лекция 1}
\label{sec:lecture-1}

\subsection{Программа курса}
\label{sec:program}
\begin{enumerate}
\item Представления группы Лоренца
\item Метод индуцированных представлений, представления группы Пуанкаре
\item Алгебраический анализ представлений
\item Представления групп Ли
\item Метод орбит, геометрическое квантование
\end{enumerate}

\subsection{Литература}
\bibliographystyle{gost780}
\bibliography{groups}{}


Книга автора курса: \cite{book:10358}, \cite{book:1244385}, книга, с подробным разъяснением
материала первой части курса: \cite{book:855864}, подробный двухтомник для физиков:
\cite{book:130300,book:130301}, хорошая обзорная книга, охватывающая дополнительные разделы:
\cite{book:872634}.

\subsection{Группы и Представления групп}
\label{sec:intro}

\begin{equation}
  \label{eq:1}
  (U\psi,U\varphi)=(\psi,\varphi)
\end{equation}
$U$ -- унитарный линейный оператор. Если же $ (U\psi,U\varphi)=(\psi,\varphi)^{*}$, то $U$ --
антиунитарный, антилинейный.

Пусть $T_{1},T_{2}$ -- преобразования, тогда или
\begin{equation}
  \label{eq:2}
  U(T_{2})U(T_{1})=U(T_{2}T_{1})
\end{equation}
или
\begin{equation}
  \label{eq:2}
  U(T_{2})U(T_{1})=e^{i\varphi(T_{1},T_{2})} U(T_{2}T_{1}).
\end{equation}
Во втором случае представление называется проективным.

Пусть $G$ -- группа, элемент $g\in G$. Обозначим через $D(G,V)$ представление группы $G$ линейными
операторами в линейном пространстве $V$, $D(g)\in GL(V)$. Действие группы на каком-то множестве,
не являющемся линейным пространством будем называть ``реализацией''. 

\begin{definition}
  {\it Топологическая группа} -- группа, являющаяся топологическим пространством. 
\end{definition}
\begin{definition}
  {\it Группа Ли} -- группа, являющаяся гладким (аналитическим) многообразием
\end{definition}

{\it Определяющим} называется представление, использующееся для задания группы. 

Рассмотрим множество $M$ и отображение $g\in G\to O(g)\in Aut(M)$, т.ч. $O(g_{1}g_{2}^{-1})=O(g_{1})
O^{-1}(g_{2})$. Тогда для функций $f\in Fun(M,k)$, где $k$ -- поле, $k=\mathbb{R}, \mathbb{C}$ --
характеристики 0 (характеристика $p$ -- числа по модулю простого числа $p$), отображения
\begin{equation}
  \label{eq:3}
  \begin{array}{l}
  f(\cdot)\to f(\cdot\; g)\\
  f(\cdot)\to f(g\; \cdot)    
  \end{array}
\end{equation}
задают представления $G$. 
\begin{exercise}
  Проверить, что это действительно представление
\end{exercise}

Пусть $V_{1},V_{2}$ -- линейные пространства и существует отображение $\zeta$:
\begin{equation}
  \label{eq:4}
    \begin{diagram}
    \node{V_{1}}\arrow{s,r}{D_{1}} \arrow{e,t}{\zeta} \node{V_{2}}\arrow{s,r}{D_{2}}\\
    \node{V_{1}}\arrow{e,t}{\zeta} \node{V_{2}}
  \end{diagram}
\end{equation}
Если диаграмма коммутативна, то есть $\forall g, v$ верно $D_{2}(g)\zeta v=\zeta D_{1}(g) v$, тогда
$\zeta$ называется сплетающим оператором представлений $D_{1}, D_{2}$. Если существует обратное
отображение $\zeta^{-1}$, то $\zeta D_{1} \zeta^{-1}=D_{2}$ и представления $D_{1}$ и $D_{2}$
эквивалентны.

\begin{definition}
  {\it Категория} $C$ состоит из объектов категории $Ob(C)$ и отображений -- морфизмов $f\in
  Hom(C)$, $f:a\to b$. Морфизмы должны быть определена композиция $f\cdot g$ и для любого $a\in
  Ob(C)$ должно существовать тождественное отображение $1_{a}:a\to a$, причем $1_{b}\cdot f=f\cdot
  1_{a}$.
\end{definition}

Представления группы образуют категорию, а сплетающие операторы являются морфизмами в ней. 

Прямая сумма представлений $D_{1}, D_{2}$:
\begin{equation}
  \label{eq:5}
  \begin{pmatrix}
    D_{1}(g) & 0 \\
    0 & D_{2}(g)
  \end{pmatrix} 
  \begin{pmatrix}
    v_{1}\\ v_{2}
  \end{pmatrix}
\end{equation}
Если
\begin{equation}
  \label{eq:6}
  D(g)=
  \begin{pmatrix}
    D_{1}(g) & C(g) \\
    0 & D_{2}(g)
  \end{pmatrix} 
  \begin{pmatrix}
    v_{1}\\ v_{2}
  \end{pmatrix},
\end{equation}
то $V_{2}$ является инвариантным подпространством, $D_{2}$ -- подпредставлением $D$, а
$D_{\downarrow V_{2}}=D_{2}$ -- ограничение представления $D$ на $V_{2}$. Представление $D$
называется приводимым, так как есть инвариантное подпространство. В случае топологических групп
инвариантное подпространство должно быть замкнутым. Если дополнение инвариантного подпространства
инвариантно, то представление называется вполне приводимым и $D=D_{1}\oplus \dots\oplus D_{n}$ (в
случае конечномерных представлений).

\begin{theorem}
  Всякое унитарное представление вполне приводимо
\end{theorem}

Какие задачи нужно уметь решать в теории представлений:
\begin{enumerate}
\item Находить спектр унитарных представлений группы
\item Раскладывать представления на неприводимые
\item Раскладывать представления группы на представления подгруппы
  $H\subset G,\quad D_{G\downarrow H}=D_{1}(H)\oplus \dots D_{n}(H)$
\item Строить индуцированные представления $D_{H\uparrow G}$
\item Раскладывать тензорное произведение на неприводимые
  $D_{1}\otimes D_{2}=D_{1}'\oplus\dots \oplus D_{n}'$, в теории поля такое разложение описывает
  возможные каналы распада после столкновения двух частиц. 
\end{enumerate}
\begin{definition}
  Алгебра Ли $\gf$ группы Ли $G$ -- это касательное пространство к групповому многообразию. В
  алгебре Ли определена скобка $[\;,\;]$, экспоненциальное отображение задает локальные координаты
  на группе в окрестности единицы: $X\in \gf, \; e^{tX}\in G$, $t$ -- координаты. 
\end{definition}
\begin{definition}
  Присоединенное представление группы $G$ -- это представление на алгебре Ли $\gf$, для $g\in G, \;
  X\in \gf$ задающееся формулой
  \begin{equation}
    \label{eq:7}
    Ad_{g} X=\left.\frac{d}{dt}g e^{tX} g^{-1}\right|_{t=0}
  \end{equation}

\end{definition}

\begin{exercise}
  Проверить, что это -- представление
\end{exercise}

Введем на алгебре $\gf$ скалярное произведение, тогда $\gf^{*}$ -- пространство линейных
функционалов над $\gf$, -- можно отождествить с $\gf$: $(v,\cdot)\equiv v^{*}$. Соответственно,
можно определить коприсоединенное представление по правилу
\begin{equation}
  \label{eq:8}
  (Ad^{*} Y,X)=(Y,Ad^{-1}X)
\end{equation}

\begin{theorem}
  Унитарные представления группы $G$ находятся в однозначном соответствии с орбитами
  коприсоединенного представления $\gf^{*}/G$.
\end{theorem}

\begin{example}
  $G=\mathbb{R}^{n}$ -- группа Ли, $\gf=(\mathbb{R}^{n},[\;,\;]=0)$; орбиты -- точки в
  $\mathbb{R}^{n}$. Пусть $g_{1}\in G,\quad r\in \mathbb{R}^{n}$, тогда
  $D^{r}(g_{1})=e^{i(r,g_{1})}$. Существует представление $\tilde D(g_{1}): e^{i(r,g')}\to
  e^{i(r,g'-g_{1})}$. 
Левое действие $G$ на функциях $f:\mathbb{R}^{n}\to k,\quad D:f(\cdot)\to
  f(g^{-1}\cdot)$ раскладывается на неприводимые представления вида $\tilde D$, что соответствует
  разложению функции $f$ в ряд Фурье. Спектр представлений -- это орбиты коприсоединенного
  представления, то есть каждая точка пространства $\mathbb{R}^{n}$ задает неприводимое представление.
\end{example}

\subsubsection{Индуцированные представления}
\label{sec:induced-reps}

Рассмотрим $f\in Fun(G,k)$. По отношению к действию тривиальной подгруппы ${e}\subset G$ все функции
инвариантны.

Пусть $H\subset G,\; h\in H, g\in G$, рассмотрим функции $f:G\to V$, для которых выполняется
$f(gh)=D(h^{-1})\cdot f(g),\; D(h^{-1})\in D(H,V)$. Назовем пространство таких функций $f\in
Fun(G,H,V)$. Наша задача по $Fun(G,H,V)$ восстанавливать представление группы $G$. 

Обозначим через $D^{L}(g)\cdot f=f(g^{-1}\cdot)$ левый сдвиг, а через  $D^{R}(g)\cdot f=f(\cdot
g^{-1})$ -- правый. Тогда
\begin{equation}
  \label{eq:9}
  D^{L}(g)f(g'h)=f(g^{-1}g' h)=D(h^{-1}) f(g^{-1} g')=D(h^{-1})D^{L}(g)f(g')
\end{equation}

Покажем, что $Fun(G,H,V)$ инвариантно по отношению к $D^{L}$. Действие $D^{L}$ на $Fun(G,H,V)$ --
индуцированное представление $D_{H\uparrow G}$. Введем классы эквивалентности $H\backslash
G=\{H\}\cdot g_{x}$, $G/H=\{g_{x}\{H\}\}=X$, $x$ -- класс, $g_{x}$ -- представитель класса.
Рассмотрим функции на классах $\varphi(x)=f(g_{x})$, $f(g\cdot g_{x}h)=D(h^{-1})f(g_{x})$. То есть
если мы определим действие на факторпространстве $D(H)$ на $\varphi$ на $X$, то затем его можно
поднять до представления всей группы. 

\section{Лекция 2}
\label{sec:lecture-2}


От функций на группе переходим к функциям на классах смежности. 
\end{document}
