\documentclass[a4paper,12pt]{article}
\usepackage[utf8x]{inputenc}
\usepackage[russian]{babel}
\usepackage{ucs}
\usepackage[unicode,verbose]{hyperref}
\usepackage{amsmath,amssymb,amsthm}
\usepackage{mathtools}
\usepackage{pb-diagram}
\usepackage{multicol}
\usepackage{cmap}
\usepackage{color}
\usepackage{graphicx}
\pagestyle{plain}

%\usepackage{verbatim} 
\newenvironment{comment}
{\par\noindent{\bf TODO}\\}
{\\\hfill$\scriptstyle\blacksquare$\par}

\newtheorem{statement}{Statement}
\newtheorem{lemma}{Lemma}
\newtheorem{theorem}{Theorem}
\theoremstyle{definition}
\newtheorem{corollary}{Corollary}[theorem]
\theoremstyle{definition}
\newtheorem{mynote}{Замечание}[section]
\theoremstyle{definition}
\newtheorem{definition}{Definition}
\newcommand{\go}{\stackrel{\circ }{\mathfrak{g}}}
\newcommand{\ao}{\stackrel{\circ }{\mathfrak{a}}}
\newcommand{\co}[1]{\stackrel{\circ }{#1}}
\newcommand{\pia}{\pi_{\mathfrak{a}}}
\newcommand{\piab}{\pi_{\mathfrak{a}_{\bot}}}
\newcommand{\af}{\mathfrak{a}}
\newcommand{\afb}{\mathfrak{a}_{\bot}}

\title{Конформная теория поля и критические явления\\
\small{Академический университет, кафедра теоретической физики}
}
\author{Антон Назаров\\
  \small{СПбГУ, физический факультет,}\\
  \small{ кафедра физики высоких энергий и элементарных частиц;}\\
  \texttt{antonnaz@gmail.com}
}
\date{Весенний семестр 2015 года}
\begin{document}
\maketitle
\thispagestyle{empty}
\begin{abstract}
Программа курса
\end{abstract}
\tableofcontents

\section{Программа курса}
\label{sec:program}

\begin{enumerate}
\item Лекция 1
  \begin{itemize}
  \item Программа курса
  \item Литература
  \item Фазовые переходы. Основные понятия
  \item Классические статистические модели. Связь с теорией поля. Квантовые статистические модели.
    Классическая и квантовая теория поля. Симметрии. Теорема Нётер
  \item Критические явления
  \end{itemize}

\item Лекция 2. Двумерная модель Изинга
  \begin{itemize}
  \item Низкотемпературное и высокотемпературное разложение
  \item Дуальность Крамерса-Ванье
  \item Решение Вдовиченко
  \end{itemize}
\item Лекция 3. Скейлинг в критических моделях
  \begin{itemize}
  \item Процедура Каданова
  \item Ренормгруппа в решеточных моделях
  \item Ренормгруппа в непрерывных моделях
  \item Уравнение Каллана-Симанчика
  \item Ренормгруппа в двух измерениях
  \end{itemize}
\item Лекция 4. Конформная группа.
\item Лекция 5
  \begin{itemize}
  \item Инварианты конформных преобразований
  \item Явный вид конформных преобразований полей в $d\geq 3$.
  \end{itemize}
\item Лекция 6
  \begin{itemize}
  \item Корреляторы квазипримарных полей
  \item Тождества Уорда
  \end{itemize}
\item Лекция 7. Конформная инвариантность в двумерной теории.
  \begin{itemize}
  \item Глобальные конформные преобразования
  \item Тождества Уорда в двумерной конформной теории
  \end{itemize}
\item Лекция 8. Алгебраическая конструкция двумерной конформной теории поля
  \begin{itemize}
  \item Алгебра локальных полей. Операторное разложение
  \item Свободный бозон
  \item Центральный заряд. Алгебра Вирасоро
  \end{itemize}
\item Лекция 9. Классификация минимальных моделей
  \begin{itemize}
  \item Детерминант Каца
  \item Значения центрального заряда и соответствие с решеточными моделями
  \end{itemize}
\item Лекция 10. Представление кулоновского газа для минимальных моделей
\item Лекция 11. Уравнения на корреляционные функции в минимальных моделях
  \begin{itemize}
  \item Конформный предел модели Изинга
  \item Свободный фермион
  \item Выражение для четырехточечной функции в модели Изинга
  \end{itemize}
\item Лекция 12. Модели с дополнительной симметрией
  \begin{itemize}
  \item Модели Весса-Зумино-Новикова-Виттена
  \item Алгебры Каца-Муди
  \item $Z_{N}$-парафермионы
  \end{itemize}
\item Лекция 13. Поправки от конечного размера системы и теория с границей
\end{enumerate}
\end{document}
