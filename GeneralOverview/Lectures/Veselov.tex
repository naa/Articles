\documentclass[a4paper,12pt]{article}
\usepackage[utf8x]{inputenc}
\usepackage[russian]{babel}
\usepackage{ucs}
\usepackage[unicode,verbose]{hyperref}
\usepackage{amsmath,amssymb,amsthm}
%\usepackage{mathtools}
\usepackage{pb-diagram}
\usepackage{multicol}
\usepackage{cmap}
\usepackage{color}
\usepackage{graphicx}
\pagestyle{plain}

% \usepackage{verbatim} 
\newenvironment{comment}
{\par\noindent{\bf TODO}\\}
{\\\hfill$\scriptstyle\blacksquare$\par}

\newtheorem{statement}{Statement}
\newtheorem{lemma}{Lemma}
\newtheorem{theorem}{Theorem}
\theoremstyle{definition}
\newtheorem{corollary}{Corollary}[theorem]
\theoremstyle{definition}
\newtheorem{mynote}{Замечание}[section]
\theoremstyle{definition}
\newtheorem{definition}{Definition}
\newcommand{\go}{\stackrel{\circ }{\mathfrak{g}}}
\newcommand{\ao}{\stackrel{\circ }{\mathfrak{a}}}
\newcommand{\co}[1]{\stackrel{\circ }{#1}}
\newcommand{\pia}{\pi_{\mathfrak{a}}}
\newcommand{\piab}{\pi_{\mathfrak{a}_{\bot}}}
\newcommand{\af}{\mathfrak{a}}
\newcommand{\afb}{\mathfrak{a}_{\bot}}

\begin{document}
Напомним, что наша система состоит из частиц в точках $q_{i}$, гамильтониан системы имеет вид
\begin{equation}
  \label{eq:1}
  H=\frac{1}{2}\sum p_{i}^{2}+ \sum_{i=1}^{n-1} e^{q_{i}-q_{i+1}}
\end{equation}
Матрица Лакса
\begin{equation}
  \label{eq:2}
  L=
  \left(
    \begin{array}{cccc}
      a_{1} & b_{1} & \dots & 0 \\
      b_{1} & \ddots & \ddots & \vdots \\
      \vdots & \ddots & \ddots & b_{n-1} \\
      0 & \dots & b_{n-1} & a_{n}
    \end{array}
  \right)
\end{equation}
\begin{equation}
  \label{eq:3}
  B=
  \left(
    \begin{array}{cccc}
      0 & b_{1} & \dots & 0 \\
      b_{1} & \ddots & \ddots & \vdots \\
      \vdots & \ddots & \ddots & b_{n-1} \\
      0 & \dots & b_{n-1} & 0
    \end{array}
  \right),
\end{equation}
где
\begin{equation}
  \label{eq:4}
  a_{i}=\frac{1}{2} p_{i}, \quad b_{i}=\frac{1}{2}e^{q_{i}-q_{i+1}}
\end{equation}
Представление Лакса описывает цепочку Тоды, динамика определяется следующей леммой.
\begin{lemma}
  \begin{equation}
    \label{eq:5}
      \dot{L}=[B,L] \quad\Longleftrightarrow\quad \left\{
          \begin{array}{l}
            (L-\lambda E)x=0\\
            \dot{x}=B x,\quad |x|=1
          \end{array}
          \right.
  \end{equation}
$B$ описывает эволюцию собственного вектора $L$.
\begin{proof}
  $\Leftarrow )$

  \begin{equation*}
    Lx=\lambda x, \quad \dot{L}x+L\dot{x}=\lambda\dot{x}
  \end{equation*}
  \begin{equation*}
    \dot{L}x+L B  x=\lambda B x = B\lambda x= BL x
  \end{equation*}
  следовательно
  \begin{equation*}
    (\dot{L}+LB-BL)x=0\quad\Rightarrow \quad \dot{L}=[B,L]
  \end{equation*}

  $\Rightarrow )$

  Продифференцируем по времени равенство $(L-\lambda E)x=0$:
  \begin{equation*}
    \dot{L}x+(L-\lambda E)\dot{x}=0, \quad \dot{L}=BL-LB
  \end{equation*}
  То есть
  \begin{equation*}
    BLx-LBx+(L-\lambda E)\dot{x}=0
  \end{equation*}
  Первые два члена равны $-(L-\lambda E)Bx$, то есть равенство можно переписать, как
  \begin{equation*}
    (L-\lambda E)(\dot x -Bx)=0
  \end{equation*}
  Значит $\dot x-Bx$ -- тоже собственный вектор $L$, но мы знаем, что спектр $L$ -- простой, поэтому
  \begin{equation*}
    \dot x -Bx= \mu x
  \end{equation*}
  Умножим последнее равенство скалярно на $x$ и воспользуемся тем, что $B^{T}=-B$, а значит $(Bx,x)=(x,-Bx)=-(Bx,x)=0$, получаем
  \begin{equation*}
    (\dot x, x)-(B x, x)=\mu(x,x)
  \end{equation*}
  Первый член равен нулю, так как норма $|x|=1$, второй тоже равен нулю, следовательно и $\mu=0$, ч.т.д.
\end{proof}
\end{lemma}
Общая схема метода обратной задачи рассеяния:
\begin{equation*}
  \begin{diagram}
    \node{p(0),q(0)}\arrow{e,t}{?} \arrow{s,l}{S} \node{p(t),q(t)}\\
    \node{ \text{спектральные данные:}\;L,\lambda,\xi}\arrow{e,t}{\text{простая}}\node{\lambda,\xi(t)}\arrow{n,r}{\mbox{обратная задача}}
  \end{diagram}
\end{equation*}
\begin{theorem}
  Динамика цепочки Тоды описывается явно формулами
  \begin{eqnarray}
    \label{eq:6}
    \lambda (t)=\lambda(0)\\
    \xi(t)=\frac{e^{\Lambda t}\xi(0)}{\left|e^{\Lambda t}\xi(0)\right|}\\
    \Lambda=\left(
      \begin{array}{ccc}
        \lambda_{1}&\dots& 0 \\
        0 & \ddots & 0 \\
        0 & \dots & \lambda_{n}
      \end{array}
      \right)
  \end{eqnarray}
  \begin{proof}
    
  \end{proof}
\end{theorem}
\end{document}
