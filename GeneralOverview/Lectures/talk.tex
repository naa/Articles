\documentclass[a4paper,12pt]{article}
\usepackage[utf8x]{inputenc}
\usepackage[russian]{babel}
\usepackage{ucs}
\usepackage[unicode,verbose]{hyperref}
\usepackage{amsmath,amssymb,amsthm}
\usepackage{mathtools}
\usepackage{pb-diagram}
\usepackage{multicol}
\usepackage{cmap}
\usepackage{color}
\usepackage{graphicx}
\pagestyle{plain}

%\usepackage{verbatim} 
\newenvironment{comment}
{\par\noindent{\bf TODO}\\}
{\\\hfill$\scriptstyle\blacksquare$\par}

\newtheorem{statement}{Statement}
\newtheorem{lemma}{Lemma}
\newtheorem{theorem}{Theorem}
\theoremstyle{definition}
\newtheorem{corollary}{Corollary}[theorem]
\theoremstyle{definition}
\newtheorem{mynote}{Замечание}[section]
\theoremstyle{definition}
\newtheorem{definition}{Определение}
\newtheorem{exercise}{Упражнение}
\newtheorem{example}{Пример}
\newcommand{\go}{\stackrel{\circ }{\mathfrak{g}}}
\newcommand{\ao}{\stackrel{\circ }{\mathfrak{a}}}
\newcommand{\co}[1]{\stackrel{\circ }{#1}}
\newcommand{\pia}{\pi_{\mathfrak{a}}}
\newcommand{\piab}{\pi_{\mathfrak{a}_{\bot}}}
\newcommand{\gf}{\mathfrak{g}}
\newcommand{\af}{\mathfrak{a}}
\newcommand{\afb}{\mathfrak{a}_{\bot}}

\title{Записки с семинара ``Конформная теория поля''}
\author{Антон Назаров}

\begin{document}
\maketitle
\tableofcontents
\section{14 марта 2011}
\label{sec:14--2011}

\subsection{Классическая теория поля}
\label{sec:qft}
Напоминание про теорию поля.

Лагранжева формулировка классической теории поля.

 Действие
\begin{equation}
  S=\int d^dx \mathcal{L}
\end{equation}
зависит от полей $\varphi$ и констант связи $\vec u=(u_1,\dots,u_n)$.
Например:
\begin{equation}
  S[\varphi]=\int d^d x\left( -\frac{(\partial \varphi)^2}{2}-\frac{\mu \varphi^2}{2}-\frac{u\varphi^4}{24}+h\varphi\right),
\end{equation}


\subsection{Симметрии и генераторы}
\label{sec:symmetry}

При преобразованиях координат
\begin{equation}
  \label{eq:303}
  x\to x'
\end{equation}
поля тоже преобразуются, то есть у них не только меняется аргумент, но и само поле. Тип поля (скалярное, векторное, спинорное) -- это вид такого преобразования.
\begin{equation}
  \label{eq:304}
  \varphi(x)\to \varphi'(x')=F(\varphi(x))
\end{equation}

При этом действие тоже преобразуется:
\begin{equation}
  \label{eq:305}
  S'=\int d^{d}x' \mathcal{L}(\varphi'(x'),\partial_{\mu} \varphi'(x'))=\int d^{d}x \left|\frac{\partial x'}{\partial x}\right| \mathcal{L}(F(\varphi(x)),\frac{\partial x^{\nu}}{\partial x'^{\mu}} \partial_{\nu}F(\varphi(x)))
\end{equation}

\begin{example}
Трансляции:

\begin{equation}
  \label{eq:306}
  x'^{\mu}=x^{\mu}+a^{\mu},\quad \varphi'(x+a)=\varphi(x),\quad S'=S
\end{equation}
  
\end{example}
\begin{example}

Поворот (преобразования Лоренца):
\begin{equation}
  x'^{\mu}=m^{\mu}_{\nu} x^{\nu}, \quad \varphi'(m x)=\Lambda \varphi(x),\quad \Lambda \mbox{--представление группы} 
\end{equation}
  
\end{example}
Теперь рассмотрим инфинитезимальные преобразования. 
Если действие инвариантно относительно каких-либо преобразований, то говорят, что в теории есть симметрия. В этом случае действие должно быть стационарно по отношению к инфинитезимальным преобразованиям. Мы будем считать, что инфинитезимальные параметры преобразований зависят от координат, то есть рассматривать не только глобальные, но и локальные преобразования. 

\begin{equation}
  \label{eq:165}
  x'^{\mu}=x^{\mu}+\omega_a \frac{\delta x^{\mu}}{\delta \omega_a}
\end{equation}
($\omega_{a}$ -- бесконечно малые параметры).

Наше поле преобразуется так:
\begin{equation}
  \label{eq:166}
  \varphi'(x')=\varphi(x)+\omega_a \frac{\delta F}{\delta \omega_a} (x).
\end{equation}
Генератор преобразования определяется следующим равенством:
\begin{equation}
  \label{eq:167}
  \delta_{\omega} \varphi(x)=\varphi'(x)-\varphi(x)\equiv -i\omega_a G_a \varphi(x)
\end{equation}
(здесь нет суммирования по $a$). Действие генератора на поле:
\begin{equation}
  \label{eq:168}
  iG_a \varphi=\frac{\delta x^{\mu}}{\delta\omega_a} \partial_{\mu}\varphi-\frac{\delta F}{\delta \omega_a}
\end{equation}

\begin{example}
Если мы предположим, что поле $\varphi$ -- такое поле, которое не меняется при конформных преобразованиях, то есть $F(\varphi)=\varphi$, то мы получим следующий вид для генераторов:
\begin{eqnarray}
  \label{eq:169}
  \mbox{трансляция}  & x'^{\mu}=x^{\mu}+\omega^{\mu}& \quad P_{\mu}=-i\partial_{\mu}\\
  \mbox{поворот} & x'^{\mu}=x^{\mu}+\omega^{\mu}_{\nu}x^{\nu} & \quad L_{\mu\nu}=i(x_{\mu}\partial_{\nu}-x_{\nu}\partial{\mu})
\end{eqnarray}
Если же при поворотах поле преобразуется $\varphi'(x')=\Lambda\varphi(x)$, то при инфинитезимальных преобразованиях $\varphi'(x')=\varphi(x)-\frac{i}{2}\omega_{\mu\nu}S^{\mu\nu}$ и генератор принимает вид
\begin{equation}
  L_{\mu\nu}=i(x_{\mu}\partial_{\nu}-x_{\nu}\partial{\mu})+S_{\mu\nu}
\end{equation}
  
\end{example}

\subsection{Теорема Нётер}
\label{sec:noether}

Теперь рассмотрим произвольные инфинитезимальные преобразования, при которых действие не меняется. Инфинитезимальные параметры зависят от точки в пространстве, то есть мы рассматриваем локальные преобразования. Преобразования, для которых такой зависимости нет, называются глобальными. 
Якобиан такого преобразования
\begin{eqnarray}
  \frac{\partial x'^{\nu}}{\partial x^{\mu}} = \delta^{\mu}_{\nu}+\partial_{\mu}\left(\omega_a \frac{\delta x^{\nu}}{\delta \omega_a}\right)\\
  \frac{\partial x^{\nu}}{\partial x'^{\mu}} = \delta^{\mu}_{\nu}-\partial_{\mu}\left(\omega_a \frac{\delta x^{\nu}}{\delta \omega_a}\right)
\end{eqnarray}
\begin{equation}
  \left|\frac{\partial x'^{\nu}}{\partial x^{\mu}}\right| = 1+\partial_{\mu}\left(\omega_a \frac{\delta x^{\mu}}{\delta \omega_a}\right)
\end{equation}
Вариация действия
\begin{multline}
  0=\delta_{\omega} S=\int \mathcal{L}(\varphi',\partial_{\mu}\varphi',x'^{\mu})dx'-\int \mathcal{L}(\varphi,\partial_{\mu}\varphi,x)dx=\\
  \int d^{d}x\left[ \left( 1+\partial_{\mu}\left(\omega_a \frac{\delta x^{\mu}}{\delta \omega_a}\right)\right) \mathcal{L} \left(\varphi+\delta_{\omega}\varphi,
    \partial_{\mu}\varphi+\partial_{\mu}\delta_{\omega}\varphi\right)\right]-\mathcal{L}(\varphi,\partial_{\mu}\varphi)=
\end{multline}
\begin{multline}
      \int d^{d}x\left( \frac{\partial \mathcal{L}}{\partial \varphi} \delta_{\omega}\varphi +\frac{\partial \mathcal{L}}{\partial(\partial_{\mu}\varphi)}\delta_{\omega}\partial_{\mu}\varphi +\partial_{\mu}\left(\omega_a \frac{\delta x^{\mu}}{\delta \omega_a}\right)\mathcal{L}\right)
\end{multline}

Члены, не содержащие производных от $\omega_{a}$ зануляются в случае наличия глобальной симметрии.

Первые два члена переписываются с использованием уравнений Эйлера-Лагранжа и интегрирования по частям:
\begin{equation}
  \partial_{\mu}\left(\frac{\partial \mathcal{L}}{\partial (\partial_{\mu}\varphi)}\right)\delta_{\omega}\varphi+\frac{\partial \mathcal{L}}{\partial (\partial_{\mu}\varphi)}\delta_{\omega}\partial_{\mu}\varphi=
    \partial_{\mu}\left(\frac{\partial \mathcal{L}}{\partial(\partial_{\mu}\varphi)}\delta_{\omega}\varphi\right)
\end{equation}
В итоге имеем:
\begin{equation}
  0=\int d^{d}x \partial_{\mu}\left(\frac{\partial \mathcal{L}}{\partial(\partial_{\mu}\varphi)}\delta_{\omega}\varphi +\delta^{\mu}_{\nu}\mathcal{L} \frac{\delta x^{\nu}}{\delta \omega_{a}}\right)
\end{equation}
Если подставить явный вид вариации поля и ввести обозначение
\begin{equation}
  j^{\mu}_{a}=\left(\frac{\partial \mathcal{L}}{\partial(\partial_{\mu}\varphi)}\partial_{\nu}\varphi -\delta^{\mu}_{\nu}\mathcal{L}\right) \frac{\delta x^{\nu}}{\delta \omega_{a}}-\frac{\partial \mathcal{L}}{\partial(\partial_{\mu}\varphi)}\frac{\delta F}{\delta w_{a}},
\end{equation}
то для вариации действия выходит:
\begin{equation}
  \delta_{\omega}S=-\int d^{d}x j^{\mu}_{a}\partial_{\mu}\omega_{a}
\end{equation}
$j^{\mu}_{a}$ называется Нётеровским током, соответствующим данной симметрии. Проинтегрируем по частям и получим:
\begin{equation}
    \delta_{\omega}S=\int d^{d}x (\partial_{\mu} j^{\mu}_{a})\omega_{a}
\end{equation}
Если поля удовлетворяют уравнениям движения, то действие инвариантно относительно любой вариации полей, то есть вариация действия должна зануляться для любого $\omega(x)$. Тогда
\begin{equation}
  \partial_{\mu}j^{\mu}_{a}=0
\end{equation}
Таким образом каждой симметрии соответствует ток $j_{\mu}^{a}$ и сохраняющийся заряд:
\begin{equation}
  Q_{a}=\int d^{d-1}x\; j^{0}_{a}
\end{equation}

\begin{example}
  
 Рассмотрим инфинитезимальное масштабное преобразование $\lambda=1+\omega$. При этом вариация $x_{\nu}$ будет $\delta_{\omega}x_{\nu}=\omega x_{\nu}$.  При инфинитезимальных преобразованиях $x\to x+dl x$ вариация плотности лагранжиана дается Нётеровским током:
\begin{equation}
  \label{eq:74}
  \delta\mathcal{L}=\partial_{\mu}J^{\mu}dl
\end{equation}
\begin{equation}
  \label{eq:75}
  J^{\mu}=x_{\nu}T^{\mu\nu}
\end{equation}
Здесь $T^{\mu\nu}$ - тензор энергии-импульса.  Через $\Theta$ мы обозначим след тензора энергии-импульса $\Theta=\partial_{\mu}J^{\mu}=T_{\mu}^{\mu}$. Он задает вариацию действия
\begin{equation}
  \label{eq:87}
  \delta S=dl \int d^d x \Theta(x)
\end{equation}
Если теория масштабно-инвариантна, то $\partial_{\mu}J^{\mu}=T_{\mu}^{\mu}=\Theta=0$. В масштабно-инвариантной теории тензор энергии-импульса бесследовый.
\end{example}

\section{21 марта 2011}
\label{sec:21--2011-1}

\subsection{Конформная алгебра в $d\geq 3$}
\label{sec:conformal}

 Сейчас мы обсудим, какие ограничения накладывает на теорию конформная инвариантность, а затем покажем, что в двух измерениях она следует из масштабной, трансляционной и вращательной инвариантностей.

Сперва рассмотрим конформную группу в произвольном числе измерений $d$. Метрический тензор обозначим через $g_{\mu\nu},\; \mu,\nu=1,\dots,d$. Конформными называются преобразования $x\to x'$, сохраняющие метрический тензор с точностью до масштаба:
\begin{equation}
  \label{eq:123}
  g'_{\mu\nu}(x')=\Lambda(x) g_{\mu\nu}(x).
\end{equation}
Заметим, что группа Пуанкаре является подгруппой конформной группы с $\Lambda(x)=1$, а также что конформные преобразования сохраняют углы. 

Рассмотрим инфинитезимальные преобразования
\begin{equation}
  \label{eq:144}
  x^{\mu}\to x'^{\mu}=x^{\mu}+\epsilon^{\mu}.
\end{equation}
Метрический тензор преобразуется следующим образом:
\begin{equation}
  \label{eq:145}
  g'_{\mu\nu}=\frac{\partial x^{\alpha}}{\partial x'^{\mu}}\frac{\partial x^{\beta}}{\partial x'^{\nu}} g_{\alpha\beta}=(\delta^{\alpha}_{\mu}-\partial_{\mu}\epsilon^{\alpha})(\delta^{\beta}_{\nu}-\partial_{\nu} \epsilon^{\beta})g_{\alpha\beta}=g_{\mu\nu}-(\partial_{\mu}\epsilon_{\nu}+\partial_{\nu}\epsilon_{\mu})
\end{equation}
Перепишем условие (\ref{eq:123})в таком виде:
\begin{equation}
  \label{eq:146}
  g'_{\mu\nu}(x')=g_{\mu\nu}(x)-f(x)g_{\mu\nu}(x)
\end{equation}
Отсюда вытекает условие на вид преобразований:
\begin{equation}
  \label{eq:147}
  \partial_{\mu}\epsilon_{\nu}+\partial_{\nu}\epsilon_{\mu}=f(x)g_{\mu\nu}(x).
\end{equation}
Для простоты рассмотрим преобразования, действующие на плоскую метрику $g_{\mu\nu}(x)=\eta_{\mu\nu}$, кроме того, с учетом приложений к статистической физике, будем работать в евклидовом пространстве, а не в пространстве Минковского. Так что $\eta=\mathrm{diag}(1,1,\dots,1),\quad \eta_{\mu\nu}=\delta_{\mu\nu}$. В этом случае условие \eqref{eq:147} перепишется в простом виде
\begin{equation}
  \label{eq:148}
  f(x)=\frac{2}{d}\partial_{\rho}\epsilon^{\rho}.
\end{equation}
Теперь подставим $g_{\mu\nu}=\eta_{\mu\nu}$ в уравнение \eqref{eq:147} и продиффернцируем:
\begin{equation}
  \label{eq:149}
  \partial_{\rho} \partial_{\mu}\epsilon_{\nu}+\partial_{\rho}\partial_{\nu}\epsilon_{\mu}=\eta_{\mu\nu}\partial_{\rho} f.
\end{equation}
Переставим два раза значки и скомбинируем три уравнения в одно:
\begin{equation}
  \label{eq:150}
  2\partial_{\mu}\partial_{\nu}\epsilon_{\rho}=\eta_{\mu\rho}\partial_{\nu} f+\eta_{\nu\rho}\partial_{\mu}f-\eta_{\mu\nu}\partial_{\rho}f
\end{equation}
Свернем это уравнение с $\eta^{\mu\nu}$ и получим
\begin{equation}
  \label{eq:151}
  2\partial^2 \epsilon_{\rho}=(2-d)\partial_{\rho}f
\end{equation}
Теперь продифференцируем его по $x^{\nu}$ и поменяем значок $\rho$ на $\mu$:
\begin{equation}
  \label{eq:152}
  2\partial^2 \partial_{\nu} \epsilon_{\mu}=(2-d)\partial_{\mu}\partial_{\nu} f
\end{equation}
Сравним полученное равенство с результатом применения оператора $\partial^2$ к уравнению \eqref{eq:149}:
\begin{equation}
  \label{eq:153}
  \partial^2 \partial_{\mu}\epsilon_{\nu}+\partial^2 \partial_{\nu}\epsilon_{\mu}=\eta_{\mu\nu}\partial^2 f
\end{equation}
Из равенств \eqref{eq:152}, \eqref{eq:153} следует, что
\begin{equation}
  \label{eq:154}
  (2-d)\partial_{\mu}\partial_{\nu}f=\eta_{\mu\nu}\partial^2 f.
\end{equation}
Свернув с $\eta^{\mu\nu}$ получим
\begin{equation}
  \label{eq:155}
  (d-1)\partial^2 f =0.
\end{equation}
Сразу можно отметить, что при $d=1$ любое гладкое преобразование будет конформным. Рассмотрим случай $d\geq 3$. Функция $f(x)$ должна иметь вид
\begin{equation}
  \label{eq:156}
  f(x)=A+B_{\mu}x^{\mu}.
\end{equation}
Тогда из \eqref{eq:148} получаем для $\epsilon$
\begin{equation}
  \label{eq:157}
  \epsilon_{\mu}=a_{\mu}+b_{\mu\nu}x^{\nu} +c_{\mu\nu\rho}x^{\nu}x^{\rho},\quad c_{\mu\nu\rho}=c_{\mu\rho\nu}
\end{equation}
Так как равенства \eqref{eq:147}, \eqref{eq:148}, \eqref{eq:150} должны выполняться для любых $x^{\mu}$, то мономы в $\epsilon$ можно рассматривать независимо. На $a_{\mu}$ не возникает никаких ограничений. Этот член соответствует трансляциям. Теперь подставляем линейный член в \eqref{eq:147}, \eqref{eq:148} и получаем условие
\begin{equation}
  \label{eq:158}
  b_{\mu\nu}+b_{\nu\mu}=\frac{2}{d}b^{\lambda}_{\lambda}\eta_{\mu\nu}
\end{equation}
То есть $b_{\mu\nu}$ можно записать в виде
\begin{equation}
  \label{eq:159}
  b_{\mu\nu}=\alpha \eta_{\mu\nu} +m_{\mu\nu},\quad m_{\mu\nu}=-m_{\nu\mu}
\end{equation}
Первый член соответствует масштабному преобразованию, а второй - повороту. 
В результате подстановки квадратичного члена $\epsilon$ в \eqref{eq:148}, \eqref{eq:150} получаем следующее условие на $c_{\mu\nu\rho}$:
\begin{equation}
  \label{eq:160}
  c_{\mu\nu\rho}=\eta_{\mu\rho} h_{\nu} +\eta_{\mu\nu}h_{\rho}-\eta_{\nu\rho}h_{\mu}, \quad h_{\mu}=\frac{1}{d} c^{\alpha}_{\alpha\mu}
\end{equation}
Ему соответствует преобразование
\begin{equation}
  \label{eq:161}
  x'^{\mu}=x^{\mu}+2 (x^{\nu}h_{\nu})x^{\mu} -h^{\mu} x^{\nu}x_{\nu}
\end{equation}
Такое преобразование называется специальным конформным преобразованием.
Это преобразование можно естественно интерпретировать, если переписать в виде
\begin{equation}
  \label{eq:162}
  \frac{x'^{\mu}}{x'^2}=\frac{x^{\mu}}{x^2}-h^{\mu}.
\end{equation}
Видно, что специальное конформное преобразование --- это инверсия, трансляция и обратная инверсия. 

Соответствующие конченые конформные преобразования имеют вид
\begin{eqnarray}
  \label{eq:163}
  x'^{\mu}=x^{\mu}+a^{\mu}&\quad \mbox{--- трансляция}\\
  x'{\mu}=\alpha x^{\mu} &\quad \mbox{--- растяжение}\\
  x'^{\mu}=m^{\mu}_{\nu} x^{\nu} &\quad \mbox{--- поворот}\\
  x'^{\mu}=\frac{x^{\mu}-h^{\mu}x^2}{1-2h_{\mu}x^{\mu}+h^2 x^2}  & \quad \begin{array}{c}\mbox{--- специальное конформное}\\ \mbox{ преобразование}\end{array}
\end{eqnarray}

Теперь выпишем вид генераторов конформных преобразований для скалярного поля. 
Напомним, что при произвольном конечном преобразовании скалярное поле преобразуется как
\begin{equation}
  \label{eq:164}
  \Phi'(x')=F(\Phi(x)).
\end{equation}
При соответствующем инфинитезимальном преобразовании
\begin{equation}
  \label{eq:10}
  x'^{\mu}=x^{\mu}+\omega_a \frac{\delta x^{\mu}}{\delta \omega_a}
\end{equation}
скалярное поле преобразуется так:
\begin{equation}
  \label{eq:11}
  \Phi'(x')=\Phi(x)+\omega_a \frac{\delta F}{\delta \omega_a} (x).
\end{equation}
Генератор преобразования определяется следующим равенством:
\begin{equation}
  \label{eq:12}
  \delta_{\omega} \Phi(x)=\Phi'(x)-\Phi(x)\equiv -i\omega_a G_a \Phi(x)
\end{equation}
(здесь нет суммирования по $a$). Из \eqref{eq:166} получаем действие генератора на скалярное поле:
\begin{equation}
  \label{eq:13}
  iG_a \Phi=\frac{\delta x^{\mu}}{\delta\omega_a} \partial_{\mu}\Phi-\frac{\delta F}{\delta \omega_a}
\end{equation}

Если мы предположим, что поле $\Phi$ такое поле, которое не меняется при конформных преобразованиях, то есть $F(\Phi)=\Phi$, то мы получим следующий вид для генераторов:
\begin{eqnarray}
  \label{eq:14}
  \mbox{трансляция} & \quad P_{\mu}=-i\partial_{\mu}\\
  \mbox{поворот} & \quad L_{\mu\nu}=i(x_{\mu}\partial_{\nu}-x_{\nu}\partial{\mu})\\
  \mbox{растяжение}& \quad D=-ix^{\mu}\partial_{\mu}\\
  \begin{array}{r}
  \mbox{специальное конформное}    \\
  \mbox{преобразование}
  \end{array}
  & \quad K_{\mu}=-i(2x_{\mu}x^{\nu}\partial_{\nu}-x^2\partial_{\mu})
\end{eqnarray}
Отсюда легко найти коммутационные соотношения алгебры конформных преобразований в случае $d\geq 3$:
\begin{eqnarray}
  \label{eq:170}
  \left[D,P_{\mu}\right]=i P_{\mu}\\
  \left[D,K_{\mu}\right]=-i K_{\mu}\\
  \left[K_{\mu},P_{\nu}\right]=2i (\eta_{\mu\nu}D - L_{\mu\nu})\\
  \left[K_{\rho},L_{\mu\nu}\right]=i(\eta_{\rho\mu}P_{\nu}-\eta_{\rho\nu}P_{\mu})\\
 \left[L_{\mu\nu},L_{\rho\sigma}\right]=i(\eta_{\nu\rho}L_{\mu\sigma}+\eta_{\mu\sigma}L_{\nu\rho}-\eta_{\mu\rho}L_{\nu\sigma}-\eta_{\nu\sigma}L_{\mu\rho})
\end{eqnarray}
Остальные коммутаторы равны нулю.
Чтобы понять, о какой алгебре идет речь, переопределим генераторы следующим образом. Введем генераторы $J_{ab}, \; a,b=-1,0,\dots,d, J_{ab}=-J_{ba}$:
\begin{eqnarray}
  \label{eq:171}
  J_{\mu\nu}=L_{\mu\nu}\\
  J_{-1,0}=D\\
  J_{-1,\mu}=\frac{1}{2}(P_{\mu}-K_{\mu})\\
  J_{0,\mu}=\frac{1}{2}(P_{\mu}+K_{\mu})
\end{eqnarray}
Коммутационные соотношения для таких генераторов запишутся в виде
\begin{equation}
  \label{eq:172}
  \left[J_{ab},J_{cd}\right]=i(\eta_{ad}J_{bc}+\eta_{bc}J_{ad}-\eta_{ac}J_{bd}-\eta_{bd}J_{ac}),
\end{equation}
где $\eta_{ab}=\mathrm{diag}(-1,1,\dots,1)$. Видно, что мы получили алгебру $so(d+1,1)$. В случае пространства Минковского была бы $so(d,2)$.

\subsection{Локальные конформные преобразования в двумерной теории}
\label{sec:2d-cft}

Мы получили общее условие на вид инфинитезимальных конформных преобразований:
\begin{equation}
  \label{eq:223}
  \partial_{\mu}\epsilon_{\nu}+\partial_{\nu}\epsilon_{\mu}=\frac{2}{d} \partial_{\rho}\epsilon^{\rho} \eta_{\mu\nu}.
\end{equation}
Теперь у нас $d=2$, $\mu,\nu=0,1$ и $\eta_{\mu\nu}=\delta_{\mu\nu}$, так как мы работаем в евклидовой теории. Расписывая компоненты уравнения \eqref{eq:223}, получаем
\begin{eqnarray}
  \label{eq:224}
  \partial_{0} \epsilon_{1}+ \partial_{1}\epsilon_{0}=0& \Rightarrow & \quad\partial_{0} \epsilon_{1}=- \partial_{1}\epsilon_{0}\\
  2\partial_{0}\epsilon_{0}=\partial_{0}\epsilon_{0}+\partial_{1}\epsilon_{1}& \Rightarrow& \quad \partial_{0} \epsilon_{0}= \partial_{1}\epsilon_{1}
\end{eqnarray}
То есть мы получили уравнения Коши-Римана. Введем комплексные координаты
\begin{eqnarray}
  \label{eq:225}
  z=x_{0}+i x_{1}\\
  \bar z = x_{0}-i x_{1}\\
  \partial = \partial_{z}=\frac{1}{2}(\partial_{0}-i\partial_{1}) \\
  \bar \partial=\partial_{\bar z}=\frac{1}{2} (\partial_{0}+i \partial_{1})\\
  \epsilon=\epsilon_{0}+i \epsilon_{1}\\
  \bar \epsilon=\epsilon_{0}-i \epsilon_{1}, 
\end{eqnarray}
тогда уравнения \eqref{eq:224} можно переписать в виде
\begin{eqnarray}
  \label{eq:226}
   \bar \partial \epsilon=0\\
  \partial\bar \epsilon=0.
\end{eqnarray}
Решениями будут любые голоморфные и антиголоморфные функции: $\epsilon=\epsilon(z)$ -- голоморфная функция и $\bar \epsilon=\bar \epsilon(\bar z)$ -- антиголоморфная. Таким образом мы видим, что алгебра локальных конформных преобразований в двумерном случае оказывается бесконечномерной алгеброй преобразований
\begin{eqnarray}
  \label{eq:227}
  z\to f(z)\\
  \bar z \to \bar f(\bar z)
\end{eqnarray}
Введем в алгебре конформых преобразований следующий базис:
\begin{eqnarray}
  \label{eq:228}
  z'=z+\epsilon_{n}(z)\\
  \bar z'=\bar z+\bar \epsilon_{n}(\bar z)\\
  \epsilon_{n}(z)= -z^{n+1}\\
  \bar \epsilon_{n}(\bar z)=-\bar z^{n+1}
\end{eqnarray}
Тогда соответствующие генераторы будут равны
\begin{eqnarray}
  \label{eq:229}
  l_{n}=-z^{n+1}\partial\\
  \bar l_{n}=-\bar z^{n+1}\bar \partial 
\end{eqnarray}
Легко видеть, что коммутационные соотношения имеют вид
\begin{eqnarray}
  \label{eq:230}
  \left[l_{m},l_{n}\right]=(m-n) l_{m+n}\\
  \left[\bar l_{m},\bar l_{n}\right]=(m-n) \bar l_{m+n}\\  
  \left[l_{n},\bar l_{m}\right]=0
\end{eqnarray}
Мы видим, что алгебра распадается в прямую сумму $\mathcal{A} \oplus \bar{\mathcal{A}}$, каждая компонента --- это алгебра Витта (Witt algebra). Оказывается удобно продолжить теорию на случай независимых $z,\bar z$. Тогда теория распадется на два независимых сектора. Условие же $z^{*}=\bar z$ можно наложить в самом конце. Такая процедура соответствует комплексному продолжению всех функций от $x_{0},x_{1}$ на $x_{0},x_{1}\in \mathbb{C}^{2}$. Заметим, что вещественная плоскость сохраняется подалгеброй, натянутой на генераторы $l_{n}+\bar l_{n}$ и $i(l_{n}-\bar l_{n})$.

\subsection{Глобальные конформные преобразования}
\label{sec:global-conformal}

Глобальными называются те преобразования, которые определены на всей сфере Римана $S^{2}=\mathbb{C}^{2}\cup \{\infty\}$. Понятно, что это может быть только при $n\geq -1$. Кроме того, чтобы рассмотреть окрестность точки $z=\infty$ можно сделать преобразование координат $z=-\frac{1}{w}$. Тогда
\begin{equation}
  \label{eq:231}
  l_{n}=-z^{n+1}\partial = - \left(-\frac{1}{w}\right)^{n+1}\left(\frac{\partial z}{\partial w}\right) \partial_{w}=-\left(-\frac{1}{w}\right)^{n-1}\partial_{w}.
\end{equation}
Это выражение должно быть хорошо определено при $w\to 0$, то есть $n\leq 1$. Значит глобальные конформные преобразования генерируются $l_{\pm 1},l_{0},\bar l_{\pm 1},\bar l_{0}$. Заметим, что генераторы $l_{-1},\bar l_{-1}$ порождают трансляции, $i(l_{0}-\bar l_{0})$ -- вращения, $l_{0}+\bar l_{0}$ -- масштабные преобразования. Конечные глобальные преобразования имеют вид
\begin{equation}
  \label{eq:232}
  z\to \frac{az+b}{cz+d}, \quad ad-bc=1,\quad a,b,c,d\in \mathbb{C}
\end{equation}
(И аналогично для $\bar z$).
Если собрать коэффициенты $a,b,c,d$ в матрицу $A=
\begin{pmatrix}
  a & b\\
  c & d
\end{pmatrix}$, то ясно, что мы имеем дело с группой $SL_{2}(\mathbb{C})/Z_{2}\approx SO(3,1)$. Факторизация по $Z_{2}$ соответствует тому, что изменение знака у $a,b,c,d$ разом не меняет преобразования. Эта группа называется также группой проективных конформных преобразований. 

Трансляции, дилатации и вращения в матричном виде записываются следующим образом:
\begin{eqnarray}
  \label{eq:233}
  &\mbox{трансляции} \quad x\to x+a, B=a^{0}+i a^{1},\quad &
  \begin{pmatrix}
    1 & B\\
    0 & 1
  \end{pmatrix}\\
  &\mbox{вращения}\quad &
  \begin{pmatrix}
    e^{i \Theta/2} & 0\\
    0 & e^{-i \Theta/2}
  \end{pmatrix}\\
  & \mbox{дилатации}\quad &
  \begin{pmatrix}
    \lambda & 0\\
    0 & \lambda^{-1}
  \end{pmatrix}\\
  &\begin{array}{r}
  \mbox{специальные конформные}    \\
  \mbox{преобразования}
  \end{array} 
  \quad
  C=h_{0}-i h_{1}\quad &
  \begin{pmatrix}
    1 & 0\\
    C & 1
  \end{pmatrix}
\end{eqnarray}

Глобальные конформные преобразования образуют группу, локальные же преобразования не обратимы, поэтому по отношению к ним говорят только об алгебре.

Глобальные преобразования полезны для описания физических состояний. Допустим, мы работаем в базисе собственных состояний операторов $l_{0}, \bar l_{0}$, соответствующие собственные значения $h,\bar h$ -- независимые, вещественные, называются конформными весами или голоморфной и антиголоморфной размерностями. Так как $l_{0}+\bar l_{0}$ -- генератор дилатации, то скейлинговая размерность $\Delta=h+\bar h$, а поскольку $i(l_{0}-\bar l_{0})$ порождает вращения, то спин $s=h-\bar h$.

\subsection{Центральное расширение алгебры Витта}
\label{sec:witt-algebra-central-extension}

Напомним несколько определений (см \cite{fuks1986cohomology},\cite{fuks1984}, \cite{feigin1988}).
\begin{definition}
Будем рассматривать алгебру Ли $\gf$ над полем $k=\mathbb{C},\mathbb{R}$. Векторное пространство $A$ над полем $k$ называется {\it модулем над $\gf$} или {\it $\gf$-модулем}, если задано билинейное отображение $\mu:\gf\times A\to A$, такое что $\mu([X,Y],a)=\mu(X,\mu(Y,a))-\mu(Y,\mu(X,a))$ для $X,Y\in \gf, a\in A$. Далее мы будем опускать символ $\mu$ и писать $X a= \mu(X,a)$.
\end{definition}
\begin{definition}
  {\it $n$-мерная коцепь с коэффициентами в $A$} -- это кососимметричный $n$-линейный функционал на $\gf$ со значениями в $A$. Пространство $n$-коцепей $C^{n}(\gf;A)=\mathrm{Hom}(\wedge^{n}\gf;A)$.
\end{definition}
Заметим, что элементы $\gf$ действуют на $C^{n}(\gf;A)$.
\begin{definition}
  Внешний дифференциал $d=d_{n}:C^{n}(\gf;A)\to C^{n+1}(\gf;A)$ определяется формулой
  \begin{multline}
    \label{eq:1}
    (dc) (X_{1},\dots, X_{n+1})=\sum_{1\leq s<t\leq n+1} (-1)^{s+t-1} c([X_{s},X_{t}],X_{1},\dots,\hat X_{s},\dots,\hat X_{t},\dots X_{n+1})\\
    +\sum_{1\leq s\leq n+1} (-1)^{s}X_{s} c(X_{1},\dots \hat X_{s},\dots X_{n+1})
  \end{multline}
\end{definition}
Оставляем в качестве упражнения проверку того факта, что последовательность
\begin{equation}
  \label{eq:2}
  \dots\stackrel{d_{n}}{\longleftarrow} C^{n}(g;A)\stackrel{d_{n-1}}{\longleftarrow} C^{n-1}(\gf;A)\leftarrow \dots \leftarrow C^{1}(\gf;A) \stackrel{d_{0}}{\longleftarrow} C^{0}(\gf;A)\leftarrow 0
\end{equation}
точна.Тогда $\{C^{n}(\gf;A),d_{n}\}=C^{*}(\gf;A)$ есть комплекс.
\begin{definition}
Соответствующие когомологии называются {\it когомологиями алгебры $\gf$ с коэффициентами в $A$} и обозначаются через $H^{n}(\gf;A)=\mathrm{Ker}\; d_{n}/\mathrm{Im}\; d_{n-1}$. 
  
\end{definition}
Заметим, что поле $k$ может рассматриваться как тривиальный $\gf$-модуль. В этом случае второй член в формуле \eqref{eq:1} исчезает и используют сокращенные обозначения $C^{n}(\gf), H^{n}(\gf)$.
\begin{definition}
  Определим, заодно, и гомологии. Пространство $n$-мерных цепей $C_{n}(\gf;A)=A\otimes \wedge^{n}\gf$, дифференциал $\partial=\partial_{n}:C_{n}(\gf;A)\to C_{n-1}(\gf;A)$ определяется формулой
  \begin{multline}
    \label{eq:3}
    \partial(a\otimes (X_{1}\wedge \dots \wedge X_{n})) =\\ \sum_{1\leq s<t\leq n+1} (-1)^{s+t-1} a\otimes ([X_{s},X_{t}]\wedge X_{1} \wedge\dots\wedge\hat X_{s}\wedge \dots\wedge \hat X_{t}\wedge \dots\wedge X_{n+1})\\
    +\sum_{1\leq s\leq n+1} (-1)^{s}(X_{s} a)\otimes (X_{1}\wedge\dots\wedge \hat X_{s}\wedge\dots\wedge X_{n+1})
  \end{multline}
  Аналогично определяется точная последовательность, комплекс и группа гомологий $H_{n}(\gf;A)$.
\end{definition}
\begin{definition}
  {\it Одномерным центральным расширением алгебры $\gf$ } называется точная последовательность
  \begin{equation}
    \label{eq:4}
    0\to k \to \tilde \gf\to \gf\to 0,
  \end{equation}
  такая что образ $k\to \tilde\gf$ содержится в центре $\tilde\gf$.
\end{definition}
Заметим, что всякий 2-коцикл $c\in C^{2}(\gf;A)$ определяет центральное расширение $\gf$:
\begin{equation}
  \label{eq:5}
  0\to k\stackrel{\lambda\to (0,\lambda)}{\longrightarrow}\tilde\gf=\gf\oplus k \stackrel{(X,\lambda)\to X}{\longrightarrow} \gf\to 0
\end{equation}
Скобка Ли в алгебре $\tilde\gf$, которая равна $\gf\oplus k$ как векторное пространство, определяется равенством
\begin{equation}
  \label{eq:6}
  [(X,\lambda),(Y,\mu)]=([X,Y],c(X,Y))
\end{equation}
Тождество Якоби для такой скобки равносильно тому, что $c$-- 2-коцикл. Когомологичным коциклам отвечают эквивалентные расширения. 

Два расширения $\tilde \gf$ и $\tilde{\tilde\gf}$ называются эквивалентными, если существует изоморфизм $I:\tilde\gf\to \tilde{\tilde \gf}$ такой, что диаграмма коммутирует:
\begin{equation}
  \label{eq:7}
  \begin{diagram}
    \node 0 \arrow{e,t}{}  \node{\gf} \arrow{e,t}{} \arrow{s,l}{id} \node{\tilde\gf}\arrow{s,l}{I} \arrow{e,t}{} \node{k}\arrow{s,r}{id} \arrow{e,t}{} \node{0}\\
    \node 0 \arrow{e,t}{}  \node{\gf} \arrow{e,t}{} \node{\tilde{\tilde\gf}} \arrow{e,t}{} \node{k} \arrow{e,t}{} \node{0}
  \end{diagram}
\end{equation}
\begin{exercise}
 Построить изоморфизм $I$.   
\end{exercise}


Таким образом пространство $H^{2}(\gf)$ -- это множество классов 1-мерных центральных расширений $\gf$. Нуль в $H^{2}(\gf)$  соответствует тривиальному расширению. 

Вернемся к алгебре Витта \eqref{eq:230}. Существует коцикл
\begin{equation}
  \label{eq:8}
  c(l_{n},l_{m})=\frac{1}{12}(m^{3}-m)\delta_{-n,m}
\end{equation}
\begin{exercise}
  Проверить, что $c$ -- коцикл алгебры Витта. 
\end{exercise}
Соответствующее центральное расширение алгебры Витта называется алгеброй Вирасоро. Коммутационные соотношения этой алгебры имеют вид
\begin{eqnarray}
  \label{eq:9}
  [L_{n},c]=0\\
  \left[L_{n},L_{m}\right]=(n-m)L_{n+m} +\frac{c}{12} (m^{3}-m) \delta_{-n,m}
\end{eqnarray}
Можно показать, что $H^{2}(Witt)\cong \mathbb{C}$ и все нетривиальные коциклы пропорциональны $c$. 
(Доказательство есть в \cite{fuks1986cohomology}, \cite{schottenloher2008mathematical}).
\bibliography{talk}{}
\bibliographystyle{utphys}

\section{28 марта 2011}
\label{sec:28--2011}
\begin{itemize}
\item Броуновское движение
\item Интеграл по путям
\item Квантовая механика
\item Статистическая физика
\item Функциональный интеграл в квантовой теории поля
\item Тождества Уорда
\item Конформная аномалия.
\end{itemize}

\section{11 апреля 2011}
\label{sec:11--2011}

\subsection{Тождества Уорда}
\label{sec:ward-identities}

Напомним, о чем идет речь. В произвольной теории поля с полями $\Phi(x)$ инфинитезимальное преобразование может быть записано при помощи генераторов
\begin{equation}
  \label{eq:198}
  \Phi'(x)=\Phi(x)-i\omega_{a} G_{a} \Phi(x)
\end{equation}
Вариация действия вычисляется через ток, соответствующий данному преобразовании:
\begin{equation}
  \label{eq:199}
  j^{\mu}_{a}=\left(\frac{\partial \mathcal{L} }{\partial (\partial_{\mu}\Phi)} \partial_{\nu}\Phi -\delta^{\mu}_{\nu}\mathcal{L}\right) \frac{\delta x^{\nu}}{\delta \omega_{a}} - \frac{\partial \mathcal{L} }{\partial (\partial_{\mu} \Phi)}\frac{\delta F}{\delta \omega_{a}}
\end{equation}
\begin{equation}
  \label{eq:200}
  \delta S =\int d^{d}x \partial_{\mu}j^{\mu}_{a} \omega_{a}
\end{equation}
Обозначим через $X$ произведение локальных полей, входящих в коррелятор
\begin{equation}
  \label{eq:201}
  X=\Phi(x_{1})\dots \Phi(x_{n}).
\end{equation}
Коррелятор дается функциональным интегралом
\begin{equation}
  \label{eq:202}
  \langle X\rangle = \frac{1}{Z}\int \mathcal{D} \Phi X e^{-S[\Phi]}
\end{equation}
Сделаем замену переменных $\Phi\to \Phi'$ в функциональном интеграле. При этом можно предполагать, что ``мера интегрирования'' не меняется $\mathcal{D}\Phi'=\mathcal{D}\Phi$ \cite{difrancesco1997cft}:
\begin{equation}
  \label{eq:203}
  \langle X\rangle = \frac{1}{Z} \int \mathcal{D} \Phi' (X+\delta X)e^{-S[\Phi]-\int d^{d}x \partial_{\mu}j^{\mu}_{a}\omega_{a}(x)}
\end{equation}
Раскладываем выражение в ряд до первого порядка по $\omega_{a}(x)$:
\begin{equation}
  \label{eq:204}
  \langle X\rangle=\frac{1}{Z} \left( \int \mathcal{D} \Phi X e^{-S[\Phi]}+\int \mathcal{D} \Phi \delta X e^{-S[\Phi]} -\int \mathcal{D}\Phi \int d^{d}x \partial_{\mu}j^{\mu}_{a}\omega_{a}(x) X e^{-S[\Phi]}\right)
\end{equation}
В результате имеем
\begin{equation}
  \label{eq:205}
  \langle \delta X\rangle=\int d^{d}x \partial_{\mu}\langle j^{\mu}_{a}(x) X\rangle \omega_{a}(x)
\end{equation}
С другой стороны, мы можем вычислить $\langle\delta X\rangle$ используя \eqref{eq:198}:
\begin{multline}
  \label{eq:206}
  \langle\delta X\rangle=-i\sum_{i=1}^{n}\langle \Phi(x_{1})\dots G_{a} \Phi(x_{i}) \dots \Phi(x_{n})\rangle \omega_{a}(x_{i})=\\
  -i \int d^{d}x \omega_{a}(x) \sum_{i=1}^{n} \langle \Phi(x_{1}) \dots G_{a} \Phi(x_{i}) \dots \Phi(x_{n})\rangle \delta (x-x_{i})
\end{multline}
Окончательно общий вид тождеств Уорда выглядит следующим образом:
\begin{equation}
  \label{eq:207}
  \partial_{\mu}\langle j^{\mu}_{a}\Phi(x_{1}) \dots \Phi(x_{n})\rangle=-i\sum_{i=1}^{n}\delta (x-x_{i}) \langle\Phi(x_{1})\dots G_{a} \Phi(x_{i})\dots \Phi(x_{n})\rangle
\end{equation}
Рассмотрим некоторые примеры, которые понадобятся нам для конформной теории. Мы используем полученный нами явный вид генераторов \eqref{eq:169}.

Тождество Уорда для трансляций:
\begin{equation}
  \label{eq:208}
  \partial_{\mu}\langle T^{\mu}_{\nu} X\rangle =-i \sum_{i} \delta(x-x_{i}) \frac{\partial}{\partial x_{i}^{\nu}} \langle X \rangle
\end{equation}
Теперь рассмотрим вращения. Если тензор энергии-импульса симметризован (это можно сделать в большинстве случаев), то отвечающий вращениям ток дается выражением
\begin{equation}
  \label{eq:209}
  j^{\mu\nu\rho}=T^{\mu\nu}x^{\rho}-T^{\mu\rho}x^{\nu}.
\end{equation}
Тогда тождество Уорда может быть записано в виде
\begin{equation}
  \label{eq:210}
  \partial_{\mu}\langle (T^{\mu\nu}x^{\rho}-T^{\mu\rho}x^{\nu}) X\rangle=\sum_{i }\delta(x-x_{i}) \left( (x^{\nu}_{i}\partial^{\rho}_{i}-x^{\rho}_{i}\partial^{\nu}_{i})\langle X\rangle-i S^{\nu\rho}_{i}\langle X\rangle\right).
\end{equation}
Его можно упростить с использованием тождества для трансляций \eqref{eq:208}:
\begin{equation}
  \label{eq:211}
  \langle (T^{\rho\nu}-T^{\nu\rho}) X\rangle=-i \sum_{i} \delta(x-x_{i}) S^{\nu\rho}_{i}\langle X\rangle.
\end{equation}
Тензор энергии-импульса в квантовой теории симметричен в корреляционных функциях, если его положение не совпадает с положениями других полей в корреляторе. 

Наконец, для масштабных преобразований ток имеет вид $T^{\mu}_{\nu}x^{\nu}$, действие генератора дается выражением \eqref{eq:181} и тождества Уорда выглядят так:
\begin{equation}
  \label{eq:212}
  \partial_{\mu}\langle T^{\mu}_{\nu} x^{\nu} X\rangle=-\sum_{i} \delta(x-x_{i}) \left( x^{\nu}_{i}\frac{\partial}{\partial x^{\nu}_{i}} \langle X\rangle+\Delta_{i} \langle X\rangle\right) 
\end{equation}
Его тоже можно упростить:
\begin{equation}
  \label{eq:213}
  \langle T^{\mu}_{\mu} X \rangle = -i \sum_{i} \delta(x-x_{i}) \Delta_{i}\langle X\rangle
\end{equation}

\subsection{Тождества Уорда в двумерной конформной теории}
\label{sec:ward-2d-cft}
В прошлой лекции мы получили общий вид тождеств Уорда для трансляций, поворотов и масштабных преобразований. Выпишем их здесь.
\begin{eqnarray}
  \label{eq:236}
  \partial_{\mu}\langle T^{\mu}_{\nu} X\rangle = -\sum_{i} \delta(x-x_{i}) \partial_{\nu}^{(i)}\langle X\rangle\\
  \langle (T^{\mu\nu}-T^{\nu\mu}) X\rangle =i\sum_{i}\delta(x-x_{i}) S^{\mu\nu}_{(i)}\langle X\rangle\\
  \langle T^{\mu}_{\mu} X\rangle=-\sum_{i}\delta(x-x_{i}) \Delta_{(i)} \langle X\rangle
\end{eqnarray}
Прежде чем мы начнем переписывать тождества в комплексных координатах обсудим следующий вспомогательный математический факт: для голоморфных (антиголоморфных) функций можно использовать следующее представление для $\delta$-функции:
\begin{equation}
  \label{eq:237}
  \delta(x)=\frac{1}{\pi} \bar \partial \frac{1}{z} = \frac{1}{\pi}\partial \frac{1}{\bar z}
\end{equation}
Заметим, что
\begin{equation}
  \label{eq:238}
  \int_{M} d^{2}x \partial_{\mu}F^{\mu}=\int_{\partial M} d\xi_{\mu} F^{\mu}=\frac{1}{2i} \oint_{\partial M} (-dz F^{\bar z}+d\bar z F^{z})
\end{equation}
Для голоморфной функции
\begin{equation}
  \label{eq:239}
  \frac{1}{\pi}\int_{M} d^{2}x \bar \partial\left( \frac{f(z)}{z}\right)=\frac{1}{2\pi i} \int_{\partial M} dz \frac{f(z)}{z}=f(0)=\int_{M} d^{2}x \delta(x) f(z)
\end{equation}
Теперь перепишем тождества Уорда \eqref{eq:236} в комплексных координатах:
\begin{eqnarray}
  \label{eq:240}
  2\pi \partial \langle T_{\bar z z } X\rangle +2\pi \bar \partial \langle T_{zz}X\rangle = -\sum_{i} \bar \partial \frac{1}{z-w_{i}} \partial_{w_{i}} \langle X \rangle\\
  2\pi \partial \langle T_{\bar z \bar z } X\rangle +2\pi \bar \partial \langle T_{z\bar z}X\rangle = -\sum_{i} \bar \partial \frac{1}{\bar z-w_{i}} \partial_{\bar w_{i}} \langle X \rangle\\
  2 \langle T_{z\bar z} X 1\rangle +2 \langle T_{\bar z z }X\rangle =- \sum_{i}\delta(x-x_{i}) \Delta_{i} \langle X\rangle\\
  -2 \langle T_{z\bar z} X 1\rangle +2 \langle T_{\bar z z }X\rangle =- \sum_{i}\delta(x-x_{i}) s_{i} \langle X\rangle\\
\end{eqnarray}
Если сложить и вычесть два последних равенства и вспомнить определение $h, \bar h$, то получим
\begin{eqnarray}
  \label{eq:241}
  2\pi \langle T_{\bar z z} X\rangle = -\sum_{i} \bar \partial \frac{1}{z-w_{i}} h_{i} \langle X\rangle\\
  2\pi \langle T_{z \bar z} X\rangle = -\sum_{i}  \partial \frac{1}{\bar z-\bar w_{i}} \bar h_{i} \langle X\rangle
\end{eqnarray}
Введем обозачения
\begin{eqnarray}
  \label{eq:242}
  T=-2\pi T_{zz}\\
  \bar T = -2 \pi T_{\bar z \bar z},
\end{eqnarray}
и подставим уравнения \eqref{eq:241} в первые два уравнения \eqref{eq:240}:
\begin{eqnarray}
  \label{eq:243}
  \bar \partial \left( \langle T X\rangle -\sum_{i} \left( \frac{1}{z-w_{i}} \partial_{w_{i}} \langle X\rangle + \frac{h_{i}}{(z-w_{i})^{2}} \langle X\rangle \right) \right) =0\\
  \partial \left( \langle \bar T X\rangle -\sum_{i} \left( \frac{1}{\bar z-\bar w_{i}} \partial_{\bar w_{i}} \langle X\rangle + \frac{\bar h_{i}}{(\bar z-\bar w_{i})^{2}} \langle X\rangle \right) \right) =0
\end{eqnarray}
То есть выражения в скобках (анти)голоморфны. Следовательно
\begin{equation}
  \label{eq:244}
  \langle T X\rangle =\sum_{i} \left( \frac{1}{z-w_{i}} \partial_{w_{i}} \langle X\rangle + \frac{h_{i}}{(z-w_{i})^{2}} \langle X\rangle \right) +\mbox{регулярные члены}
\end{equation}
Это пример операторного разложения, которое мы будем обсуждать далее. 

Заметим, что вариация коррелятора при конформных преобразованиях дается тензором энергии-импульса:
\begin{equation}
  \label{eq:245}
  \delta_{\epsilon}\langle X\rangle =\int_{M} d^{2}x \partial_{\mu} \langle T^{\mu\nu}(x)\epsilon_{\nu}(x) X \rangle
\end{equation}
В комплексных координатах она перепишется так:
\begin{equation}
  \label{eq:246}
  \delta_{\epsilon,\bar \epsilon} \langle X\rangle = \frac{1}{2\pi i} \oint_{C} dz \epsilon(z) \langle T(z) X\rangle +\frac{1}{2\pi i} \oint_{C}  d\bar z \bar \epsilon(\bar z) \langle \bar T(\bar z) X\rangle
\end{equation}
Здесь интеграл берется по контуру, внутри которого находятся все аргументы $X$.
Вообще говоря, это выражение для вариации верно для коррелятора любых полей, не только примарных.

Применимость тождеств Уорда основывается на регулярности тензора энергии-импульса. Он должен быть везде хорошо определен. В частности, $T(0)$ должен быть конечен. Кроме того, если вычислить вариацию постоянного поля в точке $z=\infty$:
\begin{equation}
  \label{eq:247}
  \delta_{\epsilon} \langle 1\rangle =-\frac{1}{2\pi i} \oint_{C}dz \epsilon(z) \langle T(z)\rangle =0
\end{equation}
где контур обходит точку $\infty$. Для специальных конформных преобразований $\epsilon(z)\sim z^{2}$, поэтому $T(z)\sim z^{-4},\quad z\to \infty$.

\subsection{Алгебра локальных полей. Операторное разложение}
\label{sec:ope}

В набор локальных полей теории входят не только те поля, которые входят в плотность лагранжиана, но и их производные, произведения и так далее.
Мы предполагаем, что набор локальных полей $A$ является полным, то есть существует некий базис $\{ A_{j}\}$, по которому можно разложить любое поле. Локальные поля образуют линейное пространство.

Кроме того, мы предполагаем, что имеет место операторное разложение
\begin{equation}
  \label{eq:248}
  A_{i}(x)A_{j}(y) =\sum_{k} C^{k}_{ij} (x,y) A_{k}(y),
\end{equation}
здесь $C_{ij}^{k}(x,y)$ -- функция. Равенство \eqref{eq:248} надо понимать в смысле корреляционных функций
\begin{equation}
  \label{eq:249}
  \langle A_{i}(x) A_{j}(y) X\rangle = \sum_{k} C^{k}_{ij} (x,y) \langle A_{k}(y) X\rangle.
\end{equation}
Тогда набор локальных полей образует алгебру. 

Рассмотрим четырехточечную функцию и воспользуемся операторным разложением \eqref{eq:248}:
\begin{equation}
  \label{eq:250}
  \langle A_{1}(x_{1}) A_{2}(x_{2}) A_{3}(x_{3}) A_{4}(x_{4}) \rangle =\sum_{k,l} C^{k}_{12} (x_{1}-x_{2}) D_{kl}(x_{2}-x_{4}) C^{l}_{34}(x_{3}-x_{4}) 
\end{equation}
Здесь мы воспользовались трансляционной инвариантностью, а также ввели обозначение для пропагатора
\begin{equation}
  \label{eq:251}
  \langle A_{i}(x_{i}) A_{j}(x_{j}) \rangle =D_{ij}(x_{i}-x_{j}).
\end{equation}
Однако операторным разложением можно воспользоваться и в другом порядке, тогда для четырехточечной функции получим:
\begin{equation}
  \label{eq:252}
  \langle A_{1}(x_{1}) A_{2}(x_{2}) A_{3}(x_{3}) A_{4}(x_{4}) \rangle =\sum_{k,l} C^{k}_{13} (x_{1}-x_{3}) D_{kl}(x_{3}-x_{4}) C^{l}_{24}(x_{2}-x_{4})   
\end{equation}
Приравняв правые части уравнений \eqref{eq:252},\eqref{eq:250} можно получить систему уравнений на коэффициенты $C^{k}_{ij}$. Если решить эти уравнения, то в принципе можно вычислять любые корреляторы, если известен набор полей и их конформных размерностей. Такой подход называется конформным бутстрапом. 

Корреляционные функции имеют особенности при совпадении положений полей. Именно характер этих особенностей важен при описании критического поведения (критические индексы и т.п.).
Для произведения $T$ и примарного поля $\varphi$ с конформными размерностями $h,\bar h$ имеет место операторное разложение \eqref{eq:244}:
\begin{eqnarray}
  \label{eq:253}
  T(z) \varphi(w,\bar w)\sim \frac{h}{(z-w)^{2}} \varphi(w,\bar w)+\frac{1}{z-w} \partial_{w} \varphi(w,\bar w)\\
  \bar T(\bar z) \varphi(w,\bar w)\sim \frac{\bar h}{(\bar z-\bar w)^{2}} \varphi(w,\bar w)+\frac{1}{\bar z-\bar w} \partial_{\bar w} \varphi(w,\bar w)  
\end{eqnarray}

\subsection{Свободный бозон}
\label{sec:free-boson}

В качестве примера конформной теории рассмотрим свободное бозонное поле $\varphi$. Действие имеет вид
\begin{equation}
  \label{eq:254}
  S=\frac{1}{2} g \int d^{2}x \partial_{\mu}\varphi \partial^{\mu}\varphi,
\end{equation}
а пропагатор дается равенством
\begin{equation}
  \label{eq:255}
  \langle\varphi(x) \varphi(y)\rangle =-\frac{1}{4\pi g} \ln (x-y)^{2}+\text{const}
\end{equation}
В комплексных координатах пропагатор запишется следующим образом:
\begin{equation}
  \label{eq:256}
  \langle \varphi(z,\bar z) \varphi(w,\bar w)\rangle =-\frac{1}{4\pi g} \left( \ln(z-w) + \ln(\bar z -\bar w)\right) +\text{const}
\end{equation}
Дифференцируя получаем
\begin{eqnarray}
  \label{eq:257}
  \langle\partial_{z} \varphi(z,\bar z) \partial_{w}\varphi(w,\bar w)\rangle =-\frac{1}{4\pi g} \frac{1}{(z-w)^{2}}\\
  \langle\partial_{\bar z} \varphi(z,\bar z) \partial_{\bar w}\varphi(w,\bar w)\rangle =-\frac{1}{4\pi g} \frac{1}{(\bar z-\bar w)^{2}}  
\end{eqnarray}
То есть имеет место операторное разложение полей $\partial \varphi$:
\begin{equation}
  \label{eq:259}
  \partial \varphi(z) \partial \varphi(w) \sim -\frac{1}{4 \pi g} \frac{1}{(z-w)^{2}}
\end{equation}

Тензор энергии-импульса:
\begin{equation}
  \label{eq:258}
  T_{\mu\nu}=g \left(\partial_{\mu}\varphi \partial_{\nu} \varphi -\frac{1}{2} \eta_{\mu\nu}\partial_{\rho}\varphi \partial^{\rho} \varphi \right)
\end{equation}
После квантования получаем для оператора $T(z)$ следующее выражение:
\begin{equation}
  \label{eq:260}
  T(z) =-2\pi g :\partial \varphi \partial \varphi : =-2\pi g \lim_{w\to z} \left(\partial \varphi(z) \partial \varphi(w) -\langle\partial \varphi(z) \partial \varphi(w)\rangle\right)
\end{equation}
Теперь мы можем вычислить операторное разложение тензора энергии-импульса и локального поля $\partial \varphi$:
\begin{equation}
  \label{eq:261}
  T(z)\partial \varphi(w)=-2\pi g :\partial \varphi(z) \partial \varphi(z): \partial \varphi(w) \sim -4\pi g:\partial \varphi(z)  \partial \overbracket{ \varphi(z): \partial \varphi } (w) \sim \frac{\partial \varphi (z)}{(z-w)^{2}}
\end{equation}
Для вычисления мы использовали теорему Вика (см. \cite{vasiliev1998}, \cite{difrancesco1997cft}). Раскладывая в ряд по $(z-w)$ получаем окончательно для сингулярной части операторного разложения
\begin{equation}
  \label{eq:262}
  T(z) \partial \varphi(w) \sim \frac{\partial \varphi(w) }{(z-w)^{2}} +\frac{\partial_{w}^{2} \varphi(w)}{z-w}
\end{equation}
Теперь мы видим, что поле $\partial \varphi$ -- примарное, с конформной размерностью $h=1$. 


Аналогично вычислим операторное разложение $T$  с самим собой.
\begin{multline}
  \label{eq:263}
  T(z) T(w) =4 \pi^{2}g^{2}:\partial \varphi(z) \partial \varphi(z): :\partial \varphi(w)\partial \varphi(w): \\
  \sim \frac{1/2}{(z-w)^{4}} -\frac{4\pi g:\partial \varphi(z) \partial \varphi(w):}{(z-w)^{2}} \\
  \sim \frac{1/2}{(z-w)^{4}}+ \frac{2T(w)}{(z-w)^{2}} +\frac{\partial T(w)}{z-w}
\end{multline}

\subsection{Центральный заряд. Алгебра Вирасоро}
\label{sec:virasoro}

В силу обсуждения после формулы \eqref{eq:247}, ренормгруппового анализа \ref{c-theorem} и вычисления функции Швингера  \eqref{eq:221} мы имеем следующий общий вид операторного разложения для $T(z)T(w)$:
\begin{equation}
  \label{eq:264}
  T(z)T(w) \sim \frac{c/2}{(z-w)^{4}}+ \frac{2T(w)}{(z-w)^{2}} +\frac{\partial T(w)}{z-w}
\end{equation}
Здесь $c$ -- параметр теории, называемый центральным зарядом. Он зависит от физической природы теории и не определяется из соображений конформной инвариантности. В теории свободного бозона $c=1$, для фермиона $c=1/2$. Член $\frac{c/2}{(z-w)^{4}}$ называется аномальным. Если этого члена нет, то $T$--квазипримарное поле c $h=2$. 

Сравнивая выражение \eqref{eq:264} и \eqref{eq:221} мы видим, что константа $A$ в функции Швингера
\begin{equation}
  \label{eq:265}
  A=\frac{c}{4\pi^{2}}
\end{equation}
Покажем, как преобразуется тензор энергии-импульса при конформных преобразованиях. Используем формулу \eqref{eq:246}:
\begin{multline}
  \label{eq:266}
  \delta_{\epsilon} T(w) =-\frac{1}{2\pi i} \oint_{C} dz \epsilon(z) T(z) T(w) =-\frac{1}{12} c \partial_{w}^{3}\epsilon(w) -2 T(w) \partial_{w} \epsilon(w) -\epsilon(w) \partial_{w}T(w) 
\end{multline}
Соответствующее конечное преобразование $z\to w(z)$ имеет вид
\begin{eqnarray}
  \label{eq:267}
  T\to T'(w) =\left(\frac{dw}{dz}\right)^{-2} \left(T(z) -\frac{c}{12} \left\{ w,z\right\}\right)\\
  \left\{ w,z\right\}=\frac{\left(\frac{d^{3}w}{dz^{3}}\right) } {\frac{dw}{dz}} -\frac{3}{2}\left(\frac{\frac{d^{2}w}{dz^{2}}}{\frac{dw}{dz}}\right)^{2}
\end{eqnarray}
В качестве упражнения предлагаем проверить это выражение обратным переходом к инфинитезимальному преобразованию.

Разложим $T(z)$ в ряд Лорана.
\begin{eqnarray}
  \label{eq:268}
  T(z)=\sum_{n\in \mathbb{z}}z^{-n-2}L_{n}\\
  T(z)=\sum_{n\in \mathbb{z}}\bar z^{-n-2}\bar L_{n}
\end{eqnarray}
При масштабном преобразовании $z\to z/\lambda$ оператор $T(z)\to \lambda^{2}T(z/\lambda)$, соответственно компоненты $L_{-n}\to \lambda^{n}L_{-n}$.
\begin{equation}
  \label{eq:269}
  L_{n}=\oint \frac{dz}{2\pi i} z^{n+1} T(z)
\end{equation}
Выведем коммутационные соотношения для операторов $L_{n}$:
\begin{equation}
  \label{eq:270}
  [L_{n},L_{m}]=\left( \oint \frac{dz}{2\pi i}\oint \frac{dw}{2\pi i}-\oint \frac{dw}{2\pi i}\oint \frac{dz}{2\pi i}\right) z^{n+1} T(z)w^{m+1}T(w)
\end{equation}
Используем операторное разложение \eqref{eq:264}:
\begin{multline}
  \label{eq:271}
  [L_{n},L_{m}]=\oint \frac{dz}{2\pi i} \oint \frac{dw}{2\pi i} z^{n+1} w^{m+1}\left( \frac{c/2}{(z-w)^{4}} +\frac{2T(w)}{(z-w)^{2}}+\frac{\partial T(w) }{z-w} +\mbox{рег. члены}\right) =\\
  \oint \frac{dw}{2\pi i} \left(\frac{c}{12} (n+1)n(n-1)w^{n-2}w^{m+1} + 2(n+1)w^{n}w^{m+1}T(w)+w^{n+1}w^{m+1}\partial T(w)\right)
\end{multline}
Интегрируем последний член по частям, внеинтегральный член зануляется, а под интегралом остается $(n-m)w^{n+m+1}T(w)$:
\begin{equation}
  \label{eq:272}
    [L_{n},L_{m}]=\oint \frac{dw}{2\pi i} \left(\frac{c}{12} (n+1)n(n-1)w^{m+n-1}+ (n-m)w^{n+m+1}T(w)\right)
\end{equation}
Используя определение \eqref{eq:269} и вычисляя вычет подинтегрального выражения при $n=-m$, в итоге получаем коммутационное соотношение
\begin{equation}
  \label{eq:273}
    [L_{n},L_{m}]=(n-m)L_{m+n}+\frac{c}{12}(n^{3}-n)\delta_{m+n,0}
\end{equation}
Это коммутационные соотношения алгебры Вирасоро.

Легко показать, что в результате действия оператора $L_{-n}$ на поле $\varphi$  с конформной размерностью $(h,\bar h)$ получается поле с размерностью $(h-n,\bar h)$. Все поля в теории разбиваются на конформные семейства, соответствующие примарным полям:
\begin{equation}
  \label{eq:274}
  [\varphi]=\{L_{-n_{1}}L_{-n_{2}}\dots L_{-n_{N}}\bar L_{-m_{1}}\dots \bar L_{-m_{M}}\varphi\}
\end{equation}
Вычисление корреляторов вторичных полей сводится к корреляторам примарных. Алгебра локальных полей оказывается прямой суммой конформных семейств
\begin{equation}
  \label{eq:275}
  \mathcal{A} =\bigoplus_{\alpha}[\varphi_{\alpha}]
\end{equation}
{\it Минимальными} называются теории с конечным числом примарных полей. Чтобы полностью задать такую теорию нужно определить примарные поля и их конформные размерности.

\bibliography{2dqft}{}
\bibliographystyle{utphys}

\end{document}
%%% Local Variables: 
%%% mode: latex
%%% TeX-master: t
%%% End: 
