\documentclass[a4paper]{article}
\usepackage{verbatim}
\usepackage{amsmath,amssymb,amsthm}
\newcommand{\gfh}{\hat{\mathfrak{g}}}
\newcommand{\gf}{\mathfrak{g}}
\newcommand{\af}{\mathfrak{a}}


\title{Boundary condition changing operators and singular vectors of Virasoro and
affine Lie algebra modules}

\author{Anton Nazarov}
\begin{document}
\maketitle
\begin{abstract}
We consider the behaviour of CFT correlation functions with respect to
Schramm-Loewner evolution with additional Brownian motion on gauge
group. Using the analogy with lattice models we consider
correlation functions that satisfy martingale conditions and contain
boundary condition changing operators and primary fields.
Martingale conditions can be rewritten as algebraic relations on
boundary condition changing operators and interpreted from
representation-theoretic point of view.
\end{abstract}

\section{Title}
\label{sec:title}

Hello! 

It's a great pleasure for me to be here in Simons Center for geometry and physics. My name is Anton
Nazarov. I am work in St. Petersburg state university and my talk is called ``Boundary condition
changing operators and singular vectors of Virasoro and affine Lie algebra modules''.

This talk is based upon my letter to JETP letters, some previous conference talks and my PhD thesis. 

\section{Plan of the talk}
\label{sec:plan-talk}

Plan of my talk is here. First of all I will remind you how to study martingales of Schramm-Loewner
evolution using conformal field theory in two dimensions. 

Then I will talk of generalized Schramm-Loewner evolution with additional random gauge
transformations. This process is connected with Wess-Zumino-Novikov-Witen models of conformal field
theory. 

To generalize this study even further we will add the interaction with pure gauge field. Then gauge
fixing reduces some degrees of freedom and we obtain coset models of conformal field theory and
martingale conditions on certain class of observables in these models. 

I will discuss observables that can have natural interpretation as the limits of lattice
observables. These observables contain boundary condition changing operators. Martingale condition
can be written as an algebraic relation on boundary fields. So in Wess-Zumino-Witten and coset models
we can use representation theory to interpret martingale condition. 

Consider some simple two-dimensional lattice model, for example, Ising
model on hexagonal lattice. Let us study the behavior of domain walls.
In order to do it we need to look at the model on the bounded domain.
We take upper half-plane. Then impose boundary condition in such a way
that we have a domain wall from zero to infinity.

In critical model there is continuous limit. It is given by a solution
of Schramm-Loewner stochastic differential equation. 

\section{Domain wall in Ising model}
\label{sec:domain-wall-ising}

I want to remind you of Schramm-Loewner evolution. Consider Ising model on triangular lattice on the
upper half-plane  with the following boundary condition: Spins up on half of the boundary and down
on another half. Then we have a domain boundary growing from zero. We can parametrize this growth
with parameter $t$ and prove that in scaling limit in critical point it tends to SLE trace. 
Consider curve up to some value of parameter $t_{I}$ and make a cut along it. Then conformal map
$g_{t}$ maps the domain with the cut to the upper half plane. For Schramm-Loewner evolution this map
satisfies following differential equation. It will be more convenient for me to write it another
way. Solution of the equation gives us conformally-invariant probability measure on trajectories
$\gamma_{t}$ in upper half-plane. 

\section{CFT description}
\label{sec:cft-description}

How can we describe this using conformal field theory? We can consider lattice observables in upper
half plane. Expectation values of these observables can be obtained as an average of expectation
values in domains with all possible cuts. For a finite lattice it can be written as a sum over all
possible trajectories of average values of ${\cal O}$ in slit domains multiplied by probabilities of
trajectories. 

But the expectation value of the observable in upper half-plane does not depend on parametrization
of cuts, so the expectation value in slit domain should be a martingale with respect to growth of
cut. 

Now we consider continuous limit in critical point. Expectation value of lattice observable is then
given by correlation function of conformal field theory. 

Since we consider CFT with boundary we need to take into account boundary conditions. So we insert
boundary condition changing operators at the tip of the trace and at the infinity. Using conformal
map we can rewrite it on the whole half-plane. 

We also assume that limit of lattice observable is given by a product of primary fields in points
$z_{1}, \dots, z_{n}$ or $w_{1},\dots,w_{n}$ after the conformal map. 

\section{SLE martingales}
\label{sec:sle-martingales}

Now I want to write martingale condition and analyze correlation functions that satisfy it. 

Use Ito formula for stochastic differential. First term here gives us this contribution, second term
depends on primary fields $\varphi_{i}$. So the differential consists of powers of field
transformation generators. For minimal models these generators are derivatives, so we obtain this
differential equation on correlation function. 

We can rewrite this equation as algebraic relation on boundary condition changing operator. Then we
can establish correspondence between SLE parameter $\kappa$ and CFT paramters $c$ and $h_{\phi}$. 

\section{SLE martingales-2}
Let's generalize this analysis to non-minimal models. Consider Wess-Zumino-Novikov-Witten model. The
action of this model is written for the field $\gamma$ that lives on complex plane with infinity and
takes values in Lie group $G$ or its representation. The action consists of two termss. First term
is the action of non-linear sigma model and the second term is topological. It is added to restore
the conformal invariance after quantization. The model has gauge invariance and two conserved
currents. Conformal invariance is manifest in independent conservation of holomorphic and
antiholomorphic currents. 

We can use Ward identities to get commutation relation for the coefficients in power series
expansion for the currents.
\section{Primary fields}
\label{sec:primary-fields}

We see that coefficients in mode expansion of currents satisfy commutation relations for generators
of affine Lie algebra $\gfh$. Sugawara construction gives us generators of Virasoro algebra in terms
of affine Lie algebra generators. 

So Virasoro algebra is embedded into the universal enveloping algebra of affine Lie algebra. 
Full chiral algebra is a semidirect product of Virasoro and affine Lie algebra with these
commutation relations. Note that central charge is determined by the level of affine Lie algebra
representation. 

Primary fields with respect to full chiral algebra are labeled by highest weights of affine Lie
algebra representations. Conformal weight for primary field is given by this formula. 

\section{SLE and WZNW models}
\label{sec:sle-wznw-models}

Consider and observable in Wess-Zumino-Novikov-Witten model of the same form as in minimal models. 

Now we have an additional degree of freedom since we can do gauge transformations. We augment
Schramm-Loewner evolution with random walk on the group $G$. 

This idea was proposed in papers by Gruzberg and Rassmussen. 

Now we add the differential of random gauge transform to field transformation generator. 
Full generator look like this. Here the basis elements of Lie algebra representation and $d \Theta$
are independent Brownian motions. We have new parameter $\tau$. 

\section{SLE martingales in WZNW-models}
\label{sec:sle-martingales-wznw}

Again we use Ito formula to obtain the differential equation on martingales. 

Here I use this notation for ${\cal L}_{-n}$ and ${\cal J}^{a}_{-n}$. 

The differential equation holds for all observables so for all combinations of primary fields in
correlation function. Then it can be rewritten as an algebraic relation on boundary condition
changing operator. ????

Correlation function of arbitrary primary fields with this operator is zero. Then this state is
level 2 null-state and action of raising operators on it is zero. So we obtain algebraic relations
connecting $\kappa, \tau$ and $k, h_{\phi}$. In this way we get a classification of primary boundary
condition changing operators which can sit at the tip of SLE trace. 

\section{Null states and singular weights of affine Lie algebra module}
\label{sec:null-states-singular}

Can we answer simple question: are there level two null states in a given affine Lie algebra Verma
module? Yes, we can. All null states belong to Verma modules of singular weights. 

Consider Verma module of highest weight $\lambda$ and let $\lambda$ be integral dominant. Then the
set of singular weights is $\{\omega (\lambda+\rho)-\rho\}$ for $\omega\in W_{\gfh}$.

\section{Gauged WZNW-action and coset construction}
\label{sec:gauged-wznw-action}

We can generalize our analysis even further and consider coset models of conformal field theory.
These models can be constructed by adding the interaction with pure gauge fields to
Wess-Zumino-Novikov-Witten action. The fields $\alpha, \bar \alpha$ takes values in some subalgebra
$\af$ of Lie algebra $\gf$.  So we have this action. 

Now the current has an additional term. If we fix gauge some degrees of freedom are restricted. 

Algebraic structure is determined by two algebras $\af\subset\gf$ and generators of Virasoro algebra
are given by difference of Sugawara expressions. 

Note that affine subalgebra generators commute with Virasoro algebra. 

\section{Primary fields}
\label{sec:primary-fields-1}

In coset models we can label primary fields by pairs of weights. First weight is weight of algebra
$\gfh$ and the second for subalgebra $\af$. Branching functions for these pairs must be non-trivial.
Some pairs are equivalent but it is possible to take this equivalence into account. 

Conformal weight is equal to the difference of these expressions that are similar to conformal
weights in Wess-Zumino-Novikov-Witten model. 

We also have an analogue of Knizhnik-Zamolodchikov equations that can be useful later. 

\section{SLE on coset space}
\label{sec:sle-coset-space}

Let me analyze the martingale condition in coset models. When I fix gauge in the model some degrees
of freedom are frozen, so I allow only certain directions for additional random walk. 


\end{document}
