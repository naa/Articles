\documentclass[12pt]{article}
\usepackage{amsmath,amssymb,amsthm,amsfonts}
\usepackage{multicol}
\usepackage{color}
\usepackage{hyperref}
\usepackage{graphicx}

\newcommand{\co}[1]{\stackrel{\circ }{#1}}
\newcommand{\gf}{\mathfrak{g}}
\newcommand{\nfp}{\mathfrak{n}^{+}}
\newcommand{\nfm}{\mathfrak{n}^{-}}
\newcommand{\af}{\mathfrak{a}}
\newcommand{\uf}{\mathfrak{u}}
\newcommand{\sfr}{\mathfrak{s}}
\newcommand{\aft}{\widetilde{\mathfrak{a}}}
\newcommand{\afb}{\mathfrak{a}_{\bot}}
\newcommand{\hf}{\mathfrak{h}}
\newcommand{\hfb}{\mathfrak{h}_{\bot}}
\newcommand{\pf}{\mathfrak{p}}

\newcommand{\gfh}{\hat{\mathfrak{g}}}
\newcommand{\afh}{\hat{\mathfrak{a}}}
\newcommand{\sfh}{\hat{\mathfrak{s}}}
\newcommand{\bff}{\mathfrak{b}}
\newcommand{\hfg}{\hf_{\gf}}

\begin{document}
\title{Schramm-Loewner evolution in tricritical Ising model}
\author{Anton Nazarov}%% $^{1,2}$}

%% \address{
%%   $^1$ Department of High-Energy and Elementary Particle Physics, 
%%   Faculty of Physics, \\ SPb State University
%%   198904, Saint-Petersburg, Russia}
%% \address{
%%   $^{2}$ Chebyshev Laboratory,
%%   Faculty of Mathematics and Mechanics, \\ SPb State University
%%   199178, Saint-Petersburg, Russia\\
%% anton.nazarov@hep.phys.spbu.ru}
%% 
%% 
\maketitle

\begin{abstract}
  We consider restricted solid-on-solid realization of the tricritical Ising model on upper
  half-plane. We use conformally-invariant boundary conditions  and integrability of the model to
  show the coincidence of lattice interfaces to Schramm-Loewner evolution traces in critical point.
  We emphasize the similarity and differences with the Ising model. 
\end{abstract}

\section{Introduction}
\label{sec:introduction}

Tricritical Ising model is the dilute Ising model in the tricritcal point.
The 2-Dimensional tricritical Ising model is the simplest known statistical model to exhibit
supersymmetry. This simple model has a long history of theoretical and experimental study. 

Several lattice realization of tricritical Ising model are possible. We consider the dilute Ising model and the $A_4$
Restricted Solid-on-Solid (RSOS) model. The dilute Ising model is described by the Hamiltonian 
\begin{equation}
  \label{eq:1}
  H = -\beta \sum_{<i,j>}\sigma_i\sigma_j - \mu \sum_{i}(\sigma_i)^2  
\end{equation}
We overview the phase diagram and present restricted solid-on-solid realization of the model in
section \ref{sec:rsos-real-ising}. This model has the tricritical point $T$. 


In recent
years theoretical progress in the study of critical behavior was connected with the description of
geometrical objects -- interfaces. It was achieved by the use of Scharmm-Loewner evolution. To study
the behavior of critical interfaces we consider the model on the bounded domain. 

The behaviour of the model in critical
point is described by conformal field theory (CFT) with central charge $c=\frac{7}{10}$
\cite{friedan1985superconformal}. 

Shramm-Loewner evolution is the stochastic differential equation that describes the growth of
conformally-invariant cut (some details here).

We use CFT data to establish the value of Schramm-Loewner evolution parameter $\kappa$. The relation
$c=\frac{(8-3\kappa)(\kappa-6)}{2\kappa}$ gives us possible values of $\kappa=\frac{16}{5},\; 5$.
The model is also simplest example of supersymmetric conformal field theory. Schramm-Loewner
evolution on superspace was introduced in \cite{nagi2005stochastic,rasmussen2004stochastic}. The
relationship between the central charge and SLE parameter in this case is
$c=\frac{15}{2}-3\left(\kappa+\frac{1}{\kappa}\right)$. It gives us
$\kappa=\frac{3}{5},\frac{5}{3}$. (? c different for different sectors? )

We consider the model on the domain with a boundary (upper half-plane). So we need to impose
conformally invariant boundary conditions. Boundary states satisfy Cardy conditions and were
classified by Chim \cite{chim1996boundary}. Conformal weights for the boundary fields are $h=0,\;
\frac{1}{10},\; \frac{3}{5},\; \frac{3}{2},\; \frac{3}{80},\; \frac{7}{16}$.

We review the Schramm-Loewner evolution and conformal field theory model for the tricritical Ising
model in section \ref{sec:conf-field-theory}. 

The weights $h=\frac{1}{10},\; \frac{7}{16}$ satisfy the relationship between the parameter $\kappa$
and conformal weight of boundary condition changing operator in SLE martingale: $(3\kappa-8) h =c$.
If we consider SLE on a superspace we have the equation $h=\frac{6\kappa-3}{16}$ and get the weights
$h=\frac{7}{6}?, \frac{1}{10}$ in the Neveu-Schwartz sector and $h=\frac{3}{80}, \frac{7}{16}$ in
the Ramond-Ramond sector.

The classification of boundary states and SLE martingale condition suggests us the proper choice of
integrable boundary conditions in RSOS realization of the model. For example, the following boundary
satisfy it:
\begin{equation}
  \label{eq:2}
  \underline{\begin{array}{llllllllll}
    2 & & 2 & & 4 & & 4 & & 4\\
    &  3 & & 3 & & 3 & & 3 & &
  \end{array}}
\end{equation}
There is obviously an interface on double lattice delimiting $2$-s from $4$-s. We analyze the
boundary conditions in RSOS model and  introduce the interfaces in section
\ref{sec:bound-cond-interf}. 

 Then we
select lattice observable that is martingale with respect to the interface growth.
This observable is a generalization of Smirnov-Hongler parafermionic observable []:
\begin{equation}
  \label{eq:3}
  f = \sum e^{-\mbox{winding}} e^{-Kl}
\end{equation}

In the section \ref{sec:discr-holom-observ} we show that this observable satisfy discrete s-holomorphicity
condition and is a solution of discrete boundary problem. In critical limit this gives us the
holomorphic function crucial for the proof of the convergence of the interface to SLE trace.

We list unsolved problems and future directions in conclusion \ref{sec:conclusion}.

\section{Statement of the main theorem}
\label{sec:results}


\section{RSOS realization of the Ising and tricritical Ising model}
\label{sec:rsos-real-ising}

\section{Conformal field theory and Scramm-Loewner evolution}
\label{sec:conf-field-theory}

\section{Boundary conditions and interfaces}
\label{sec:bound-cond-interf}

\section{Discrete holomorphic observable and proof of convergence}
\label{sec:discr-holom-observ}


\section{Conclusion}
\label{sec:conclusion}



\bibliography{bibliography}{} 
\bibliographystyle{utphys}

\end{document}
