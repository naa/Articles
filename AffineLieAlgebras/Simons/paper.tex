\documentclass[12pt]{article}
\usepackage{amsmath,amssymb,amsthm,amsfonts}
\usepackage{multicol}
\usepackage{color}
\usepackage{hyperref}
\usepackage{graphicx}

\newtheorem{theorem}{Theorem}

\newcommand{\co}[1]{\stackrel{\circ }{#1}}
\newcommand{\gf}{\mathfrak{g}}
\newcommand{\nfp}{\mathfrak{n}^{+}}
\newcommand{\nfm}{\mathfrak{n}^{-}}
\newcommand{\af}{\mathfrak{a}}
\newcommand{\uf}{\mathfrak{u}}
\newcommand{\sfr}{\mathfrak{s}}
\newcommand{\aft}{\widetilde{\mathfrak{a}}}
\newcommand{\afb}{\mathfrak{a}_{\bot}}
\newcommand{\hf}{\mathfrak{h}}
\newcommand{\hfb}{\mathfrak{h}_{\bot}}
\newcommand{\pf}{\mathfrak{p}}

\newcommand{\gfh}{\hat{\mathfrak{g}}}
\newcommand{\afh}{\hat{\mathfrak{a}}}
\newcommand{\sfh}{\hat{\mathfrak{s}}}
\newcommand{\bff}{\mathfrak{b}}
\newcommand{\hfg}{\hf_{\gf}}

\begin{document}
\title{Schramm-Loewner evolution in tricritical Ising model}
\author{Anton Nazarov}%% $^{1,2}$}

%% \address{
%%   $^1$ Department of High-Energy and Elementary Particle Physics, 
%%   Faculty of Physics, \\ SPb State University
%%   198904, Saint-Petersburg, Russia}
%% \address{
%%   $^{2}$ Chebyshev Laboratory,
%%   Faculty of Mathematics and Mechanics, \\ SPb State University
%%   199178, Saint-Petersburg, Russia\\
%% anton.nazarov@hep.phys.spbu.ru}
%% 
%% 
\maketitle

\begin{abstract}
  We consider restricted solid-on-solid realization of the tricritical Ising model on upper
  half-plane. We use conformally-invariant boundary conditions  and integrability of the model to
  show the coincidence of lattice interfaces to Schramm-Loewner evolution traces in critical point.
  We emphasize the similarity and differences with the Ising model. 
\end{abstract}

\section{Introduction}
\label{sec:introduction}

Tricritical Ising model is the dilute Ising model in the tricritcal point.
The 2-Dimensional tricritical Ising model is the simplest known statistical model to exhibit
supersymmetry. This simple model has a long history of theoretical and experimental study. 

Several lattice realization of tricritical Ising model are possible. We consider the dilute Ising model and the $A_4$
Restricted Solid-on-Solid (RSOS) model. The dilute Ising model is described by the Hamiltonian 
\begin{equation}
  \label{eq:1}
  H = -\beta \sum_{<i,j>}\sigma_i\sigma_j - \mu \sum_{i}(\sigma_i)^2  
\end{equation}
We overview the phase diagram and present restricted solid-on-solid realization of the model in
section \ref{sec:rsos-real-ising}. This model has the tricritical point $T$. 


In recent
years theoretical progress in the study of critical behavior was connected with the description of
geometrical objects -- interfaces. It was achieved by the use of Scharmm-Loewner evolution. To study
the behavior of critical interfaces we consider the model on the bounded domain. 

The behaviour of the model in critical
point is described by conformal field theory (CFT) with central charge $c=\frac{7}{10}$
\cite{friedan1985superconformal}. 

Shramm-Loewner evolution is the stochastic differential equation that describes the growth of
conformally-invariant cut (some details here).

We use CFT data to establish the value of Schramm-Loewner evolution parameter $\kappa$. The relation
$c=\frac{(8-3\kappa)(\kappa-6)}{2\kappa}$ gives us possible values of $\kappa=\frac{16}{5},\; 5$.
The model is also simplest example of supersymmetric conformal field theory. Schramm-Loewner
evolution on superspace was introduced in \cite{nagi2005stochastic,rasmussen2004stochastic}. The
relationship between the central charge and SLE parameter in this case is
$c=\frac{15}{2}-3\left(\kappa+\frac{1}{\kappa}\right)$. It gives us
$\kappa=\frac{3}{5},\frac{5}{3}$. (? c different for different sectors? )

We consider the model on the domain with a boundary (upper half-plane). So we need to impose
conformally invariant boundary conditions. Boundary states satisfy Cardy conditions and were
classified by Chim \cite{chim1996boundary}. Conformal weights for the boundary fields are $h=0,\;
\frac{1}{10},\; \frac{3}{5},\; \frac{3}{2},\; \frac{3}{80},\; \frac{7}{16}$.

We review the Schramm-Loewner evolution and conformal field theory model for the tricritical Ising
model in section \ref{sec:conf-field-theory}. 

The weights $h=\frac{1}{10},\; \frac{7}{16}$ satisfy the relationship between the parameter $\kappa$
and conformal weight of boundary condition changing operator in SLE martingale: $(3\kappa-8) h =c$.
If we consider SLE on a superspace we have the equation $h=\frac{6\kappa-3}{16}$ and get the weights
$h=\frac{7}{6}?, \frac{1}{10}$ in the Neveu-Schwartz sector and $h=\frac{3}{80}, \frac{7}{16}$ in
the Ramond-Ramond sector.

The classification of boundary states and SLE martingale condition suggests us the proper choice of
integrable boundary conditions in RSOS realization of the model. For example, the following boundary
satisfy it:
\begin{equation}
  \label{eq:2}
  \underline{\begin{array}{llllllllll}
    2 & & 2 & & 4 & & 4 & & 4\\
    &  3 & & 3 & & 3 & & 3 & &
  \end{array}}
\end{equation}
There is obviously an interface on double lattice delimiting $2$-s from $4$-s. We analyze the
boundary conditions in RSOS model and  introduce the interfaces in section
\ref{sec:bound-cond-interf}. 

 Then we
select lattice observable that is martingale with respect to the interface growth.
This observable is a generalization of Smirnov-Hongler parafermionic observable []:
\begin{equation}
  \label{eq:3}
  f = \sum e^{-\mbox{winding}} e^{-Kl}
\end{equation}

In the section \ref{sec:discr-holom-observ} we show that this observable satisfy discrete s-holomorphicity
condition and is a solution of discrete boundary problem. In critical limit this gives us the
holomorphic function crucial for the proof of the convergence of the interface to SLE trace.

We list unsolved problems and future directions in conclusion \ref{sec:conclusion}.

\subsection{Statement of the main theorem}
\label{sec:results}

\begin{theorem}
  For tricritical Ising model on upper half plane there exists s-holomorphic observable $f$ given by the
  formula. It satisfies discrete boundary problem with boundary conditions. In the limit it
  converges to analytic function. Correlation function of something $F$ can be calculated as $F=f+if$.
\end{theorem}
\section{RSOS realization of the Ising and tricritical Ising model}
\label{sec:rsos-real-ising}

Tricritical Ising model was introduced as a modification of the Ising model \cite{} (See Malte Henkel ``Conformal invariance and critical phenomena'' 430, 408). (pp 14, 221, 340).

Blume-Capel model. Critical exponents [430]. Mathematica program for CFT: Lassig Mussardo CPC 66, 71 (1991) [423].

 The model is
the dilute Ising model with the Hamiltonian 
\begin{equation}
  \label{eq:1}
  H = -\beta \sum_{<i,j>}\sigma_i\sigma_j - \mu \sum_{i}(\sigma_i)^2.
\end{equation}
Here $\mu$ is chemical potential of vacancies,
if $\mu$ tends to zero we recover ordinary Ising model without vacant sites. We consider the model on some domain
$\Omega$ in complex plane. The model has three vacuum states with the spins on all sites equal to
$+1$, $-1$ and $0$. In tricritical point the vacuum is degenerate. 

(Phase diagram here)

(In tricritical point this model is conformally-invariant and described by CFT)

There is another lattice model in the same universality class. It is $A_4$ Restricted Solid-on-Solid
(RSOS) model. 

In restricted solid-on-solid models spins on the sites take values labeled by the
nodes of simply-laced Dynkin diagram. The values at the adjacent sites must be adjacent on the
Dynkin diagram.

The weight of the configuration $C$ is equal to the product of
face weights and boundary weights:
\begin{equation}
  \label{eq:7}
  W(c)=\prod_{f\in \mathrm{faces}} W(f,C) \prod_{b\in \mbox{\small{boundary
      faces}}} B(b,C)
\end{equation}

Face weight is determined by the Perron-Frobenius
eigenvector $\psi$ of the adjacency matrix $G$ of Dynkin diagram. Denote by $2\cos\lambda$ be the
highest eigenvalue of $G$, then $\sum_{b\in {\cal G}} G_{ab}\psi_b = 2\cos(\lambda) \psi_a $ and 
by the Perron-Frobenius theorem $\psi_a \neq 0$. 

Bolzmann weight for a face with vertex labels $a,b,c,d$ is given by the formula
\begin{equation}
  \label{eq:5}
    W\left.\left(
    \begin{array}{cc}
      d & c\\
      a & b
    \end{array}\right| u \right)
= s_1 (-u)\delta_{ac} + \frac{s_0 (u) \psi_a^{1/2}
  \psi_c^{1/2}}{\psi_b} \delta_{bd},
\end{equation}
where
  \begin{equation}
    \label{eq:6}
    s_r(u)=\frac{\sin(r\lambda+u)}{\sin\lambda},\quad \frac{\lambda}{\pi}\not\in\mathbb{Z}.
  \end{equation}

Bulk weights satisfy Yang-Baxter equation \cite{behrend2001integrable}
that leads to integrability of the model using transfer matrix
approach. 

Bolzmann weights for the lattice edges can differ from face weights.
Choice of these weights corresponds to different boundary conditions.
Conformally invariant boundary conditions for $A_4$ RSOS realization
of the tricritical Ising model were identified in
\cite{chim1996boundary}, boundary weights were listed in
\cite{behrend2001integrable}. 

Behrend and Pearce use following labeling of boundary conditions for
rank $g$ model:
$\left\{(r,a)\ |\ r\in\{1,\dots,g-1\}, a\in {\cal G}\right\}$. For the
$(1,a)$ boundary condition boundary weights are given by the equation
\begin{equation}
  \label{eq:8}
  B^{1,a}\left.\left( c
    \begin{array}{ll}
      a & 1\\
      a & 1
    \end{array}\right| u, \xi \right)= \frac{s_0(\xi+u) s_1(\xi-u)
  \psi^{1/2}_c } {s_0(2\xi) \psi_a^{1/2}}
\end{equation}

So in the $(1, a)$ boundary condition the state of every alternate
boundary spin is fixed to be $a$, while each other boundary spin,
with state $c$ adjacent to $a$ on ${\cal G}$, is associated with a
weight proportional to $\psi_c^{1/2}$. 

These boundary weights satisfy the boundary Yang-Baxter equation.

%%%%%%%%%%
For the Ising and tricritical Ising models we limit ourselves with $r=1$
boundary states. 
%%%%%%%%%%


Ising model is realized using $A_{3}$ diagram. If we enumerate the nodes by $1,2,3$ the sites with
values $1$ and $3$ correspond to positive and negative spins and the sublattice with $2$-s is frozen
and do not play any role. 

For RSOS realization of Ising model we have:
\begin{eqnarray}
  \label{eq:10}
  2 \cos\lambda = \sqrt{2}\\
  \lambda=\frac{\pi}{4}\\
  \psi = (1,\sqrt{2},1)
\end{eqnarray}
Values on even sublattice sites are fixed to $2$ and odd sublattice
values $1,3$ can be mapped to Ising spins $-1,+1$. Critical value for
Ising model is $K_C=\frac{1}{2}\log (1+\sqrt{2})$.

So we obtain Boltzmann weights:
\begin{equation}
  \label{eq:9}
  \begin{array}{lllll}
    V_{-} &=& W\left.\left(
    \begin{array}{ll}
      2 & 3\\
      1 & 2
    \end{array}\right| u \right) &=&   W\left.\left(
    \begin{array}{ll}
      2 & 1\\
      3 & 2
    \end{array}\right| u
\right)=\frac{1}{\sqrt{2}} \frac{\sin(u)}{\sin(\lambda)}=\sin(u)\\
H_{-}&=&W\left.\left(
    \begin{array}{ll}
      1 & 2\\
      2 & 3
    \end{array}\right| u\right) &=&  W\left.\left(
    \begin{array}{ll}
      3 & 2\\
      2 & 1
    \end{array}\right| u
\right)=-\frac{\sin(u-\lambda)}{\sin(\lambda)}=-\sqrt{2}\sin\left(u-\frac{\pi}{4}\right)\\

V_{+}&=&W\left.\left(
    \begin{array}{ll}
      2 & 1\\
      1 & 2
    \end{array}\right| u
\right)&=&   W\left.\left(
    \begin{array}{ll}
      2 & 3\\
      3 & 2
    \end{array}\right| u
\right)= \frac{\sin\left( u\right) }{\sqrt{2}\,\sin\left(
    \lambda\right) }-\frac{\sin\left( u-\lambda\right) }{\sin\left(
    l\right) } = \sin(u)-\sqrt{2}\sin\left(u-\frac{\pi}{4}\right)\\
H_{+}&=&W\left.\left(
    \begin{array}{ll}
      1 & 2\\
      2 & 1
    \end{array}\right| u
\right)&=&   W\left.\left(
    \begin{array}{ll}
      3 & 2\\
      2 & 3
    \end{array}\right| u
\right)= \frac{\sqrt{2}\sin\left( u\right) }{\sin\left(
    \lambda\right) }-\frac{\sin\left( u-\lambda\right) }{\sin\left(
    l\right) } = 2\sin(u)-\sqrt{2}\sin\left(u-\frac{\pi}{4}\right)
\end{array}
\end{equation}
Here we have denoted by $V_{\pm}, H_{\pm}$ Boltzmann weights for
vertical and horizontal adjacent spins with positive ($+$) or negative
($-$) product. The case $u=\lambda/2=\frac{\pi}{8}$ is isotropic
$\frac{V_+}{V_-}=\frac{H_+}{H_-}= 1+\sqrt{2}=e^{2 K_C}$.


%% The crucial object in the proof of convergence of lattice correlations
%% to correlation functions in critical limit is parafermionic observable
%% of Smirnov, Hongler \cite{1202.2838v2}.
%% 
%% It can be defined as follows: consider Ising configuration on a double cover of domain $\Omega$
%% with singularity at $z$. 
%% 





  We are interested in the case $A_4$, so we have
$2\cos\lambda=\frac{\sqrt{\sqrt{5}+3}}{\sqrt{2}}$,
$\lambda=0.62831853071796$ and $\psi=\left[1,2\cos\lambda,
  4\cos^2\lambda-1, 2cos\lambda-\frac{1}{2\cos\lambda}\right]$ or $\psi=\left[1,\frac{\sqrt{\sqrt{5}+3}}{\sqrt{2}},\frac{\sqrt{5}+1}{2},\frac{\sqrt{\sqrt{5}+3}\,\left( \sqrt{2}\,\sqrt{5}-\sqrt{2}\right) }{4}\right]$.
The isotropic case $u=\lambda/2$ with boundary parameters
$u=\xi_1=\xi_2=\lambda/2$ corresponds to the  conformal point. Then we have
$s_1(-u)=\frac{\sin\lambda/2}{\sin\lambda}=\frac{1}{2\cos\lambda/2}$ and $s_0(u)=\frac{1}{2\cos\lambda/2}$.

\begin{itemize}
\item Boltzmann weights for $A_{4}$
\item Boundary weights ?
\item Critical point, isotropicity
\item Loop description (two types of loops, duality, similarity with dilute Ising model) 
\end{itemize}

\section{Conformal field theory and Scramm-Loewner evolution}
\label{sec:conf-field-theory}



In recent
years theoretical progress in the study of critical behavior was connected with the description of
geometrical objects -- interfaces. It was achieved by the use of Scharmm-Loewner evolution \cite{schramm2000scaling}.

The behaviour of the dilute Ising model in tricritical
point is described by conformal field theory (CFT) with central charge $c=\frac{7}{10}$
\cite{friedan1985superconformal}. 

Shramm-Loewner evolution is the stochastic differential equation that describes the growth of
conformally-invariant cut

\begin{equation}
  \label{eq:6}
  
\end{equation}


We use CFT data to establish the value of Schramm-Loewner evolution parameter $\kappa$. The relation
$c=\frac{(8-3\kappa)(\kappa-6)}{2\kappa}$ gives us possible values of $\kappa=\frac{16}{5},\; 5$.
The model is also simplest example of supersymmetric conformal field theory. Schramm-Loewner
evolution on superspace was introduced in \cite{nagi2005stochastic,rasmussen2004stochastic}. The
relationship between the central charge and SLE parameter in this case is
$c=\frac{15}{2}-3\left(\kappa+\frac{1}{\kappa}\right)$. It gives us
$\kappa=\frac{3}{5},\frac{5}{3}$. (? c different for different sectors? )

We consider the model on the domain with a boundary (upper half-plane). So we need to impose
conformally invariant boundary conditions. Boundary states satisfy Cardy conditions and were
classified by Chim \cite{chim1996boundary}. Conformal weights for the boundary fields are $h=0,\;
\frac{1}{10},\; \frac{3}{5},\; \frac{3}{2},\; \frac{3}{80},\; \frac{7}{16}$.

We review the Schramm-Loewner evolution and conformal field theory model for the tricritical Ising
model in section \ref{sec:conf-field-theory}. 

The weights $h=\frac{1}{10},\; \frac{7}{16}$ satisfy the relationship between the parameter $\kappa$
and conformal weight of boundary condition changing operator in SLE martingale: $(3\kappa-8) h =c$.
If we consider SLE on a superspace we have the equation $h=\frac{6\kappa-3}{16}$ and get the weights
$h=\frac{7}{6}?, \frac{1}{10}$ in the Neveu-Schwartz sector and $h=\frac{3}{80}, \frac{7}{16}$ in
the Ramond-Ramond sector.

\section{Boundary conditions and interfaces}
\label{sec:bound-cond-interf}


The classification of boundary states and SLE martingale condition suggests us the proper choice of
integrable boundary conditions in RSOS realization of the model. For example, the following boundary
satisfy it:
\begin{equation}
  \label{eq:2}
  \underline{\begin{array}{llllllllll}
    2 & & 2 & & 4 & & 4 & & 4\\
    &  3 & & 3 & & 3 & & 3 & &
  \end{array}}
\end{equation}
There is obviously an interface on double lattice delimiting $2$-s from $4$-s. We analyze the
boundary conditions in RSOS model and  introduce the interfaces in section
\ref{sec:bound-cond-interf}. 

\section{Discrete holomorphic observable and proof of convergence}
\label{sec:discr-holom-observ}

 Then we
select lattice observable that is martingale with respect to the interface growth.
This observable is a generalization of Smirnov-Hongler parafermionic observable []:
\begin{equation}
  \label{eq:3}
  f = \sum e^{-\mbox{winding}} e^{-Kl}
\end{equation}

In the section \ref{sec:discr-holom-observ} we show that this observable satisfy discrete s-holomorphicity
condition and is a solution of discrete boundary problem. In critical limit this gives us the
holomorphic function crucial for the proof of the convergence of the interface to SLE trace.

\section{Conclusion}
\label{sec:conclusion}

We list unsolved problems and future directions in conclusion \ref{sec:conclusion}.


\section*{Acknowledgements}
\label{sec:acknowledgements}

This research was started during my stay at Simons center for geometry and physics in Stony Brook
university. I thank Simons foundation for the financial support during my stay, the organizers of
the program ``Conformal geometry'' and Simons center staff for their hospitality. I also thank
Christian Hagendorff(?) and Dmitry Chelkak for fruitful discussions. 

This work  is supported by
the Chebyshev Laboratory (Department of Mathematics and Mechanics,
Saint-Petersburg State University) under the grant 11.G34.31.0026
of the Government of the Russian Federation and by the Dynasty foundation. 

I am grateful to my wife Alexandra for her constant support and care. 

\bibliography{bibliography}{} 
\bibliographystyle{utphys}

\end{document}
