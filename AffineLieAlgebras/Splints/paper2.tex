\documentclass[12pt]{article}
\usepackage{amsfonts}

%%%%%%%%%%%%%%%%%%%%%%%%%%%%%%%%%%%%%%%%%%%%%%%%%%%%%%%%%%%%%%%%%%%%%%%%%%%%%%%%%%%%%%%%%%%%%%%%%%%
\usepackage{amsmath,amssymb,amsthm}
\usepackage{multicol}
\usepackage{color}
\usepackage{hyperref}
\usepackage{graphicx}
\usepackage[english]{babel}

%TCIDATA{OutputFilter=Latex.dll}
%TCIDATA{LastRevised=Mon Oct 10 18:16:16 2011}
%TCIDATA{<META NAME="GraphicsSave" CONTENT="32">}
%TCIDATA{CSTFile=article.cst}

\newtheorem{Def}{Definition}[section]
\newtheorem{theorem}{Theorem}
\newtheorem{statement}{Statement}
\newtheorem{Cnj}[Def]{Conjecture}
\newtheorem{Prop}[Def]{Property}
\newtheorem{example}{Example}[section]

\newcommand{\co}[1]{\stackrel{\circ }{#1}}
\newcommand{\gf}{\mathfrak{g}}
\newcommand{\nfp}{\mathfrak{n}^{+}}
\newcommand{\nfm}{\mathfrak{n}^{-}}
\newcommand{\af}{\mathfrak{a}}
\newcommand{\aft}{\widetilde{\mathfrak{a}}}
\newcommand{\afb}{\mathfrak{a}_{\bot}}
\newcommand{\hf}{\mathfrak{h}}
\newcommand{\hfb}{\mathfrak{h}_{\bot}}
\newcommand{\pf}{\mathfrak{p}}

\newcommand{\gfh}{\hat{\mathfrak{g}}}
\newcommand{\afh}{\hat{\mathfrak{a}}}
\newcommand{\bff}{\mathfrak{b}}
\newcommand{\hfg}{\hf_{\gf}}

%\input{tcilatex}

\begin{document}
\title{On affine extension of splint root systems}



\author{V.D.~Lyakhovsky$^1$, A.A.~Nazarov$^{1,2}$ \\
  {\small $^1$ Department of High-energy and elementary particle physics, SPb State University}\\
  {\small 198904, Saint-Petersburg, Russia}\\
  {\small e-mail: lyakh1507@nm.ru}\\
  {\small$^{2}$ Chebyshev Laboratory,}\\
  {\small Department of Mathematics and Mechanics, SPb State University}\\
  {\small 199178, Saint-Petersburg, Russia}\\
  {\small email: antonnaz@gmail.com}}
\maketitle

\begin{abstract}
Splint of root system of simple Lie algebra appears naturally in
the study of (regular) embeddings of reductive subalgebras. It can
be used to derive branching rules. Application of
splint properties drastically simplifies calculations of
branching coefficients. We study affine extension of splint root system of simple Lie algebra.
\end{abstract}

\section{Introduction}
\label{sec:introduction}

Consider Lie algebra $\gf$ and its irreducible modules $L^{\mu}_{\gf}$. 
Splints of root systems were introduced by D. Richter in paper \cite{richter2008splints} to describe cases of drastic simplification of module reduction into the sum of subalgebra modules. In paper \cite{2011arXiv1111.6787L} we've proved that branching coefficients coincide with multiplicities (more details here). 

In present note we generalize our analysis to affine extension of $\gf$. We consider affine Lie algebra $\gfh$ with the underlying simple finite-dimensional Lie algebra $\gf$. We assume that root system $\Delta_{\gf}$ admits splint $\Delta_{\gf}=\Delta_{\af_{1}}\cup \Delta_{\af_{2}}$. It is well-known \cite{kac1990idl} that irreducible $\gfh$-module can be decomposed into the sum of irreducible $\gf$-modules.
\begin{equation}
  \label{eq:1}
  L_{\gfh}^{\mu}=\bigoplus_{\nu=(\co{\nu},n,k)} b^{\mu}_{\co{\nu}}(n) L^{\co{\nu}}_{\gf}
\end{equation}
Then each of these $\gf$-modules in this decomposition admits decomposition to $\af_{1}$ modules with branching coefficients equal to weight multiplicities of $\af_{2}$ irreducible module \cite{2011arXiv1111.6787L}. 
\begin{equation}
  \label{eq:2}
  L^{\co{\nu}}_{\gf} = \bigoplus b^{\mu}_{\af_{1}\co{\nu}}(n) L^{\co{\nu}}_{\gf}
\end{equation}
We state that it is possible to reorganize weights in such a way that branching coefficients for the reduction $\gfh\downarrow \af_{1}$  are expressed in terms of weight multiplicities of $\hat\af_{2}$. 
\section{Main theorem}
\label{sec:main-theorem}
\begin{theorem}
  Consider reduction of affine Lie algebra $\gfh$-module $L^{\mu}_{\gfh}$ to finite-dimensional subalgebra $\af_{1}$. Then for branching coefficients following relation holds:
  \begin{equation}
    \label{eq:3}
    b^{\mu}_{\nu}=\sum_{?} m^{\nu}_{\hat\af_{2}}
  \end{equation}
\end{theorem}
\begin{proof}
  At first we use equation \eqref{eq:1}, then substitute to equation \eqref{eq:2}, reorganize weights and obtain the desired result. 
\end{proof}

\section{Examples}
\label{sec:examples}
The most simple example is constructed starting with splint $\Delta_{A_{2}}=\Delta_{A_{1}\oplus A_{1}}\cup \Delta_{A_{1}}$.

Then we consider the case $\gfh=\hat B_{2}$.
\section*{Conclusion}
\label{sec:conclusion}
Further question is to generalize this analysis to affine sublagebras and apply the results on branching coefficients to the study of CFT coset models. 

%%  
%%  
%%  In the present paper we provide alternative proof of the main statement of paper [LN-2011-Splints] which uses graded commutative structure on vector space of universal enveloping algebra $U(n^{-})$. 
%%  
%%  Con
%%  
%%  
%%  Consider decomposition of simple Lie algebra $\gf=\nfp\oplus \hf \oplus \nfm$. Verma module $M^{\mu}=U(\gf)\underset{U(\bff_{+})}{\otimes} D^{\mu}(\bff_{+})$. $S(\nfm)=\{x_{1}\dots x_{l}:x_{i}\in\nfm\}_{l=0}^{\infty}$
%%  \begin{theorem}
%%    Fix some ordering $\alpha_{1},\dots,\alpha_{s}$ on the set of positive roots $\Delta^{+}$ and obtain the ordering on the basis $f_{1},\dots,f_{s}$ of $\nfm$.  Then monomials $f_{1}^{n_{1}} \dots f_{s}^{n_{s}}$ form a basis in $U(\nfm)\sim M^{\lambda}$ and $U(\nfm)\sim \mathbb{Z}^{s}_{\geq 0}$. 
%%  \end{theorem}
%%  \begin{statement}
%%    Let $\Delta^{+}_{\gf}=\Delta^{+}_{\af_{1}}\cup \Delta^{+}_{\af_{2}}$ be a splint positive root system of simple Lie algebra $\gf$. Then $M^{\lambda}_{\gf}\sim M^{\lambda}_{\af_{1}}\oplus M^{\lambda}_{\af_{2}}$.
%%    \begin{proof}
%%      Basis of $M^{\lambda}_{\gf}$ is given by $f_{1}^{n_{1}}\dots f_{s_{1}}^{n_{s_{1}}} f_{s_{1}+1}^{n_{s_{1}+1}}\dots f_{s_{1}+s_{2}}^{n_{s_{1}+s_{2}}}$, where $f_{1},\dots, f_{s_{1}}\in \nfm_{\af_{1}}$ and $f_{s_{1}+1},\dots,f_{s_{1}+s_{2}}\in \nfm_{\af_{2}}$
%%    \end{proof}
%%  \end{statement}
%%  Now consider irreducible $\gf$-module $L_{\gf}^{\mu}=M^{\mu}_{\gf}/I^{\mu}_{\gf}$t
%%  
\bibliography{bibliography}{}
\bibliographystyle{utphys}

\end{document}
