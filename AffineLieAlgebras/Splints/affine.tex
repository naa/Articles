\documentclass[12pt]{article}
%% \textheight=24cm
%% \textwidth=16.5cm
%% \topmargin=-1.5cm
%% \oddsidemargin=-0.25cm
%%
\usepackage{amsfonts}

%%%%%%%%%%%%%%%%%%%%%%%%%%%%%%%%%%%%%%%%%%%%%%%%%%%%%%%%%%%%%%%%%%%%%%%%%%%%%%%%%%%%%%%%%%%%%%%%%%%
\usepackage{amsmath,amssymb,amsthm}
\usepackage{multicol}
\usepackage{color}
\usepackage{hyperref}
\usepackage{graphicx}
\usepackage[english]{babel}

\newtheorem{Def}{Definition}[section]
\newtheorem{theorem}{Theorem}
\newtheorem{statement}{Statement}
\newtheorem{Cnj}[Def]{Conjecture}
\newtheorem{Prop}[Def]{Property}
\newtheorem{example}{Example}[section]

\newcommand{\co}[1]{\stackrel{\circ }{#1}}
\newcommand{\gf}{\mathfrak{g}}
\newcommand{\nfp}{\mathfrak{n}^{+}}
\newcommand{\nfm}{\mathfrak{n}^{-}}
\newcommand{\af}{\mathfrak{a}}
\newcommand{\uf}{\mathfrak{u}}
\newcommand{\sfr}{\mathfrak{s}}
\newcommand{\aft}{\widetilde{\mathfrak{a}}}
\newcommand{\afb}{\mathfrak{a}_{\bot}}
\newcommand{\hf}{\mathfrak{h}}
\newcommand{\hfb}{\mathfrak{h}_{\bot}}
\newcommand{\pf}{\mathfrak{p}}

\newcommand{\gfh}{\hat{\mathfrak{g}}}
\newcommand{\afh}{\hat{\mathfrak{a}}}
\newcommand{\sfh}{\hat{\mathfrak{s}}}
\newcommand{\bff}{\mathfrak{b}}
\newcommand{\hfg}{\hf_{\gf}}

\begin{document}
\title{On affine extension of splint root systems}



\author{V.D.~Lyakhovsky$^1$, A.A.~Nazarov$^{1,2}$ \\
  {\small $^1$ Department of High-energy and elementary particle physics, SPb State University}\\
  {\small 198904, Saint-Petersburg, Russia,}
  {\small e-mail: lyakh1507@nm.ru}\\
  {\small$^{2}$ Chebyshev Laboratory,}
  {\small Department of Mathematics and Mechanics, SPb State University}\\
  {\small 199178, Saint-Petersburg, Russia}
  {\small email: antonnaz@gmail.com}}
\date{}
\maketitle

\begin{abstract}
Splint of root system of simple Lie algebra appears naturally in
the study of (regular) embeddings of reductive subalgebras. It can
be used to derive branching rules. Application of
splint properties drastically simplifies calculations of
branching coefficients. We study affine extension of splint root system of simple Lie algebra and obtain relations on theta and branching functions.
\end{abstract}

\section{Introduction}
\label{sec:introduction}

The equivalence of coset models $g_2/su(3)$ and $su(3)\times
su(3)/su(3)$ is well-known for more then twenty years \cite{Dunbar:1992gh}. 
They are also equivalent to $su(2)\times su(2)/su(2)$ coset model with
non-diagonal modular invariant. In paper \cite{Dunbar:1992gh} the equivalence
is derived from the coincidence of conformal weights and central
charges in these models.

Note that irreducible Virasoro modules are uniquely specified by central
charge and conformal weights and characters
of Virasoro modules are proportional to branching functions. So the branching
coefficients for the affine Lie algebra embeddings $\hat A_2\to \hat G_2$ and
$\hat A_2\to \hat A_2 \otimes \hat A_2$ should coincide.

In present text we do not rely on properties of Virasoro algebra modules to
establish this equality of branching functions. We use structure of affine Lie
algebras.


\section{Branching coefficients, branching funcitons and characters}
\label{sec:branch-coeff-branch}

For affine Lie algebra $\gfh$ the specialized character can be computed as
\begin{equation}
  \label{eq:1}
  \mathrm{ch}_{\mu} (\tau, \{z^i \}) = \mathrm{tr}\left( e^{2\pi i \tau L_0
    +\frac{c}{24} - h_{\mu } + \sum_i z_i H^i_0} \right)= \mathrm{tr} \left(
  q^{L_0 +c/24-h_{\mu}} \prod_i \omega_i^{H^i_0}\right)
\end{equation}
Here $H^i$ are the elements of the Cartan subalgera of $\gf$, $q=e^{2\pi i
  \tau}$, $\omega_i = e^{2\pi i z_i}$, $h_{\mu} = \frac{(\mu,\mu+2\rho)}{2(k+h^{\vee}_{\gf})}$. 
Virasoro character for conformal family of primary field $\varphi_{\mu}$ for Wess-Zumino-Novikov-Witten model with affine Lie
algebra $\gfh$ is equal to restricted character multiplied by
$q^{h_{\mu}-c/24}$:
\begin{equation}
  \label{eq:2}
  \chi_{\mu} (\tau) = q^{h_{\mu}-c/24} \mathrm{ch} _\mu (\tau,0)
\end{equation}

Now consider branching coefficients computation for the embedding
$\hat{su(3)}\to \hat g_2$. Construct the injection
fan. All the multipliers, corresponding to long and imaginary roots of
$\hat g_2$ are reduced. So we are left with
$\prod_{\alpha\in \Delta^+_{short} (g_2)}\prod_{n=1}^{\infty}
(1-exp(-\alpha - n\delta))$
We have Weyl denominator of affine $su(3)$ on short roots, but without
purely imaginary roots.

To construct singular element we can present the Weyl group elements
of $\hat g_2 as s_1\dots s_k t_{\lambda} s'_1\dots s'_l$,
where $s_i$ are reflection w.r.t. long roots, $s'$ are reflections w.r.t.
short roots and $t_{\lambda}$ are usual translations
that change grade. The translation can be presented as a combination
of translations on the short roots (?).
So in each grade of singular element we have singular weights of $\hat
A_2$ on short roots, additionally
translated and reflected by Weyl group of $A_2$ on long roots.

This data allows us to compute branching coefficients.

Then we need to compare this picture with $su(3)\times su(3)/su(3)$
model. The injection fan is similar to previous
case, but we have the imaginary roots, since rank of algebra is 4 and
rank of subalgebra is 2.
In singular element we also have some additional translations. The
question is, how do these translations compensate
extra imaginary roots in calculation?

The case of non-diagonal modular invariants for $su(2)\times
su(2)/su(2)$ model is also unclear for me.


\section{Splints and affine root systems}
\label{sec:splints-affine-root}

\begin{Def}
  Embedding $\phi$ of a root system $\Delta_1$ into a root system $\Delta$ is a bijective map of
  roots of $\Delta_{1}$ to a (proper) subset of $\Delta$ that commutes with vector composition law
  in $\Delta_{1}$ and $\Delta$.
\begin{equation*}
\phi:\Delta_1 \longrightarrow \Delta, \quad \phi \circ (\alpha + \beta) =\phi \circ \alpha + \phi \circ \beta,\,\,\, \alpha,\beta \in \Delta_1
\end{equation*}
\end{Def}

Note that the image $Im(\phi)$ must not inherit the root system properties except the addition rules
equivalent to the addition rules in $\Delta_{1}$ (for pre-images). Two embeddings $\phi_1$ and
$\phi_2$ can splinter $\Delta$ when the latter can be presented as a disjoint union of images
$Im(\phi_1)$ and $Im(\phi_2)$.

$\phi$ induces an injection of formal algebras $:{\mathcal{E}}_0
\hookrightarrow \mathcal{E}$ and for the image ${\mathcal{E}}%
_i=Im_{\phi}\left( {\mathcal{E}}_0\right)$ one can consider its inverse $%
\phi^{-1}:{\mathcal{E}}_i \longrightarrow {\mathcal{E}}_0$.

\begin{Def}
  A root system $\Delta $ ''splinters'' as $(\Delta _{1},\Delta _{2})$ if there are two embeddings
  $\phi _{1}:\Delta _{1}\hookrightarrow \Delta $ and $%
  \phi _{2}:\Delta _{2}\hookrightarrow \Delta $ where (a) $\Delta $ is the disjoint union of the
  images of $\phi _{1}$ and $\phi _{2}$ and (b) neither the rank of $\Delta _{1}$ nor the rank of
  $\Delta _{2}$ exceeds the rank of $%
  \Delta $.
\end{Def}

It is equivalent to say that $(\Delta_1,\Delta_2)$ is a "splint'' of $\Delta$ and we shall denote
this by $\Delta \approx (\Delta_1,\Delta_2)$. Each component $\Delta_1$ and $\Delta_2$ is a "stem''
of the splint.


We consider the case when one of the stems $\Delta _{1}=\Delta _{\frak{a}}$ is a root subsystem. As shown in paper \cite{2011arXiv1111.6787L} the second stem $\Delta _{\frak{s}}:=\Delta
_{2}=\Delta \setminus \Delta _{\frak{a}}$ can be translated into a product
$\prod_{\beta \in \Delta _{\frak{s}}^{+}}\left( 1-e^{-\beta }\right)=-\sum_{\gamma \in P}s(\gamma )e^{-\gamma }\quad   \label{splint product}$
and it defines an injection fan $\Gamma _{\frak{a}%
\hookrightarrow \frak{g}}$ \cite{lyakhovsky1996rra,ilyin812pbc,2010arXiv1007.0318L}.


Using straightforward calculation we can easily obtain an identity on theta-functions



definition of Dedekind eta-function
$\eta(\tau)=q^{1/24}\prod_{n=1}^{\infty}(1-q^{n})$, where
$q=e^{2\pi i \tau}$ we can rewrite this identity as the relation
imposed on theta-functions $\Theta^{(\gfh)}_{\widehat{\lambda}=(\lambda,k,0)}(\tau,z)=\sum_{\xi\in Q_{\gf}+\frac{\lambda}{k}}e^{2\pi i k \left(\frac{1}{2} (\xi,\xi) \tau + (\xi,z)\right)}$:
\begin{multline}
  \label{eq:5}
  \eta(\tau)^{\mathrm{dim}(\af)}\prod_{\alpha\in\Delta_{1}^{+}}\frac{\Theta^{(\hat A_{1})}_{\alpha}(\tau,z)}{\eta(\tau)} \eta(\tau)^{\mathrm{dim}(\sfr)}\prod_{\beta\in\Delta_{2}^{+}}\frac{\Theta^{(\hat A_{1})}_{\phi\circ \beta}(\tau,z))}{\eta(\tau)}=\\
\eta(\tau)^{\mathrm{rank}(\af)+\mathrm{rank}(\sfr)-\mathrm{rank}(\gf)}
\eta(\tau)^{\mathrm{dim}(\gf)}\prod_{\alpha\in\Delta^{+}}\frac{\Theta^{(\hat A_{1})}_{\alpha}(\tau,z))}{\eta(\tau)}
\end{multline}
Here $z\in P_{\geq 0}\otimes \mathbb{C}$. Using Weyl denominator identity this relation can be
rewritten as a non-trivial relation connecting theta-functions of algebras $\gfh,\hat\sfr,\afh$:
\begin{equation}
  \label{eq:6}
  \left(\sum_{v\in W_{\af}}\epsilon(v) \Theta^{(\afh)}_{v\rho_{\af}}(\tau,z)\right)
  \cdot \left(\sum_{u\in W_{\sfr}}\epsilon(u) \Theta^{(\hat{\sfr})}_{\phi\circ(u\rho_{\sfr})}(\tau,z)\right)= 
  \left(\sum_{w\in W}\epsilon(w) \Theta^{(\gfh)}_{w\rho_{\gf}}(\tau,z)\right)
\end{equation}


\bibliography{bibliography}{}
\bibliographystyle{utphys}

\end{document}