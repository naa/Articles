\documentclass[12pt]{article}
\usepackage{amsfonts}

%%%%%%%%%%%%%%%%%%%%%%%%%%%%%%%%%%%%%%%%%%%%%%%%%%%%%%%%%%%%%%%%%%%%%%%%%%%%%%%%%%%%%%%%%%%%%%%%%%%
\usepackage{amsmath,amssymb,amsthm}
\usepackage{multicol}
\usepackage{color}
\usepackage{hyperref}
\usepackage{graphicx}
\usepackage[russian]{babel}
\usepackage[utf8]{inputenc}

\newtheorem{Def}{Определение}[section]
\newtheorem{theorem}{Теорема}
\newtheorem{statement}{Утверждение}
\newtheorem{Cnj}[Def]{Гипотеза}
\newtheorem{Prop}[Def]{Свойство}
\newtheorem{example}{Пример}[section]
\newtheorem{axiom}{Аксиома}

\newcommand{\co}[1]{\stackrel{\circ }{#1}}
\newcommand{\gf}{\mathfrak{g}}
\newcommand{\nfp}{\mathfrak{n}^{+}}
\newcommand{\nfm}{\mathfrak{n}^{-}}
\newcommand{\af}{\mathfrak{a}}
\newcommand{\uf}{\mathfrak{u}}
\newcommand{\sfr}{\mathfrak{s}}
\newcommand{\aft}{\widetilde{\mathfrak{a}}}
\newcommand{\afb}{\mathfrak{a}_{\bot}}
\newcommand{\hf}{\mathfrak{h}}
\newcommand{\hfb}{\mathfrak{h}_{\bot}}
\newcommand{\pf}{\mathfrak{p}}

\newcommand{\gfh}{\hat{\mathfrak{g}}}
\newcommand{\afh}{\hat{\mathfrak{a}}}
\newcommand{\sfh}{\hat{\mathfrak{s}}}
\newcommand{\bff}{\mathfrak{b}}
\newcommand{\hfg}{\hf_{\gf}}

\begin{document}
\title{Вертексные алгебры, связанные с представлениями аффинных алгебр Ли}

\maketitle

\begin{abstract}
  В этих заметках приводится мотивация и определения теории вертексных алгебр.
\end{abstract}

\section{Мотивация}

Конформную теорию поля в двух измерениях можно определить, как теорию, корреляционные функции в
которой удовлетворяют некоторым аксиомам. Например, следующая аксиоматика была приведена в моей
диссертаций. 

Корреляционные функции обычно понимают следующим образом. Пусть $\Omega\in\mathbb{H}$ -- вакуум,
тогда $n$-точечные корреляционные функции в квантовой теории поля даются выражением
\begin{equation}
  \label{eq:50}
  G_{i_{1},\dots ,i_{n}}(z_{1},\dots,z_{n}):=\left<\Omega|\varphi_{i_{1}}(z_{1})\dots \varphi_{i_{n}}(z_{n})|\Omega\right>, \quad |z_{n}|>\dots > |z_{1}|,
\end{equation}
где $\varphi_{i_{k}}$ -- поля в теории. Функции $G_{i_{1},\dots,i_{n}}$ можно аналитически
продолжить на пространство $M_{n}=\{(z_{1},\dots,z_{n})\in \mathbb{C}^{n}: z_{i}\neq z_{j},\; i\neq
j\}$. $M_{n}^{+}$ состоит из наборов, где $\mathrm{Re}\;z_{i}>0$ для любого $i$. Введем
последовательность пространств тест-функций $S_{n}^{+}$, где $S_{0}^{+}=\mathbb{C}$, а
$S_{n}^{+}=\{f\in S(\mathbb{C}^{n}): \mathrm{supp}(f)\subset M^{+}_{n}\} $. Теперь мы можем
определить корреляционные функции не обращаясь к понятию полей.

Пусть $B_{0}$ -- индексное множество (счетное). Последовательности произвольной длины
$(i_{1},\dots,i_{n})\in B_{0}^{n}$ образуют пространство мультииндексов $i\in B$. Корреляционные
функции $G_{i_{1},\dots, i_{n}}:M_{n}\to \mathbb{R}$ удовлетворяют следующим аксиомам.
\begin{axiom}
  (Аксиома локальности) Для всех $(i_{1},\dots,i_{n})\in B_{0}^{n}$, $(z_{1},\dots, z_{n})\in M_{n}$
  и $\pi\in S_{n}$ --- перестановок множества из $n$ элементов верно равенство
  \begin{equation}
    \label{eq:51}
    G_{i_{1},\dots,i_{n}}(z_{1},\dots,z_{n})=G_{i_{\pi(1)},\dots i_{\pi(n)}}(z_{\pi(1)},\dots, z_{\pi(n)})
  \end{equation}
\end{axiom}
Рассмотрим группу движений двумерного пространства $E_{2}$, генераторами которой являются повороты
$r_{\alpha}:\mathbb{C}\to\mathbb{C}, \; z\to e^{i\alpha}z,\; \alpha\in \mathbb{R}$ и трансляции
$t_{a}:\mathbb{C}\to\mathbb{C},\; z\to z+a,\; a\in\mathbb{C}$.
\begin{axiom}
  (Аксиома ковариантности) Для любого индекса $i\in B_{0}$ существуют независимые конформные веса
  $h_{i},\bar h_{i}\in \mathbb{R}$, такие, что для всех преобразований $w\in E_{2}$
  \begin{multline}
    \label{eq:52}
    G_{i_{1},\dots,i_{n}}(z_{1},\dots,z_{n},\bar z_{1},\dots, \bar z_{n})=\\
    \prod_{j=1}^{n}\left(\frac{dw(z_{j})}{dz}\right)^{h_{i_{j}}}\left(\overline{\frac{dw(z_{j})}{dz}}\right)^{\bar{h}_{i_{j}}}
    G_{i_1,\dots i_n}(w_{1},\dots, w_{n},\bar w_{1},\dots,\bar w_{n}),
  \end{multline}
  где $w_{i}=w(z_{i})$, а $s_{i}=h_{i}-\bar h_{i}, d_{i}=h_{i}+\bar h_{i}$ -- конформный спин и
  скейлинговая размерность.
\end{axiom}
\begin{axiom}
  (Положительность по отношению к отражениям). Обозначим через $\Theta:\mathbb{C}\to\mathbb{C}$
  отображение $z=t+i y\to \Theta(z)= -t+i y$. Тогда аксиома утверждает, что существует инволюция
  $\star:B_{0}\to B_{0}$, $\star^{2}=\mathrm{id}_{B_{0}}$, которая продолжается на $B$ ($\star:i\to
  i^{*}$), и выполняются свойства
  \begin{enumerate}
  \item Верно равенство
    \begin{equation}
      \label{eq:53}
      G_{i}(z)=G_{i^{*}}(\Theta(z))=G_{i^{*}}(-z^{*})
    \end{equation}
  \item Обозначим через $\underline{S}^{+}$ пространство последовательностей тест-функций
    $\underline{f}=(f_{i})_{i\in B}, f_{i}\in S^{+}_{n}$. Тогда
    \begin{multline}
      \label{eq:54}
      \left<\underline{f},\underline{f}\right>=\\
      \sum_{i,j\in B}\sum_{n,m}\int_{M_{n+m}}G_{i^{*} j}(\Theta(z_{1}),\dots ,\Theta(z_{n}),w_{1},\dots,w_{m}) f_{i}(z)^{*}f_{j}(w) d^{n}z d^{m}w 
      \geq 0,\\ \forall \underline{f}\in \underline{S}^{+}
    \end{multline}
  \end{enumerate}
\end{axiom}
Эта аксиома позволяет восстановить гильбертово пространство $\mathbb{H}$, так как она дает
положительную полуопределенную форму $H$ на $\underline{S}^{+}$. То есть мы можем определить
$\mathbb{H}$ как пополнение $\underline{S}^{+}$ факторизованное по $\mathrm{ker}\, H$ с
произведением \eqref{eq:54}. Мы можем построить и полевые операторы. Для $j\in B_{0}$ определим
$\varphi_{j}$ как операторно-значную обобщенную функцию. Пусть $f\in S^{+}$,
$\underline{g}\in\underline{S}^{+}$, а через $[\underline{g}]$ обозначим класс эквивалентности
$\underline{g}$ по отношению к ядру $H$. Определим $\varphi_{j}(f)([\underline{g}])$ как класс
эквивалентности $\underline{g}\times f$, такой, что
\begin{equation}
  \label{eq:55}
  \begin{array}{l}
    \underline{g}\times f=((\underline{g}\times f)_{i_{1},\dots,i_{n+1}});\quad\quad (i_{1},\dots,i_{n+1})\in B\\
    (\underline{g}\times f)_{i_{1},\dots,i_{n+1}}(z_{1},\dots,z_{n+1}):=g_{i_{1},\dots,i_{n}}(z_{1},\dots,z_{n})f(z_{n+1})\delta_{j,i_{n+1}}.
  \end{array}
\end{equation}
Можно показать, что эта конструкция порождает унитарное представление $U$ группы $E_{2}$ евклидовых
движений плоскости на гильбертовом пространстве $\mathbb{H}$. Кроме того, существует инвариантное
плотное подпространство $D\subset \mathbb{H}$, такое, что отображения
$\varphi_{j}(f):[\underline{g}]\to [\underline{g}\times f]$ определены на $D$ для всех $j\in B_{0}$
и $\varphi_{j}(f)(D)\subset D$. Также существует вакуум $\Omega\in\mathbb{H}: \Omega=[f];\;
f_{\emptyset}=1, f_{i}=0\quad \forall i\neq \emptyset$. Тогда следующая теорема определяет структуру
двумерной евклидовой теории поля
\begin{theorem}
  \begin{enumerate}
  \item Для всех $j\in B_{0}$ отображения $\varphi_{j}:S^{+}\to \mathrm{End}(D)$ линейны, $\varphi_{j}$ -- полевые операторы, $\varphi_{j}(D)\subset D, \Omega\in D$ и вакуум $\Omega$ инвариантен относительно унитарных представлений $U$ группы $E_{2}$.
  \item Поля $\varphi_{j}$ преобразуются ковариантно по отношению к представлению $U$, для $w\in E_{2}$:
    \begin{equation}
      \label{eq:57}
      U(w)\varphi_{j}(z)U(w)^{*}=\left(\frac{\partial w}{\partial z}\right)^{h_{j}}\varphi_{j}(w(z))
    \end{equation}
  \item Матричные коэффициенты $\left<\Omega|\varphi_{i_{1}}(z_{1})\dots \varphi_{i_{n}}(z_{n})|\Omega\right>$ представляются аналитическими функциями, которые при $\mathrm{Re}z_{n}>\dots>\mathrm{Re}z_{1}>0$ совпадают с корреляционными функциями
  \begin{equation}
    \label{eq:56}
    \left<\Omega|\varphi_{i_{1}}(z_{1})\dots \varphi_{i_{n}}(z_{n})|\Omega\right>=G_{i_{1},\dots,i_{n}}(z_{1},\dots,z_{n})
  \end{equation}
  \end{enumerate}
\end{theorem}
Доказательство этой теоремы приведено в книге \cite{schottenloher2008mathematical}.

Теперь добавим аксиомы, специфичные для конформной теории поля. Во-первых введем масштабную
инвариантность.
\begin{axiom}
  (Масштабная инвариантность) Корреляционная функция $G_{i}, i\in B$ преобразуется ковариантно
  \eqref{eq:52} при масштабных преобразованиях $w(z)=e^{\tau}z$, то есть
  \begin{equation}
    \label{eq:58}
    G_{i_{1},\dots,i_{n}}(z_{1},\dots,z_{n})=\left(e^{\tau}\right)^{h_{1}+\dots+h_{n}+\bar{h}_{1}+\dots+\bar{h}_{n}} 
    G_{i_{1},\dots,i_{n}}(e^{\tau} z_{1},\dots,e^{\tau} z_{n}),
  \end{equation}
  где $(z_{1},\dots,z_{n})\in M_{n},\quad h_{j}=h_{i_{j}}$
\end{axiom}
Из требований масштабной инвариантности можно вычислить двухточечные функции. 
\begin{axiom}
  (Существование тензора энергии-импульса) Среди полей $\varphi_{i},\; i\in B_{0}$ есть четыре поля
  $T_{\mu\nu},\; \mu,\nu=0,1$, такие, что $T_{\mu\nu}=T_{\nu\mu},\quad
  T_{\mu\nu}^{*}=T_{\nu\mu}(\Theta(z)),\quad \partial_{0} T_{\mu 0}+\partial_{1}T_{\mu 1}=0$,
  скейлинговая размерность поля $d(T_{\mu\nu})=h_{\mu\nu}+\bar{h}_{\mu\nu}=2$, конформный спин
  $s(T_{00}-T_{11}\pm 2i T_{01})=\pm 2$.
\end{axiom}
Можно показать, что $\mathrm{tr} T_{\mu\nu}=0$ и $T=T_{00}-i T_{01}$ не зависит от $\bar z$, то есть
$\bar \partial T=0$. Операторы
\begin{equation}
  \label{eq:59}
    L_{n}=\oint \frac{dz}{2\pi i} z^{n+1} T(z)
\end{equation}
удовлетворяют коммутационным соотношениям алгебры Вирасоро. 

Примарными называются поля $\varphi_{i}, i\in B_{0}$, такие, что
\begin{equation}
  \label{eq:60}
  [L_{n}, \varphi_{i}(z)]=z^{n+1}\partial \varphi_{i}(z)+h_{i}(n+1)z^{n}\varphi_{i}(z),\quad \forall n\in\mathbb{Z}
\end{equation}
Для каждого примарного поля $\varphi_{i}$ можно определить конформное семейство $[\varphi_{i}]$,
состоящее из полей вторичных $\varphi_{i}^{\alpha}(z)=L_{-\alpha_{1}}(z)\dots
L_{-\alpha_{n}}(z)\varphi_{i}(z)$, где $L_{-n}(z)=\frac{1}{2\pi i}\oint\frac{T(\xi)}{(\xi-z)^{n+1}}
d\xi$. Заметим, что $L_{n}(0)=L_{n}$ и корреляционные функции вторичных полей могут быть выражены
через корреляционные функции примарных. Последняя аксиома определяет операторное разложение
\begin{axiom}
  (Операторное разложение). Корреляционные функции примарных полей при $z_{i}\to z_{j}$
  удовлетворяют уравнению:
\begin{multline}
  \label{eq:61}
  \left<\Omega|\varphi_{i_{1}}(z_{1})\dots\varphi_{i}(z_{i})\dots \varphi_{j}(z_{j})\dots \varphi_{i_{n}}(z_{n})|\Omega\right>=\\
  \sum_{k\in B_{0}}C_{ijk} (z_{i}-z_{j})^{h_{k}-h_{i}-h_{j}} \left<\Omega|\varphi_{i_{1}}(z_{1})\dots\varphi_{k}(z_{k})\dots \varphi_{i_{n}}(z_{n})|\Omega\right>\\
  +\mbox{регулярные члены}
\end{multline}

\end{axiom}

Если дополнительно предположить, что операторное разложение ассоциативно (так называемый
``конформный бутстрап''), то вся теория определяется набором примарных полей $\varphi_{j}, j\in
B_{0}$, их размерностями и коэффициентами операторного разложения $C_{ijk}$.

Для формализации конформной теории поля нам нужна алгебраическая конструкция, которая будет
удовлетворять этому (или какому-то другому) набору аксиом. Предлагается следующее определение
вертексной алгебры:

\begin{Def}
  Вертексная алгебра -- это градуированное векторное пространство
  $V=\sum_{n\in \mathbb{Z}_{+}} V_{n}$, содержащее выделенный элемент
  $1$ или  $\Omega$, снабженное эндоморфизмом $T:V\to V$ и линейным отображением
  $Y:V\otimes V\to \mathrm{End}V((z)): (a,b)\to Y(a,z)b=\sum_n a (n) b
  z^{-n-1}$ таким, что выполняются следующие свойства:
  \begin{itemize}
  \item {\it Единица} $\forall a\in V, Y(1,z) a =a$ и $Y(a,z) 1 \in a+z V z$
  \item {\it Трансляции} $T(1)=0$ and $\forall a,b\in V$:
    $$Y(a,z)T b - T Y(a,z) b =
    \frac{d}{dz} Y(a,z) b$$
  \item {\it Четырехточечная функция}. $\forall a,b,c\in V \exists
    X(a,b,c;z,w)\in V[[z,w]][z^{-1},w^{-1},(z-w)^{-1}]: Y(a,z)Y(b,w)c,
    Y(b,w)Y(a,z)c, Y(Y(a,z-w)b,w)c$ are expansions of $X(a,b,c;z,w)$
    in $V((z))((w)), V((w))((z)), V((w))((z-w))$
  \item {\it Градуировка} $V=\bigoplus_{n=0}^{\infty} V_n$
  \item {\it Элемент Вирасоро} $\omega\in V_2: Y(w,z)=\sum_{n\in
      \mathbb{Z}} L_n z^{-n-2}$ удовлетворяет $\forall a\in V_n$
    соотношениям $L_0 a =n a$,
    $Y(L_{-1}a,z)=\frac{d}{dz}Y(a,z)=[Y(a,z),T]$,
    $[L_m,L_n]a=(m-n)L_{m+n} a + \delta_{m+n,0} \frac{m^3 - m}{12}ca$,
    где $c$ -- центральный заряд .
  \end{itemize}

\end{Def}


Так как мы работаем с формальными рядами, введем следующие обозначения.
Пусть $f(z)=\sum f_n z^{-n-1}$ тогда $\mathrm{Res} f(z)= f_0$. Определим $i_{z_1,\dots z_n}
f(z_1,\dots,z_n)$ как разложение $f$ в ряд в области $|z_1|>\dots>|z_n|$. 
Тогда $i_{z,w}(z-w)^{-n} = \sum_{k=1}^{\infty} C_n^k (-1)^k z^{-n-k} w^k$ и $i_{w,z}(z-w)^{-n} = \sum_{k=1}^{\infty} C_n^k (-1)^{n+k} w^{-n-k} z^{k} $ 

\subsection{Связь вертексных алгебр и конформной теории поля}
 Примарные поля
$\varphi_j$ можно понимать в смысле вертексных алгебр: $\varphi_j (z)
=  \sum a(n) z^{-n-1} \in \mathrm{End} D [[z^{\pm}]]$. Если
рассматривать пространство $V\subset D$, натянутое на элементы $\left\{a^1
(-k_1) \dots a^n (-k_n) 1 \right\} \cup \{1\}$, то $\varphi_j (z) \in
\mathrm{End} V [[z^{\pm}]]$. 
$\varphi_j (z) = Y(j,z)$.

С другой стороны, корреляционные функции можно определить как
матричные элементы вертексной алгебры: $\left<v',Y(a_1,z_1)\dots
  Y(a_n,z_n) v\right> \;\mbox{for}\; v\in V, v'\in V^*$

\section{Вертексные алгебры и аффинные алгебры Ли}
\label{sec:affine-lie-algebras}
Данный раздел следует статье \cite{frenkel1992vertex}.

Рассмотрим аффинную алгебру Ли. Пусть $\gf$ -- простая конечномерная
алгера Ли, $\hf$ -- подалгебра Картана, $\Delta$ -- корневая система
$\gf$, $\Delta^+$ -- набор положительных корней, $\Theta$ --
максимальный корень и $(\Theta,\Theta)=2$. 
Тогда $\gfh=\gf\otimes \mathbb{C} [t,t^{-1}]\oplus \mathbb{C}{\bf k}$
и $[a\otimes t^n, b\otimes t^m]=[a,b]\otimes t^{m+n} +{\bf k}\cdot
(a,b)\delta_{m+n,0}$, где $a,b\in \gf$. 

Пусть $\gfh_+=\gf\otimes \mathbb{C}[t]t$, $\gfh_- =\gf\otimes
\mathbb{C}[t^{-1}]t^{-1}$, тогда $\gfh=\gfh_+\oplus \gfh_- \oplus \gf\oplus
\mathbb{C}{\bf k}$. 
Обозначим $a\otimes t^n$ через $a(n)$.

Для произвольного $\gf$-модуля $V$ построим индуцированный $\gfh$-модуль $\hat{V}_k$ следующим
образом: рассмотрим $V$ как $\gfh_+ \oplus\gf\oplus \mathbb{C}{\bf k}$-модуль с
действием $\gfh_+ V=0; \; {\bf k} V=k V,\; k\in \mathbb{Z}$, тогда
\begin{equation}
  \label{eq:1}
  \hat{V}_k=U(\gfh) \bigotimes_{U(\gfh_+)\oplus \gf\oplus\mathbb{C}{\bf k}} V
\end{equation}
Для старшего веса $\mu\in \hf^* $ обозначим через $M^{\mu}_{k}$ $\gfh$-модуль
$\hat{L}^{\mu}_k$. Пусть $J^{\mu}_k$ -- максимальный подмодуль $M^{\mu}_k$,
тогда $L^{\mu}_k=M^{\mu}_k/J^{\mu}_k$ -- неприводимый модуль аффинной алгебры Ли
$\gfh$ со старшим весом $(\mu,k,0)$.

Если $\mu=0$, то $M^0_k\cong U(\gfh_-)$ как векторное пространство. 

Градуировка в алгебре $\mathrm{deg} a(n)=-n, \; \mathrm{deg} {\bf k}=0$ 
продолжается на $\hat{V}_k$ и $U(\gfh)$.

Покажем, что $M^0_k$ при $k\neq h^{\vee}$ можно рассматривать как вертексную
алгебру и все $\hat{V}_k$ являются ее модулями. 

$M^0_k=\bigoplus_{n=0}^{\infty} M^0_k(n)$ и $M^0_k(0)\cong \mathbb{C}$,
$M^0_k(1)=\gf$. Введем $Y(1,z)=Id$, где $1\in M^0_k$ и для всех $a\in \gf\subset
M^0_k$ определим
\begin{equation}
  \label{eq:2}
  Y(a,z)=a(z)=\sum_{n=-\infty}^{\infty} a(n) z^{-n-1}
\end{equation}
На физическом языке эти вертексные операторы называются токами $J^a(z)$.

Элементы $a(z)$ принадлежат пополненному пространству $\tilde{U}(\gfh,k)<z>$, формальных
рядов с коэффициентами в $U(\gfh)$, где действие ${\bf k}$ равно $k
\mathrm{Id}$.

Введем действие $\gfh$ на $\tilde{U}(\gfh,k)<z>$ следующим образом:
\begin{equation}
  \label{eq:3}
  a(n)\bullet b(z)=\mathrm{Res}_w \left(a(w)b(z)i_{w,z}(w-z)^n - b(z)a(w) i_{z,w}(w-z)^n\right)
\end{equation}
Это эквивалентно формальному определению контурного интеграла $\frac{1}{2\pi i }
\oint_C a(w) b(z) (w-z)^n dw$

Итак, мы наделили  $\tilde{U}(\gfh,k)<z>$ структурой $\gfh$-модуля, при этом
$1$ является вектором старшего веса, $a(n)\bullet 1=0$ при $n\geq 0$ и
$a(n)\bullet 1=\frac{1}{(-n-1)!}\left(\frac{d}{dz}\right)^{-n-1}a(z)$ при $n<0$. 

То есть $Y(\cdot, z): a_1 (-i_1)\dots a_n(-i_n)1\to a_1(-i_1)\bullet\dots\bullet
a_n(-i_n)\bullet 1$ -- гомоморфизм $\gfh$-модулей $M^0_k$ и
$\tilde{U}(\gfh,k)<z>$. Будем называть $Y(b,z)$ вершинным оператором,
соответствующим $b$.

Элемент Вирасоро $\omega=\frac{1}{2(k+h^{\vee})}\sum_i u_i(-1) u_i(-1)$, где
$u_i (i=1,\dots,\mathrm{dim}\gf)$ -- ортогональный базис по отношению к форме
Картана-Киллинга. 

Случай $k=-h^{\vee}$ выделенный, в этом случае нет элемента Вирасоро.

Можно показать, что отображение
 $$Y_{\hat{V}_k}(\cdot,z):M^0_k\to
\tilde{U}(\gfh,k)<z> \to \mathrm{End}(\hat{V}_k)[[z,z^{-1}]]$$
является представлением вертексной операторной алгебры $M^0_k$.

\subsection{Корреляционные функции}
\label{sec:correlation-functions}

\subsection{Вертексные алгебры, связанные с неприводимыми представлениями}
\label{sec:voa-irreps}



\begin{itemize}
\item Singular element of simple Lie algebra irreducible module, VOA corresponding to that module.
  What are the relations in VOA on singular vectors? Is there BGG-resolution?
\item Affine Lie algebra, parabolic Verma modules?
\item Splints?
\item Coset vertex operator algebras and branching
\end{itemize}

\bibliography{bibliography}{}
\bibliographystyle{utphys}

\end{document}
