\documentclass[12pt]{article}
\usepackage{amsfonts}

%%%%%%%%%%%%%%%%%%%%%%%%%%%%%%%%%%%%%%%%%%%%%%%%%%%%%%%%%%%%%%%%%%%%%%%%%%%%%%%%%%%%%%%%%%%%%%%%%%%
\usepackage{amsmath,amssymb,amsthm}
\usepackage{multicol}
\usepackage{color}
\usepackage{hyperref}
\usepackage{graphicx}
\usepackage[english]{babel}

\newtheorem{Def}{Definition}[section]
\newtheorem{theorem}{Theorem}
\newtheorem{statement}{Statement}
\newtheorem{Cnj}[Def]{Conjecture}
\newtheorem{Prop}[Def]{Property}
\newtheorem{example}{Example}[section]

\newcommand{\co}[1]{\stackrel{\circ }{#1}}
\newcommand{\gf}{\mathfrak{g}}
\newcommand{\nfp}{\mathfrak{n}^{+}}
\newcommand{\nfm}{\mathfrak{n}^{-}}
\newcommand{\af}{\mathfrak{a}}
\newcommand{\uf}{\mathfrak{u}}
\newcommand{\sfr}{\mathfrak{s}}
\newcommand{\aft}{\widetilde{\mathfrak{a}}}
\newcommand{\afb}{\mathfrak{a}_{\bot}}
\newcommand{\hf}{\mathfrak{h}}
\newcommand{\hfb}{\mathfrak{h}_{\bot}}
\newcommand{\pf}{\mathfrak{p}}

\newcommand{\gfh}{\hat{\mathfrak{g}}}
\newcommand{\afh}{\hat{\mathfrak{a}}}
\newcommand{\sfh}{\hat{\mathfrak{s}}}
\newcommand{\bff}{\mathfrak{b}}
\newcommand{\hfg}{\hf_{\gf}}

\begin{document}
\title{On singular elements in conformal field theory. \\ Singular elements in vertex operator algebras}



\author{V.D.~Lyakhovsky$^1$, A.A.~Nazarov$^{1,2}$ \\
  {\small $^1$ Department of High-energy and elementary particle physics, SPb State University}\\
  {\small 198904, Saint-Petersburg, Russia,}
  {\small e-mail: lyakh1507@nm.ru}\\
  {\small$^{2}$ Chebyshev Laboratory,}
  {\small Department of Mathematics and Mechanics, SPb State University}\\
  {\small 199178, Saint-Petersburg, Russia}
  {\small email: antonnaz@gmail.com}}
\date{}
\maketitle

\begin{abstract}
  We consider singular elements of simple finite-dimensional and affine Lie algebra modules. We
  construct corresponding vertex operator algebras. We consider decompositions of singular elements
  connected with branching, splints and parabolic Verma modules and describe vertex counterparts. 
\end{abstract}

\section{Introduction}
\begin{itemize}
\item Singular element of simple Lie algebra irreducible module, VOA corresponding to that module.
  What are the relations in VOA on singular vectors? Is there BGG-resolution?
\item Affine Lie algebra, parabolic Verma modules?
\item Splints?
\item Coset vertex operator algebras and branching
\end{itemize}

\begin{Def}
  Vertex algebra: $V$ -- vector space with identity $1$, endomorphism
  $T:V\to V$ and linear multiplication map $Y:V\otimes V\to V((z)):
  (a,b)\to Y(a,z)b=\sum_n a_n b z^{-n-1}$,
  such that
  \begin{itemize}
  \item {\it Identity} $\forall a\in V, Y(1,z) a =a$ and $Y(a,z) 1 \in a+z V z$
  \item {\it Translation} $T(1)=0$ and $\forall a,b\in V$:
    $$Y(a,z)T b - T Y(a,z) b =
    \frac{d}{dz} Y(a,z) b$$
  \item {\it Four point function}. $\forall a,b,c\in V \exists
    X(a,b,c;z,w)\in V[[z,w]][z^{-1},w^{-1},(z-w)^{-1}]: Y(a,z)Y(b,w)c,
    Y(b,w)Y(a,z)c, Y(Y(a,z-w)b,w)c$ are expansions of $X(a,b,c;z,w)$
    in $V((z))((w)), V((w))((z)), V((w))((z-w))$
  \item {\it Grading} $V=\bigoplus_{n=0}^{\infty} V_n$
  \item {\it Virasoro element} $\omega\in V_2: Y(w,z)=\sum_{n\in
      \mathbb{Z}} L_n z^{-n-2}$ satisfies $\forall a\in V_n$ the
      relations $L_0 a =n a$,
      $Y(L_{-1}a,z)=\frac{d}{dz}Y(a,z)=[Y(a,z),T]$,
      $[L_m,L_n]a=(m-n)L_{m+n} a + \delta_{m+n,0} \frac{m^3 -
        m}{12}ca$, $c$ -- central charge.
  \end{itemize}

\end{Def}
\end{document}
