\documentclass[12pt]{article}
\usepackage{amsfonts}

%%%%%%%%%%%%%%%%%%%%%%%%%%%%%%%%%%%%%%%%%%%%%%%%%%%%%%%%%%%%%%%%%%%%%%%%%%%%%%%%%%%%%%%%%%%%%%%%%%%
\usepackage{amsmath,amssymb,amsthm}
\usepackage{multicol}
\usepackage{color}
\usepackage{hyperref}
\usepackage{graphicx}
\usepackage[english]{babel}

\newtheorem{Def}{Definition}[section]
\newtheorem{theorem}{Theorem}
\newtheorem{statement}{Statement}
\newtheorem{Cnj}[Def]{Conjecture}
\newtheorem{Prop}[Def]{Property}
\newtheorem{example}{Example}[section]

\newcommand{\co}[1]{\stackrel{\circ }{#1}}
\newcommand{\gf}{\mathfrak{g}}
\newcommand{\nfp}{\mathfrak{n}^{+}}
\newcommand{\nfm}{\mathfrak{n}^{-}}
\newcommand{\af}{\mathfrak{a}}
\newcommand{\uf}{\mathfrak{u}}
\newcommand{\sfr}{\mathfrak{s}}
\newcommand{\aft}{\widetilde{\mathfrak{a}}}
\newcommand{\afb}{\mathfrak{a}_{\bot}}
\newcommand{\hf}{\mathfrak{h}}
\newcommand{\hfb}{\mathfrak{h}_{\bot}}
\newcommand{\pf}{\mathfrak{p}}

\newcommand{\gfh}{\hat{\mathfrak{g}}}
\newcommand{\afh}{\hat{\mathfrak{a}}}
\newcommand{\sfh}{\hat{\mathfrak{s}}}
\newcommand{\bff}{\mathfrak{b}}
\newcommand{\hfg}{\hf_{\gf}}

\begin{document}
\title{On singular elements in conformal field theory. \\ Singular elements in vertex operator algebras}



\author{V.D.~Lyakhovsky$^1$, A.A.~Nazarov$^{1,2}$ \\
  {\small $^1$ Department of High-energy and elementary particle physics, SPb State University}\\
  {\small 198904, Saint-Petersburg, Russia,}
  {\small e-mail: lyakh1507@nm.ru}\\
  {\small$^{2}$ Chebyshev Laboratory,}
  {\small Department of Mathematics and Mechanics, SPb State University}\\
  {\small 199178, Saint-Petersburg, Russia}
  {\small email: antonnaz@gmail.com}}
\date{}
\maketitle

\begin{abstract}
  We consider singular elements of simple finite-dimensional and affine Lie algebra modules. We
  construct corresponding vertex operator algebras. We consider decompositions of singular elements
  connected with branching, splints and parabolic Verma modules and describe vertex counterparts. 
\end{abstract}

\section{Introduction}
The notion of vertex operator algebra \cite{borcherds1986vertex} formalizes algebraic structure that
underlies two-dimensional conformal field theory \cite{kac1998vertex,frenkel1988vertex}. Vertex
algebras associated to representations of affine Lie algebras \cite{frenkel1992vertex} form an
important class corresponding to Wess-Zumino-Novikov-Witten \cite{witten1984nab,Walton:1999xc} and
coset models \cite{Goddard198588} of conformal field theory. These vertex algebras are used to
reveal modular properties of formal characters of affine Lie algebra modules \cite{zhu1996modular}. 

In present paper we consider vertex operator algebras corresponding to coset models of conformal
field theory. Properties of these algebras are connected with the problem of reduction of affine Lie
algebra module into the sum of affine Lie subalgebra modules. 

We demonstrate that recurrent relations on branching coefficients for affine Lie algebra modules are
related to the structure of Virasoro modules. Our method is based upon the structure of singular
elements of affine Lie algebra modules.

\section{Vertex operator algebra for coset models}
\label{sec:vert-oper-algebra}

Recall the definition of vertex operator algebra \cite{borcherds1986vertex,kac1998vertex,frenkel2001vertex,frenkel1988vertex}.
\begin{Def}
  Vertex algebra is a  vector space $V$ with identity $1$, equipped with an endomorphism
  $T:V\to V$ and a linear multiplication map $Y:V\otimes V\to V((z)):
  (a,b)\to Y(a,z)b=\sum_n a_n b z^{-n-1}$,
  such that following relations are satisfied:
  \begin{itemize}
  \item {\it Identity} $\forall a\in V, Y(1,z) a =a$ and $Y(a,z) 1 \in a+z V z$
  \item {\it Translation} $T(1)=0$ and $\forall a,b\in V$:
    $$Y(a,z)T b - T Y(a,z) b =
    \frac{d}{dz} Y(a,z) b$$
  \item {\it Four point function}. $\forall a,b,c\in V \exists
    X(a,b,c;z,w)\in V[[z,w]][z^{-1},w^{-1},(z-w)^{-1}]: Y(a,z)Y(b,w)c,
    Y(b,w)Y(a,z)c, Y(Y(a,z-w)b,w)c$ are expansions of $X(a,b,c;z,w)$
    in $V((z))((w)), V((w))((z)), V((w))((z-w))$
  \item {\it Grading} $V=\bigoplus_{n=0}^{\infty} V_n$
  \item {\it Virasoro element} $\omega\in V_2: Y(w,z)=\sum_{n\in
      \mathbb{Z}} L_n z^{-n-2}$ satisfies $\forall a\in V_n$ the
      relations $L_0 a =n a$,
      $Y(L_{-1}a,z)=\frac{d}{dz}Y(a,z)=[Y(a,z),T]$,
      $[L_m,L_n]a=(m-n)L_{m+n} a + \delta_{m+n,0} \frac{m^3 -
        m}{12}ca$, $c$ -- central charge.
  \end{itemize}

\end{Def}

Consider affine Lie algebra $\gfh$, which is an affine extension of (semi)simple Lie algebra $\gf$.
Let $\omega_{0},\dots,\omega_{r}$ be fundamental weights of $\gfh$. Then $L^{k\omega_{0}}_{\gfh}$ is vertex operator algebra and its irreducible modules are $L^{\mu}_{\gfh}$ such that $\mu<k$.

In this case Virasoro element is  $\omega=\frac{1}{2(k+h^{\vee})}\sum_i u_i(-1) u_i(-1)$, where
$u_i (i=1,\dots,\mathrm{dim}\gf)$  is an orthogonal basis in $\gfh$ w.r.t. Cartan-Killing form. 

Vertex operator algebra structure for coset models of conformal field theory is given by the following statements \cite{frenkel1992vertex}:
\begin{statement}
\label{thm:cosets}
Let  $V=\bigoplus_{n=0}^{\infty} V_n$ -- vertex operator algebra such that $\mathrm{dim}V_0=1$ and $W\subset V$ is a subalgebra, denote by $\omega,\omega_1$ Virasoro elements of  $V$ and $W$ correspondingly. If the condition  $L_1\omega_1 =0$ is satisfied then $W'=\{b:
  a(n)b=0 \forall n\geq 0, a\in W\}$ is vertex operator subalgebra of  $V$ with Virasoro element $\omega'=\omega-\omega_1$.
\end{statement}
Note that for $\af\subset\gf$ and $V=L^0_{(\gf)}k, W=L^0_{(\af)k}$ the algebra
$W'$ is vertex operator algebra of  $G/A$-coset-model. 

The algebra  $W'$ is called {\it commutant} of $W$ and denoted as $\mathrm{Com}(W)\equiv W'$.
\begin{statement}
  If the conditions of previous statement are satisfied and  $Y(\omega_1,z)=\sum_n
  L_{(1)n} z^{-n-2}$ then $\mathrm{Com}(W)=\{b: L_{(1)-1}b=0\}$. 
\end{statement}


\begin{itemize}
\item Singular element of simple Lie algebra irreducible module, VOA corresponding to that module.
  What are the relations in VOA on singular vectors? Is there BGG-resolution?
\item Affine Lie algebra, parabolic Verma modules?
\item Splints?
\item Coset vertex operator algebras and branching
\end{itemize}


\bibliography{bibliography}{}
\bibliographystyle{utphys}

\end{document}
