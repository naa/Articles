\documentclass[a4paper]{article}

\begin{document}
Thank you. I am happy to be here. My talk is about representation theory of affine Lie algebras. I want to emphasize the role of singular elements in structure of irreducible modules. Here is the outline of my talk. At first I will remind you of Wess-Zumino-Novikov-Witten models and coset construction in two-dimensional conformal field theory. Then we will see the appearance of affine Lie algebras. I will remind you of Weyl-Kac formula and use singular element decomposition to introduce recurrent relation on branching coefficients. At last we introduce splints of root  systems and corresponding decompositions of singular elements and consider consequences for representation theory. 

Wess-Zumino-Witten action has this form. The first term is the action of non-linear sigma model. It is written for the field on complex plane which takes values in some semisimple Lie group Gi. Conformal invariance is lost after the quantization, so we need to add topological term to cancel the anomaly. This Wess-Zumino term is written in term of continuation of field gi to 3-manifold B which is bounded by two-sphere. Such continuation is not unique and this expression is not single-valued but the exponent of it is if k is integral. 

Conserved currents for this model are written here. After the introduction of complex variables z and z bar the sectors can be studied independently. We have gauge invariance, where omega and omega bar are independent. Ward identities for gauge transformations are here. 

We do series expansion of the currents and substitute it to Ward identities. We obtain commutation relations of affine Lie algebra g hat. 
Virasoro generators are given by Sugawara construction, so Virasoro algebra is embedded into universal enveloping algebra of g hat. 

We see that full chiral algebra of the model is semidirect product of affine Lie algebra and Virasoro algebra with this commutation relations. 
Note that we have grading here. L zero is essentially an energy operator. 

All the fields are organised as families. Each family consists of primary field and its descendants. Primary fields are defined by this operator product expansion with currents. Primary fields are labelled by highest weights of affine Lie algebra representations. From field-state correspondence we see that these fields are also primary with respect to Virasoro generators and have these conformal dimensions. 

Singular vectors in Hilbert space are those which are destroyed by all raising operators. Hilbert space consists of highest-weight modules of affine Lie algebra. Let us study the structure of these modules. 

The simplest modules are Verma modules. If we introduce formal characters we see that characters of such modules has this form. The numerator consists of single exponent of highest weight and the denominator can be written as the product over positive roots or as the sum over Weyl group orbit. Verma module has unique maximal submodule and non-trivial factormodule is irreducible. We can think of it as of universal enveloping algebra of lowering operators modulo some ideal generated by singular element. 
It can be seen from Weyl-Kac character formula. Here psi is singular element and R is Weyl denominator. Also character of irreducible module can be presented as a combination of Verma module characters. 

Let us now move to coset construction. We add interaction with pure gauge fields to the action. These fields take values in some subalgebra of algebra g. The currents now has two components. From Ward identities we see that we essentially have two Wess-Zumino-Novikov-Witten models. The generators of Virasoro algebra are given by the difference of Sugawara expression. 

Primary fields are now labeled with pairs of weights mu and nu for algebras g and a. But some pairs are equivalent. The equivalence is given by action of so-called simple currents. Conformal weights of primary field are differences of conformal weights in Wess-Zumino-Witten models. 

The modules of affine Lie algebra g hat can be decomposed as a sum of tensor product of subalgebra and Virasoro modules. So we need to study the branching rules to describe Virasoro modules. 

We can rewrite the decomposition of modules as the decomposition of formal characters. Branching coefficients are dimensions of certain energy subspaces of Virasoro modules. 

In order to compute branching coefficients we want to multiply by denominator, equate coefficients before the exponents and rearrange sum  as a recurrent relation. But we have technical difficulty - some components of denominator after projection can give us zero. We will isolate such roots. 

Consider roots that are orthogonal to root system of subalgebra a. It can be seen that such roots form root system and determine subalgebra b which we call orthogonal to a. For such an orthogonal pair of subalgebras we can state following lemma. The singular element of algebra g module divided by the denominator of orthogonal subalgebra can be decomposed into the sum over equivalence classes of Weyl group and coefficients are equal to dimensions of orthogonal subalgebra modules. 

Now we can write recurrent relation for signed branching coefficients. Here is a recurrent term and elements of our decomposition appear as initial data for the recursion. The recursion is governed by the certain set of weights which appear in the expansion of denominator. We call this set an injection fan. It is universal for all modules. 

Now consider simple example. Here is root system of algebra B two. This is highest weight of module. Here are weights of singular element. Here is positive root system of algebra, here is root of subalgebra. Here is root of orthogonal subalgebra. Then we show the modules of orthogonal subalgebra and their dimensions. Calculate branching coefficients recurrently using this initial data. 

Now consider affine extension. Here is injection fan and singular element with the dimensions of modules of orthogonal subalgebra. And here is result of recurrent computation. 

Let us study another decomposition. We will use pair of algebras but one of them is not required to be subalgebra. Instead we will look at the root system of algebra as the union of images of two embeddings. These embeddings are not required to preserve the angles between the roots. We call this situation ``splint''. Let first component of splint be root system of some subalgebra a. Then we can clearly see that splint determine injection fan. We also has a decomposition of singular elements. 

Then it is possible to show that branching coefficients coincide with weight multiplicities of modules of algebra s. Consider as an example Lie algebra B2. Its root system can be seen as the union of root system of A1 and root system of A2 with changed angles. We see that singular element can also be presented as combination of singular elements of A2 with changed angles. So when we calculate branching coefficients to A1 we see that they coincide with weight multiplicities of A2 modules. 

\end{document}
