\documentclass{article}
\usepackage{graphicx}
\usepackage{amsmath,amssymb,amsthm} 
\usepackage{pb-diagram}
\usepackage{ucs}
\usepackage[utf8x]{inputenc}
\usepackage[russian]{babel}
\usepackage{epstopdf}
\usepackage{multicol}
\usepackage{cancel}

\usepackage{amsfonts}

%%%%%%%%%%%%%%%%%%%%%%%%%%%%%%%%%%%%%%%%%%%%%%%%%%%%%%%%%%%%%%%%%%%%%%%%%%%%%%%%%%%%%%%%%%%%%%%%%%%

%\newtheorem{theorem}{Theorem}
%% \newtheorem{acknowledgement}[theorem]{Acknowledgement}
%% \newtheorem{algorithm}[theorem]{Algorithm}
%% \newtheorem{axiom}[theorem]{Axiom}
%% \newtheorem{case}[theorem]{Case}
%% \newtheorem{claim}[theorem]{Claim}
%% \newtheorem{conclusion}[theorem]{Conclusion}
%% \newtheorem{condition}[theorem]{Condition}
%% \newtheorem{conjecture}[theorem]{Conjecture}
%% \newtheorem{mycorollary}[theorem]{Corollary}
%% \newtheorem{mycriterion}[theorem]{Criterion}
%% \newtheorem{mydefinition}[theorem]{Definition}
%% \newtheorem{myexample}[theorem]{Example}
%% \newtheorem{myexercise}[theorem]{Exercise}
%% \newtheorem{mynotation}[theorem]{Notation}
%% \newtheorem{myproblem}[theorem]{Problem}
%% \newtheorem{myproposition}[theorem]{Proposition}
%% \newtheorem{myremark}[theorem]{Remark}
%% \newtheorem{mysolution}[theorem]{Solution}
%% \newtheorem{mysummary}[theorem]{Summary}
%% \newenvironment{myproof}[1][Proof]{\textbf{#1.} }{\ \rule{0.5em}{0.5em}}


\newcommand{\go}{\stackrel{\circ }{\mathfrak{g}}}
\newcommand{\ao}{\stackrel{\circ }{\mathfrak{a}}}
\newcommand{\co}[1]{\stackrel{\circ }{#1}}
\newcommand{\pia}{\pi_{\mathfrak{a}}}
\newcommand{\piab}{\pi_{\mathfrak{a}_{\bot}}}
\newcommand{\gf}{\mathfrak{g}}
\newcommand{\gfh}{\hat{\mathfrak{g}}}
\newcommand{\af}{\mathfrak{a}}
\newcommand{\afh}{\hat{\mathfrak{a}}}
\newcommand{\bff}{\mathfrak{b}}
\newcommand{\afb}{\mathfrak{a}_{\bot}}
\newcommand{\hf}{\mathfrak{h}}
\newcommand{\hfg}{\hf_{\gf}}
\newcommand{\hfb}{\mathfrak{h}_{\bot}}
\newcommand{\pf}{\mathfrak{p}}
\newcommand{\aft}{\widetilde{\mathfrak{a}}}
\newcommand{\sfr}{\mathfrak{s}}
%\pagestyle{plain}

\theoremstyle{definition} \newtheorem{Def}{Определение}
\newcommand{\tr}{\hat\triangleright} \newcommand{\trc}{\triangleright}
\newcommand{\adk}{a^{\dagger}_{\kappa}} \newcommand{\ak}{a_{\kappa}}
\def\bF{\mbox{$\overline{\cal F}$}} \def\F{\mbox{$\cal F$}}

\begin{document}
Здравствуйте! Меня зовут Антон Назаров, я из петербургского университета. Мой доклад называется ``Сингулярные элементы модулей аффинных алгебр Ли  в моделях конформной теории поля''. Он основан на тексте кандидатской диссертации. 
Сперва я расскажу про физическую мотивацию, напомню про WZNW и coset-модели, а потом буду говорить про наши результаты. У нас вышло несколько статей, которые доступны в виде препринтов. Что-то я уже рассказывал здесь летом на конференции SQS. 

Модели Весса-Зумино-Новикова-Виттена выделяются из числа моделей двумерной конформной теории поля тем, что у них есть привычная формулировка на языке действия. Действие в модели имеет вид суммы. Первый член -- это просто нелинейная сигма-модель. Поле g живет на комплексной плоскости с бесконечностью (два-сфере) и принимает значение в простой компактной группе Ли. 
В самой нелинейной сигма-модели конформная инвариантность теряется при квантовании. Но конформная инвариантность восстанавливается за счет второго члена Весса-Зумино. Это топологический член, который определен на трехмерном многообразии, границей которого является наша два-сфера. Он строится из полей g с тильдой -- продолжений поля g. Естественно, такое продолжение не единственно, но если мы возьмем два разных продолжения, то мы получим три-сферу. Так как третья группа гомологий любой компактной конечномерной группы Ли равна Z, при целых k экспонента действия определена однозначно. 

То есть у нас хорошая модель, в которой есть конформная инвариантность. Перейдем к комплексным переменным. У нас есть два сохраняющихся тока J и J бар. Заметим, что они принимают значения в алгебре Ли g, соответствующей группе G. Есть калибровочная инвариантность относительно таких преобразований. Соответствующие тождества Уорда имеют такой вид. Теперь если мы подставим сюда сами токи J и разложим их по генераторам и степеням z, то мы получим коммутационные соотношения аффинной алгебры g хат. 
Конформная инвариантность в двумерной теории поля связана с алгеброй Вирасоро. Здесь у нас алгебра Вирасоро реализуется с помощью конструкции Сугавары, то есть она вкладывается в универсальную обертывающую аффинной алгебры Ли.

Полная киральная алгебра -- это полупрямое произведение аффинной алгебры Ли и алгебры Вирасоро. Примарными по отношению к ней называются локальные поля, которые ведут себя как поле g и имеют операторное разложение такого вида. 
Примарные поля можно нумеровать старшими весами представлений g с крышкой. Их конформные веса даются следующим выражением. 
Состояния можно получать из соответствия между полями и состояниями. Тогда сингулярными векторами называются такие, что действие на них повышающих операторов равно нулю. Конечно, векторы старшего веса являются сингулярными, но не только они. Модули, которые получаются действием понижающих операторов, отщепляются. То есть корреляторы между с полями равны нулю. 
Теперь я напомню про coset-модели. Их можно формулировать как WZNW-модели с дополнительным чисто калибровочным полем, принимающим значение в (представлениях) подалгебры a алгебры g. Теперь у нас такие токи и для калибровочных полей выполняется такое равенство.  В результате интегрирования этого калибровочного поля мы избавляемся от степеней свободы, связанных с подалгеброй a. Здесь тоже есть конформная инвариантность и генераторы алгебры Вирасоро даются разностями выражений Сугавары соответствующих алгебрам g и a. 
Заметим, что генераторы алгебры a  с крышкой коммутируют с генераторами алгебры Вирасоро. Сразу можно заметить, что сингулярные по отношению к a с крышкой векторы образуют модули алгебры Вирасоро. Тогда функции ветвления оказываются характерами модулей алгебры Вирасоро. (Я здесь упростил некоторые детали).

В coset-моделях примарные поля нумеруются парами весов алгебр g и a, для которых функции ветвления не равны нулю. Некоторые пары эквивалентны. Соотношение эквивалентности дается умножением на так называемые простые токи, состоящие из g и a-компонент с равными конформными весами. Тогда конформный вес примарного поля дается разностью конформных весов для двух WZNW-моделей. 

Теперь перейдем к изучению структуры сингулярных элементов. Именно сингулярные элементы определяют свойства модулей аффинных алгебр Ли. 
Простейший тип модулей -- это модули Верма. Их можно представить как произведение универсальной обертывающей алгебры и одномерного представления подалгебры Бореля. Тогда формальный характер можно записать вот в таком виде. Здесь мы воспользовались тождеством знаменателей. Заметим, что знаменатель у нас всегда одинаковый, а в числителе как-раз и стоит сингулярный элемент. Неприводимый модуль можно получить из модуля Верма факторизацией по максимальному подмодулю. Известная формула Вейля-Каца дает нам следующее выражение для характеров неприводимых модулей. Как ее можно интерпретировать: в числителе сингулярный элемент, а знаменатель тот же самый. Для неприводимых модулей сингулярный элемент дается сдвинутой орбитой группы Вейля. Эту формулу можно рассматривать как выражение характера неприводимого модуля через характеры модулей Верма вследствие резольвенты Бернштейна-Гельфанда-Гельфанда. 

Теперь поговорим о разложениях сингулярных элементов. Рассмотрим аффинную или конечномерную алгебру g и ее редуктивную подалгебру a. Тогда мы можем разложить модули алгебры g на модули подалгебры a. Здесь я для определенности написал неприводимые модули, хотя мог бы рассматривать модули Верма. 
Для характеров я могу написать следующее разложение. Здесь пи - это проекция в алгебрах Картана, ее нужно добавить, если ранг a меньше ранга g. Проблема тут в том, чтобы найти коэффициенты этого разложения, которые называются коэффициентами ветвления. Можно считать в лоб - построить модуль g, потом последовательно строить вычитать модули a, но это неэффективно. 
Для этого мы хотим домножить на знаменатель, а потом ввести упорядочение. Но в случае немаксимальной подалгебры из-за проекции мы можем случайно домножить на ноль, поэтому нам нужен промежуточный шаг. Введем еще одну подалгебру b, которую мы также будем обозначать как a ортогональное. Эта подалгебра определяется набором корней g, ортогональных корневой системе a. Тогда у нас есть ``ортогональная пара подалгебр''. 
Еще введем такое обозначение для разностей векторов Вейля подалгебр и проекций вектора Вейля алгебры. 
Тогда верна следующая лемма о разложении сингулярного элемента. 
Здесь мы фактически построили модули Верма алгебры b из весов сингулярного элемента модуля g и собрали комбинации характеров модулей Верма в характеры неприводимых модулей. После проекции у нас от этих неприводимых модулей остаются только размерности. 
Теперь мы можем написать рекуррентное соотношение для сингулярных коэффициентов ветвления k. Эти сингулярные коэффициенты ветвления равны обычным внутри главной камеры Вейля. Рекурсия идет по множеству весов, которое получается из разложения знаменателя. В качестве начальных данных рекурсии выступают сингулярные веса с соответствующими размерностями. Заметим, что множество гамма нужно построить один раз для пары алгебр a и g.

Сейчас я покажу простой пример. Рассмотрим алгебру B2 (so(5)) и подалгебру A1 (so(3)), построенную на длинном корне. Вот сингулярный элемент модуля алгебры. Ортогональная подалгебра -- это тоже A1. Справа построены модули ортогональной подалгебры и указаны их размерности.
Дальше веер вложения строится из этих двух корней. После раскрытия скобок остается три элемента, выкидываем ноль, начинаем с этой двойки, получаем здесь два, тут один и все, вот оно, наше ветвление. 
Теперь аффинный пример. Возьмем аффинное расширение этого же примера. У нас получается вот такой веер и вот такой сингулярный элемент с соответствующими размерностями. В результате рекуррентного вычисления получается следующая картина. Смотрим в главную камеру Вейля, показанную черными линиями и видим наши коэффициенты ветвления. Можно их рассматривать как коэффициенты в разложении функций ветвления по степеням параметра q. 
Покажем связь нашего разложения с обобщенной резольвентой Бернштейна-Гельфанда-Гельфанда. Заметим, что ортогональная подалгебра по определению всегда регулярна. Можно считать, что она определяется подмножеством простых корней I. Возьмем сингулярный элемент и построим модули ортогональной подалгебры, аналогично прошлому примеру. Теперь подействуем на эти модули таким знаменателем. В результате мы получаем обобщенные или параболические модули Верма. 
Здесь я привел рисунок параболических модулей Верма, соответствующих нашему разложению модулей B2 на модули A1. Мы можем записать характер неприводимого модуля в виде чередующейся суммы характеров параболических модулей Верма. Здесь пунктиром я показал те модули, которые входят в сумму с положительным знаком, а точечками -- с отрицательным. 
Теперь я перейду к немного другому разложению сингулярных элементов, которое следует из разложения корневой системы, но не на ортогональные компоненты. Введем вложение корневых систем так, чтобы сохранялась сумма. Это вложение индуцирует вложение формальных алгебр. Заметим, что мы не требуем сохранения углов между корнями. 
Назовем сплинтом или расщеплением корневой системы ее разложение в объединение образов двух вложений, причем таких, что ранги прообразов не превышают ранга образа. Если одна из компонент расщепления является корневой системой подалгебры, то есть эта компонента вложена метрически, то вторая компонента определяет веер вложения. Веер вложения оказывается образом сингулярного элемента второй алгебры под действием phi. 
Оказывается, что в таком случае и сингулярный элемент модуля можно разложить в сумму образов сингулярных элементов модулей второй алгебры. При этом у этих модулей старший вес имеет те же индексы Дынкина. В результате коэффициенты ветвления модулей алгебры g на модули подалгебры a1 оказываются равны кратностям весов в модулях второй алгебры. При этом сами образы модулей оказываются деформированными, с неправильными углами. Посмотрите на картинку. В данном случае корневая система алгебры B2 раскладывается в объединение корневой системы подалгебры A1 и образа корневой системы алгебры A2 с ``неправильными'' углами. То есть вот такой модуль раскладывается с вот такими коэффициентами. Еще один красивый пример -- это алгебры G2 и разложение ее корневой системы на корневые системы двух алгебр A2. На рисунке пунктиром показаны соответствующие сингулярные элементы. Здесь -- коэффициенты ветвления. 
Есть классификация сплинтов, она не очень большая, но что интересно, включает случай, описывающийся базисом Гельфанда-Цетлина. 
Теперь рассмотрим аффинное расширение этой ситуации. Здесь у нас есть подобное разложение для знаменателей Вейля, но у нас разное количество чисто мнимых корней, поэтому полностью конечномерный случай не воспроизводится. Зато мы можем переписать разложение знаменателя в произведение в виде тождества, связывающего тета-функции, соответствующие трем разным алгебрам g, s и a. 
Рассмотрим ветвление модулей аффинной алгебры на модули конечномерной подалгебры. В этом случае мы можем завести функции ветвления. Коэффициенты ветвления для этого случая табулированы в книжке Касса, Муди и Патеры. 
Такие функции ветвления оказываются связаны с q-размерностью, то есть являются модулярными формами. 
Если мы введем порядок на множестве весов, то мы можем записать связь струнных функций с функциями ветвления в матричном виде. При этом матрица M содержит кратности весов в модулях g (по строчкам), а обратная матрица $M^{-1}$ -- рекуррентные соотношения, то есть коэффициенты веера $s(\Gamma)$. 
Теперь рассмотрим подалгебру a алгебры g и предположим, что есть сплинт, то есть корневая система g раскладывается в объединение корневой системы a и образа корневой системы s. 
Тогда мы можем записать характер через функции ветвление на модули a, а для них воспользоваться соотношением, следующим из существования сплинта. В результате мы связываем функции ветвления g с крышкой на g и g с крышкой на a. 

То есть мы продемонстрировали возникновение сингулярных элементов модулей аффинных алгебр Ли в coset-моделях конформной теории поля. Показали, что сингулярные элементы определяют структуру модуля, а их разложение позволяет вычислять функции ветвления. 
Кроме того, мы продемонстрировали связь разложения сингулярных элементов с обобщенной резольвентой Бернштейна-Гельфанда-Гельфанда. Затем мы показали, что в случае существования расщепления коэффициенты ветвления совпадают с кратностями весов в модуле алгебры, которая не является подалгеброй g. 
Еще мы обсудили следствия сплинта для аффинных алгебр Ли.
\end{document}
