\intro

%
% Используемые далее команды определяются в файле common.tex.
%

% Актуальность работы


%%Аффинные алгебры Ли были открыты Виктором Кацем \cite{kac1968simple} и Робертом Муди \cite{moody1968new} в 1967 году в результате отказа от требования положительной определенности матрицы Картана, определяющего полупростые конечномерные алгебры Ли. Аффинные алгебры Ли отличаются тем, что матрица Картана, задающая коммутационные соотношения, лишь положительно полуопределена. Такие алгебры могут быть реализованы как центральные расширения алгебр петель, связанных с полупростыми алгебрами Ли.  После появления в 1984 году конформной теории поля \cite{belavin1984ics}, аффинные алгебры Ли приобрели большое значение в физике, так как квантование теорий поля естественным образом приводит к центральным расширениям алгебр симметрии. Теория представлений аффинных алегбр Ли играет определяющую роль при изучении важных классов моделей конформной теории поля -- моделей Весса-Зумино-Новикова-Виттена и coset-моделей. 
%%
%%
% Методы, используемые для изучения аффинных алгебр Ли, тесно связаны с теорией представлений конечномерных алгебр Ли, широко используемой в различных разделах физики, в том числе в теории элементарных частиц. Большое число 


\actualitysection
\actualitytext

% Цель диссертационной работы
% \objectivesection
% \objectivetext

% Научная новизна
 \noveltysection
 \noveltytext

% Практическая ценность
% \valuesection
% \valuetext

% Результаты и положения, выносимые на защиту
\resultssection
\resultstext

% Апробация работы
\approbationsection
\approbationtext

% Публикации
\pubsection
\pubtext

% Личный вклад автора
% \contribsection
% \contribtext

% Структура и объем диссертации
\structsection
\structtext

% Краткое содержание работы
%\contentsection
%\contenttext

Глава \ref{cha:CFT} носит обзорный характер. В ней мы даем аксиоматическую формулировку конформной теории поля, описываем модели Весса-Зумино-Новикова-Виттена и coset-модели. Затем мы демонстрируем роль аффинных алгебр в описании этих моделей и приводим основные понятия теории представлений, использующиеся в диссертации. Кроме того, мы обсуждаем конформную теорию поля на области с границей, так как она оказывается связана со стохастическим описанием решеточных моделей. 

Основной проблемой данной диссертации является изучение редукции модулей аффинных и конечномерных алгебр Ли на модули подалгебр, вычисление коэффициентов ветвления. 

В главе \ref{cha:affine-lie-algebras} мы показываем, что структура сингулярного элемента определяет свойства модуля алгебры Ли, доказываем лемму о разложении сингулярного элемента и выводим основное рекуррунтное соотношение на коэффициенты ветвления. Основные результаты данной главы опубликованы в работе \citemy{2010arXiv1007.0318L}. 

В следующей главе \ref{cha:BGG} мы проясняем связь ветвления с (обобщенной) резольвентой Бернштейна-Гельфанда-Гельфанда. Результаты третьей главы опубликованы в работах \citemy{2011arXiv1102.1702L,2010LyakhovskyNazarovMQFT}

Глава \ref{cha:splints} посвящена сплинтам -- расщеплением корневой системы алгебры Ли в объединение образов корневых систем двух алгебр, не обязательно являющихся подалгебрами данной алгебры. Если одна из алгебр является подалгеброй, то сплинт приводит к резкому упрощению в вычислении коэффициентов ветвления -- они совпадают с кратностями весов в модуле другой алгебры. Основная часть главы посвящена доказательству этого факта. Кроме того, сплинт корневой системы простой конечномерной алгебры Ли приводит к возникновению новых соотношений на струнные функции и функции ветвления соответствующего аффинного расширения. Эти соотношения обсуждаются в разделе \ref{sec:splints-affine}. Данные результаты опубликованы в статьях \citemy{2011arXiv1111.6787L,2012arXiv1204.1855L}.

Заключительная глава \ref{cha:applications} посвящена практическим приложениям результатов диссертации. В разделе \ref{sec:SLE} мы  описываем применение алгебраических методов к проблеме поиска соответствия между квантовополевым и решеточным описанием критического поведения. Эти результаты были опубликованы нами в работах \citemy{NazarovJETPletters,2011arXiv1112.4354N}. Раздел \ref{cha:computational-methods} представляет собой описание пакета {\bf Affine.m}, предназначенного для вычислений в теории представлений аффинных и конечномерных алгебр Ли и реализованного с использованием методов диссертации. Вычислительным методам посвящены наши работы \citemy{2011arXiv1107.4681N,NazarovACSM2009,Nazarov2008}.

%%% Local Variables: 
%%% mode: latex
%%% TeX-master: "thesis"
%%% End: 
