\intro

%
% Используемые далее команды определяются в файле common.tex.
%

% Актуальность работы
\actualitysection
\actualitytext

% Цель диссертационной работы
\objectivesection

\objectivetext

% Научная новизна
\noveltysection
\noveltytext

% Практическая ценность
\valuesection
\valuetext

% Результаты и положения, выносимые на защиту
\resultssection
\resultstext

% Апробация работы
\approbationsection
\approbationtext

% Публикации
\pubsection
\pubtext

% Личный вклад автора
\contribsection
\contribtext

% Структура и объем диссертации
\structsection
\structtext
Диссертация состоит из трех глав. В первой главе рассматривается теория представлений аффинных алгебр Ли. Обсуждаются проблемы построения представлений и вычисления правил ветвлений представлений алгебры на представления подалгебры. В разделе \ref{sec:branching}  выводятся рекуррентные соотношения для коэффициентов ветвления, в разделе \ref{sec:bgg} обсуждается связь ветвлений представлений с обобщенной резольвентой Бернштейна-Гельфанда-Гельфанда. Во второй главе рассматриваются практические алгоритмы для вычисления кратностей и коэффициентов ветвления и описывается пакет {\bf Affine.m} для системы компьютерной алгебры {\it Mathematica}, разработанный с использованием этих алгоритмов. В третьей главе рассмотрены приложения к моделям конформной теории поля. 
%%% Local Variables: 
%%% mode: latex
%%% TeX-master: "thesis"
%%% End: 
