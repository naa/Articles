\intro

%
% Используемые далее команды определяются в файле common.tex.
%

% Актуальность работы


Аффинные алгебры Ли были открыты Виктором Кацем \cite{kac1968simple} и Робертом Муди \cite{moody1968new} в 1967 году в результате отказа от требования положительной определенности матрицы Картана. Это требование определяет полупростые конечномерные алгебры Ли. Аффинные алгебры Ли отличаются тем, что матрица Картана, задающая коммутационные соотношения, лишь положительно полуопределена. Оказывается такие алгебры могут быть реализованы как центральные расширения алгебр петель, связанных с полупростыми алгебрами Ли. Эта особенность структуры аффинных алгебр Ли привела к тому, что они приобрели особое значение в физике после появления в 1984 году конформной теории поля \cite{belavin1984ics}. Ведь квантование теорий поля естественным образом приводит к центральным расширениям алгебр симметрии. Теория представлений аффинных алегбр Ли играет определяющую роль при изучении важных классов моделей конформной теории поля -- моделей Весса-Зумино-Новикова-Виттена и coset-моделей. 


% Методы, используемые для изучения аффинных алгебр Ли, тесно связаны с теорией представлений конечномерных алгебр Ли, широко используемой в различных разделах физики, в том числе в теории элементарных частиц. Большое число 


%\actualitysection
%\actualitytext

% Цель диссертационной работы
% \objectivesection

% \objectivetext

% Научная новизна
% \noveltysection
% \noveltytext

% Практическая ценность
% \valuesection
% \valuetext

% Результаты и положения, выносимые на защиту
\resultssection
\resultstext

% Апробация работы
\approbationsection
\approbationtext

% Публикации
\pubsection
\pubtext

% Личный вклад автора
%\contribsection
%\contribtext

% Структура и объем диссертации
\structsection
\structtext
%%% Local Variables: 
%%% mode: latex
%%% TeX-master: "thesis"
%%% End: 
