\documentclass[12pt]{disser}
\usepackage[russian]{babel}
\usepackage[utf8]{inputenc}
\usepackage{amssymb}

%%%%%%%%%%%%%%%%%%%%%%%%%%%%%%%%%%%%%%%%%%%%%%%%%%%%%%%%%%%%%%%%%%%%%%%%%%%%%%%%%%%%%%%%%%%%%%%%%%%%
\usepackage{amsmath}
\usepackage{amsmath,amssymb,amsthm}
\usepackage{multicol}
\usepackage{color}
\usepackage{graphicx}
\usepackage{hyperref}

\newtheorem{statement}{Statement}
\newtheorem{theorem}{Theorem}
\newtheorem{corollary}{Corollary}[theorem]
\newtheorem{lemma}{Lemma}
\newtheorem{mynote}{Note}[section]
\theoremstyle{definition}
\newtheorem{definition}{Definition}
\newtheorem{remark}{Remark}
\newtheorem{example}{Example}
\newcommand{\go}{\stackrel{\circ }{\mathfrak{g}}}
\newcommand{\ao}{\stackrel{\circ }{\mathfrak{a}}}
\newcommand{\co}[1]{\stackrel{\circ }{#1}}
\newcommand{\pia}{\pi_{\mathfrak{a}}}
\newcommand{\piab}{\pi_{\mathfrak{a}_{\bot}}}
\newcommand{\gf}{\mathfrak{g}}
\newcommand{\af}{\mathfrak{a}}
\newcommand{\aft}{\widetilde{\mathfrak{a}}}
\newcommand{\afb}{\mathfrak{a}_{\bot}}
\newcommand{\hf}{\mathfrak{h}}
\newcommand{\hfb}{\mathfrak{h}_{\bot}}
\newcommand{\pf}{\mathfrak{p}}
%\input{tcilatex}

\begin{document}
\title{Бесконечномерные алгебры симметрии в моделях квантовой теории поля}
\author{А.А. Назаров$^{1,2}$\\
{\small $^1$ Theoretical Department, SPb State University}\\
{\small 198904, Sankt-Petersburg, Russia}\\
{\small$^{2}$ Chebyshev Laboratory,}\\
{\small Department of Mathematics and Mechanics, SPb State University}\\
{\small 199178, Saint-Petersburg, Russia}\\
{\small email: antonnaz@gmail.com}}

\maketitle

\begin{abstract}
\end{abstract}

\tableofcontents

\chapter{Введение}
\label{cha:intro}

\chapter{Аффинные алгебры Ли}
\label{cha:affine-lie-algebras}

\section{Конечномерные алгебры Ли}
\label{sec:finite-dimensional}
Основные определения: корни, веса, кратности.

\section{Теория представлений аффинных алгебр Ли}
\label{sec:representation-theory}
Определение аффинных алгебр. Корни, веса, кратности. Категория О.

\section{Ветвления}
\label{sec:branching}
Веер вложения. Алгоритм.

\section{БГГ-резольвента}
\label{sec:bgg}
Категория О.
Резольвента Бернштейна-Гельфанда-Гельфанда для представлений аффинных алгебр Ли. Параболические модули и их связь с подалгебрами и ветвлением.

\section{Проблемы вычислений}
\label{sec:computations}
Производительность, скорость алгоритмов. 

\chapter{Конформная теория поля}
\label{cha:cft}

Подробное изложение \cite{difrancesco1997cft}. 

\section{Общие свойства}
\label{sec:cft-general}
Алгебра Вирасоро. Представления алгебры Вирасоро. Модулярная инвариантность.

\section{Модели Весса-Зумино-Новикова-Виттена}
\label{sec:wzw}
Связь с аффинными алгебрами Ли.
Конформные вложения и недиагональные модулярные инварианты. 

\section{Coset-модели}
\label{sec:coset-models}
Ветвления. Функции ветвления и статсуммы. 
Калибровочные WZW-модели. 

\section{SLE}
\label{sec:sle}
SLE на WZW-моделях. Граничные условия. Парафермионы. SLE на coset-моделях.

\chapter{Заключение}
\label{cha:conclusion}


\bibliography{bibliography}{}
\bibliographystyle{utphys}

\end{document}
