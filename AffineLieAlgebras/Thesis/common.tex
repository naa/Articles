% Общие поля титульного листа диссертации и автореферата
\institution{Санкт-Петербургский Государственный Университет}

\topic{Правила ветвления аффинных алгебр Ли и приложения в моделях конформной теории поля}

\author{Назаров Антон Андреевич}

\specnum{01.04.02}
\spec{Теоретическая физика}

\sa{Ляховский Владимир Дмитриевич}
\sastatus{д.~ф.-м.~н., проф.}

\city{Санкт-Петербург}
\date{\number\year}

% Общие разделы автореферата и диссертации

\mkcommonsect{actuality}{Актуальность работы}{%
 Проблема вычисления коэффициентов ветвления для представлений алгебр Ли стоит уже многие десятилетия. Она актуальна для различных физических приложений. Вместе с тем, в отличие от кратностей весов не существует особенно эффективных алгоритмов.                                     
}

  
\mkcommonsect{objective}{Цель диссертационной работы}{%
 Разработка рекуррентного подхода к функциям ветвления аффинных алгебр Ли, его связь с проблемами теории представлений и его приложения в моделях конформной теории поля.
}
 
\mkcommonsect{novelty}{Научная новизна}{%
  Предложен эффективный алгоритм для вычисления коэффициентов ветвления, показана его связь с резольвентой Бернштейна-Гельфанда-Гельфанда.
}
 
\mkcommonsect{value}{Практическая значимость}{%
  Результаты работы                                  
}


\mkcommonsect{results}{%
На защиту выносятся следующие основные результаты и положения:}{%
\begin{itemize}
\item Продемонстрирована роль сингулярных элементов в описании структуры модулей аффинных алгебр Ли 
\item Из разложения сингулярных элементов получены новые рекуррентные соотношения на коэффициенты ветвления представлений аффинных алгебр Ли на представления произвольных редуктивных подалгебр
\item Показана связь процедуры редукции с обобщенной резольвентой Бернштейна-Гельфанда-Гельфанда
\item Выявлена связь расщепления корневой системы алгебры с разложением сингулярных элементов модулей алгебры в комбинацию сингулярных элементов модулей подалгебр
\item Показано, что наличие расщепления приводит к существенному упрощению при вычислении коэффициентов ветвления и ведет к новым соотношениям на функции ветвления
\item Предложено обобщение стохастического процесса Шрамма-Лёвнера на случай систем с калибровочной инвариантностью, соответствующих coset-моделям конформной теории поля
\item Продемонстрирована роль сингулярных элементов в построении мартингалов стохастического процесса Шрамма-Лёвнера, то есть проиллюстрировано применение алгебраических методов теории представлений аффинных алгебр Ли в изучении критического поведения в двумерных решеточных моделях
\item Реализованы различные алгоритмы для вычислений в теории представлений конечномерных и аффинных алгебр Ли
\end{itemize}

}

\mkcommonsect{approbation}{Апробация работы}{%
  Материалы диссертации докладывались на семинарах кафедры физики высоких энергий и элементарных частиц СПбГУ, на семинарах в лаборатории имени П.Л. Чебышева математико-механического факультета СПбГУ, на  международном семинаре молодых ученых ``Workshop on Advanced Computer Simulation Methods''  27 - 29 апреля 2009 (Санкт-Петербург),  на международных конференциях:  ``Модели квантовой теории поля (MQFT-2010)'' 18-22 октября 2010 (Санкт-Петербург), ``Supersymmetries and Quantum Symmetries - 2011'', 18-23 июля 2011 (Дубна), ``Quantum Theory and Symmetries (QTS-7)'', 7-13 августа 2011 (Прага). 
}

\mkcommonsect{pub}{Публикации.}{%
Материалы диссертации опубликованы в $10$ печатных работах, из них $4$ статьи в
рецензируемых журналах~\citemy{2010arXiv1007.0318L,2011arXiv1102.1702L,2011arXiv1111.6787L,NazarovJETPletters}, $5$ статей в
сборниках тезисов и трудов конференций \cite{2011arXiv1112.4354N,2010LyakhovskyNazarovMQFT,Nazarov2008,NazarovACSM2009,2012arXiv1204.1855L}, и в препринте \cite{2011arXiv1107.4681N}.
}

\mkcommonsect{contrib}{Личный вклад автора}{%
  
}

\mkcommonsect{struct}{Структура и объем диссертации}{%
Диссертация состоит из шести глав. 

Глава \ref{cha:CFT} является вводной. В ней мы даем аксиоматическую формулировку конформной теории поля, описываем модели Весса-Зумино-Новикова-Виттена и coset-модели. Затем мы демонстрируем роль аффинных алгебр в описании этих моделей и приводим основные понятия теории представлений, использующиеся в диссертации. Кроме того, мы обсуждаем конформную теорию поля на области с границей, так как она оказывается связана со стохастическим описанием решеточных моделей. 

Основной проблемой данной диссертации является изучение редукции модулей аффинных и конечномерных алгебр Ли на модули подалгебр, вычисление коэффициентов ветвления. В главе \ref{cha:affine-lie-algebras} мы вводим основные понятия теории представлений аффинных алгебр Ли и выводим основное рекуррунтное соотношение на коэффициенты ветвления.

В следующей главе \ref{cha:BGG} мы проясняем связь ветвления с (обобщенной) резольвентой Бернштейна-Гельфанда-Гельфанда. 

Глава \ref{cha:splints} посвящена сплинтам -- расщеплением корневой системы алгебры Ли в объединение образов корневых систем двух алгебр, не обязательно являющихся подалгебрами данной алгебры. Если одна из алгебр является подалгеброй, то сплинт приводит к резкому упрощению в вычислении коэффициентов ветвления -- они совпадают с кратностями весов в модуле другой алгебры. Основная часть главы посвящена доказательству этого факта. Кроме того, сплинт корневой системы простой конечномерной алгебры Ли приводит к возникновению новых соотношений на струнные функции и функции ветвления соответствующего аффинного расширения. Эти соотношения обсуждаются в разделе 

Заключительная глава \ref{cha:applications} посвящена практическим приложениям результатов диссертации. В разделе \ref{sec:SLE} мы  описываем применение алгебраических методов к проблеме поиска соответствия между квантовополевым и решеточным описанием критического поведения. Раздел \ref{cha:computational-methods} представляет собой описание пакета {\bf Affine.m}, предназначенного для вычислений в теории представлений аффинных и конечномерных алгебр Ли и реализованного с использованием методов диссертации.

}

%%
%% End of file
%%% Local Variables: 
%%% mode: latex
%%% TeX-master: "thesis"
%%% End: 
