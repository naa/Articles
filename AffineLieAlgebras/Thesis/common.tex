% Общие поля титульного листа диссертации и автореферата
\institution{Санкт-Петербургский Государственный Университет}

\topic{Правила ветвления аффинных алгебр Ли и приложения в моделях конформной теории поля}

\author{Назаров Антон Андреевич}

\specnum{01.04.02}
\spec{Теоретическая физика}

\sa{Ляховский Владимир Дмитриевич}
\sastatus{д.~ф.-м.~н., проф.}

\city{Санкт-Петербург}
\date{\number\year}

% Общие разделы автореферата и диссертации

\mkcommonsect{actuality}{Актуальность работы}{%

  Последние тридцать лет конформная теория поля в двух измерениях привлекает большое внимание исследователей \cite{belavin1984ics}. Эта теория используется для описания критического поведения в двумерных статистических системах. Благодаря наличию бесконечномерной алгебры симметрии двумерная конформная теория поля может быть сформулирована аксиоматически. Помимо математической красоты теория обладает огромной практической ценностью -- с ее использованием было получено большое количество результатов и численных предсказаний в изучении критического поведения в двумерных системах \cite{difrancesco1997cft,henkel1999conformal}. Методы двумерной конформной теории поля с успехом применяются также при изучении эффекта Кондо \cite{cox1998exotic,affleck1993exact} и дробного квантового эффекта Холла \cite{moore1991nonabelions}. 

Поиски строгого математического доказательства для предсказаний двумерной конформной теории поля \cite{cardy1992critical} в последние годы привели к большому количеству новых идей и результатов в дискретном комплексном анализе \cite{smirnov2001critical,duminil2011conformal,smirnov2010discrete}.

Теория представлений бесконечномерных алгебр Ли является важным инструментом изучения моделей конформной теории поля. Помимо алгебры Вирасоро, наличие которой обязательно в двумерной конформной теории поля, большую роль играют аффинные алгебры Ли. Изучение аффинных алгебр Ли было начато Виктором Кацем и Робертом Муди в 1960-х годах с попытки обобщения классификации простых конечномерных алгебр Ли на бесконечномерный случай \cite{kac1968simple,moody1968new}. Первоначально интерес к этим алгебрам был связан с модулярными свойствами характеров их модулей \cite{kac1984infinite,macdonald1971affine}. После возникновения двумерной конформной теории поля были предложены модели Весса-Зумино-Новикова-Виттена \cite{witten1984nab}, а затем и coset-модели \cite{Goddard198588}, в которых теория представлений аффинных алгебр Ли играет определяющую роль. 

Моделям Весса-Зумино-Новикова-Виттена и coset-моделям  посвящены тысячи работ. Большой интерес вызывает и теория представлений аффинных алгебр Ли. Вместе с тем, в теории представлений многие вопросы по-прежнему не имеют простых решений. Например, проблема вычисления коэффициентов ветвления для представлений алгебр Ли стоит уже многие десятилетия. Она актуальна для различных физических приложений в coset-моделях конформной теории поля. Вместе с тем, в отличие от проблемы вычисления кратностей весов, для вычисления коэффициентов ветвления не существовало особенно эффективных алгоритмов. 
}

  
\mkcommonsect{objective}{Цели и задачи работы}{%

 Разработка рекуррентного подхода к функциям ветвления аффинных алгебр Ли, его связь с проблемами теории представлений и его приложения в моделях конформной теории поля.
}
 
\mkcommonsect{novelty}{Научная новизна и практическая значимость}{%

  В диссертации впервые решены следующие задачи:
  \begin{itemize}
  \item Получено эффективное рекуррентное соотношение для коэффициентов ветвления модулей аффинных и конечномерных алгебр Ли на модули не максимальных подалгебр.
  \item Показано, что наличие расщепления приводит к существенному упрощению при вычислении коэффициентов ветвления.
  \item Показана связь процедуры редукции с обобщенной резольвентой Бернштейна-Гельфанда-Гельфанда.
  \item Предложено обобщение стохастического процесса Шрамма-Лёвнера на случай систем с калибровочной инвариантностью, соответствующих coset-моделям конформной теории поля.
  \end{itemize}
  Алгоритм вычисления коэффициентов ветвления реализован в пакете  {\bf Affine.m} для популярной системы компьютерной алгебры {\it Mathematica}. Пакет {\bf Affine.m} может быть использован для решения задач теории представлений конечномерных и аффинных алгебр Ли, возникающих в различных областях физики, начиная от изучения атомных и молекулярных спектров и заканчивая конформной теорией поля и интегрируемыми системами.
}
 
\mkcommonsect{value}{Практическая значимость}{%
}

\mkcommonsect{results}{%

На защиту выносятся следующие основные результаты и положения:}{%
\begin{itemize}
\item Продемонстрирована роль сингулярных элементов в описании структуры модулей аффинных алгебр Ли 
\item Из разложения сингулярных элементов получены новые рекуррентные соотношения на коэффициенты ветвления представлений аффинных алгебр Ли на представления произвольных редуктивных подалгебр
\item Показана связь процедуры редукции с обобщенной резольвентой Бернштейна-Гельфанда-Гельфанда
\item Выявлена связь расщепления корневой системы алгебры с разложением сингулярных элементов модулей алгебры в комбинацию сингулярных элементов модулей подалгебр
\item Показано, что наличие расщепления приводит к существенному упрощению при вычислении коэффициентов ветвления и ведет к новым соотношениям на функции ветвления
\item Предложено обобщение стохастического процесса Шрамма-Лёвнера на случай систем с калибровочной инвариантностью, соответствующих coset-моделям конформной теории поля
%\item Продемонстрирована роль сингулярных элементов в построении мартингалов стохастического процесса Шрамма-Лёвнера, то есть проиллюстрировано применение алгебраических методов теории представлений аффинных алгебр Ли в изучении критического поведения в двумерных решеточных моделях
\item Разработан пакет программ {\bf Affine.m}, реализующий различные алгоритмы для вычислений в теории представлений конечномерных и аффинных алгебр Ли
\end{itemize}

}

\mkcommonsect{approbation}{Апробация работы}{%

  Материалы диссертации докладывались на семинарах кафедры физики высоких энергий и элементарных частиц СПбГУ, на семинарах в лаборатории имени П.Л. Чебышева математико-механического факультета СПбГУ, на  международном семинаре молодых ученых ``Workshop on Advanced Computer Simulation Methods''  27 - 29 апреля 2009 (Санкт-Петербург),  на международных конференциях:  ``Модели квантовой теории поля (MQFT-2010)'' 18-22 октября 2010 (Санкт-Петербург), ``Supersymmetries and Quantum Symmetries - 2011'', 18-23 июля 2011 (Дубна), ``Quantum Theory and Symmetries (QTS-7)'', 7-13 августа 2011 (Прага). 
}

\mkcommonsect{pub}{Публикации.}{%

Материалы диссертации опубликованы в $10$ печатных работах, из них $5$ статей в
рецензируемых журналах~\citemy{2010arXiv1007.0318L,2011arXiv1102.1702L,2011arXiv1111.6787L,NazarovJETPletters,2011arXiv1107.4681N}, $5$ статей в
сборниках тезисов и трудов конференций \citemy{2011arXiv1112.4354N,2010LyakhovskyNazarovMQFT,Nazarov2008,NazarovACSM2009,2012arXiv1204.1855L}.
}

\mkcommonsect{contrib}{Личный вклад автора}{%
  
}

\mkcommonsect{struct}{Структура и объем диссертации}{%

Диссертация состоит из введения и шести глав, содержит 157 страниц и 30 рисунков. Список литературы включает 151 наименование. 

}

%%  \mkcommonsect{content}{Содержание работы}{%
%%  
%%  }

%%
%% End of file
%%% Local Variables: 
%%% mode: latex
%%% TeX-master: "thesis"
%%% End: 
