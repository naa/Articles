% Общие поля титульного листа диссертации и автореферата
\institution{Санкт-Петербургский Государственный Университет}

\topic{Бесконечномерные алгебры симметрии в моделях квантовой теории поля}

\author{Назаров Антон Андреевич}

\specnum{01.04.02}
\spec{Теоретическая физика}

\sa{Ляховский Владимир Дмитриевич}
\sastatus{д.~ф.-м.~н., проф.}

\city{Санкт-Петербург}
\date{\number\year}

% Общие разделы автореферата и диссертации
\mkcommonsect{actuality}{Актуальность работы}{%
Проблема вычисления коэффициентов ветвления для представлений алгебр Ли стоит уже многие десятилетия. Она актуальна для различных физических приложений. Вместе с тем, в отличие от кратностей весов не существует особенно эффективных алгоритмов.                                     
}

\mkcommonsect{objective}{Цель диссертационной работы}{%
Разработка рекуррентного подхода к функциям ветвления аффинных алгебр Ли, его связь с проблемами теории представлений и его приложения в моделях конформной теории поля.
}

\mkcommonsect{novelty}{Научная новизна}{%
Предложен эффективный алгоритм для вычисления коэффициентов ветвления, показана его связь с резольвентой Бернштейна-Гельфанда-Гельфанда.
}

\mkcommonsect{value}{Практическая значимость}{%
Результаты работы                                  
}

\mkcommonsect{results}{%
На защиту выносятся следующие основные результаты и положения:}{%
Текст раздела
}

\mkcommonsect{approbation}{Апробация работы}{%
Текст раздела
}

\mkcommonsect{pub}{Публикации.}{%
Материалы диссертации опубликованы в $N$ печатных работах, из них $n_1$ статей в
рецензируемых журналах~\citemy{Ivanov_1999_Journal_17_173,
Petrov_2001_Journal_23_12321,Sidorov_2002_Journal_32_1531}, $n_2$ статей в
сборниках трудов конференций и $n_3$ тезисов докладов.
}

\mkcommonsect{contrib}{Личный вклад автора}{%
Текст раздела
}

\mkcommonsect{struct}{Структура и объем диссертации}{%
Текст раздела
}
