% Общие поля титульного листа диссертации и автореферата
\institution{Санкт-Петербургский Государственный Университет}

\topic{Бесконечномерные алгебры симметрии в моделях квантовой теории поля}

\author{Назаров Антон Андреевич}

\specnum{01.04.02}
\spec{Теоретическая физика}

\sa{Ляховский Владимир Дмитриевич}
\sastatus{д.~ф.-м.~н., проф.}

\city{Санкт-Петербург}
\date{\number\year}

% Общие разделы автореферата и диссертации

\mkcommonsect{actuality}{Актуальность работы}{%
 Проблема вычисления коэффициентов ветвления для представлений алгебр Ли стоит уже многие десятилетия. Она актуальна для различных физических приложений. Вместе с тем, в отличие от кратностей весов не существует особенно эффективных алгоритмов.                                     
}

  
\mkcommonsect{objective}{Цель диссертационной работы}{%
 Разработка рекуррентного подхода к функциям ветвления аффинных алгебр Ли, его связь с проблемами теории представлений и его приложения в моделях конформной теории поля.
}
 
\mkcommonsect{novelty}{Научная новизна}{%
  Предложен эффективный алгоритм для вычисления коэффициентов ветвления, показана его связь с резольвентой Бернштейна-Гельфанда-Гельфанда.
}
 
\mkcommonsect{value}{Практическая значимость}{%
  Результаты работы                                  
}


\mkcommonsect{results}{%
На защиту выносятся следующие основные результаты и положения:}{%
\begin{itemize}
\item Получены новые рекуррентные соотношения на коэффициенты ветвления представлений аффинных алгебр Ли на представления произвольных редуктивных подалгебр
\item Показана связь процедуры редукции с обобщенной резольвентой Бернштейна-Гельфанда-Гельфанда
\item Показано, что наличие сплинта приводит к существенному упрощению при вычислении коэффициентов ветвления и ведет к новым соотношениям на функции ветвления
\item Продемонстрирована роль алгебраических методов теории представлений аффинных алгебр Ли в изучении критического поведения в двумерных решеточных моделях
\item Реализованы различные алгоритмы для вычислений в теории представлений конечномерных и аффинных алгебр Ли
\end{itemize}

}

\mkcommonsect{approbation}{Апробация работы}{%
  Материалы диссертации докладывались на семинарах кафедры физики высоких энергий и элементарных частиц СПбГУ, на семинарах в лаборатории имени П.Л. Чебышева математико-механического факультета СПбГУ, на  международном семинаре молодых ученых ``Workshop on Advanced Computer Simulation Methods''  27 - 29 апреля 2009 (Санкт-Петербург),  на международных конференциях:  ``Модели квантовой теории поля (MQFT-2010)'' 18-22 октября 2010 (Санкт-Петербург), ``Supersymmetries and Quantum Symmetries - 2011'', 18-23 июля 2011 (Дубна), ``Quantum Theory and Symmetries (QTS-7)'', 7-13 августа 2011 (Прага). 
}

\mkcommonsect{pub}{Публикации.}{%
Материалы диссертации опубликованы в $8$ печатных работах, из них $3$ статьи в
рецензируемых журналах~\citemy{2010arXiv1007.0318L,2010LyakhovskyNazarovTMF}, $2$ статьи в
сборниках трудов конференций, 2 препринта и $1$ тезисы доклада.
}

\mkcommonsect{contrib}{Личный вклад автора}{%
Текст раздела
}

\mkcommonsect{struct}{Структура и объем диссертации}{%
Диссертация состоит из шести глав. 

Глава \ref{cha:CFT} является вводной. В ней мы даем обзор конформной теории поля, моделей Весса-Зумино-Новикова-Виттена и coset-моделей. Мы приводим аксиоматическую формулировку конформной теории поля, а также обсуждаем БРСТ-подход к описанию калибровочных моделей Весса-Зумино-Новикова-Виттена, приводящий к реализации coset-конструкции. 

Основной проблемой данной диссертации является изучение редукции модулей аффинных и конечномерных алгебр Ли на модули подалгебр, вычисление коэффициентов ветвления. В главе \ref{cha:affine-lie-algebras} мы вводим основные понятия теории представлений аффинных алгебр Ли и выводим основное рекуррунтное соотношение на коэффициенты ветвления.

В следующей главе \ref{cha:BGG} мы проясняем связь ветвления с (обобщенной) резольвентой Бернштейна-Гельфанда-Гельфанда. 

Глава \ref{cha:splints} посвящена сплинтам. 

В главе \ref{sec:SLE} мы описываем применение алгебраических методов к проблеме поиска соответствия между квантовополевым и решеточным описанием критического поведения. 

Заключительная глава \ref{cha:computational-methods} представляет собой описание пакета {\bf Affine.m}, предназначенного для вычислений в теории представлений аффинных и конечномерных алгебр Ли.

}

%%
%% End of file
%%% Local Variables: 
%%% mode: latex
%%% TeX-master: "thesis"
%%% End: 
