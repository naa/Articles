% Общие поля титульного листа диссертации и автореферата
\institution{Санкт-Петербургский Государственный Университет}

\topic{Правила ветвления аффинных алгебр Ли\\ и приложения в моделях конформной теории поля}

\author{Назаров Антон Андреевич}

\specnum{01.04.02}
\spec{Теоретическая физика}

\sa{Ляховский Владимир Дмитриевич}
\sastatus{д.~ф.-м.~н., проф.}

\city{Санкт-Петербург}
\date{\number\year}

% Общие разделы автореферата и диссертации

\mkcommonsect{actuality}{Актуальность работы}{%

  Последние тридцать лет конформная теория поля в двух измерениях привлекает большое внимание исследователей. Эта теория используется для описания критического поведения в двумерных статистических системах. Благодаря наличию бесконечномерной алгебры симметрии двумерная конформная теория поля может быть сформулирована аксиоматически. Помимо математической красоты теория обладает огромной практической ценностью -- с ее использованием было получено большое количество результатов и численных предсказаний в изучении критического поведения в двумерных системах \cite{difrancesco1997cft,henkel1999conformal}. Методы двумерной конформной теории поля с успехом применяются также при изучении эффекта Кондо \cite{cox1998exotic,affleck1993exact} и дробного квантового эффекта Холла \cite{moore1991nonabelions}. 

Поиски строгого математического доказательства для предсказаний двумерной конформной теории поля \cite{cardy1992critical} в последние годы привели к большому количеству новых идей и результатов в дискретном комплексном анализе \cite{smirnov2001critical,duminil2011conformal,smirnov2010discrete}.

Теория представлений бесконечномерных алгебр Ли является важным инструментом изучения моделей конформной теории поля. Помимо алгебры Вирасоро, наличие которой обязательно в двумерной конформной теории поля, большую роль играют аффинные алгебры Ли. Изучение аффинных алгебр Ли было начато Виктором Кацем и Робертом Муди в 1960-х годах с попытки обобщения классификации простых конечномерных алгебр Ли на бесконечномерный случай \cite{kac1968simple,moody1968new}. Первоначально интерес к этим алгебрам был связан с модулярными свойствами характеров их модулей \cite{kac1984infinite,macdonald1971affine}. После возникновения двумерной конформной теории поля были предложены модели Весса-Зумино-Новикова-Виттена \cite{witten1984nab}, а затем и coset-модели \cite{Goddard198588}, в которых теория представлений аффинных алгебр Ли играет определяющую роль. 

Моделям Весса-Зумино-Новикова-Виттена, coset-моделям и теории представлений аффинных алгебр Ли  посвящены тысячи работ. Однако многие проблемы по-прежнему не имеют простых решений. Например, задача вычисления коэффициентов ветвления для представлений алгебр Ли стоит уже многие десятилетия. Она актуальна для различных физических приложений в coset-моделях конформной теории поля. При этом, в отличие от проблемы вычисления кратностей весов, для вычисления коэффициентов ветвления не существовало особенно эффективных алгоритмов. 
}

  
\mkcommonsect{objective}{Цели и задачи работы}{%
 Разработка рекуррентного подхода к функциям ветвления аффинных алгебр Ли, его связь с проблемами теории представлений и его приложения в моделях конформной теории поля.
}
 
\mkcommonsect{novelty}{Научная новизна и практическая значимость.}{%
  В диссертации впервые решены следующие задачи:
  \begin{itemize}[topsep=0pt, itemsep=-1ex]
  \item Получено эффективное рекуррентное соотношение для коэффициентов ветвления модулей аффинных и конечномерных алгебр Ли на модули не максимальных подалгебр.   Алгоритм вычисления коэффициентов ветвления реализован в пакете  {\bf Affine.m} для популярной системы компьютерной алгебры {\it Mathematica}.
  \item Установлена прямая связь инъективного сплинта и ветвлений. Доказано, что при определенных условиях  кратности весов вспомогательного модуля иньективного сплинта совпадают с коэффициентами ветвления в редукции на вложенную подалгебру. Наличие расщепления приводит к существенному упрощению при вычислении коэффициентов ветвления. 
%  \item Показано, что наличие расщепления приводит к существенному упрощению при вычислении коэффициентов ветвления.
  \item Исследована связь процедуры редукции с обобщенной резольвентой Бернштейна-Гельфанда-Гельфанда (БГГ).  Показано, что разложение сингулярного элемента определяет как коэффициенты ветвления, так и обобщенную БГГ-резольвенту, так как действие веера вложения на компоненты разложения порождает обобщенные модули Верма, которые образуют точную последовательность.
  \item Построена модель обобщенного стохастического процесса Шрамма-Лёвнера для систем с калибровочной инвариантностью, соответствующих coset-моделям конформной теории поля.
  \end{itemize}
Отметим, что  пакет {\bf Affine.m} может быть использован для решения задач теории представлений конечномерных и аффинных алгебр Ли, возникающих в различных областях физики, начиная от изучения атомных и молекулярных спектров и заканчивая конформной теорией поля и интегрируемыми системами.
}
 
%\mkcommonsect{value}{Практическая значимость}

\mkcommonsect{results}{На защиту выносятся следующие результаты и положения:}{%
\begin{itemize}[topsep=0pt, itemsep=-1ex]
\item Получены новые рекуррентные соотношения на коэффициенты ветвления представлений аффинных алгебр Ли на представления произвольных редуктивных подалгебр, с использованием разложения сингулярных элементов
\item Установлено, что разложение сингулярного элемента определяет как коэффициенты ветвления, так и обобщенную БГГ-резольвенту, так как действие веера вложения на компоненты разложения порождает обобщенные модули Верма, которые образуют точную последовательность
\item  Доказано, что при определенных условиях  кратности весов вспомогательного модуля иньективного сплинта совпадают с коэффициентами ветвления в редукции на вложенную подалгебру. Наличие расщепления приводит к существенному упрощению при вычислении коэффициентов ветвления.
\item Показано, что условие для мартингала, определяющее классификацию операторов изменения граничных условий в наблюдаемых стохастического процесса Шрамма-Лёвнера, задает ограничения на структуру сингулярных элементов представлений аффинной алгебры Ли, порожденных граничными состояниями. Изучение структуры сингулярных элементов существенно упрощает поиск операторов смены граничных условий. Построена модель обобщеннного стохастического процесса Шрамма-Лёвнера для систем с калибровочной инвариантностью, соответствующих coset-моделям конформной теории поля и показано, что такое обобщение совместно с coset-реализацией минимальных моделей.
%\item Продемонстрирована роль сингулярных элементов в построении мартингалов стохастического процесса Шрамма-Лёвнера, то есть проиллюстрировано применение алгебраических методов теории представлений аффинных алгебр Ли в изучении критического поведения в двумерных решеточных моделях
\item Разработан пакет программ {\bf Affine.m}, реализующий различные алгоритмы для вычислений в теории представлений конечномерных и аффинных алгебр Ли
\end{itemize}

}

\mkcommonsect{approbation}{Апробация работы}{%
  Материалы диссертации докладывались на трех международных конференциях, а также на семинарах кафедры физики высоких энергий и элементарных частиц СПбГУ, на семинарах в лаборатории имени П.Л. Чебышева математико-механического факультета СПбГУ, на семинаре лаборатории теоретической физики ОИЯИ (Дубна).

%  международном семинаре молодых ученых ``Workshop on Advanced Computer Simulation Methods''  27 - 29 апреля 2009 
%(Санкт-Петербург),  на международных конференциях:  ``Модели квантовой теории поля (MQFT-2010)'' 18-22 октября 2010 
%(Санкт-Петербург), ``Supersymmetries and Quantum Symmetries - 2011'', 18-23 июля 2011 (Дубна), ``Quantum Theory and 
%Symmetries (QTS-7)'', 7-13 августа 2011 (Прага). 
}

\mkcommonsect{pub}{Публикации.}{%
Материалы диссертации опубликованы в $10$ печатных работах, из них $5$ статей в
рецензируемых журналах~\citemy{2010arXiv1007.0318L,2011arXiv1102.1702L,2011arXiv1111.6787L,NazarovJETPletters,2011arXiv1107.4681N}, $5$ статей в
сборниках тезисов и трудов конференций \citemy{2011arXiv1112.4354N,2010LyakhovskyNazarovMQFT,Nazarov2008,NazarovACSM2009,2012arXiv1204.1855L}.
}

\mkcommonsect{contrib}{Личный вклад автора.}{%
  Все основные результаты и выносимые на защиту положения получены автором самостоятельно. Личный вклад автора в работы с соавтором составляет $50$ процетнов, в работы без соавторов -- $100$ процентов. 
}

\mkcommonsect{struct}{Структура и объем диссертации}{%
Диссертация состоит из введения и шести глав, содержит 160 страниц и 30 рисунков. Список литературы включает 151 наименование. 

}

%%  \mkcommonsect{content}{Содержание работы}{%
%%  
%%  }

%%
%% End of file
%%% Local Variables: 
%%% mode: latex
%%% TeX-master: "thesis"
%%% End: 
