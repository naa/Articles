\conclusion
\label{cha:conclusion}


В данной диссертации были решены следующие задачи, возникающие в конформной теории поля и при изучении критических явлений. 

Во-первых, мы рассмотрели проблему разложения модуля аффинной алгебры Ли в прямую сумму модулей редуктивной подалгебры. Данная задача возникает в конформной теории поля при построении модулярно-инвариантных статсумм в ВЗНВ-моделях методом конформных вложений и статсумм в coset-моделях рациональной конформной теории поля. Мы показали, что техника веера вложения может использоваться для работы с произвольными редуктивными подалгебрами, как максимальными, так и не максимальными. Использование данной техники и разложение сингулярного элемента позволили нам вывести рекуррентные соотношения на коэффициенты ветвления, которые можно решать последовательно. Эти соотношения позволяют эффективно вычислять коэффициенты ветвления, что было продемонстрировано на различных примерах.

Во-вторых  мы исследовали связь ветвления и обобщенной резольвенты Бернштейна-Гельфанда-Гельфанда. Мы показали, что действие веера вложения на компоненты разложения сингулярного элемента порождает обобщенные модули Верма. Эти модули образуют точную последовательность обобщенной резольвенты Бернштейна-Гельфанда-Гельфанда. Характер исходного неприводимого модуля выражается через характеры обобщенных модулей Верма посредством формулы Вейля-Верма. Коэффициенты в этой формуле равняются коэффициентам разложения сингулярного элемента. 
Таким образом, разложение сингулярного элемента определяет как коэффициенты ветвления, так и обобщенную резольвенту Бернштейна-Гельфанда-Гельфанда. 
Кроме того, мы показали, что в случае когда ортогональный партнер подалгебры является алгеброй $A_1$, коэффициенты ветвления  определяют обобщенную резольвенту Бернштейна-Гельфанда-Гельфанда. 

Также нами была установлена связь инъективного сплинта и ветвлений. Для сплинта $\phi:\Delta^+_{\af}\cup \Delta^+_{\sfr}\to \Delta^+_{\gf}$ доказано, что при условии включения образа  $\phi(\Phi_{\sfr}^{(\tilde \mu)})$ сингулярного элемента вспомогательного модуля алгебры $\sfr$ в главную камеру Вейля  $\bar C_{\af}$ подалгебры $\af$ кратность веса вспомогательного модуля определяет коэффициент ветвления $b_{\nu}^{(\mu)}$.  Это свойство сплинта представляет очень эффективный инструмент для вычисления коэффициентов ветвления, так как определение правил ветвления модулей старшего веса сводится к вычислению кратностей весов для модуля с теми же индексами Дынкина алгебры Ли $\mathfrak{s}$. В этом случае для вычислений коэффициентов ветвления можно применять не только общий алгоритм (раздел \label{sec:algorithm}), но и формулу Фрейденталя, позволяющую быстро вычислять кратности весов. 
Для вложений  $D_{r}\hookrightarrow B_{r}$ , $D_{r}\hookrightarrow C_{r}$ и $A_{r-1}\oplus u\left( 1\right) \hookrightarrow A_{r}$ существование сплинта ведет к появлению правил ветвления Гельфанда-Цейтлина: редукция свободна от множественности (все ненулевые коэффициенты ветвления равны 1). Кроме того, мы продемонстрировали, что сплинт для аффинных алгебр Ли ведет к новым соотношениям на тета-функции и функции ветвления для редукции на модули конечномерных подалгебр. 

Наконец, мы исследовали классификацию операторов изменения граничных условий в моделях Весса-Зумино-Новикова-Виттена, соответствующих стохастическим моделям с эволюцией Шрамма-Левнера с дополнительным блужданием на группе Ли $G$. Было показано, что условие для мартингала, определяющее классификацию операторов изменения граничных условий, задает ограничения на структуру сингулярных элементов представлений аффинной алгебры Ли, порожденных граничными состояниями. Изучение структуры сингулярных элементов существенно упрощает поиск операторов смены граничных условий, так как отсекается большое количество лишних вариантов. Далее мы обобщили эволюцию Шрамма-Левнера путем введения дополнительного случайного блуждания на факторпространстве $G/A$ и показали, что такое обобщение эволюции Шрамма-Левнера совместно с coset-реализацией минимальных моделей конформной теории поля. 

Полученные рекуррентные соотношения на коэффициенты ветвления и свойства сплинта были нами использованы при реализации пакета программ {\bf Affine.m}  для вычислений в теории представлений конечномерных и аффинных алгебр Ли. Этот пакет он может использоваться для изучения групп Вейля, корневых систем, неприводимых модулей, модулей Верма и параболических модулей Верма конечномерных и аффинных алгебр Ли. Большое число иллюстраций и вычислений в данной работе выполнены с использованием пакета  {\bf Affine.m}. Однако применения пакета не ограничиваются задачами, возникающими в конформной теории поля, он может быть полезен и при изучении интегрируемых систем, исследовании атомных и молекулярных спектров и в других областях физики, где используется теория представлений конечномерных и аффинных алгебр Ли. 

Теоретические идеи и методы данной работы могут быть в дальнейшем использованы при изучении представлений более широкого класса алгебр -- расширенных алгебр Каца-Муди. Кроме того, в теории критического поведения большой интерес вызывает задача описания нарушения конформной инвариантности при выходе из критической точки. При этом возникают квантовые теории поля типа аффинной теории Тоды, в которых алгебраическая структура частично сохраняется и теория представлений аффинных алгебр Ли остается важным инструментом. Кроме того, может быть расширена и область применения пакета {\bf Affine.m}, в частности, планируется улучшение интерфейса и непосредственная поддержка вычисления коэффициентов разложения тензорных произведений представлений на неприводимые. 


%%
%% End of file
%%% Local Variables: 
%%% mode: latex
%%% TeX-master: "thesis"
%%% End: 
