\documentclass[14pt,autoref,href,facsimile
%,fixint=false
%,times
]{disser}

\usepackage[a4paper,nohead,includefoot,mag=1000,
            margin=2cm,footskip=1cm]{geometry}
\usepackage[T2A]{fontenc}
\usepackage[utf8x]{inputenc}
\usepackage[english,russian]{babel}
\usepackage{tabularx}
\ifpdf\usepackage{epstopdf}\fi

% Поддержка нескольких списков литературы в одном документе
\usepackage{multibib}
% Создание команд для цитирования собственных работ диссертанта
% в отдельном разделе. В данном случае ссылка будет иметь вид \citemy{...}.
\newcites{my}{Список публикаций}

% Путь к файлам с иллюстрациями
\graphicspath{{figures/}}


\usepackage{amsmath}
\usepackage{amsmath,amssymb,amsthm}
\usepackage{multicol}
\usepackage{color}
%\usepackage{graphicx}
\usepackage{ucs}
\usepackage{pb-diagram}
\usepackage{enumerate}
\usepackage{appendix}
\usepackage{listings}
\usepackage{textcomp}


\definecolor{listinggray}{gray}{0.9}
\definecolor{lbcolor}{rgb}{0.9,0.9,0.9}
\lstset{
%	backgroundcolor=\color{lbcolor},
        language=Mathematica,
	tabsize=2,
	rulecolor=,
        basicstyle=\scriptsize,
        upquote=true,
%        aboveskip={1.5\baselineskip},
        columns=fixed,
        showstringspaces=false,
        extendedchars=true,
%        breaklines=true,
        prebreak = \raisebox{0ex}[0ex][0ex]{\ensuremath{\hookleftarrow}},
%        frame=single,
        showtabs=false,
        showspaces=false,
        showstringspaces=false,
        identifierstyle=\ttfamily,
        keywordstyle=\color[rgb]{0,0,1},
        commentstyle=\color[rgb]{0.133,0.545,0.133},
        stringstyle=\color[rgb]{0.627,0.126,0.941},
}

\newtheorem{statement}{Утверждение}
\newtheorem{theorem}{Теорема}
\newtheorem{axiom}{Аксиома}
\newtheorem{corollary}{Следствие}[chapter]
\newtheorem{lemma}{Лемма}
\newtheorem{mynote}{Замечание}[chapter]
%%\newtheorem{Def}{Definition}[section]
\newtheorem{Def}{Определение}[chapter]
\newtheorem{Cnj}[Def]{Гипотеза}
\newtheorem{Prop}{Свойство}
%%\newtheorem{example}{Example}[section]

\theoremstyle{definition}
\newtheorem{definition}{Определение}
\newtheorem{remark}{Замечание}[chapter]
\newtheorem{example}{Пример}[chapter]
\newtheorem{exercise}{Упражнение}
\newtheorem{conjecture}{Гипотеза}[chapter]

\newcommand{\go}{\stackrel{\circ }{\mathfrak{g}}}
\newcommand{\ao}{\stackrel{\circ }{\mathfrak{a}}}
\newcommand{\co}[1]{\stackrel{\circ }{#1}}
\newcommand{\pia}{\pi_{\mathfrak{a}}}
\newcommand{\piab}{\pi_{\mathfrak{a}_{\perp}}}
\newcommand{\gf}{\mathfrak{g}}
\newcommand{\af}{\mathfrak{a}}
\newcommand{\uf}{\mathfrak{u}}
\newcommand{\sfr}{\mathfrak{s}}
\newcommand{\aft}{\widetilde{\mathfrak{a}}}
\newcommand{\afb}{\mathfrak{a}_{\perp}}
\newcommand{\hf}{\mathfrak{h}}
\newcommand{\hfb}{\mathfrak{h}_{\perp}}
\newcommand{\pf}{\mathfrak{p}}

\newcommand{\gfh}{\hat{\mathfrak{g}}}
\newcommand{\afh}{\hat{\mathfrak{a}}}
\newcommand{\bff}{\mathfrak{b}}
\newcommand{\hfg}{\hf_{\gf}}


\begin{document}
% Включение файла с общим текстом диссертации и автореферата
% (текст титульного листа и характеристика работы).
% Общие поля титульного листа диссертации и автореферата
\institution{Санкт-Петербургский Государственный Университет}

\topic{Бесконечномерные алгебры симметрии в моделях квантовой теории поля}

\author{Назаров Антон Андреевич}

\specnum{01.04.02}
\spec{Теоретическая физика}

\sa{Ляховский Владимир Дмитриевич}
\sastatus{д.~ф.-м.~н., проф.}

\city{Санкт-Петербург}
\date{\number\year}

% Общие разделы автореферата и диссертации
\mkcommonsect{actuality}{Актуальность работы}{%
Проблема вычисления коэффициентов ветвления для представлений алгебр Ли стоит уже многие десятилетия. Она актуальна для различных физических приложений. Вместе с тем, в отличие от кратностей весов не существует особенно эффективных алгоритмов.                                     
}

\mkcommonsect{objective}{Цель диссертационной работы}{%
Разработка рекуррентного подхода к функциям ветвления аффинных алгебр Ли, его связь с проблемами теории представлений и его приложения в моделях конформной теории поля.
}

\mkcommonsect{novelty}{Научная новизна}{%
Предложен эффективный алгоритм для вычисления коэффициентов ветвления, показана его связь с резольвентой Бернштейна-Гельфанда-Гельфанда.
}

\mkcommonsect{value}{Практическая значимость}{%
Результаты работы                                  
}

\mkcommonsect{results}{%
На защиту выносятся следующие основные результаты и положения:}{%
Текст раздела
}

\mkcommonsect{approbation}{Апробация работы}{%
Текст раздела
}

\mkcommonsect{pub}{Публикации.}{%
Материалы диссертации опубликованы в $8$ печатных работах, из них $3$ статьи в
рецензируемых журналах~\citemy{2010arXiv1007.0318L,2010LyakhovskyNazarovTMF}, $2$ статьи в
сборниках трудов конференций, 2 препринта и $1$ тезисы доклада.
}

\mkcommonsect{contrib}{Личный вклад автора}{%
Текст раздела
}

\mkcommonsect{struct}{Структура и объем диссертации}{%
Текст раздела
}

%%
%% End of file
%%% Local Variables: 
%%% mode: latex
%%% TeX-master: "thesis"
%%% End: 


\title{АВТОРЕФЕРАТ\\
диссертации на соискание ученой степени\\
кандидата физико-математических наук}

\maketitle

% Внутренняя сторона обложки
\noindent
\begin{center}
Работа выполнена на кафедре физики высоких энергий и элементарных частиц физического факультета Санкт-Петербургского государственного университета.
% \emph{название организации}.
\end{center}
\vskip1ex
\begin{tabularx}{\linewidth}{lp{1cm}X}
Научный руководитель:  & & \emph{доктор физико-математических наук}, \\
                       & & \emph{профессор}, \\
                       & & \emph{Ляховский Владимир Дмитриевич}
\\
Официальные оппоненты: & & \emph{доктор физико-математических наук}, \\
                       & & \emph{профессор}, \\
                       & & \emph{Кулиш Петр Петрович}\\
                       & & \emph{кандидат физико-математических наук}, \\
                       & & \emph{ученое звание}, \\
                       & & \emph{Мудров Андрей Игоревич}
\\
Ведущая организация:   & & \emph{Объединенный институт ядерных исследований}\\
\end{tabularx}

\vskip2ex\noindent
Защита состоится \datefield{} в \rule[0pt]{1cm}{0.5pt}\, часов
на заседании совета \emph{Д 212.232.24} по защите докторских и кандидатских диссертаций при \emph{Санкт-Петербургском государственном университете}, расположенном по адресу:
\emph{Санкт-Петербург, Средний пр. В.О., д. 41/43, ауд. 305}

\vskip1ex\noindent
С диссертацией можно ознакомиться в научной библиотеке
\emph{Санкт-Петербургского государственного университета}.

\vskip1ex\noindent
Автореферат разослан \datefield{}

\vskip2ex\noindent
%Отзывы и замечания по автореферату в двух экземплярах, заверенные
%печатью, просьба высылать по вышеуказанному адресу на имя ученого секретаря
%диссертационного совета.

\vfill\noindent
Ученый секретарь\\
диссертационного совета,\\
\emph{ученая степень}, \emph{ученое звание}%
\hfill
\makeatletter
% вставка файла, содержащего факсимиле ученого секретаря
\ifDis@facsimile
  \raisebox{-4pt}{\includegraphics[width=3cm]{sec-facsimile}}\hfill
\fi%
\makeatother%
\emph{фамилия и. о.}

\clearpage

\nsection{Общая характеристика работы}

% Актуальность работы
\actualitysection
\actualitytext

% Цель диссертационной работы
%\objectivesection
%\objectivetext

% Научная новизна
\noveltysection
\noveltytext

% Практическая значимость
%\valuesection
%\valuetext

% Результаты и положения, выносимые на защиту
\resultssection
\resultstext

% Апробация работы
\approbationsection
\approbationtext

% Публикации
\pubsection
\pubtext

% Личный вклад автора
%\contribsection
%\contribtext

% Структура и объем диссертации
\structsection
\structtext

% \contentsection
% \contenttext



\nsection{Содержание работы}


\textbf{Во Введении} обоснована актуальность диссертационной работы,
сформулирована цель и аргументирована научная новизна исследований, показана
практическая значимость полученных результатов, представлены выносимые на
защиту научные положения.

\textbf{Глава 1} носит обзорный характер. В ней мы даем аксиоматическую формулировку конформной теории поля, описываем модели Весса-Зумино-Новикова-Виттена и coset-модели. Затем мы демонстрируем роль аффинных алгебр в описании этих моделей и приводим основные понятия теории представлений, использующиеся в диссертации. Кроме того, мы обсуждаем конформную теорию поля на области с границей, так как она оказывается связана со стохастическим описанием решеточных моделей. 

\textbf{В главе 2} мы показываем, что структура сингулярного элемента определяет свойства модуля алгебры Ли, доказываем лемму о разложении сингулярного элемента и выводим основное рекуррунтное соотношение на коэффициенты ветвления. Основные результаты данной главы опубликованы в работе \citemy{2010arXiv1007.0318L}. 

Формула Вейля-Каца для формальных характеров интегрируемых модулей старшего веса конечномерных и аффинных алгебр Ли имеет вид
\begin{equation}
  \label{eq:1}
  \mathrm{ch} V^{(\mu)} = \frac{\Psi^{(\mu)}}{R},
\end{equation}
где $\Psi^{(\mu)}$ -- сингулярный элемент модуля, а $R=\prod_{\alpha\in \Delta^+}(1-e^{-\alpha})^{\mathrm{mult}(\alpha)}$ -- знаменатель Вейля. Сингулярный элемент определяется набором сингулярных весов модуля и имеет разный вид для разных типов модулей старшего веса. Например, $\Psi^{(\mu)}=\sum_{w\in W} \epsilon(w) e^{w(\mu+\rho)-\rho}$ для неприводимых модулей. Заметим, что знаменатель Вейля является универсальным объектом, характеризующим корневую систему алгебры Ли, а свойства модуля определяются сингулярным элементом.

Процедура редукции состоит в разложении модуля алгебры Ли $\gf$ в сумму модулей некоторой подалгебры $\af$
\begin{equation}
  \label{eq:2}
  L_{\gf\downarrow \af}^{\mu }=\bigoplus
\limits_{\nu \in P_{\af}^{+}}b_{\nu }^{\left( \mu \right) }L_{\af}^{\nu }.
\end{equation}
Используя оператор проекции  $\pi_{\af}$ (на весовое пространство $\hf_{\af}^*$), перепишем это разложение для формальных характеров:
\begin{equation}
\label{branching1}
 \pi _{\af}\circ ch\left( L^{\mu }\right)
 =\sum_{\nu \in P_{\af}^{+}}b_{\nu }^{(\mu)}ch\left( L_{\af}^{\nu }\right) .
\end{equation}
Нас интересуют коэффициенты ветвления $b^{(\mu)}_{\nu}$.

Для любой алгебры $\gf$ и подалгебры $\af\subset \gf$ можно построить  подалгебру $\afb$ такую, что 
\begin{equation}
\label{delta-a-ort}
\Delta _{\af_{\perp }} :=\left\{ \beta \in \Delta _{\gf}|
\forall h \in \hf_{\af};  \beta\left(h \right)=0  \right\} 
\end{equation}
Обозначим через $W_{\afb}$ подгруппу группы Вейля $W$, порожденную отражениями $w _{\beta }$, соответствующими корням $\beta \in \Delta _{\afb}^{+}$ . Подсистема  $\Delta _{\af_{\perp }}$ определяет подалгебру $\af_{\perp }$ с подалгеброй Картана $\hf_{\afb}$.  Пусть
$\hf_{\perp }^{\ast }:=\left\{ \eta \in \hf_{\perp \af}^{\ast
}|\forall h \in \hf_{\af\oplus \af_{\perp}}; \eta \left( h \right)=0 \right\}$, тогда имеет место разложение подалгебры Картана $\hf=\frak{\hf_{\af}}\oplus \hf_{\afb}\oplus\hf_{\perp }$.

Для подалгебр из ортогональной пары  $\left( \af,\afb\right) $ рассмотрим соответствующие векторы Вейля $\rho _{\af}$ и $\rho _{\af_{\perp }}$, и образуем так называемые  ''дефекты'' вложения $\mathcal{D}_{\af}$ и $\mathcal{D}_{\af_{\perp }}$ :
\begin{equation}
\mathcal{D}_{\af}:=\rho _{\af}-\pi _{\af}\rho ,
\end{equation}
\begin{equation}
\label{defect-perp}
\mathcal{D}_{\af_{\perp }}:=\rho _{\af_{\perp }}-\pi_{\afb}\rho .
\end{equation}

Рассмотрим сингулярные веса  $\left\{\left( w(\mu +\rho )-\rho \right)|w  \in W \right\}$  модуля старшего веса  $L_{\gf}^{\mu }$ и их проекции на $h_{\widetilde{\af_{\perp }}}^{\ast }$ (дополнительно сдвинутые на дефект $-\mathcal{D}_{\af_{\perp }}$):
\begin{equation*}
\mu _{\widetilde{\af_{\perp }}}\left( w\right) :=\pi _{\widetilde{\frak{%
a}_{\perp }}}\circ\left[ w(\mu +\rho )-\rho \right] -\mathcal{D}_{\af_{\perp
}},\quad w\in W.
\end{equation*}
Среди весов  $\left\{\mu _{\widetilde{\af_{\perp }}}\left( w\right)|w\in W\right\}$ выберем находящиеся в главной камере Вейля $\overline{C_{\widetilde{\afb}}}$ и обозначим через $U$ множество представителей $u$ классов смежности $W/W_{\af_{\perp }}$, таких что
\begin{equation}
U:=\left\{ u\in W|\quad \mu _{\widetilde{\af_{\perp }}}\left( u\right)
\in \overline{C_{\widetilde{\af_{\perp }}}}\right\} \quad .
\label{U-def}
\end{equation}
Для множества  $U$ введем веса
\begin{equation*}
\mu _{\af}\left( u\right) :=\pi _{\af}\circ\left[ u(\mu +\rho )-\rho %
\right] +\mathcal{D}_{\af_{\perp }}.
\end{equation*}
Кроме того, мы будем пользоваться аналогичным определением
\begin{eqnarray}
\label{eq:136}
\mu _{\tilde \af}\left( u\right) :=\pi _{\tilde \af}\left[ u(\mu +\rho )-\rho \right] +\mathcal{D}_{\afb},\\
\mu _{\afb}\left( u\right) :=\pi _{\afb}\left[ u(\mu +\rho )-\rho \right] +\mathcal{D}_{\afb}.
\end{eqnarray}


Мы доказываем следующую лемму о разложении сингулярного элемента:
\begin{lemma}
\label{lemma}
Пусть $\left( \af,\afb \right)$ -- ортогональная пара редуктивных подалгебр $\gf$ и  $\widetilde{\afb}=\afb\oplus \hf_{\perp }$, $\widetilde{\af}=\af\oplus\hf_{\perp }$ ,

$L^{\mu }$ -- модуль старшего веса с сингулярным элементом $\Psi ^{\left(\mu \right)}$ ,

$R_{\af_{\perp }}$ -- знаменатель Вейля для подалгебры $\af_{\perp }$.

Тогда элемент  $\Psi ^{\left( \mu \right) }_{\left(  \af, \afb \right)}=\pi _{\af}\left( \frac{\Psi _{\gf}^{\mu }}{R_{\af_{\perp }}}\right) $ можно разложить в сумму по  $u\in U$ (см. (\ref{U-def})) сингулярных весов $e^{\mu _{\af}\left( u\right) }$ с коэффициентами $\epsilon (u)\mathrm{\dim}\left( L_{\widetilde{\afb}}^{\mu _{\widetilde{\afb}}\left( u\right) }\right) $:
\begin{equation}
\Psi ^{\left( \mu \right) }_{\left(  \af, \afb \right)}=\quad \pi _{\af}\left( \frac{\Psi^{\mu }}{R_{\af%
_{\perp }}}\right) =\sum_{u\in U}\;\epsilon (u)\mathrm{\dim }
\left( L_{\widetilde{\af_{\perp }}}^{\mu _{%
\widetilde{\af_{\perp }}}\left( u\right) }\right) e^{\mu _{\af}\left( u \right) }.
\end{equation}
\end{lemma}

Теперь мы можем ввести ``веер вложения'', который необходим для формулировки рекуррентных соотношений:
\begin{definition}
\label{fan-definition} Рассмотрим произведение
\begin{equation}
\prod_{\alpha \in \Delta ^{+}\setminus \Delta _{\afb }^{+}}\left( 1-e^{-\pi
_{\af}\alpha }\right) ^{\mathrm{mult}(\alpha )-\mathrm{mult}_{\af%
}(\pi _{\af}\alpha )}=-\sum_{\gamma \in P_{\af}}s(\gamma
)e^{-\gamma }  \label{eq:142}
\end{equation}
и носитель $\Phi _{\af\subset \gf}\subset P_{\af}$ функции $s(\gamma )=\det \left( \gamma \right) $ :
\begin{equation}
\Phi _{\af\subset \gf}=\left\{ \gamma \in P_{\af}|s(\gamma
)\neq 0\right\}   \label{eq:37}
\end{equation}
Упорядочение корней в  $\co{\Delta _{\af}}$ индуцирует естественное упорядочение весов в $P_{\af}$. Обозначим через $\gamma_{0}$ наименьший вектор $\Phi _{\af\subset \gf}$. Множество
\begin{equation}
\Gamma _{\af\rightarrow \gf}=\left\{ \xi -\gamma _{0}|\xi \in \Phi _{%
\af\subset \gf}\right\} \setminus \left\{ 0\right\}
\label{fan-defined}
\end{equation}
называется  \textit{веером вложения}.
\end{definition}
Веер вложения универсален и зависит только от вложения $\af\to\gf$ и не зависит от модуля $L^{(\mu)}$.

Введем сингулярные коэффициенты ветвления следующим образом:
\begin{eqnarray}
  \label{eq:3}
  k^{(\mu)}_{\xi}=b^{(\mu)}_{\xi} \;\text{если}\; \xi\in \bar C_{\af}\\
  k^{(\mu)}_{\xi}=\epsilon(w) b^{(\mu)}_{w(\xi+\rho_{af})-\rho_{\af}} \;\text{где}\; w\in W_{\af}:w(\xi+\rho_{\af})-\rho_{\af}\in \bar C_{\af}
\end{eqnarray}

Теперь мы можем сформулировать основную теорему, которая позволит нам рекуррентно вычислять коэффициенты ветвления.

\begin{theorem}
  Для сингулярных коэффициентов ветвления $k^{(\mu)}_{\nu}$ выполняется соотношение
  \begin{equation}
    \label{recurrent-relation}
    \begin{array}{c}
      k_{\xi }^{\left( \mu \right) }=-\frac{1}{s\left( \gamma _{0}\right) }\left(
        \sum_{u\in U} \epsilon(u)\;
        \dim \left( L_{\widetilde{\af_{\perp }}}^{\mu
        _{\widetilde{\af_{\perp }}}\left( u\right) }\right)
        \delta_{\xi-\gamma_0,\pi_{\af}(u(\mu+\rho)-\rho)}+ \right.\\
      \left.
        +\sum_{\gamma \in
          \Gamma _{\af \rightarrow \gf}}s\left( \gamma +\gamma _{0}\right) k_{\xi
          +\gamma }^{\left( \mu \right) }\right).
    \end{array}
  \end{equation}
\end{theorem}

Далее мы анализируем пары $(\af,\afb)$ для простых алгебр Ли. Оказывается, что для ``ортогональной пары'' $(\af,\afb)$, вообще говоря, $\af\oplus\afb\not\subset\gf $. В частности, для серии простых конечномерных алгебр $B_n$ существуют ``ортогональные пары'' подалгебр $(B_k,B_{n-k})$.

На основании рекуррентного соотношения (\ref{recurrent-relation}) сформулирован алгоритм вычисления коэффициентов ветвления. Остальные разделы главы 2 содержат примеры вычислений с использованием предложенного алгоритма, а также описание роли функций ветвления в формулировке конформной теории поля на торе и в coset-моделях конформной теории поля. 


\textbf{В главе 3} мы используем  разложение сингулярного элемента, чтобы показать связь ветвления с (обобщенной) резольвентой Бернштейна-Гельфанда-Гельфанда.   Результаты третьей главы опубликованы в работах \citemy{2011arXiv1102.1702L,2010LyakhovskyNazarovMQFT}.

Оказывается, что для полупростой конечномерной алгебры $\gf$  и полупростой конечномерной подалгебры $\af$ алгебра $\afb$, определенная в формуле (\ref{branching1}), является регулярной. Отношение знаменателей Вейля порождает параболические модули Верма. Сингулярный элемент  $\Psi ^{\left( \mu \right) }$ может быть разложен в сумму по  $u\in U$ (см. (\ref{U-def})) сингулярных элементов $\Psi _{\frak{a}_{\perp }}^{\mu _{\frak{a}_{\perp }}\left( u\right) }$ с коэффициентами
$\epsilon (u)e^{\mu _{\widetilde{\mathfrak{a}}}\left( u\right) }$:
\begin{equation}
\Psi ^{\left( \mu \right) }=\sum_{u\in U}\;\epsilon (u)e^{\mu _{\widetilde{%
\mathfrak{a}}}\left( u\right) }\Psi _{\frak{a}_{\perp }}^{\mu _{\frak{a}%
_{\perp }}\left( u\right) }.  \label{sing decomp main}
\end{equation}
Мы доказываем следующее утверждение, демонстрирующее, что разложение сингулярного элемента связано с разложением характера неприводимого модуля в комбинацию характеров обобщенных модулей Верма
\begin{statement}
%\bigskip
Для ортогональной подалгебры  $\frak{a}_{\perp }$ в $\frak{g}$ (являющейся ортогональным партнером редуктивной подалгебры $\frak{a}\hookrightarrow \frak{g}$) характер интегрируемого модуля старшего веса  $L^{\mu }$ может быть представлен в виде комбинации (с целочисленными коэффициентами) характеров параболических модулей Верма, распределенных по множеству весов $\mu _{\widetilde{\mathfrak{a}}}\left(
u\right)$:
\begin{equation}
\mathrm{ch}\left( L^{\mu }\right) =\sum_{u\in U}\;\epsilon (u)e^{\mu _{%
\widetilde{\frak{a}}}\left( u\right) }\mathrm{ch}M_{I}^{\mu _{\frak{a}%
_{\perp }}\left( u\right) },  \label{gen Weyl-Verma}
\end{equation}
где  $U:=\left\{ u\in W|\quad \mu _{\frak{a}_{\perp }}\left( u\right) \in
\overline{C_{\frak{a}_{\perp }}}\right\} $ и $I$ -- такое подмножество в  $S$, что $\Delta _{I}^{+}$ эквивалентно $\Delta _{\frak{a}_{\perp }}^{+}$.
\end{statement}

Связь редукции и (обобщенной) резольвенты Бернштейна-Гельфанда-Гельфанда дается следующим утверждением:
\begin{statement}
Пусть $L^{\mu }$ --  $\frak{g}$-модуль со старшим весом $\mu \in P^{+}$, и пусть регулярная подалгебра  $\afb\hookrightarrow \frak{g}$ является ортогональным партнером редуктивной подалгебры $\frak{a}\hookrightarrow \frak{g}$. Тогда разложение (\ref{sing decomp main}) определяет как обобщенную резольвенту $L^{\mu }$ по отношению к $\afb$, так и правила ветвления $L^{\mu }$ по отношению к $\afb$, так и правила ветвления $L^{\mu }$ по отношению к $\af$ .
\end{statement}

\textbf{Глава 4} посвящена сплинтам -- расщеплением корневой системы алгебры Ли в объединение образов корневых систем двух алгебр, не обязательно являющихся подалгебрами данной алгебры. Если одна из алгебр является подалгеброй, то сплинт приводит к резкому упрощению в вычислении коэффициентов ветвления -- они совпадают с кратностями весов в модуле другой алгебры. Основная часть главы посвящена доказательству этого факта. Кроме того, сплинт корневой системы простой конечномерной алгебры Ли приводит к возникновению новых соотношений на струнные функции и функции ветвления соответствующего аффинного расширения. Эти соотношения обсуждаются в разделе 4.4.
Данные результаты опубликованы в статьях \citemy{2011arXiv1111.6787L,2012arXiv1204.1855L}.


\begin{definition}
Пусть $\Delta _{0}$ и $\Delta$ -- корневые системы с соответствующими весовыми решетками $P_{0}$ и $P$. Рассмотрим отображение
\begin{equation}
\phi :\left\{
\begin{array}{l}
\Delta _{0}\hookrightarrow \Delta , \\
P_{0}\hookrightarrow P.
\end{array}
\right.
\end{equation}
Оно называется ``вложением'', если \newline
\noindent (a) оно вкладывает $\Delta _{0}$ в $\Delta $, и \newline
\noindent (b) $\phi$ действует гомоморфно по отношению к группам сложения векторов в $P_{0}$ и $P$:
\[
\phi (\gamma )=\phi (\alpha )+\phi (\beta )
\]
для любой тройки $\alpha ,\beta ,\gamma \in P_{0}$, такой, что $\gamma =\alpha+\beta $.
\end{definition}

$\phi$ индуцирует вложение формальных алгебр: ${\mathcal{E}}_0\hookrightarrow \mathcal{E}$ и для образа ${\mathcal{E}}_i=\mathrm{Im}_{\phi}\left( {\mathcal{E}}_0\right)$ можно рассмотреть обратное отображение $\phi^{-1}:{\mathcal{E}}_i \longrightarrow {\mathcal{E}}_0$.

Заметим, что нужно различать два класса вложений: когда скалярное произведение (заданное формой Киллинга) в корневом пространстве $P_0$ инвариантно по отношению к  $\phi$ и когда оно не  $\phi$-инвариантно. Вложения первого класса называются ``метрическими'', второго -- ``неметрическими''. 

\begin{definition}
Корневая система $\Delta$ ``расщепляется'' на  $(\Delta _{1},\Delta _{2})$, если существует два вложения  $\phi _{1}:\Delta _{1}\hookrightarrow \Delta $ и $\phi _{2}:\Delta _{2}\hookrightarrow \Delta $, где (a) $\Delta $ -- несвязное объединение образов $\phi _{1}$ и $\phi _{2}$, и (b) ни ранг  $\Delta _{1}$, ни ранг  $\Delta _{2}$ не превосходит ранга $\Delta $.
\end{definition}

Эквивалентно, можно сказать, что  $(\Delta_1,\Delta_2)$  -- ``сплинт'' (расщепление)  $\Delta$ и мы можем обозначить его через $\Delta \approx (\Delta_1,\Delta_2)$. Каждая из компонент  $\Delta_1$ и $\Delta_2$ называется ``стеблем'' сплинта $(\Delta_1,\Delta_2)$.

Чтобы показать связь веера вложения со сплинтом рассмотрим случай, когда один из стеблей $\Delta _{1}=\Delta _{\af}$  является подсистемой корневой системы. 

Сплинт $\Delta \approx (\Delta _{1},\Delta _{2})$ называется ``инъективным'', если $\Delta _{1}=\Delta _{\af}$, -- подсистема корневой системы $\Delta $, соответствующая регулярной редуктивной подалгебре $\af\hookrightarrow \gf$. 

В случае инъективного сплинта второй стебель $\Delta _{\sfr}:=\Delta_{2}=\Delta \setminus \Delta _{\af}$ может быть переписан как произведение (аналогично формуле (\ref{eq:142})) и определяет веер вложения  $\Gamma _{\af\hookrightarrow \gf}$. Обозначим через $\Delta_{\mathfrak{s}0}$ кообраз второго вложения $\phi:\Delta_{\mathfrak{s}0}\to \Delta_{\gf}$. Верна следующая гипотеза.

\begin{conjecture}
Каждый инъективный сплинт $\Delta \approx (\Delta _{\af},\Delta _{\sfr})$ определяет веер вложения с носителем $\left\{ \xi \right\} _{\af\rightarrow \gf}$, задающимся произведением
\begin{equation}
\prod_{\beta \in \Delta _{\sfr}^{+}}\left( 1-e^{-\beta }\right)
=-\sum_{\gamma \in P}s(\gamma )e^{-\gamma }\quad   \label{splint product}
\end{equation}
\end{conjecture}

В случае инъективного сплинта мы можем сказать, что подалгебра $\af\hookrightarrow \gf$ расщепляется $\Delta$ (и назовем $\af$ ``расщепляющей подалгеброй'' алгебры $\gf$).  Сплинты были классифицированы в работе \cite{richter2008splints}  (см. Приложение в конце главы) и первые три класса сплинтов в этой классификации инъективны. 


\begin{Prop}
\begin{equation}
\frac{e^{\rho _{\gf}}}{\prod_{\beta \in \Delta _{\sfr%
}^{+}}(1-e^{-\beta })}\left( \Psi ^{\widetilde{\mu }+\rho _{\sfr%
}}\right) =\sum_{\widetilde{\nu }\in \mathcal{N}_{\sfr}^{\widetilde{\mu }%
}}M_{\left( \sfr\right) \widetilde{\nu }}^{\widetilde{\mu }}e^{\left(
\mu -\phi \left( \widetilde{\mu }-\widetilde{\nu }\right) \right)
}=\sum_{\nu \in P_{\af}^{++}}b_{\nu }^{(\mu )}e^{\nu }.
\label{singular main-4}
\end{equation}
Любой вес с ненулевой кратностью, входящий в правую часть равенства, равен одному из старших весов в разложении. Кратность $M_{\left( \sfr
\right) \widetilde{\nu }}^{\widetilde{\mu }}$ веса  $\widetilde{\nu}\in \mathcal{N}_{\sfr}^{\widetilde{\mu }}$ определяет коэффициент ветвления  $b_{\nu }^{(\mu )}$ для старшего веса $\nu =\left( \mu-\phi \left( \widetilde{\mu }-\widetilde{\nu }\right) \right) $:
\[
b_{\left( \mu -\phi \left( \widetilde{\mu }-\widetilde{\nu }\right) \right)
}^{(\mu )}=M_{\left( \sfr\right) \widetilde{\nu }}^{\widetilde{\mu }}.
\]
\end{Prop}



Заключительная \textbf{глава 5} посвящена практическим приложениям результатов диссертации.

 В разделе 5.1 мы  описываем применение алгебраических методов к проблеме поиска соответствия между квантовополевым и решеточным описанием критического поведения. Эти результаты были опубликованы нами в работах \citemy{NazarovJETPletters,2011arXiv1112.4354N}.

Стохастический процесс, который удовлетворяет уравнению 
\begin{equation}
\label{eq:166}
  \frac{\partial g_t(z)}{\partial t} = \frac{ 2}{g_t(z)-\sqrt{\kappa}\xi_{t}} ,
\end{equation}
называется {\it эволюцией Шрамма-Левнера} на верхней полуплоскости $\mathbb{H}$. Здесь $\xi_{t}$ -- Броуновское движение. Динамика конца  $z_{t}$ критической кривой $\gamma_{t}$ (конец следа эволюции Шрамма-Левнера) описывается уравнением $z_{t}=g_{t}^{-1}(\sqrt{\kappa}\xi_{t})$. Нам удобнее использовать отображение $w_{t} (z)=g_{t}(z)-\sqrt{\kappa}\xi_{t}$. 

Мы обобщаем анализ соответствия между эволюцией Шрамма-Левнера и конформной теорией поля на случай coset-моделей \cite{Goddard198588}. Такие модели задаются алгеброй Ли $\gf$ и ее подалгеброй $\af$. 
$G/A$-coset модель конформной теории поля может быть реализована как ВЗНВ-модель (с калибровочной группой $G$), взаимодействующая с чисто калибровочными полями, с калибровочной группой $A\subset G$ \cite{gawdzki1988g,figueroa89equivalence}. Действие записывается через поля $\gamma:\mathbb{C}\to G$ and $\alpha,\bar\alpha:\mathbb{C}\to A$:
\begin{multline}
\label{eq:178}
      S_{G/A}(\gamma, \alpha)=\\
-\frac{k}{8\pi}\int_{S^2} d^2x\; {\cal K} (\gamma^{-1}\partial^{\mu}\gamma,\gamma^{-1}\partial_{\mu}\gamma)-
 \frac{k }{24\pi} \int_{B}\epsilon_{ijk} {\cal K}\left(
    \tilde \gamma^{-1}\frac{\partial \tilde \gamma}{\partial y^i},
      \left[\tilde \gamma^{-1}\frac{\partial \tilde \gamma}{\partial y^j}
      \tilde \gamma^{-1}\frac{\partial \tilde \gamma}{\partial y^k}\right]\right) d^3y+\\
+
      \frac{k}{4\pi}\int_{S^2} d^{2}z \left(\mathcal{K}(\alpha, \gamma^{-1}\bar \partial \gamma)-\mathcal{K}(\bar \alpha, (\partial \gamma ) \gamma^{-1})\right.
      \left.+\mathcal{K}(\alpha,\gamma^{-1}\bar \alpha \gamma)-\mathcal{K}(\alpha,\bar \alpha)\right).
\end{multline}
Здесь через  $\mathcal{K}$ обозначена форма Киллинга в алгебре Ли $\gf$, соответствующей группе Ли $G$.

Если мы фиксируем  $A$-калибровку, у нас останется  $G/A$ калибровочная инвариантность. Значит мы должны добавить случайные калибровочные преобразования к эволюции Шрамма-Левнера, аналогично случаю ВЗНВ-моделей  (См. \cite{bettelheim2005stochastic}).  Обозначим через $t^{a}_{i}$ ($\tilde{t}^{b}_{i}$) генераторы представления алгебры $\gf$ (соответственно, представления $\af$), соответствующего примарному полю $\varphi_{i}$.

Теперь рассмотрим наблюдаемые в присутствии следа эволюции Шрамма-Левнера. Математическое ожидание решеточной наблюдаемой $\mathcal{O}$ на верхней полуплоскости можно вычислить как сумму ожиданий этой наблюдаемой в присутствии (конечной части) траектории эволюции Шрамма-Левнера  $\gamma_{t}$ вплоть до некоторого времени $t$, умноженных на вероятность этой траектории:
\begin{equation*}
  \prec \mathcal{O} \succ_{\mathbb{H}}=\mathbb{E}\left[\prec\mathcal{O}\succ_{\gamma_{t}}\right]=\sum_{\gamma_{t}} P\left[C_{\gamma_{t}}\right] \prec \mathcal{O} \succ_{\gamma_{t}}
\end{equation*}
Решеточная наблюдаемая  $\prec \mathcal{O} \succ_{\mathbb{H}}$ не зависит от  $t$, следовательно $\prec\mathcal{O}\succ_{\gamma_{t}}$ -- мартингал. Это должно выполняться и для ее непрерывного предела, дающегося комбинацией корреляционных функций в конформной теории поля \cite{bauer2003sle}:
\begin{equation}
  \prec \mathcal{O} \succ_{\mathbb{H}_{t}}\to \mathcal{F}(\left\{z_{i}\right\})_{\mathbb{H}_{t}}=
  \frac{\left< \mathcal{O}(\{z_{i}\})\phi(z_{t})\phi^{\dagger}(\infty)\right>_{\mathbb{H}_{t}}}{\left<\phi(z_{t})\phi^{\dagger}(\infty)\right>_{\mathbb{H}_{t}}}
%% =
%%   \frac{\left<^{g_{t}}
%% \mathcal{O}\phi(\xi_{t})\phi^{\dagger}(\infty)\right>_{\mathbb{H}}}{\left<\phi(\xi_{t})\phi^{\dagger}(\infty)\right>_{\mathbb{H}}}
\label{eq:162}
\end{equation}
Мы рассматриваем теорию с границей, так что мы должны использовать модели граничной конформной теории поля и накладывать соответствующие граничные условия. В случае верхней полуплоскости корреляционные функции в граничной конформной теории поля могут быть переписаны как корреляционные функции для теории на всей плоскости, но с удвоенным числом полей.

Мы предполагаем, что $\mathcal{F}$ содержит некоторый набор примарных полей  $\varphi_{i}$ с конформными весами $h_{i}$. Так как мы рассматриваем граничную конформную теорию поля, мы должны добавить объемные поля в сопряженных точках  $\bar z_{i}$.  Кроме того, у нас есть операторы смены граничного условия   $\phi$ на конце следа эволюции Шрамма-Левнера и на бесконечности.

Рассмотрим, что происходит с наблюдаемыми при эволюции следа SLE $\gamma_{t}$ с момента   $t$ до $t+ dt$. 

Через  $\mathcal{G}_{i}$ мы обозначили  генераторы инфинитезимальных преобразований примарных $\varphi_{i}$:$d\varphi_{i}(w_{i}) = \mathcal{G}_{i}\varphi_{i}(w_{i})$. Нормируем дополнительное $\left(\dim\gf\right)$-мерное Броуновское движение следующим образом: $\mathbb  {E}\left[d\theta^{a}\; d\theta^{b}\right]=\mathcal{K}(t^{a},t^{b})dt$. Тогда генератор преобразования поля равен
\begin{equation}
  \mathcal{G}_{i}=\left(\frac{2dt}{w_{i}}-\sqrt{\kappa} d\xi_{t}\right) \partial_{w_{i}}+\frac{\sqrt{\tau}}{w_{i}}\left(\sum_{a:\mathcal{K}(t^{a},\tilde{t}^{b})=0}\left(d \theta ^{a} t^{a}_{i}\right)\right).
\label{eq:179}
\end{equation}
То есть мы фиксировали  $A$-калибровку, разрешив случайное блуждание только в направлении, ортогональном подалгебре $\af$. 

Формула Ито, дает выражение для дифференциала $\mathcal{F}$, который равняется нулю в силу условия мартингала. Это равенство можно переписать в виде следующего алгебраического условия на граничное состояние $\phi(0)\left|0\right>$:
\begin{multline}
\label{eq:182}
  \left<0\left|\phi(\infty)\varphi_{1}(w_{1})\dots\varphi_{2N}(w_{2N})\right.\right.\\
  \left(-2L_{-2}+\frac{1}{2}\kappa L_{-1}^{2}+\frac{1}{2}\tau \left(\sum_{a=1}^{\dim\gf}J^{a}_{-1}J^{a}_{-1}-\sum_{b=1}^{\dim\af}\tilde{J}^{b}_{-1}\tilde{J}^{b}_{-1}\right)\right)\\
\left.\phi(0)|0\right>=0.
\end{multline}
Так как набор  $\{\phi_{i}\}$ состоит из произвольных примарных полей, мы заключаем, что 
\begin{multline}
\label{eq:126}
|\psi>=\left(-2L_{-2}+\frac{1}{2}\kappa L_{-1}^{2}+\frac{1}{2}\tau \left(\sum_{a=1}^{\dim\gf}J^{a}_{-1}J^{a}_{-1}-\sum_{b=1}^{\dim\af}\tilde{J}^{b}_{-1}\tilde{J}^{b}_{-1}\right)\right)
\cdot\phi(0)|0>
\end{multline}
является нулевым состоянием, то есть соответствуют сингулярному весу в представлении алгебры Вирасоро. Действуя повышающими операторами мы получаем соотношения, связывающие параметры стохастического процесса и coset-модели конформной теории поля:
\begin{equation}
 (3\kappa-8)h_{(\mu,\nu)}-c+\tau (k\dim\gf-x_{e}k\dim\af) =0.
 \label{eq:184}
\end{equation}
\begin{equation}
\label{eq:185}
 -12 h_{(\mu,\nu)}+2\kappa h_{(\mu,\nu)} (2h_{(\mu,\nu)}+1) + \tau
(C_{\mu}-\tilde{C}_{\nu})=0,
\end{equation}
здесь $C_{\mu}=(\mu,\mu+2\rho)$ и $\tilde{C}_{\nu}=(\nu,\nu+2\rho_{\af})$ -- это собственные значения квадратичных операторов Казимира $\sum_{a}t^{a}t^{a}$ и $\sum_{b}\tilde{t}^{b}\tilde{t}^{b}$ алгебр Ли $\gf$ и $\af$.
Из уравнения  \eqref{eq:184},\eqref{eq:185} мы сразу получаем значения  $\kappa,\tau$ для каждой пары весов $(\mu,\nu)$ алгебр $\gf$ и $\af$. Для coset-реализаций минимальных и парафермионных моделей эти результаты совпадают с тем, что было ранее получено путем введения стохастического процесса с дополнительным дискретным случайным блужданием  \cite{santachiara2008sle}. 

Остальная часть главы представляет собой описание пакета {\bf Affine.m}, предназначенного для вычислений в теории представлений аффинных и конечномерных алгебр Ли и реализованного с использованием методов диссертации. Вычислительным методам посвящены наши работы \citemy{2011arXiv1107.4681N,NazarovACSM2009,Nazarov2008}.

%%  
%%  \textbf{Во Введении} обоснована актуальность диссертационной работы,
%%  сформулирована цель и аргументирована научная новизна исследований, показана
%%  практическая значимость полученных результатов, представлены выносимые на
%%  защиту научные положения.
%%  
%%  \textbf{В первой главе} ...
%%  
%%  Содержание первой главы.
%%  
%%  Результаты первой главы опубликованы в
%%  работе~\cite{2010arXiv1007.0318L}
%%  
%%  \textbf{Во второй главе} ...
%%  
%%  Содержание второй главы.
%%  
%%  Результаты второй главы опубликованы в
%%  работе~\citemy{Petrov_2001_Journal_23_12321}.
%%  
%%  \textbf{В третьей главе} ...
%%  
%%  Содержание третьей главы.
%%  
%%  Результаты третьей главы опубликованы в
%%  работе~\citemy{Sidorov_2002_Journal_32_1531}.
%%  
%%  \textbf{В Заключении}
%%  
% ----------------------------------------------------------------
\renewcommand\bibsection{\nsection{Список публикаций}}

% Префикс номеров ссылок на работы соискателя
\def\BibPrefix{A}
\bibliographystylemy{disser}
\bibliographymy{bibliography}

\renewcommand\bibsection{\nsection{Цитированная литература}}

\def\BibPrefix{}
\bibliographystyle{disser}
\bibliography{bibliography}
% ----------------------------------------------------------------

\end{document}
