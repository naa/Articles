\documentclass[14pt,autoref,href,facsimile
%,fixint=false
%,times
]{disser}

\usepackage[a4paper,nohead,includefoot,mag=1000,
            margin=2cm,footskip=1cm]{geometry}
\usepackage[T2A]{fontenc}
\usepackage[utf8x]{inputenc}
\usepackage[english,russian]{babel}
\usepackage{tabularx}
\ifpdf\usepackage{epstopdf}\fi

% Поддержка нескольких списков литературы в одном документе
\usepackage{multibib}
% Создание команд для цитирования собственных работ диссертанта
% в отдельном разделе. В данном случае ссылка будет иметь вид \citemy{...}.
\newcites{my}{Список публикаций}

% Путь к файлам с иллюстрациями
\graphicspath{{figures/}}


\usepackage{amsmath}
\usepackage{amsmath,amssymb,amsthm}
\usepackage{multicol}
\usepackage{color}
%\usepackage{graphicx}
\usepackage{ucs}
\usepackage{pb-diagram}
\usepackage{enumerate}
\usepackage{appendix}
\usepackage{listings}
\usepackage{textcomp}


\definecolor{listinggray}{gray}{0.9}
\definecolor{lbcolor}{rgb}{0.9,0.9,0.9}
\lstset{
%	backgroundcolor=\color{lbcolor},
        language=Mathematica,
	tabsize=2,
	rulecolor=,
        basicstyle=\scriptsize,
        upquote=true,
%        aboveskip={1.5\baselineskip},
        columns=fixed,
        showstringspaces=false,
        extendedchars=true,
%        breaklines=true,
        prebreak = \raisebox{0ex}[0ex][0ex]{\ensuremath{\hookleftarrow}},
%        frame=single,
        showtabs=false,
        showspaces=false,
        showstringspaces=false,
        identifierstyle=\ttfamily,
        keywordstyle=\color[rgb]{0,0,1},
        commentstyle=\color[rgb]{0.133,0.545,0.133},
        stringstyle=\color[rgb]{0.627,0.126,0.941},
}

\newtheorem{statement}{Утверждение}
\newtheorem{theorem}{Теорема}
\newtheorem{axiom}{Аксиома}
\newtheorem{corollary}{Следствие}[chapter]
\newtheorem{lemma}{Лемма}
\newtheorem{mynote}{Замечание}[chapter]
%%\newtheorem{Def}{Definition}[section]
\newtheorem{Def}{Определение}[chapter]
\newtheorem{Cnj}[Def]{Гипотеза}
\newtheorem{Prop}{Свойство}
%%\newtheorem{example}{Example}[section]

\theoremstyle{definition}
\newtheorem{definition}{Определение}
\newtheorem{remark}{Замечание}[chapter]
\newtheorem{example}{Пример}[chapter]
\newtheorem{exercise}{Упражнение}
\newtheorem{conjecture}{Гипотеза}[chapter]

\newcommand{\go}{\stackrel{\circ }{\mathfrak{g}}}
\newcommand{\ao}{\stackrel{\circ }{\mathfrak{a}}}
\newcommand{\co}[1]{\stackrel{\circ }{#1}}
\newcommand{\pia}{\pi_{\mathfrak{a}}}
\newcommand{\piab}{\pi_{\mathfrak{a}_{\perp}}}
\newcommand{\gf}{\mathfrak{g}}
\newcommand{\af}{\mathfrak{a}}
\newcommand{\uf}{\mathfrak{u}}
\newcommand{\sfr}{\mathfrak{s}}
\newcommand{\aft}{\widetilde{\mathfrak{a}}}
\newcommand{\afb}{\mathfrak{a}_{\perp}}
\newcommand{\hf}{\mathfrak{h}}
\newcommand{\hfb}{\mathfrak{h}_{\perp}}
\newcommand{\pf}{\mathfrak{p}}

\newcommand{\gfh}{\hat{\mathfrak{g}}}
\newcommand{\afh}{\hat{\mathfrak{a}}}
\newcommand{\bff}{\mathfrak{b}}
\newcommand{\hfg}{\hf_{\gf}}


\begin{document}
% Включение файла с общим текстом диссертации и автореферата
% (текст титульного листа и характеристика работы).
% Общие поля титульного листа диссертации и автореферата
\institution{Санкт-Петербургский Государственный Университет}

\topic{Бесконечномерные алгебры симметрии в моделях квантовой теории поля}

\author{Назаров Антон Андреевич}

\specnum{01.04.02}
\spec{Теоретическая физика}

\sa{Ляховский Владимир Дмитриевич}
\sastatus{д.~ф.-м.~н., проф.}

\city{Санкт-Петербург}
\date{\number\year}

% Общие разделы автореферата и диссертации
\mkcommonsect{actuality}{Актуальность работы}{%
Проблема вычисления коэффициентов ветвления для представлений алгебр Ли стоит уже многие десятилетия. Она актуальна для различных физических приложений. Вместе с тем, в отличие от кратностей весов не существует особенно эффективных алгоритмов.                                     
}

\mkcommonsect{objective}{Цель диссертационной работы}{%
Разработка рекуррентного подхода к функциям ветвления аффинных алгебр Ли, его связь с проблемами теории представлений и его приложения в моделях конформной теории поля.
}

\mkcommonsect{novelty}{Научная новизна}{%
Предложен эффективный алгоритм для вычисления коэффициентов ветвления, показана его связь с резольвентой Бернштейна-Гельфанда-Гельфанда.
}

\mkcommonsect{value}{Практическая значимость}{%
Результаты работы                                  
}

\mkcommonsect{results}{%
На защиту выносятся следующие основные результаты и положения:}{%
Текст раздела
}

\mkcommonsect{approbation}{Апробация работы}{%
Текст раздела
}

\mkcommonsect{pub}{Публикации.}{%
Материалы диссертации опубликованы в $8$ печатных работах, из них $3$ статьи в
рецензируемых журналах~\citemy{2010arXiv1007.0318L,2010LyakhovskyNazarovTMF}, $2$ статьи в
сборниках трудов конференций, 2 препринта и $1$ тезисы доклада.
}

\mkcommonsect{contrib}{Личный вклад автора}{%
Текст раздела
}

\mkcommonsect{struct}{Структура и объем диссертации}{%
Текст раздела
}

%%
%% End of file
%%% Local Variables: 
%%% mode: latex
%%% TeX-master: "thesis"
%%% End: 


\title{АВТОРЕФЕРАТ\\
диссертации на соискание ученой степени\\
кандидата физико-математических наук}

\maketitle

% Внутренняя сторона обложки
\noindent
\begin{center}
Работа выполнена на кафедре физики высоких энергий и элементарных частиц физического факультета Санкт-Петербургского государственного университета.
% \emph{название организации}.
\end{center}
\vskip1ex
\begin{tabularx}{\linewidth}{lp{1cm}X}
Научный руководитель:  & & \emph{доктор физико-математических наук}, \\
                       & & \emph{профессор}, \\
                       & & \emph{Ляховский Владимир Дмитриевич}
\\
Официальные оппоненты: & & \emph{доктор физико-математических наук}, \\
                       & & \emph{профессор}, \\
                       & & \emph{Кулиш Петр Петрович}\\
                       & & \emph{кандидат физико-математических наук}, \\
                       & & \emph{ученое звание}, \\
                       & & \emph{Мудров Андрей И}
\\
Ведущая организация:   & & \emph{Объединенный институт ядерных исследований}\\
\end{tabularx}

\vskip2ex\noindent
Защита состоится \datefield{} в \rule[0pt]{1cm}{0.5pt}\, часов
на заседании совета \emph{Д 212.232.24} по защите докторских и кандидатских диссертаций при \emph{Санкт-Петербургском государственном университете}, расположенном по адресу:
\emph{Санкт-Петербург, Средний пр. В.О., д. 41/43, ауд. 305}

\vskip1ex\noindent
С диссертацией можно ознакомиться в научной библиотеке
\emph{Санкт-Петербургского государственного университета}.

\vskip1ex\noindent
Автореферат разослан \datefield{}

\vskip2ex\noindent
%Отзывы и замечания по автореферату в двух экземплярах, заверенные
%печатью, просьба высылать по вышеуказанному адресу на имя ученого секретаря
%диссертационного совета.

\vfill\noindent
Ученый секретарь\\
диссертационного совета,\\
\emph{ученая степень}, \emph{ученое звание}%
\hfill
\makeatletter
% вставка файла, содержащего факсимиле ученого секретаря
\ifDis@facsimile
  \raisebox{-4pt}{\includegraphics[width=3cm]{sec-facsimile}}\hfill
\fi%
\makeatother%
\emph{фамилия и. о.}

\clearpage

\nsection{Общая характеристика работы}

% Актуальность работы
\actualitysection
\actualitytext

% Цель диссертационной работы
\objectivesection
\objectivetext

% Научная новизна
\noveltysection
\noveltytext

% Практическая значимость
%\valuesection
%\valuetext

% Результаты и положения, выносимые на защиту
\resultssection
\resultstext

% Апробация работы
\approbationsection
\approbationtext

% Публикации
\pubsection
\pubtext

% Личный вклад автора
%\contribsection
%\contribtext

% Структура и объем диссертации
\structsection
\structtext

% \contentsection
% \contenttext



\nsection{Содержание работы}


\textbf{Во Введении} обоснована актуальность диссертационной работы,
сформулирована цель и аргументирована научная новизна исследований, показана
практическая значимость полученных результатов, представлены выносимые на
защиту научные положения.

\textbf{Глава 1} носит обзорный характер. В ней мы даем аксиоматическую формулировку конформной теории поля, описываем модели Весса-Зумино-Новикова-Виттена и coset-модели. Затем мы демонстрируем роль аффинных алгебр в описании этих моделей и приводим основные понятия теории представлений, использующиеся в диссертации. Кроме того, мы обсуждаем конформную теорию поля на области с границей, так как она оказывается связана со стохастическим описанием решеточных моделей. 

\textbf{В главе 2} мы показываем, что структура сингулярного элемента определяет свойства модуля алгебры Ли, доказываем лемму о разложении сингулярного элемента и выводим основное рекуррунтное соотношение на коэффициенты ветвления. Основные результаты данной главы опубликованы в работе \citemy{2010arXiv1007.0318L}. 

Формула Вейля-Каца для формальных характеров интегрируемых модулей старшего веса конечномерных и аффинных алгебр Ли имеет вид
\begin{equation}
  \label{eq:1}
  \mathrm{ch} V^{(\mu)} = \frac{\Psi^{(\mu)}}{R},
\end{equation}
где $\Psi^{(\mu)}$ -- сингулярный элемент модуля, а $R=\prod_{\alpha\in \Delta^+}(1-e^{-\alpha})^{\mathrm{mult}(\alpha)}$ -- знаменатель Вейля. Сингулярный элемент определяется набором сингулярных весов модуля и имеет разный вид для разных типов модулей старшего веса. Например, $\Psi^{(\mu)}=\sum_{w\in W} \epsilon(w) e^{w(\mu+\rho)-\rho}$ для неприводимых модулей. Заметим, что знаменатель Вейля является универсальным объектом, характеризующим корневую систему алгебры Ли, а свойства модуля определяются сингулярным элементом.

Процедура редукции состоит в разложении модуля алгебры Ли $\gf$ в сумму модулей некоторой подалгебры $\af$
\begin{equation}
  \label{eq:2}
  L_{\gf\downarrow \af}^{\mu }=\bigoplus
\limits_{\nu \in P_{\af}^{+}}b_{\nu }^{\left( \mu \right) }L_{\af}^{\nu }.
\end{equation}
Используя оператор проекции  $\pi_{\af}$ (на весовое пространство $\hf_{\af}^*$), перепишем это разложение для формальных характеров:
\begin{equation}
\label{branching1}
 \pi _{\af}\circ ch\left( L^{\mu }\right)
 =\sum_{\nu \in P_{\af}^{+}}b_{\nu }^{(\mu)}ch\left( L_{\af}^{\nu }\right) .
\end{equation}
Нас интересуют коэффициенты ветвления $b^{(\mu)}_{\nu}$.

Для любой алгебры $\gf$ и подалгебры $\af\subset \gf$ можно построить  подалгебру $\afb$ такую, что 
\begin{eqnarray}
\label{delta-a-ort}
\Delta _{\af_{\perp }} &:&=\left\{ \beta \in \Delta _{\gf}|
\forall h \in \hf_{\af};  \beta\left(h \right)=0  \right\} , \\
\Delta _{\af_{\perp }}^{+} &:&=\left\{ \beta ^{+}\in \Delta _{\gf%
}^{+}|\forall h \in \hf_{\af};  \beta^{+}\left(h \right)=0  \right\} .
\end{eqnarray}
Обозначим через $W_{\afb}$ подгруппу группы Вейля $W$, порожденную отражениями $w _{\beta }$, соответствующими корням $\beta \in \Delta _{\afb}^{+}$ . Подсистема  $\Delta _{\af_{\perp }}$ определяет подалгебру $\af_{\perp }$ с подалгеброй Картана $\hf_{\afb}$. 

Мы доказываем следующую лемму о разложении сингулярного элемента:
\begin{lemma}
\label{lemma}
Пусть $\left( \af,\afb \right)$ -- ортогональная пара редуктивных подалгебр $\gf$ и  $\widetilde{\afb}=\afb\oplus \hf_{\perp }$, $\widetilde{\af}=\af\oplus\hf_{\perp }$ ,

$L^{\mu }$ -- модуль старшего веса с сингулярным элементом $\Psi ^{\left(\mu \right)}$ ,

$R_{\af_{\perp }}$ -- знаменатель Вейля для подалгебры $\af_{\perp }$.

Тогда элемент  $\Psi ^{\left( \mu \right) }_{\left(  \af, \afb \right)}=\pi _{\af}\left( \frac{\Psi _{\gf}^{\mu }}{R_{\af_{\perp }}}\right) $ можно разложить в сумму по  $u\in U$ (см. (\ref{U-def})) сингулярных весов $e^{\mu _{\af}\left( u\right) }$ с коэффициентами $\epsilon (u)\mathrm{\dim}\left( L_{\widetilde{\afb}}^{\mu _{\widetilde{\afb}}\left( u\right) }\right) $:
\begin{equation}
\Psi ^{\left( \mu \right) }_{\left(  \af, \afb \right)}=\quad \pi _{\af}\left( \frac{\Psi^{\mu }}{R_{\af%
_{\perp }}}\right) =\sum_{u\in U}\;\epsilon (u)\mathrm{\dim }
\left( L_{\widetilde{\af_{\perp }}}^{\mu _{%
\widetilde{\af_{\perp }}}\left( u\right) }\right) e^{\mu _{\af}\left( u \right) }.
\end{equation}
\end{lemma}

Используя эту лемму мы выводим следующее рекуррентное соотношение на сингулярные коэффициенты ветвления

\begin{theorem}
  Для сингулярных коэффициентов ветвления $k^{(\mu)}_{\nu}$ (\ref{eq:21}) выполняется соотношение
  \begin{equation}
    \label{recurrent-relation}
    \begin{array}{c}
      k_{\xi }^{\left( \mu \right) }=-\frac{1}{s\left( \gamma _{0}\right) }\left(
        \sum_{u\in U} \epsilon(u)\;
        \dim \left( L_{\widetilde{\af_{\perp }}}^{\mu
        _{\widetilde{\af_{\perp }}}\left( u\right) }\right)
        \delta_{\xi-\gamma_0,\pi_{\af}(u(\mu+\rho)-\rho)}+ \right.\\
      \left.
        +\sum_{\gamma \in
          \Gamma _{\af \rightarrow \gf}}s\left( \gamma +\gamma _{0}\right) k_{\xi
          +\gamma }^{\left( \mu \right) }\right).
    \end{array}
  \end{equation}
\end{theorem}

Затем мы формулируем рекурсивный алгоритм и показываем, он позволяет эффективно вычислять коэффициенты ветвления для конечномерных и аффинных алгебр Ли. 


В следующей \textbf{главе 3} мы используем  разложение сингулярного элемента, чтобы показать связь ветвления с (обобщенной) резольвентой Бернштейна-Гельфанда-Гельфанда. Результаты третьей главы опубликованы в работах \citemy{2011arXiv1102.1702L,2010LyakhovskyNazarovMQFT}

\textbf{Глава 4} посвящена сплинтам -- расщеплением корневой системы алгебры Ли в объединение образов корневых систем двух алгебр, не обязательно являющихся подалгебрами данной алгебры. Если одна из алгебр является подалгеброй, то сплинт приводит к резкому упрощению в вычислении коэффициентов ветвления -- они совпадают с кратностями весов в модуле другой алгебры. Основная часть главы посвящена доказательству этого факта. Кроме того, сплинт корневой системы простой конечномерной алгебры Ли приводит к возникновению новых соотношений на струнные функции и функции ветвления соответствующего аффинного расширения. Эти соотношения обсуждаются в разделе 4.4.
Данные результаты опубликованы в статьях \citemy{2011arXiv1111.6787L,2012arXiv1204.1855L}.

Заключительная \textbf{глава 5} посвящена практическим приложениям результатов диссертации. В разделе 5.1 мы  описываем применение алгебраических методов к проблеме поиска соответствия между квантовополевым и решеточным описанием критического поведения. Эти результаты были опубликованы нами в работах \citemy{NazarovJETPletters,2011arXiv1112.4354N}. Раздел 5.2 представляет собой описание пакета {\bf Affine.m}, предназначенного для вычислений в теории представлений аффинных и конечномерных алгебр Ли и реализованного с использованием методов диссертации. Вычислительным методам посвящены наши работы \citemy{2011arXiv1107.4681N,NazarovACSM2009,Nazarov2008}.

%%  
%%  \textbf{Во Введении} обоснована актуальность диссертационной работы,
%%  сформулирована цель и аргументирована научная новизна исследований, показана
%%  практическая значимость полученных результатов, представлены выносимые на
%%  защиту научные положения.
%%  
%%  \textbf{В первой главе} ...
%%  
%%  Содержание первой главы.
%%  
%%  Результаты первой главы опубликованы в
%%  работе~\cite{2010arXiv1007.0318L}
%%  
%%  \textbf{Во второй главе} ...
%%  
%%  Содержание второй главы.
%%  
%%  Результаты второй главы опубликованы в
%%  работе~\citemy{Petrov_2001_Journal_23_12321}.
%%  
%%  \textbf{В третьей главе} ...
%%  
%%  Содержание третьей главы.
%%  
%%  Результаты третьей главы опубликованы в
%%  работе~\citemy{Sidorov_2002_Journal_32_1531}.
%%  
%%  \textbf{В Заключении}
%%  
% ----------------------------------------------------------------
\renewcommand\bibsection{\nsection{Список публикаций}}

% Префикс номеров ссылок на работы соискателя
\def\BibPrefix{A}
\bibliographystylemy{disser}
\bibliographymy{bibliography}

\renewcommand\bibsection{\nsection{Цитированная литература}}

\def\BibPrefix{}
\bibliographystyle{disser}
\bibliography{bibliography}
% ----------------------------------------------------------------

\end{document}
