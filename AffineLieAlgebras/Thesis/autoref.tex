\documentclass[14pt,autoref,href,facsimile
%,fixint=false
%,times
]{disser}

\usepackage[a4paper,nohead,includefoot,mag=1000,
            margin=2cm,footskip=1cm]{geometry}
\usepackage[T2A]{fontenc}
\usepackage[utf8x]{inputenc}
\usepackage[english,russian]{babel}
\usepackage{tabularx}
\ifpdf\usepackage{epstopdf}\fi

% Поддержка нескольких списков литературы в одном документе
\usepackage{multibib}
% Создание команд для цитирования собственных работ диссертанта
% в отдельном разделе. В данном случае ссылка будет иметь вид \citemy{...}.
\newcites{my}{Список публикаций}

% Путь к файлам с иллюстрациями
\graphicspath{{figures/}}

\begin{document}
% Включение файла с общим текстом диссертации и автореферата
% (текст титульного листа и характеристика работы).
% Общие поля титульного листа диссертации и автореферата
\institution{Санкт-Петербургский Государственный Университет}

\topic{Бесконечномерные алгебры симметрии в моделях квантовой теории поля}

\author{Назаров Антон Андреевич}

\specnum{01.04.02}
\spec{Теоретическая физика}

\sa{Ляховский Владимир Дмитриевич}
\sastatus{д.~ф.-м.~н., проф.}

\city{Санкт-Петербург}
\date{\number\year}

% Общие разделы автореферата и диссертации
\mkcommonsect{actuality}{Актуальность работы}{%
Проблема вычисления коэффициентов ветвления для представлений алгебр Ли стоит уже многие десятилетия. Она актуальна для различных физических приложений. Вместе с тем, в отличие от кратностей весов не существует особенно эффективных алгоритмов.                                     
}

\mkcommonsect{objective}{Цель диссертационной работы}{%
Разработка рекуррентного подхода к функциям ветвления аффинных алгебр Ли, его связь с проблемами теории представлений и его приложения в моделях конформной теории поля.
}

\mkcommonsect{novelty}{Научная новизна}{%
Предложен эффективный алгоритм для вычисления коэффициентов ветвления, показана его связь с резольвентой Бернштейна-Гельфанда-Гельфанда.
}

\mkcommonsect{value}{Практическая значимость}{%
Результаты работы                                  
}

\mkcommonsect{results}{%
На защиту выносятся следующие основные результаты и положения:}{%
Текст раздела
}

\mkcommonsect{approbation}{Апробация работы}{%
Текст раздела
}

\mkcommonsect{pub}{Публикации.}{%
Материалы диссертации опубликованы в $8$ печатных работах, из них $3$ статьи в
рецензируемых журналах~\citemy{2010arXiv1007.0318L,2010LyakhovskyNazarovTMF}, $2$ статьи в
сборниках трудов конференций, 2 препринта и $1$ тезисы доклада.
}

\mkcommonsect{contrib}{Личный вклад автора}{%
Текст раздела
}

\mkcommonsect{struct}{Структура и объем диссертации}{%
Текст раздела
}

%%
%% End of file
%%% Local Variables: 
%%% mode: latex
%%% TeX-master: "thesis"
%%% End: 


\title{АВТОРЕФЕРАТ\\
диссертации на соискание ученой степени\\
кандидата физико-математических наук}

\maketitle

% Внутренняя сторона обложки
\noindent
\begin{center}
Работа выполнена на кафедре физики высоких энергий и элементарных частиц физического факультета Санкт-Петербургского государственного университета.
% \emph{название организации}.
\end{center}
\vskip1ex
\begin{tabularx}{\linewidth}{lp{1cm}X}
Научный руководитель:  & & \emph{доктор физико-математических наук}, \\
                       & & \emph{профессор}, \\
                       & & \emph{Ляховский Владимир Дмитриевич}
\\
Официальные оппоненты: & & \emph{доктор физико-математических наук}, \\
                       & & \emph{профессор}, \\
                       & & \emph{Кулиш Петр Петрович}\\
                       & & \emph{кандидат физико-математических наук}, \\
                       & & \emph{ученое звание}, \\
                       & & \emph{Мудров Андрей И}
\\
Ведущая организация:   & & \emph{Объединенный институт ядерных исследований}\\
\end{tabularx}

\vskip2ex\noindent
Защита состоится \datefield{} в \rule[0pt]{1cm}{0.5pt}\, часов
на заседании совета \emph{Д 212.232.24} по защите докторских и кандидатских диссертаций при \emph{Санкт-Петербургском государственном университете}, расположенном по адресу:
\emph{Санкт-Петербург, Средний пр. В.О., д. 41/43, ауд. 305}

\vskip1ex\noindent
С диссертацией можно ознакомиться в научной библиотеке
\emph{Санкт-Петербургского государственного университета}.

\vskip1ex\noindent
Автореферат разослан \datefield{}

\vskip2ex\noindent
%Отзывы и замечания по автореферату в двух экземплярах, заверенные
%печатью, просьба высылать по вышеуказанному адресу на имя ученого секретаря
%диссертационного совета.

\vfill\noindent
Ученый секретарь\\
диссертационного совета,\\
\emph{ученая степень}, \emph{ученое звание}%
\hfill
\makeatletter
% вставка файла, содержащего факсимиле ученого секретаря
\ifDis@facsimile
  \raisebox{-4pt}{\includegraphics[width=3cm]{sec-facsimile}}\hfill
\fi%
\makeatother%
\emph{фамилия и. о.}

\clearpage

\nsection{Общая характеристика работы}

% Актуальность работы
\actualitysection
\actualitytext

% Цель диссертационной работы
\objectivesection
\objectivetext

% Научная новизна
\noveltysection
\noveltytext

% Практическая значимость
%\valuesection
%\valuetext

% Результаты и положения, выносимые на защиту
\resultssection
\resultstext

% Апробация работы
\approbationsection
\approbationtext

% Публикации
\pubsection
\pubtext

% Личный вклад автора
%\contribsection
%\contribtext

% Структура и объем диссертации
\structsection
\structtext

\contentsection
\contenttext

%%  \nsection{Содержание работы}
%%  
%%  \textbf{Во Введении} обоснована актуальность диссертационной работы,
%%  сформулирована цель и аргументирована научная новизна исследований, показана
%%  практическая значимость полученных результатов, представлены выносимые на
%%  защиту научные положения.
%%  
%%  \textbf{В первой главе} ...
%%  
%%  Содержание первой главы.
%%  
%%  Результаты первой главы опубликованы в
%%  работе~\cite{2010arXiv1007.0318L}
%%  
%%  \textbf{Во второй главе} ...
%%  
%%  Содержание второй главы.
%%  
%%  Результаты второй главы опубликованы в
%%  работе~\citemy{Petrov_2001_Journal_23_12321}.
%%  
%%  \textbf{В третьей главе} ...
%%  
%%  Содержание третьей главы.
%%  
%%  Результаты третьей главы опубликованы в
%%  работе~\citemy{Sidorov_2002_Journal_32_1531}.
%%  
%%  \textbf{В Заключении}
%%  
% ----------------------------------------------------------------
\renewcommand\bibsection{\nsection{Список публикаций}}

% Префикс номеров ссылок на работы соискателя
\def\BibPrefix{A}
\bibliographystylemy{disser}
\bibliographymy{bibliography}

\renewcommand\bibsection{\nsection{Цитированная литература}}

\def\BibPrefix{}
\bibliographystyle{disser}
\bibliography{bibliography}
% ----------------------------------------------------------------

\end{document}
