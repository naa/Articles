\documentclass{article}
\usepackage{graphicx}
\usepackage{amsmath,amssymb,amsthm} 
\usepackage{pb-diagram}
\usepackage{ucs}
\usepackage[utf8x]{inputenc}
\usepackage[russian]{babel}
\usepackage{epstopdf}
\usepackage{multicol}
\usepackage{cancel}

\usepackage{amsfonts}

%%%%%%%%%%%%%%%%%%%%%%%%%%%%%%%%%%%%%%%%%%%%%%%%%%%%%%%%%%%%%%%%%%%%%%%%%%%%%%%%%%%%%%%%%%%%%%%%%%%

%\newtheorem{theorem}{Theorem}
%% \newtheorem{acknowledgement}[theorem]{Acknowledgement}
%% \newtheorem{algorithm}[theorem]{Algorithm}
%% \newtheorem{axiom}[theorem]{Axiom}
%% \newtheorem{case}[theorem]{Case}
%% \newtheorem{claim}[theorem]{Claim}
%% \newtheorem{conclusion}[theorem]{Conclusion}
%% \newtheorem{condition}[theorem]{Condition}
%% \newtheorem{conjecture}[theorem]{Conjecture}
%% \newtheorem{mycorollary}[theorem]{Corollary}
%% \newtheorem{mycriterion}[theorem]{Criterion}
%% \newtheorem{mydefinition}[theorem]{Definition}
%% \newtheorem{myexample}[theorem]{Example}
%% \newtheorem{myexercise}[theorem]{Exercise}
%% \newtheorem{mynotation}[theorem]{Notation}
%% \newtheorem{myproblem}[theorem]{Problem}
%% \newtheorem{myproposition}[theorem]{Proposition}
%% \newtheorem{myremark}[theorem]{Remark}
%% \newtheorem{mysolution}[theorem]{Solution}
%% \newtheorem{mysummary}[theorem]{Summary}
%% \newenvironment{myproof}[1][Proof]{\textbf{#1.} }{\ \rule{0.5em}{0.5em}}


\newcommand{\go}{\stackrel{\circ }{\mathfrak{g}}}
\newcommand{\ao}{\stackrel{\circ }{\mathfrak{a}}}
\newcommand{\co}[1]{\stackrel{\circ }{#1}}
\newcommand{\pia}{\pi_{\mathfrak{a}}}
\newcommand{\piab}{\pi_{\mathfrak{a}_{\bot}}}
\newcommand{\gf}{\mathfrak{g}}
\newcommand{\gfh}{\hat{\mathfrak{g}}}
\newcommand{\af}{\mathfrak{a}}
\newcommand{\afh}{\hat{\mathfrak{a}}}
\newcommand{\bff}{\mathfrak{b}}
\newcommand{\afb}{\mathfrak{a}_{\bot}}
\newcommand{\hf}{\mathfrak{h}}
\newcommand{\hfg}{\hf_{\gf}}
\newcommand{\hfb}{\mathfrak{h}_{\bot}}
\newcommand{\pf}{\mathfrak{p}}
\newcommand{\aft}{\widetilde{\mathfrak{a}}}
\newcommand{\sfr}{\mathfrak{s}}
%\pagestyle{plain}

\theoremstyle{definition} \newtheorem{Def}{Определение}
\newcommand{\tr}{\hat\triangleright} \newcommand{\trc}{\triangleright}
\newcommand{\adk}{a^{\dagger}_{\kappa}} \newcommand{\ak}{a_{\kappa}}
\def\bF{\mbox{$\overline{\cal F}$}} \def\F{\mbox{$\cal F$}}

\topmargin=-3.5cm
%\oddsidemargin=-1.5cm
%\evensidemargin=-1.5cm
%\textwidth=19cm
\textheight=28cm

\begin{document}

Спасибо! Добрый день! 
Моя диссертация называется ``Правила ветвления аффинных алгебр Ли и приложения в моделях конформной теории поля''.

Сперва о теме исследования. Конформная теория поля в двух измерениях появилась для описания критического поведения в двумерных решеточных моделях. Также двумерная конформная теория поля играет существенную роль в теории струн (если записать действие Полякова для релятивистской струны, а потом вычислить эффективное действие для полей на мировой поверхности, полученная теория будет конформной теорией поля). 

Конформная теория поля интересна еще и тем, что она является точно решаемой, то есть корреляционные функции можно найти без использования теории возмущений. Конформная теория поля выделяется еще и как квантовая теория поля со взаимодействием, допускающая аксиоматическую формулировку. 

Модели Весса-Зумино-Новикова-Виттена и coset-модели -- это два важных класса моделей конформной теории поля. Они реализуют рациональные модели конформной теории поля, то есть для любого рационального значения центрального заряда существует соответствующая ВЗНВ или coset-модель. В этих теориях имеется калибровочная инвариантность, которая сильно помогает в исследовании. 

Наличие калибровочной инвариантности приводит к возникновению аффинных алгебр Ли. Аффинные алгебры Ли -- это важный класс бесконечномерных алгебр Ли, их теория представлений хорошо развита. Таким образом, у нас есть инструмент для исследования моделей конформной теории поля. Кроме того, аффинные алгебры Ли связаны с теорией специальных функций и модулярных форм, но на этих вопросах мы не будем сейчас останавливаться. 

Задача редукции представления аффинной алгебры Ли на представления подалгебры возникает при изучении структуры пространства состояний в ВЗНВ и coset-моделях. 
Эта задача нетривиальна, ей давно занимаются, но окончательной формулы для коэффициентов ветвления пока не существует. 

Теперь я перечислю основные результаты диссертации. В диссертации впервые решены следующие задачи: получено эффективное рекуррентное соотношение для вычисления коэффициентов ветвления; создан пакет для популярной системы символьных вычислений Mathematica, позволяющий решать различные задачи теории представлений аффинных и конечномерных алгебр Ли; показано, что разложение сингулярного элемента, которое использовалось для вывода рекуррентного соотношения, определяет как коэффициенты ветвления, так и обобщенную резольвенту Бернштейна-Гельфанда-Гельфанда; установлена прямая связь расщепления корневой системы и ветвления, указано, при каких условиях кратности весов модуля вспомогательной алгебры совпадают с коэффициентами ветвления; предложено обобщение стохастического процесса Шрамма-Лёвнера на системы с калибровочной инвариантностью и исследованы требования на операторы, соответствующие изменению граничных условий в решеточных моделях. 

Отмечу, что пакет Affine.m подходит для решения задач теории представлений в различных областях физики, от изучения спектров атомов и молекул и до конформной теории поля и квантовых интегрируемых систем. 

Диссертации имеет следующую структуру. Первая глава носит обзорных характер и содержит основные понятия двумерной конформной теории поля и теории представлений аффинных алгебр Ли. 
Во второй главе вводится разложение сингулярного элементы и выводится рекуррентное соотношение на коэффициенты ветвления. Затем приводятся примеры применения рекуррентного соотношения для вычисления коэффициентов ветвления аффинных и конечномерных алгебр Ли и примеры приложений в моделях конформной теории поля. 
Третья глава посвящена исследованию связи ветвления и обобщенной резольвенты Бернштейна-Гельфанда-Гельфанда. Показано, что разложение сингулярного элемента, определяющее коэффициенты ветвления, приводит к построению обобщенных модулей Верма, образующих точную последовательность.
В четвертой главе рассматривается связь расщеплений корневых систем алгебр Ли с ветвлением. Показано, что при выполнении определенных условий на структуру сингулярного элемента, коэффициенты ветвления совпадают с кратностями весов в модуле вспомогательной алгебры. 
Пятая глава посвящена практическим приложениям. В первом разделе мы обсуждаем обобщение эволюции Шрамма-Лёвнера, использующейся для описания доменных стенок в решеточных моделях. Во втором разделе описывается компьютерная программа Affine.m. 

Перейду теперь к содержанию и объяснению основных результатов диссертации.

Модели Весса-Зумино-Новикова-Виттена имеют следующее действие. Первый член -- это действие нелинейной сигма-модели. При его квантовании конформная инвариантность нарушается, поэтому требуется второй топологический член, который восстанавливает конформную симметрию. Действие записано для поля же, которое задано на комплексной плоскости и принимает значения в группе Ли же. Так как в действии содержатся производные же, то нас, на самом деле интересует не группа, а соответствующая алгебра Ли. В этой теории есть калибровочная инвариантность, которой соответствуют сохраняющиеся токи. Если разложить эти токи по степеням зет и по базису в алгебре ли, то компоненты этого разложения будут удовлетворять коммутационным соотношениям аффинной алгебры Ли. (Аффинная алгебра Ли -- это центральное расширение алгебры петель). 
Из конформной инвариантности следует, что алгебры Вирасоро, соответствующая конформным преобразованиям, содержится в универсальной обертывающей аффинной алгебры Ли. То есть ее генераторы можно представить в виде комбинаций генераторов аффинной алгебры. Пространство состояний теории раскладывается на представления аффинной алгебры Ли. 
Coset-модели являются обобщением моделей Весса-Зумино-Новикова-Виттена. Мы добавляем к действию взаимодействие с чисто калибровочным полем, принимающим значения в некоторой подалгебре а алгебры же. 
Конформная инвариантность сохраняется, выражения для генераторов алгебры Вирасоро -- разности выражений для алгебр же и а. 
Заметим, что векторы или состояния, которые сингулярны по отношению к аффинной подалгебре а, то есть зануляются понижающими операторами, образуют модули алгебры Вирасоро. Функции ветвления, которые определяются коэффициентами ветвления, -- это характеры этих модулей алгебры Вирасоро. То есть вычисляя коэффициенты ветвления мы исследуем структуру пространства состояний coset-модели. Примарные поля в coset-моделях нумеруются парами весов алгебры и подалгебры, для которых функция ветвления отлична от нуля, то есть существуют ненулевые коэффициенты ветвления. 

Теперь несколько слов о модулях аффинных алгебр Ли, которые исследуются в данной работе. 
Во-первых, модули Верма. Это бесконечномерные модули с очень простой структурой, подобной конструкции пространства Фока - просто все состояния, которые получаются действием на вектор старшего веса операторами рождения. 

У модуля Верма есть единственный максимальный подмодуль, если по нему факторизовать, то есть выкинуть входящие в него состояния, получится неприводимый модуль. Его структура может быть записана в виде формального характера, для которого верно такое выражение. Наверху стоит сингулярный элемент, содержащий набор сингулярных весов. Внизу -- знаменатель Вейля. Характер неприводимого модуля можно представить в виде комбинации характеров модулей Верма. 
Сингулярный элемент определяет основные свойства модуля. 

Задача редукции ставится следующим образом. Пусть алгебра же и ее подалгебра а -- конечномерные или аффинные. Хотим разложить неприводимый модуль алгебры же на неприводимые модули алгебры а и вычислить коэффициенты этого разложения, то есть сколько раз каждый из неприводимых модулей входит в разложение. 

Мы получили рекуррентное соотношение на эти коэффициенты. Сейчас я его объясню. 
Рассмотрим, например, модуль алгебры эс-о-пять. Здесь приведены кратности весов в нем и его сингулярные веса пунктиром. Рассмотрим подалгебру эс-о-три, построенную на этом корне. Оказывается, что существует вспомогательная подалгебра (в данном случае - тоже эс-о-три), все корни которой ортогональны корням исходной подалгебры. Размерности неприводимых модулей этой подалгебры, построенных в сингулярных весах, являются начальными данными для рекуррентного вычисления коэффициентов ветвления. Такие модули и их размерности изображены на третьем рисунке. Для вычисления размерностей сам модуль строить не нужно, поэтому построение начальных данных -- это очень простая процедура. 

Рекуррентное соотношение имеет следующий вид: в него входят начальные точки - точки сингулярного элемента с размерностями. И сумма уже вычисленных коэффициентов ветвления, домноженных на соответствующие значения функции эс, которая вычисляется из разложения знаменателей Вейля. Она универсальна, то есть не зависит от того, какой модуль мы раскладываем, а только от того, какую алгебру и подалгебру мы рассматриваем. Набор формальных экспонент весов гамма с коэффициентами эс от гамма называется веером вложения.

Замечу, что если подалгебра -- это подалгебра Картана, то рекуррентное соотношение будет давать кратности весов в неприводимом модуле.

Вот пример применения этой рекуррентной формулы для вычисления коэффициентов ветвления для аффинных алгебр Ли. Здесь алгебра -- это аффинное расширение алгебры эс-о-пять, а подалгебра -- аффинное расширение алгебры эс-о-три. Тут показан набор весов и значений функции эс, определяющих рекурсию, а вот здесь начальные данные. На последнем рисунке результат вычисления коэффициентов ветвления. Выделена главная камера Вейля. 

Выводы ко второй главе состоят в том, что мы показали, что коэффициенты ветвления нужны для вычисления спектра возбуждений в моделях Весса-Зумино-Новикова-Виттена, функции ветвления дают статсумму для coset-моделей, мы получили новую рекуррентную формулу на коэффициенты ветвления, которая не требует лишних вычислений. В частности, не нужно строить модули, входящие в разложение. Вот на этом рисунке приведено время вычисления коэффициентов ветвления модулей эс-о-девять на модули эс-о-семь в зависимости от размерности модуля. Пунктиром - традиционный алгоритм, требующий построения модулей, входящих в разложение, сплошная -- наш рекуррентный алгоритм. 

То есть мы продемонстрировали эффективность нашего алгоритма.

Перехожу к третьей главе, которая посвящена связи ветвления и обобщенной резольвенты Бернштейна-Гельфанда-Гельфанда. 

Зачем вообще нам нужно выражать характер неприводимого модуля через характеры модулей Верма? Ведь в случае конечномерных алгебр неприводимые модули конечномерны, а модули Верма -- нет?! Посмотрите на рисунок. Здесь изображен модуль Верма алгебры эс-о-пять. Видно, что его структура довольно проста. То есть идея состоит в исследовании модуля путем погружения его в модуль большой размерности, но имеющий простую структуру.

Справа показан обобщенный или параболический модуль Верма той же алгебры. Он строится не из одного старшего веса, а из неприводимого модуля некоторой параболической подалгебры.  В данном случае это подалгебра эс-о-три, построенная на вот этом корне.  Характер обобщенного модуля Верма -- это отношение характера неприводимого модуля подалгебры к вееру вложения этой подалгебры.

Как же использовать обобщенные модули Верма для изучения неприводимого модуля? Оказывается, что характер неприводимого модуля является комбинацией с разными знаками. Вот на этом рисунке это показано. Здесь вы видите набор обобщенных модулей Верма, комбинация характеров которых дает характер вот этого неприводимого модуля. Какие же обобщенные модули здесь изображены? Эти модули построены из тех неприводимых представлений вспомогательной подалгебры, размерности которых давали нам начальные данные для вычисления коэффициентов ветвления на вот эту подалгебру!
Суммы таких модулей образуют точную последовательность обобщенной резольвенты. Здесь представлена центральная часть этой точной последовательности. 

Итак, по результатам данной главы можно сделать следующие выводы. Использование веера вложения приводит к построению параболических модулей Верма. Обобщенные модули Верма образуют точную последовательность обобщенной резольвенты Бернштейна-Гельфанда-Гельфанда. Кроме того, разложение сингулярного элемента, которое определяет коэффициенты ветвления, определяет и обобщенную резольвенту. Итак, мы установили связь ветвления и обобщенной БГГ-резольвенты.

В четвертой главе рассматриваются сплинты или расщепления корневой системы. Чтобы объяснить понятие расщепления, я начну с простого примера. Рассмотрим корневую систему алгебры эс-о-пять. Она состоит из восьми корней. Если мы выкинем корни некоторой подалгебры эс-о-три, то у нас останется шесть корней, между которыми есть те же соотношения, что и в корневой системе алгебры эс-эль-три, хотя углы и длины корней другие. Оказывается, что если мы будем раскладывать неприводимый модуль эс-о-пять на модули эс-о-три, коэффициенты ветвления совпадут с кратностями весов эс-эль-три!
Здесь приведено формальное определение расщепления. 

Из нашего рекуррентного соотношения, которое дает как коэффициенты ветвления, так и кратности весов, можно понять, почему происходит совпадение.
В каких же случаях оно имеет место?
Оказывается, что для этого должно быть выполнено следующее условие -- сингулярный элемент модуля алгебры должен представляться в виде комбинации образов сингулярных элементов вспомогательной алгебры, входящей в расщепление. Вот здесь перечислены все случаи для простых конечномерных алгебр Ли. 
В качестве еще одного примера я покажу сингулярный элемент, его разложение и коэффициенты ветвления для модуля алгебры же-два на модули подалгебры а-два. 
Итак, выводы к четвертой главе. Мы проанализировали связь расщеплений с ветвлением, доказали, что расщепление приводит к совпадению коэффициентов ветвления с кратностями весов вспомогательной алгебры. Затем мы рассмотрели аффинные расширения конечномерных алгебр Ли с расщепляющейся корневой системой и получили для них новые соотношения на тета-функции и функции ветвления аффинных алгебр на конечномерные подалгебры. Однако эти соотношения не такие простые, как простое совпадение в случае конечномерных алгебр, поэтому я не привожу их в презентации. 

Пятая глава посвящена практическим приложениям. В первом разделе мы рассматриваем обобщение стохастического процесса Шрамма-Левнера. Стохастический процесс Шрамма-Левнера служит для описания критического поведения доменных стенок в решеточных моделях. Рассмотрим решеточную модель, например, модель Изинга, на верхней полуплоскости. Пусть заданы граничные условия. Например, на половине границы спины вверх, а на половине -- вниз. Тогда у нас будет доменная стенка. Мы можем параметризовать ее параметром т. Если мы сделаем разрез вдоль стенки, то область с разрезом можно отобразить на верхнюю полуплоскость конформным отображением. Оказывается, что в критической точке в термодинамическом пределе такое отображение удовлетворяет стохастическому дифференциальному уравнению. 

Можно рассмотреть решеточную наблюдаемую в присутствии доменной стенки. Среднее значение такой наблюдаемой складывается из суммы по всем траекториям средних значений в областях с разрезом. Так как само среднее значение не зависит от параметризации стенки, то приращение среднего значения в области со стенкой при росте стенки не должно зависеть от времени. Такая наблюдаемая называется мартингалом. 

Критическое поведение решеточной модели описывается конформной теории поля, поэтому наблюдаемой должна соответствовать корреляционная функция в конформной теории поля. Она содержит операторы смены граничного условия на конце стенки и на бесконечности. Мы можем прирастить стенку, а потом сделать отображение на всю полуплоскость. При этом входящие в корреляционную функцию поля преобразуются так. Из требования определенного поведения при росте доменной стенки получается уравнение на корреляционную функцию. Это уравнение можно переписать в виде алгебраического соотношения на оператор смены граничного условия. Так можно классифицировать операторы смены граничного условия. 
В работе предложено обобщение данной конструкции. Рассмотрим аналог coset-модели, где вместе с ростом стенки производится случайное калибровочное преобразование. 
Для такого процесса тоже можно рассмотреть поведение наблюдаемых при росте стенки. Отсюда получается следующее условие на оператор смены граничного условия: такой вектор должен быть сингулярным. То есть перечислить возможные операторы смены граничного условия можно исследуя структуру сингулярных элементов в модулях аффинных алгебр Ли. Заметим, что все конформные теории поля, которые соответствуют решеточным моделям, можно получить coset-конструкцией. 

Второй раздел пятой главы посвящен описанию компьютерной программы аффайн-эм, которая использовалась для создания почти всех примеров и иллюстраций в данной презентации. Покажу еще пару примеров. Вот вычисление струнных функций для аффинной алгебры эс-эль-три. А вот модули алгебры эс-о-пять. 

Итак, в пятой главе было предложено обобщение эволюции Шрамма-Левнера, которое соответствует coset-моделям в конформной теории. Получены алгебраические соотношения на оператор смены граничных условий. Классификация таких операторов связана со структурой сингулярных элементов. Реализован пакет аффайн-эм для популярной системы символьных вычислений Математика. Этот пакет может быть полезен для решения различных задач. 

Еще раз перечислю основные результаты работы. 

Во-первых, получены новые рекуррентные соотношения на коэффициенты ветвления. 

Показана связь редукции с обобщенной резольвентой Бернштейна-Гельфанда-Гельфанда. 

Выявлена связь расщепления корневой системы алгебры с разложением сингулярных элементов модулей алгебры в комбинацию образов сингулярных элементов модулей вспомогательной алгебры. Показано, что наличие расщепления приводит к совпадению коэффициентов ветвления с кратностями весов и ведет к новым соотношениям на функции ветвления для аффинных алгебр Ли. 

Предложено обобщение стохастического процесса Шрамма-Левнера на случай coset-моделей. 

Реализован пакет аффайн-эм для вычислений в теории представлений.

Вот список публикаций в рецензируемых журналах и список других публикаций и докладов. 

Спасибо большое за внимание!

\end{document}
