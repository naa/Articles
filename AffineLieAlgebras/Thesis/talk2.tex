\documentclass{article}
\usepackage[utf8x]{inputenc}
\usepackage[russian]{babel}

\begin{document}
Здравствуйте! Сегодня я представляю свою диссертацию ``Правила ветвления аффинных алгебр Ли и приложения в моделях конформной теории поля''. Диссертация выполнена на нашей кафедре. 
Сначала я скажу о структуре диссертации. Диссертация состоит из пяти глав. Первая глава является обзорной и дает физическую мотивацию. В ней я рассказываю про конформную теорию поля в двух измерениях и о том, как в ней возникают аффинные алгебры Ли. 
Вторая глава посвящена проблемрекуррентный представления аффинной алгебры в сумму представлений подалгебры. В этой главе выведены рекуррентные соотношения на коэффициенты ветвления. 
В третьей главе показано, как процедура редукции связана с обобщенной резольвентой Бернштейна-Гельфанда-Гельфанда. 
В четвертой глав обсуждаются расщепления корневых систем простых алгебр Ли и следствия для теории представлений. Также рассматривается проблема ветвления представлений аффинной алгебры Ли на представления конечномерной подалгебры. 
Заключительная пятая глава посвящена более прикладным вещам - связи coset-моделей конформной теории поля со стохастическими процессами и компьютерной программе Affine.m 
В своем докладе я буду следовать структуре диссертации. Начну с напоминания про модели Весса-Зумино-Новикова-Виттена. Я про них уже как-то рассказывал на семинаре, но это было два года назад. 
Итак, эти модели отличаются от многих других моделей конформной теории поля тем, что их можно формулировать в терминах действия. Первый член в действии представляет собой действие нелинейной сигма модели. Поле g(x) живет на плоскости с бесконечностью и принимает значения в полупростой группе Ли G. В такой модели конформная инвариантность теряется при квантовании. Поэтому мы должны добавить топологический члена Весса-Зумино. Этот член записывается через продолжение поля на трехмерное многообразие. Мы можем думать, что наша двумерная сфера ограничивает шар, на котором задано поле g с тильдой. Такое продолжение не единственно. Так как третья группа гомологий представляет собой целые числа, то экспонента от действия определена однозначно и теория корректная. 
Перейдем к комплексным координатам z и z бар на плоскости. В теории есть сохраняющиеся токи J и J бар. Они принимают значения в конечномерной алгебре Ли g, соответствующей группе G.  То есть мы можем рассматривать голоморфный и антиголоморфный сектора независимо. Помимо конформной инвариантности у нас есть еще и калибровочная инвариантность с такими преобразованиями. Омега и омега с чертой независимы. 
Записываем тождества Уорда для таких преобразований. Если разложить токи по степеням z и генераторам алгебры Ли и подставить в тождество Уорда, мы обнаруживаем, что коэффициенты удовлетворяют коммутационным соотношениям аффинной алгебры Ли.
В конформной теории у нас должна быть алгебра Вирасоро, которая возникает при рассмотрении локальных конформных преобразований. Здесь генераторы этой алгебры можно реализовать в виде комбинаций генераторов аффинной алгебры Ли. Это так называемая конструкция Сугавары. 
Полная киральная алгебра представляет собой полупрямое произведение аффинной алгебры Ли и алгебры Вирасоро. 
Теперь мы можем ввести понятие примарного поля по отношению к полной киральной алгебре. Тогда примарные поля нумеруются старшими весами представлений аффинной алгебры Ли, действие на них понижающих операторов порождает семейства вторичных полей. 
Сингулярными называются векторы, которые зануляются повышающими операторами. Соответствующие им веса тоже называются сингулярными. 
То есть мы увидели, что аффинные алгебры Ли играют большую роль в моделях Весса-Зумино-Новикова-Виттена. 
Теперь к проблеме редукции. Самая прозрачная конструкция - это калибровочная ВЗНВ-модель. То есть мы добавляем взаимодействие с чисто калибровочным полем со значениями в некоторой подалгебре a алгебры g. В результате фиксации калибровки соответствующие степени свободы исчезают и остается coset-модель. 
В ней такие токи и тождества Уорда. Генераторы алгебры Вирасоро даются разностями выражений для алгебр g и a. Аналогично и для конформных размерностей и центральных зарядов. 
Заметим, что по отношению к генераторам подалгебры все векторы оказываются сингулярными. Функции ветвления пропорциональны характерам модулей алгебры Вирасоро. В coset-моделях примарные поля нумеруются парами весов - алгебры и подалгебры. Некоторые пары эквивалентны. То есть в coset-моделях нам нужно знать правила ветвления для модулей. 
Перейдем теперь ко второй главе диссертации, в которой выводятся новые рекуррентные соотношения на коэффициенты ветвления. Заметим, что для различных типов модулей алгебр Ли верны примерно аналогичные формулы Вейля для характеров. В числителе стоит сингулярный элемент, состоящий из экспонент сингулярных весов со знаками, а в знаменателе - произведение скобок, отвечающих положительным корням. В случае модулей Верма в числителе всего один член. Для неприводимых модулей верна такая формула. Заметим также, что характер неприводимого модуля может быть выражен в виде комбинации характеров модулей Верма. Это следствие существования резольвенты Бернштейна-Гельфанда-Гельфанда, которую мы обсуждаем в главе 3. 
Чтобы получить формулу для коэффициентов ветвления мы воспользуемся разложением сингулярного элемента. То есть мы хотим домножить формулу разложения характера алгебры в сумму характеров подалгебры на знаменатель. Но тут можно случайно домножить на ноль, если некоторые корни проектируются в ноль. Выделим их отдельно. Оказывается, что они порождают подалгебру, которую мы называем ортогональным партнером. 
Действуя знаменателем на сингулярный элемент алгебры g и проектируя мы получаем следующее разложение, включающее размерности модулей ортогонального партнера. 
Теперь мы можем домножить на дельта делить на дельта а и сравнить коэффициенты при одинаковых экспонентах. В результате получается рекуррентное соотношение на коэффициенты ветвления. 
Это соотношение новое, оно эффективно, так как для вычисления коэффициентов ветвления нам не нужно строить ни самого модуля, ни модулей подалгебры, в сумму которых он раскладывается. 
Продемонстрируем на примере, как все это работает. Вот сингулярный элемент фундаментального представления алгебры B2. Хотим раскладывать на модули подалгебры A1. Размерности вот таких модулей ортогональной алгебры А1 войдут в формулу. 
Теперь аффинное расширение. Вот веер вложения - отношение знаменателей Вейля алгебры и подалгебры, деленное на знаменатель Вейля ортогонального партнера. Вот проекция сингулярного элемента. Ставя веер в каждую точку рекуррентно вычисляем коэффициенты ветвления. При этом можно ограничиться главной камерой Вейля. 
Выводы к главе 2. Коэффициенты ветвления нужны для вычисления спектра на торе. Функции ветвления дают статсумму в coset-моделях. 
Мы вывели совершенно новую формулу для коэффициентов ветвления и продемонстрировали ее эффективность. Посмотрите на этот рисунок, здесь показано время работы программы в зависимости от необходимого числа коэффициентов ветвления. 
Теперь переходим к главе 3. 
В этой главе мы исследуем связь ветвления и обобщенной резольвенты Бернштейна-Гельфанда-Гельфанда. Пусть у нас то же, что и в предыдущей главе, но при разложении представления мы не делаем проекции. Тогда действие знаменателя ортогонального партнера на сингулярный элемент даст нам набор его неприводимых модулей. Когда мы на них подействуем веером, мы получим так параболические модули Верма. Эти модули входят в точную последовательность. На этой иллюстрации показана ее центральная часть. Эта глава относительно короткая, вот ее выводы. 
В четвертой главе мы исследуем понятие расщепления или сплинта корневой системы, введенное Дэвидом Рихтером в 2008 году. Мы будем рассматривать вложения корневых систем, при которых могут не сохраняться углы между корнями, а лишь аддитивные свойства. 
Мы будем говорить, что корневая система расщепляется, если есть два вложения двух других корневых систем. И она оказывается объединением образов этих вложений. 
Все расщепления для простых конечномерных алгебры Ли были классифицированы в работе Рихтера. 
Посмотрим, что происходит с ветвлением при наличии сплинта. 
Пусть есть алгебры g, ее регулярная подалгебра a и алгебра s. И пусть есть вложение корневой системы s в корневую систему g. То есть корневая система g представляет собой объединение корневой системы a и образа корневой системы s.
Тогда веер вложения определяется корневой системой s. Более того, сингулярный элемент представления алгебры g может быть разложен в комбинацию образов сингулярных элементов модулей алгебры s. Тогда коэффициенты ветвления совпадают с кратностями весов в представлении алгебры s. 
Продемонстрирую пару простых примеров. Рассмотрим алгебру B2 и подалгебру A1. Сингулярный элемент разложился в комбинацию образов сингулярных элементов алгебры A2. Коэффициенты ветвления совпадают с кратностями. Аналогично для алгебры G2. 
Теперь мы можем рассмотреть аффинное расширение расщепленной корневой системы. В этом случае веер вложения уже не получается - не хватает мнимых корней. То есть коэффициенты ветвления уже не совпадают с кратностями. Зато мы можем получить связь тета-функций аффинных алгебр a, s и g. 
Еще мы можем рассмотреть градуированное ветвление модуля аффинной алгебры Ли на модули конечномерной подалгебры. В этом случае мы можем ввести градуированные функции ветвления. Для этих функций в случае существования сплинта можно написать матричные соотношения. 
Введем порядок на множестве корней. 
Тогда мы можем записать кратности весов в матрицу M. Потом мы воспользуемся существованием сплинта и напишем матричное соотношение на функции ветвления. 
Выводы к четвертой главе. 
Мы проанализировали связь сплинта с веером вложения, показали, что расщепление определяет веер вложения. 
Определили, в каких случаях расщепление приводит к совпадению коэффициентов ветвления с кратностями весов. 
Получили новые соотношения на тета-функции и градуированные функции ветвления. 
Теперь я перехожу к пятой главе, которая посвящена более прикладным вещам. Первая тема здесь - это связь конформной теории поля и эволюции Шрамма-Левнера. Эволюция Шрамма-Левнера - это стохастический процесс. 
\end{document}
