
\documentclass[12pt]{article}
\usepackage{amssymb}

%%%%%%%%%%%%%%%%%%%%%%%%%%%%%%%%%%%%%%%%%%%%%%%%%%%%%%%%%%%%%%%%%%%%%%%%%%%%%%%%%%%%%%%%%%%%%%%%%%%%
\usepackage{amsmath}
\usepackage{amsmath,amssymb,amsthm}
\usepackage{multicol}
\usepackage{color}
\usepackage{graphicx}
\usepackage{hyperref}

\newtheorem{statement}{Statement}
\newtheorem{theorem}{Theorem}
\newtheorem{corollary}{Corollary}[theorem]
\newtheorem{lemma}{Lemma}
\newtheorem{mynote}{Note}[section]
\theoremstyle{definition}
\newtheorem{definition}{Definition}
\newtheorem{remark}{Remark}
\newtheorem{example}{Example}
\newcommand{\go}{\stackrel{\circ }{\mathfrak{g}}}
\newcommand{\ao}{\stackrel{\circ }{\mathfrak{a}}}
\newcommand{\co}[1]{\stackrel{\circ }{#1}}
\newcommand{\pia}{\pi_{\mathfrak{a}}}
\newcommand{\piab}{\pi_{\mathfrak{a}_{\bot}}}
\newcommand{\gf}{\mathfrak{g}}
\newcommand{\af}{\mathfrak{a}}
\newcommand{\aft}{\widetilde{\mathfrak{a}}}
\newcommand{\afb}{\mathfrak{a}_{\bot}}
\newcommand{\hf}{\mathfrak{h}}
\newcommand{\hfb}{\mathfrak{h}_{\bot}}
\newcommand{\pf}{\mathfrak{p}}
%\input{tcilatex}

\begin{document}

\title{{\bf Affine.m} -- {\it Mathematica} package for Lie algebras of finite, affine and extended affine type}
\author{A A Nazarov$^{1,2}$\\
{\small $^1$ Theoretical Department, SPb State University}\\
{\small 198904, Sankt-Petersburg, Russia}\\
{\small$^{2}$ Chebyshev Laboratory,}\\
{\small Department of Mathematics and Mechanics, SPb State University}\\
{\small 199178, Saint-Petersburg, Russia}\\
{\small email: antonnaz@gmail.com}}

\maketitle

\begin{abstract}
  We present {\it Mathematica} package for computations in the representation theory of Lie algebras of finite, affine and extended affine types. 
\end{abstract}


\section{Introduction}
\end{document}
