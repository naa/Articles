
\documentclass[12pt]{article}
\usepackage{amssymb}

%%%%%%%%%%%%%%%%%%%%%%%%%%%%%%%%%%%%%%%%%%%%%%%%%%%%%%%%%%%%%%%%%%%%%%%%%%%%%%%%%%%%%%%%%%%%%%%%%%%%
\usepackage{amsmath}
\usepackage{amsmath,amssymb,amsthm}
\usepackage{multicol}
\usepackage{color}
\usepackage{graphicx}
\usepackage{hyperref}

\newtheorem{statement}{Statement}
\newtheorem{theorem}{Theorem}
\newtheorem{corollary}{Corollary}[theorem]
\newtheorem{lemma}{Lemma}
\newtheorem{mynote}{Note}[section]
\theoremstyle{definition}
\newtheorem{definition}{Definition}
\newtheorem{remark}{Remark}
\newtheorem{example}{Example}
\newcommand{\go}{\stackrel{\circ }{\mathfrak{g}}}
\newcommand{\ao}{\stackrel{\circ }{\mathfrak{a}}}
\newcommand{\co}[1]{\stackrel{\circ }{#1}}
\newcommand{\pia}{\pi_{\mathfrak{a}}}
\newcommand{\piab}{\pi_{\mathfrak{a}_{\bot}}}
\newcommand{\gf}{\mathfrak{g}}
\newcommand{\af}{\mathfrak{a}}
\newcommand{\aft}{\widetilde{\mathfrak{a}}}
\newcommand{\afb}{\mathfrak{a}_{\bot}}
\newcommand{\hf}{\mathfrak{h}}
\newcommand{\hfb}{\mathfrak{h}_{\bot}}
\newcommand{\pf}{\mathfrak{p}}
%\input{tcilatex}

\begin{document}

\title{{\bf Affine.m} -- {\it Mathematica} package for Lie algebras of finite, affine and extended affine type}
\author{A A Nazarov$^{1,2}$\\
{\small $^1$ Theoretical Department, SPb State University}\\
{\small 198904, Sankt-Petersburg, Russia}\\
{\small$^{2}$ Chebyshev Laboratory,}\\
{\small Department of Mathematics and Mechanics, SPb State University}\\
{\small 199178, Saint-Petersburg, Russia}\\
{\small email: antonnaz@gmail.com}}

\maketitle

\begin{abstract}
  We present {\it Mathematica} package for computations in the representation theory of Lie algebras of finite, affine and extended affine types. 
\end{abstract}


\section{Introduction}

Representation theory of finite, affine and extended affine Lie algebras is of central importance for different areas of mathematical and theoretical physics. Simple Lie algebras appear as the symmetries of classical and quantum systems, e.g. of gauge quantum field theories. Tensor product decomposition problem for simple Lie algebras appears in classification of particles, branching problem -- in the study of symmetry breaking, for example in great unification models. 

Major role of representation theory for the study of quantum and classical integrable systems is well known. Here, for example, tensor product decomposition is required for the computation of spectra and eigenstates of Hamiltonian. Quantum deformation of simple Lie algebras appears naturally in the study of integrable systems. It is possible to introduce $q$-analogues of multiplicities and branching coefficients. If $q$ is not root of unity category of $U_{q}(\gf)$-modules is equivalent to the non-deformed case.  Since Weyl character formula is preserved by $q$-deformation in this case, recurrent approach is applicable. 

Affine Lie algebras, WZW and coset models. 

Extended affine Lie algebras currently attract a lot of attention. These algebras has applications in supergravity 
\end{document}
