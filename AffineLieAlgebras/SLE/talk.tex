\documentclass{article}
\begin{document}
Hello!
It's a pleasure for me to be here in Prague. 

My talk is devoted to the correspondence between Schramm-Loewner evolution and conformal field theory.

First of all I need to remind you of Schramm-Loewner evolution.
Consider Ising model on triangular lattice on upper half plane. Let we have spins down on one half of the boundary and spins up on another half. Then in any configuration we get an interface delimiting two clusters and connecting zero and infinity. 
Next we consider continuous limit of lattice model. We can think of it as of stochastic process which satisfy this stochastic equation. Here gt or wt is conformal map which maps SLE trace up to time t to the boundary. 
Now we can look at the observables in the presence of SLE trace. Expectation value of lattice observable on upper half-plane can be calculated as the sum of expectation values of this observable in presence of SLE trace multiplied by the probability of this trajectory. 
Since lattice observable does not depend on the choice of time t, observables in presence of SLE trace should be martingales. 
In continuous limit lattice observable tends to CFT correlation function of this form. It contains some set of primary fields. And we have boundary condition changing operators at the tip of SLE trace and in the infinity. 
We can use conformal map w of z to rewrite this expression in upper half plane. 
Now we want to consider evolution from t to t plus dt. First factor give us this expression. Primary field under conformal map are changed in this way.
We denote generator of this transform by gi.
Now for expectation value of martingale we have this expression. We use Ito calculus to calculate increment of our correlation function. It should be zero so we get this equation.
We can see that it can be rewritten as the action of Virasoro generators. Since we have arbitrary observable, this equation is equivalent to the requirement for boundary condition changing operator to have level two null state. 
In case of minimal model we can see that boundary condition chaining operator is primary field phi one two or phi two one. 
Now we want to generalize this analysis to rational conformal field theories. First we consider Wess-Zumino-Novikov-Witten model. The action for this model can be written in terms of map from complex plane with infinity or two-sphere to some Lie group G. 
The first term is just non-linear sigma-model. Here K is Killing form on algebra g. The second term is written in terms of continuation from two sphere to three-dimensional manifold which has two sphere as the boundary. Since this continuation is non-unique we get the requirement for k to be integer. 
Currents of this model have this form. 
Model possess gauge invariance with omega and omega bar in dependant. If we consider infinitesimal gauge transformation we get this Ward identities. Then we can get operator product expansion for currents. 
If we expand currents to modes we get commutation relations of affine Lie algebra g hat. 
This model has conformal invariance which can be seen from 
Sugawara construction. This is way to embed Virasoro algebra into the universal enveloping algebra of affine Lie algebra g hat. 
Full chiral algebra of the model is semidirect product of affine and Virasoro algebra. Its commutation relations are here. 
Primary fields are labeled by highest weights of affine Lie algebra representations. Here we see how Virasoro and affine Lie algebra generators act on primary fields.

Now we want to study Schramm-Loewner evolution in this case. Similarly to minimal case we need to study this correlation function. 
Again we can use conformal map w of z to rewrite it on the whole upper half-plane. 
We consider evolution from time t to t plus dt. We get the same expression from the first factor. But when we consider fields we need to add random gauge transformation. 
Since we have boundary we are working with boundary CFT so we have only half of gauge transformations due to Cardy conditions. The generator of field transformation have this form. We can interpret last term as Brownian motion in Lie group G.
Then we use Ito calculus to get this equation from martingale condition. Again we can rewrite it as algebraic requirement for  field  which correspond to boundary condition changing operator.
If we act on this state with raising operators we should get zero. We use commutation relations and get equations which connects parameters of random motion with level of affine Lie algebra representation. To study this equations we need to use Knizhnik-Zamolodchikov equations. Also we get some constraints on possible boundary condition changing operators. 

Now we want to generalize this analysis even further and study coset models of conformal field theory. This models can be realized as gauged Wess-Zumino-Novikov-Witten models. We add gauge fields taking values in subalgebra a of Lie algebra g. The term which describes interaction with gauge fields has this form. 
The current of this model is here. 
We have this expression for the product of gauge field and primary fields. Here hv is dual Coxeter number of Lie algebra h.
Remember that for Wess-Zumino-Witten model current we have this operator product expansion. So we see that algebraic structure is connected with the pair of algebras g and a. Virasoro generators are given by the difference of Sugawara expressions for this algebras. 
Primary fields are labeled by pairs of weights of algebra and subalgebra, but some pairs are equivalent. This equivalence relation is given by the action of simple currents corresponding to affine Lie algebras g and a such, that their conformal weights are equal. 
For conformal weight of primary fields we have this expression. 
One can also write analogues of Knizhnik-Zamolodchikov equations.

So for SLE transformation we get direct sum of random gauge transformation in algebras g and a. Martingale condition can be written in this form. So again we get algebraic relation on boundary condition changing operator. 

We can compare this result with the simplest possible coset model su two divided by u one. It is parafermion theory. It is well-known that parafermionic field can be written as product of su two primary field and Vertex operator of bosonic field. 
Parafermionic current can be seen as the combination of su(2) and u(1) currents and the martingale equation from previous slide can be written in this form. This relation coincide with the relation which can be obtained from the study SLE in  parafermionic models. 

Next steps in the study of SLE martingales for coset models  are the following. We need to act on the relation with raising operators and get relations which connects level of representation and parameters of random motion. Then we use Knizhnik-Zamolodchikov equations to rewrite the relations in simpler form and get conditions on possible boundary condition changing operators. 
Boundary states in coset models were actively studied. One of the ideas is to consider direct sum WZNW-model and use equivalence relations. This study is essentially the study representation of fusion algebra, Verlinde formula is one of key ingredients in this study. So we hope to compare the classification of boundary states obtained in this way with our condition on boundary condition changing operator. 

Now we see that it is possible to match SLE observables and CFT correlation functions. Martingale conditions can be seen as the constraints on possible Verma modules of affine Lie algebra corresponding to boundary condition changing operator. This Verma module must have level two null state. 
We can use primary fields of direct sum WZNW model and equivalence relations to study coset models. 
Martingale conditions should be compared with the classification of boundary states with other methods.

Thank you for your attention. 

\end{document}
