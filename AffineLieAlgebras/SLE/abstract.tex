\documentclass[a4paper,12pt]{article}
\usepackage{ucs}
\usepackage[utf8x]{inputenc}
\usepackage[russian]{babel}

\usepackage{latexsym}
\usepackage{amssymb}
\usepackage[T1]{fontenc}
\usepackage{mathptmx}

\topmargin=-1.5cm
\oddsidemargin=-1.5cm
\textwidth=17cm
\textheight=28cm

\author{Антон Назаров}
\title{Эволюция Шрамма-Лёвнера и coset-модели конформной теории поля}
\date{28 сентября 2011, 15:00}

\begin{document}
\pagestyle{empty}
\maketitle

  Эволюция Шрамма-Лёвнера описывает непрерывный предел интерфейсов в критической точке решеточных
  моделей. При этом значения корреляционных функций вычисляются в области с разрезом, порожденным
  эволюцией Шрамма-Лёвнера. С другой стороны существует описание в терминах минимальных моделей
  конформной теории поля. Там значения вычисляются как корреляторы некоторых наборов примарных
  операторов с добавлением оператора смены граничного условия на конце разреза. Условие, что
  определенная величина является мартингалом можно записать как дифференциальное уравнение на
  коррелятор, которое, в свою очередь, можно переписать как набор алгебраических соотношений,
  наложенных на оператор смены граничного условия. Если решить эти соотношения, получится набор
  возможных операторов смены граничных условий, который совпадает с предсказаниями граничной
  конформной теории поля, следующими из физических соображений.

  Помимо минимальных моделей конформной теории поля существуют и более сложные. Важный класс таких
  моделей -- это модели Весса-Зумино-Новикова-Виттена, которые обладают дополнительной симметрией,
  связанной с алгебрами Каца-Муди. Для таких моделей не известно решеточной реализации, однако можно
  рассмотреть эволюцию Шрамма-Лёвнера с дополнительным случайным движением по группе Ли, записать и
  решить условия на мартингалы. Это было сделано в работе Антона Алексеева, Андрея Быцко и
  Константина Изъюрова \cite{alekseev2010sle}. 

  Coset-конструкция конформной теории поля позволяет строить еще более широкий класс моделей на
  основе моделей Весса-Зумино-Новикова-Виттена. В частности, таким образом можно реализовать все
  минимальные модели конформной теории поля. В данном докладе мы попробуем обобщить эволюцию
  Шрамма-Лёвнера на этот случай путем рассмотрения дополнительного случайного блуждания на
  фактор-группе группы Ли. Так как этот подход должен воспроизводить известные результаты для
  минимальных моделей, мы рассмотрим в качестве примера coset-модели, соответствующие модели Изинга
  и парафермионам, эволюция Шрамма-Лёвнера для которых была рассмотрена Раулем Сантачиарой в работе
  \cite{santachiara2008sle}.


\bibliography{bibliography}{}
\bibliographystyle{utphys}

\end{document}
