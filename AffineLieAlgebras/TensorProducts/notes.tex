\documentclass[12pt]{article}

\usepackage{latexsym,amsmath,amssymb}
%\usepackage{amsfonts}
\usepackage{pb-diagram}
%%%%%%%%%%%%%%%%%%%%%%%%%%%%%%%%%%%%%%%%%%%%%%%%%%%%%%%%%%%%%%%%%%%%%%%%%%%%%%%%%%%%%%%%%%%%%%%%%%% 
%\usepackage{setstack}
\usepackage{multicol}
\usepackage{color}
\usepackage{hyperref}
\usepackage{graphicx}

\usepackage{ucs} \usepackage[utf8x]{inputenc}
\usepackage[english,russian]{babel}


\newtheorem{Def}{Definition}[section]
\newtheorem{theorem}{Theorem}
\newtheorem{statement}{Statement}
\newtheorem{Cnj}[Def]{Conjecture}
\newtheorem{Prop}[Def]{Property}
\newtheorem{example}{Example}[section]


\newcommand{\go}{\stackrel{\circ }{\mathfrak{g}}}
\newcommand{\ao}{\stackrel{\circ }{\mathfrak{a}}}
\newcommand{\co}[1]{\stackrel{\circ }{#1}}
\newcommand{\pia}{\pi_{\mathfrak{a}}}
\newcommand{\piab}{\pi_{\mathfrak{a}_{\bot}}}
\newcommand{\gf}{\mathfrak{g}}
\newcommand{\gfh}{\hat{\mathfrak{g}}}
\newcommand{\af}{\mathfrak{a}}
\newcommand{\afh}{\hat{\mathfrak{a}}}
\newcommand{\bff}{\mathfrak{b}}
\newcommand{\afb}{\mathfrak{a}_{\bot}}
\newcommand{\hf}{\mathfrak{h}}
\newcommand{\hfg}{\hf_{\gf}}
\newcommand{\hfa}{\hf_{\af}}
\newcommand{\hfb}{\mathfrak{h}_{\bot}}
\newcommand{\pf}{\mathfrak{p}}
\newcommand{\aft}{\widetilde{\mathfrak{a}}}
\newcommand{\sfr}{\mathfrak{s}}

\newcommand{\cR}{\check{R}}

%%%%%%%%%% Young tableaux %%%%%%%%%%%%%%%%%%%%%%%%%%%%

\newdimen\tableauside\tableauside=1.0ex
\newdimen\tableaurule\tableaurule=0.4pt
\newdimen\tableaustep
\def\phantomhrule#1{\hbox{\vbox to0pt{\hrule height\tableaurule width#1\vss}}}
\def\phantomvrule#1{\vbox{\hbox to0pt{\vrule width\tableaurule height#1\hss}}}
\def\sqr{\vbox{%
  \phantomhrule\tableaustep
  \hbox{\phantomvrule\tableaustep\kern\tableaustep\phantomvrule\tableaustep}%
  \hbox{\vbox{\phantomhrule\tableauside}\kern-\tableaurule}}}
\def\squares#1{\hbox{\count0=#1\noindent\loop\sqr
  \advance\count0 by-1 \ifnum\count0>0\repeat}}
\def\tableau#1{\vcenter{\offinterlineskip
  \tableaustep=\tableauside\advance\tableaustep by-\tableaurule
  \kern\normallineskip\hbox
    {\kern\normallineskip\vbox
      {\gettableau#1 0 }%
     \kern\normallineskip\kern\tableaurule}%
  \kern\normallineskip\kern\tableaurule}}
\def\gettableau#1 {\ifnum#1=0\let\next=\null\else
  \squares{#1}\let\next=\gettableau\fi\next}

\begin{document}

\title{Заметки про квантовые группы, интегрируемые системы и все такое}

%% \author{V.D.~Lyakhovsky$^1$, A.A.~Nazarov$^{1,2}$, O.V.~Postnova$^{1}$}
%% \address{ $^1$ Department of High-energy and elementary particle physics,  St Petersburg State University, 198904, Saint-Petersburg, Russia}
%% \ead{ $^{2}$ e-mail: antonnaz@gmail.com}
%% 


\begin{abstract}

  We show the connection of formulae for tensor product multiplicities with cluster algebras. 
  \noindent{\it Keywords\/}: Lie algebra, tensor product, multiplicity, cluster algebra, grading
\end{abstract}



\section{Introduction}
\label{sec:introduction}
\cite{di2013quantum}
\bibliography{bibliography}{} 
\bibliographystyle{iopart-num}

\subsection{Алгебры Хопфа}
\label{sec:hopf}
\begin{Def}
  {\it Алгебра Хопфа $(H,\mu,1_H,\Delta,\epsilon,S)$} - это
  векторное пространство$H$ над полем $k$, наделенное
  структурой алгебры с умножением $\mu:H\otimes H\to H$ и
  единицей $1_H$, коалгебры с коумножением $\Delta:H\to
  H\otimes H$ и коединицей $\epsilon:H\to k$, а так же
  антиподом $S$ - линейным отображением $S:H\to H$. Причем все
  эти структуры должны быть согласованы друг с другом.

  чтобы следующие диаграммы были коммутативны:
  \[
  \begin{diagram}
    \node{H\otimes H} \arrow{e,t}{\mu}
    \arrow{s,r}{\Delta\otimes \Delta} \node{H}
    \arrow{e,t}{\Delta}
    \node{H\otimes H} \\
    \node{H\otimes H\otimes H\otimes H}
    \arrow[2]{e,t}{id\otimes \tau\otimes id}
    \node[2]{H\otimes H\otimes H\otimes H}
    \arrow{n,r}{\mu\otimes \mu}
  \end{diagram}
  \]
  \[
  \begin{diagram}
    \node{H} \arrow{e,t}{\epsilon} \node{k} \node{k}
    \arrow{e,r}{\eta} \arrow{se,r}{\eta\otimes\eta}
    \node{H}\\
    \node{H\otimes H} \arrow{n,r}{\mu}
    \arrow{ne,r}{\epsilon\otimes\epsilon} \node[3]{H\otimes
      H} \arrow{n,r}{\Delta}
  \end{diagram}
  \]
  \[
  \begin{diagram}
    \node{H} \arrow{e,t}{\epsilon} \arrow{s,r}{\Delta}
    \node{k} \arrow{e,t}{\eta}
    \node{H}\\
    \node{H\otimes H} \arrow[2]{e,t}{id\otimes S,\ S\otimes
      id} \node[2]{H\otimes H} \arrow{n,r}{\mu}
  \end{diagram}
  \]
  , где $\eta$ - отображение $k\to H,\quad \eta(a)=a\ 1_{H}$

  Подробное определение есть, например, в книге Маджида
  \cite{Majid}.
\end{Def}
Универсальную обертывающую алгебру $U(g)$ любой алгебры Ли $g$ можно наделить структурой
(тривиальной) алгебры Хопфа следующим образом: $\Delta(f)=f\otimes 1 + 1 \otimes f,\
\epsilon(f)=0,\ S(f)=-f,\ f\in g$

На векторных пространствах алгебры Хопфа действуют так же, как
и обычные алгебры. Однако если векторное пространство само
наделено структурой алгебры, то действие алгебры Хопфа должно
быть согласовано со структурой алгебры на векторном
пространстве, чтобы оно могло называться модуль-алгеброй.
\begin{Def}
  \label{ModuleAlgebra}
  {\it $H$-модуль-алгеброй} называется алгебра $A$, на которой
  действие алгебры Хопфа $H$ согласовано с алгебраической
  структурой:
  \begin{eqnarray}
    h\triangleright (a\cdot b) = \sum (h_{(1)}\triangleright a) \cdot (h_{(2)}\triangleright b) \\
    h\triangleright 1 = \epsilon(h)1
  \end{eqnarray} 
  , где $a,b\in A,\ h\in H,\ \Delta(h)=\sum h_{(1)}\otimes
  h_{(2)}$

  Наглядно определение можно представить следующими
  диаграммами:
  \[
  \begin{diagram}
    \node{H\otimes A \otimes A} \arrow{e,t}{\cdot}
    \arrow{s,r}{\Delta\otimes id\otimes id} \node{H\otimes A}
    \arrow{e,t}{\triangleright} \node{A} \node{A\otimes A}
    \arrow{w,t}{\cdot}
    \\
    \node{H\otimes H \otimes A \otimes A} \arrow[3]{e,t}{id
      \otimes \tau \otimes id} \node[3]{H\otimes A \otimes H
      \otimes A} \arrow{n,l}{\triangleright \otimes
      \triangleright}
  \end{diagram}
  \]

  \[
  \begin{diagram}
    \node{H} \arrow{e,t}{\epsilon} \arrow{s,l}{\eta} \node{k}
    \arrow{e,t}{\eta} \node{A}
    \\
    \node{H\otimes A} \arrow{ene,r}{\triangleright}
  \end{diagram}
  \]

\end{Def}

Если $H$ действует на пространствах $V$ и $W$, то на тензорном
произведении $V\otimes W$ действие $h\in H$ дается формулой
\begin{equation}
  \label{HopfAlgebraActionOnTensorProduct}
  h\triangleright (v\otimes w) = \sum (h_{(1)}\triangleright v) \otimes (h_{(2)} \triangleright w)
\end{equation}
, где $\Delta(h)=\sum h_{(1)}\otimes h_{(2)}$


\subsection{$U_q(\mathfrak{sl}_2)$}
\label{sec:u_q_sl_2}

\subsection{Yangians}
\label{sec:yangians}

For any finite-dimensional semisimple Lie algebra $\af$, Drinfeld defined an infinite-dimensional
Hopf algebra $Y(\af)$, called the Yangian of $\af$. This Hopf algebra is a deformation
of the universal enveloping algebra $U(\af[z])$ of the Lie algebra of polynomial loops
of $\af$ given by explicit generators and relations. The relations can be encoded by identities
involving a rational $R$-matrix. Replacing it with a trigonometric $R$-matrix, one
arrives at affine quantum groups, defined in the same paper of Drinfeld.

In the case of the $\mathfrak{gl}_{N}$, the Yangian admits a simpler description in terms of a single ``ternary'' (or ``RTT'') relation on the matrix generators due to Faddeev and coauthors. 
The Yangian $Y(\mathfrak{gl}_{N})$ is defined to be the algebra generated by elements $t_{ij}^{(p)}$ with $1 \leq i, j \leq N$ and $p \geq 0$, subject to the relations:

\begin{equation}
  \label{eq:1}
 [t_{ij}^{(p+1)}, t_{kl}^{(q)}] -  [t_{ij}^{(p)}, t_{kl}^{(q+1)}]= -(t_{kj}^{(p)}t_{il}^{(q)} - t_{kj}^{(q)} t_{il}^{(p)}).
\end{equation}


Defining $t_{ij}^{(-1)}=\delta_{ij}$, setting $ T(z) = \sum_{p\ge -1} t_{ij}^{(p)} z^{-p+1}$

and introducing the $R$-matrix $R(z) = I + z^{-1} P$ on $\mathbb{C}^{N}\otimes \mathbb{C}^{N}$,
where $P$ is the operator permuting the tensor factors, the above relations can be written more
simply as the ternary relation:

$\displaystyle{ R_{12}(z-w) T_{1}(z)T_{2}(w) = T_{2}(w) T_{1}(z) R_{12}(z-w).}$

The Yangian becomes a Hopf algebra with comultiplication $\Delta$, counit $\epsilon$ and antipode $S$ given by
\begin{equation}
  \label{eq:2}
  (\Delta \otimes \mathrm{id})T(z)=T_{12}(z)T_{13}(z), \,\, (\varepsilon\otimes \mathrm{id})T(z)= I, \,\, (s\otimes \mathrm{id})T(z)=T(z)^{-1}.
\end{equation}


At special values of the spectral parameter $(z-w) $, the $R$-matrix degenerates to a rank one
projection. This can be used to define the quantum determinant of $T(z) $, which generates the
center of the Yangian.

The {\it twisted Yangian} $Y^{-}(\mathfrak{gl}_{2N})$, introduced by G. I. Olshansky, is the sub-Hopf algebra generated by the coefficients of

$\displaystyle{ S(z)=T(z)\sigma T(-z),}$

where $\sigma$ is the involution of $\mathfrak{gl}_{2N}$ given by

$\displaystyle{\sigma(E_{ij}) = (-1)^{i+j}E_{2N-j+1,2N-i+1}.}$
Quantum determinant is the center of Yangian.


\section{Yangians and their representations}

\subsection{Yangians}

\subsubsection{Definition of $Y(\gf)$: algebra and co-algebra}

Let the simple Lie algebra $\gf$  be generated\footnote{To avoid
becoming mired in detail, we have used anti-Hermitian generators
(hence no `$i$'), and, more importantly, compactness (hence an
inner product proportional to $\delta_{ab}$, so that we won't have
to distinguish `up' from `down' indices).} by $\{I_a\}$,
$a=1,\ldots,$dim$\,\gf$, with structure constants $f_{abc}$,
\begin{equation}\label{Y1}\left[ I_a , I_b \right] = f_{abc} I_c \,,\end{equation}
 and (trivial) coproduct\footnote{The enveloping algebra $U\gf$ of $\gf$
consists of (powers, polynomials and) series in the $I_a$ subject
to the Lie bracket, regarded as a commutator.} $\Delta:
U\gf\rightarrow U\gf\otimes U\gf$, \begin{equation}\label{Y1del}
 \Delta(I_a) = I_a \otimes 1
+ 1 \otimes I_a \,. \end{equation}
 For those new to it, the
coproduct is a generalization of the usual rule for addition of
spin. As such, the principal constraints on $\Delta$ are that it
be coassociative, \begin{equation}\label{coassoc} (\Delta\otimes
1)\Delta(x)=(1\otimes\Delta)\Delta(x)\end{equation} for all $x\in\gf$ (so that
the action of $x$ on a 3-particle state is unique), and that it
 be a homomorphism, \begin{equation}\label{hom} \Delta([x,y])=[\Delta(x),\Delta(y)]
  \end{equation} for all $x,y\in\gf$ (so that multiparticle states carry
 representations of the symmetry algebra). As we shall see below,
 there are non-trivial ways in which this can be achieved.

The Yangian \cite{drinf1,drinf2} $Y(\gf)$ is the enveloping algebra
generated by these and a second set of generators $\{J_a\}$, in
the adjoint representation of $\gf$ so that \begin{equation}\label{Y2} \left[
I_a , J_b \right] =   f_{abc} J_c \,, \end{equation} but with a non-trivial
coproduct \begin{equation}\label{Y2del} \Delta:Y(\gf)\rightarrow Y(\gf)\otimes
Y(\gf)\,,\qquad \Delta(J_a) = J_a \otimes 1 + 1 \otimes J_a +
{\alpha\over 2}f_{abc} I_c \otimes I_b\, \end{equation} for a parameter
$\alpha\in{\mathbb C}$. Note that (\ref{coassoc},\,\ref{hom}) hold
for all the $I_a,J_a$.

The commutator $[J_{a},J_{b}]$ is not fully specified, but is
constrained by the requirement that $\Delta$ be a homomorphism (as
explained in the next subsection): \begin{equation}\label{Y3}
[J_a,[J_b,I_c]]-[I_a,[J_b,J_c]] = \alpha^2
a_{abcdeg}\{I_d,I_e,I_g\} \,, \end{equation} where \begin{equation}
 a_{abcdeg}={1\over{24}}
 f_{adi} f_{bej}f_{cgk}f_{ijk}
 \hspace{0.1in},\hspace{0.15in}\{x_1,x_2,x_3\}=
 \sum_{i\neq j\neq k}x_{i}x_{j}x_{k} \,,
\end{equation} and \begin{equation}\label{Y4} [[J_a,J_b],[I_l,J_m]] +
[[J_l,J_m],[I_a,J_b]] = \alpha^2 \left( a_{abcdeg}f_{lmc} +
a_{lmcdeg}f_{abc} \right) \left\{ I_d,I_e,J_g \right\} \,. \end{equation} For
$\gf=a_1$,  (\ref{Y3}) is trivial, while for $\gf\neq a_1$,
(\ref{Y3}) implies (\ref{Y4}), which is thus
redundant\footnote{Throughout these lectures, we write specific
$\gf$ as $a_n={\mathfrak{su}}_{n+1}$, $b_n={\mathfrak{so}}_{2n+1}$,
$c_n={\mathfrak{sp}}_n$, $d_n={\mathfrak{so}}_{2n}$, along with
$e_6,e_7,e_8,f_4,g_2$.}.

In the original sense of the word `quantum' in `quantum group',
the parameter $\alpha$ is proportional to $\hbar$: it measures the
deformation of the `auxiliary' Lie algebra required to make the
quantum inverse scattering method work. In the next lecture, by
contrast, we shall see $Y(\gf)$ appearing explicitly as a charge
algebra, with $\alpha=1$ and $\hbar$ making a conventional
appearance on the right-hand side of each commutator.

Finally, there are also other structures on which we place less
emphasis but which make $Y(\gf)$ a Hopf algebra and which we give
for completeness: a co-unit \begin{equation} \epsilon: Y(\gf)\rightarrow
{\mathbb C}\,,\qquad \epsilon (I_a)=0,\quad\epsilon(J_a)=0\end{equation}
(physically a one-dimensional vacuum representation, trivial for
$Y(\gf)$), and an antipode \begin{equation} \label{antipode}s:Y(\gf)\rightarrow
Y(\gf)\,,\qquad s(I_a)=-I_a,\quad s(J_a)=-J_a+{1\over 2}f_{abc}I_c
I_b\,,\end{equation} an anti-automorphism (and physically a
$PT$-transformation).

\subsubsection{Drinfeld's `terrific' relation}

 Drinfeld
called the relations (\ref{Y3}) and (\ref{Y4}) `terrific'
\cite{drinf2}, and it is worth explaining their origin and
significance further (see also \cite{gron}). First, the left-hand
side of (\ref{Y3}) is a little more intuitive if we instead write
it as \begin{equation}\label{Y3'} [J_a,[J_b,I_c]]-[I_a,[J_b,J_c]] =
f_{d(ab}[J_{c)},J_d], \end{equation} where $(abc)$ means `the sum of $abc$
and cycles thereof'. One way of viewing $Y(\gf)$ is as a
deformation of the polynomial algebra $\gf[z]$: if $\alpha=0$, then
the algebra reduces to that of $I_a$ and $J_a\equiv z I_a$,
whereupon (\ref{Y3}) is just the Jacobi identity
$f_{d(ab}f_{c)de}=0$. So a natural way to think about $Y(\gf)$ is
as a graded algebra, in which $I_a$ has grade zero and $J_a$ and
$\alpha$ each grade one, and (\ref{Y3}) and (\ref{Y4}) are viewed
as constraints on the construction of higher-grade elements. For
example, suppose we define a grade-two element \begin{equation} K_a \equiv
{1\over c_A} f_{abc}[J_c,J_b] \qquad{\rm (where\;\;}
f_{abc}f_{dcb}=c_A \delta_{ad}\;). \end{equation} Then if we write\begin{equation}
[J_b,J_c]=f_{bcd}K_d + X_{bc}\,, \end{equation} (\ref{Y3}) fixes $X_{bc}$:
for suppose not, that there exists another possible $X'_{bc}$.
Then, setting $Y_{bc}=X'_{bc}-X_{bc}$, (\ref{Y3}) implies that
$f_{d(ab}Y_{c)d}=0$, and thus (equivalent to the statement
 that the second cohomology $H^2(\gf)=0$) that $Y_{cd}=f_{cde}Z_e$
 for some $Z_e$. But then
$c_A Z_a = f_{abc} f_{cbd}Z_d = f_{abc}Y_{cb}=c_A(K_a-K_a)=0$, so
$Z_a=0$ and so $X'_{bc}=X_{bc}$.

The origin of (\ref{Y3}) lies in first postulating $\Delta(J_a)$
in (\ref{Y2del}) and then requiring that this be a homomorphism.
To see this, first let $u_{ab}$ be such that $u_{ab}=-u_{ba}$ and
\begin{equation}\label{u1} u_{ab} [ I_a,I_b ] = 0 \,. \end{equation} Now compute \begin{equation}
u_{ab}\left(\Delta \left( [J_{a},J_{b}] \right) - 1\otimes
[J_a,J_b] - [J_a,J_b] \otimes 1 \right) \,. \end{equation} The parts of this
expression involving $J$ disappear because of (\ref{u1}), whilst
the remainder is
\begin{equation}gin{equation}\label{u2}
{1\over 2}u_{ab}f_{ade}f_{bgh}f_{dgk} \left( I_k \otimes I_e I_h +
I_e I_h\otimes I_k \right) \,.
\end{equation}
Because of (\ref{u1}), or $f_{abc}u_{ab}=0$, we may write \begin{equation}
u_{ab}= v_{dea} f_{deb} - v_{deb} f_{dea}   \end{equation} (equivalent to the
second homology $H_2(\gf)=0$). Requiring $\Delta$ to be a
homomorphism for all $v$, and using the Jacobi identity twice, we
obtain (\ref{Y3}).




\subsubsection{The $R$-matrix}

$Y(\gf)$ is closely related to the Yang-Baxter equation, which has
a rich literature in its own right (see \cite{jimbo}). A nice way
to see this, of which we give a sketch here, is to define a new
object, the monodromy matrix, \begin{equation} T(\lambda) \equiv \exp\left(
-{1\over \lambda}t^aI_a + {1\over \lambda^2}t^aJ_a-{1\over
\lambda^3}t^aK_a+\ldots\right), \end{equation} where $\lambda\in{\mathbb C}$
is a new, `spectral' parameter. The $\ldots$ denote higher terms,
of an appropriate-grade element constructed by repeated
commutation of $J$s, and the $t^a$ are a second set of generators
of $\gf$ (commuting with $Y(\gf)$), to be thought of as matrices
(perhaps in the defining representation of $\gf$, where this
exists) with elements $t^a_{ij}$. Thus, overall, $T$ is a matrix,
with entries which are elements of $Y(\gf)$.

The significance of $T$ lies in the fact that \begin{equation}\label{Tdel}
\Delta(T_{ij}(\lambda))=T_{ik}(\lambda) \otimes
T_{kj}(\lambda)\,.\end{equation} The first few terms are easily checked using
(\ref{Y1del},\,\ref{Y2del}): at order $\lambda^{-2}$, for example,
the non-trivial terms in the $\Delta(J_a)$ on the left are matched
on the right not only with the order-$\lambda^{-2}$ terms in each
exponential but also with the cross-terms from multiplying the
order-$\lambda^{-1}$ term in each $T$.

Now, $Y(\gf)$ has an (outer) automorphism \begin{equation}\label{auto}
L_\mu\,:\quad I_a\mapsto I_a\,,\qquad J_a\mapsto J_a+\mu I_a
\qquad(\mu\in{\mathbb C})\end{equation} (equivalent to $z\mapsto z+\mu$ in
the polynomial algebra if $\alpha=0$), whose action on $T$ is \begin{equation}
T(\lambda)\mapsto T(\lambda+\mu)\,.\end{equation} Let us consider the
intertwiners $\cR$, which are required to satisfy \begin{equation}\label{R}
\cR(\nu-\mu)\,. \;L_\mu\!\times\! L_\nu\left(\Delta(x)\right)
=L_\nu\!\times\! L_\mu\left(\Delta(x)\right) .\,\cR(\nu-\mu)\end{equation}
for any $x\in Y(\gf)$. (Strictly, we should only take
representations of this, and our intertwiner, conventionally
written $\cR$, and then an `$R$-matrix', is often written
$\cR={\bf P}R$ where ${\bf P}$ permutes the two module elements in
the tensor product---but we wish to defer all discussion of
representations to the next section.)

Thus \begin{equation} \label{RTT}\cR(\nu-\mu)\, T_{ik}(\mu)\otimes T_{kj}(\nu)
= T_{ik}(\nu)\otimes T_{kj}(\mu) \,\cR(\nu-\mu)\,.\end{equation}  There are
then two maps \begin{equation} T_{ik}(\mu)\otimes T_{kl}(\nu)\otimes
T_{lj}(\lambda) \mapsto T_{ik}(\lambda)\otimes T_{kl}(\nu)\otimes
T_{lj}(\mu)\,,\end{equation} and their equivalence \begin{equation}\label{YBE}
\cR(\lambda-\nu)\otimes 1\,.\, 1\otimes \cR(\lambda-\mu)\,.\,
\cR(\nu-\mu)\otimes 1 = 1\otimes\cR(\nu-\mu)\,.\,
\cR(\lambda-\mu)\otimes 1\,.\,1\otimes \cR(\lambda-\nu)\end{equation} is the
Yang-Baxter equation (YBE), illustrated schematically in fig.1.

This equation is familiar from 1+1D $S$-matrix theory, where it is
the condition for consistent factorization of the multiparticle
$S$-matrix into two-particle factors.
\begin{figure}[htb]
\hspace{32mm} \setlength{\unitlength}{1mm}
\begin{picture}(120,30)(15,15)
\put(8,-6){\line(1,3){17}} \put(98,-6){\line(1,3){17}}
\put(44,-6){\line(-2,3){34}} \put(114,-6){\line(-2,3){34}}
\put(-3,-2){\line(5,3){43}} \put(127,38){\line(-5,-3){43}}
\put(60,20){\makebox(0,0){=}}
\put(19,5){\makebox(0,0){\footnotesize $\cR(\nu\!-\!\mu)$}}
\put(37,16){\makebox(0,0){\footnotesize$\cR(\lambda\!-\!\mu)$}}
\put(28,31){\makebox(0,0){\footnotesize$\cR(\lambda\!-\!\nu)$}}
\put(111,10){\makebox(0,0){\footnotesize$\cR(\lambda\!-\!\nu)$}}
\put(88,21){\makebox(0,0){\footnotesize$\cR(\lambda\!-\!\mu)$}}
\put(116,25){\makebox(0,0){\footnotesize$\cR(\nu\!-\!\mu)$}}
\end{picture}\vspace{20mm}
\caption{The Yang-Baxter equation}
\end{figure}
Each line in the figure will carry a representation of $Y(\gf)$.
The simplest case, in which this is ${\mathbb C}^2$, yielded the
first solution of this equation, due to Yang \cite{Yang},
\begin{equation}\label{firstR}
\cR(\mu) = \left(\begin{array}{cccc} 1+\mu &  0 & 0 & 0\\
                      0 & 1 & \mu & 0\\
                       0& \mu & 1 & 0\\
                  0  &0 & 0& 1+\mu
\end{array}\right) \,.
\end{equation} $Y({\frak sl}_2)$ can be built from it, and it was in honour
of this that Drinfeld named the Yangian.

 A theorem of Belavin and Drinfeld \cite{beldr} is that
(subject to certain technical conditions) as
$\mu\rightarrow\infty$ all YBE solutions which are rational
functions of $\mu$ (and we shall see that this is so for the
Yangian $\cR(\mu)$ in the next section) are of the asymptotic form
\begin{equation} \cR(\mu) = {\bf P}\left( 1\otimes 1 + {1\over \mu} I_a\otimes
I_a + {\cal O}(\mu^{-2})\right),\end{equation} This leads not only to the
uniqueness of $Y(\gf)$, but also to the possibility of
rediscovering much about Lie algebras and their representations
purely by studying YBE solutions \cite{cher2}.

\subsection{Representations of Yangians}

Since $Y(\gf)\supset \gf$, and representations (`reps') of $Y(\gf)$
will also be reps of $\gf$, the representation theory of $\gf$ is a
good starting point. Recall that, for a Lie algebra $\gf$ of rank
$r$, there are $r$ distinguished, `fundamental' irreducible
representations (`irreps'). The story is similar for $Y(\gf)$,
which also has $r$ fundamental (and finite-dimensional) irreps
\cite{gron,drinf88}.

However, a rep which is  $Y(\gf)$-irreducible may be
$\gf$-reducible, and this is typically the case for the fundamental
irreps of $Y(\gf)$, whose $\gf$-components are the corresponding
fundamental irrep of $\gf$ and (generally) some others. These
decompositions appeared incrementally in the literature
\cite{ogiev86,chari91,KR}; for a full enumeration for simply-laced
$\gf$ see \cite{kleber96}.

\subsubsection{$Y(\gf)$-reps which are $\gf$-irreps}

The simplest situation is clearly when a $\gf$-irrep is extensible
to a $Y(\gf)$-irrep, and there are no other components. Drinfeld
enumerated the cases for which this occurs \cite{drinf1}. Starting
from an irrep $\rho$ of $\gf$, he constructed a rep $\tilde{\rho}$
of $Y(\gf)$ by setting
\begin{equation}\label{rep}
\tilde\rho(I_a) = \rho(I_a) \,, \qquad \tilde\rho(J_a) = 0 \,.
\end{equation}
Now, although $\tilde\rho$ is clearly consistent with
(\ref{Y1},\,\ref{Y2}), it is not, in general, consistent with
(\ref{Y3}) (we specialize here to $\gf\neq a_1$, and so do not
consider (\ref{Y4}) separately). Consistency is only possible for
irreps in which the right hand side of (\ref{Y3}) vanishes. This
is the case for the following irreps.

 Let $n_i$ be the coefficient of the simple
root ${\alpha}_i\in{\mathbb R}^r$ ($i=1,\ldots,r$) in the
expansion of the highest root ${\alpha}_{\rm max}$ of $\gf$, and
let $k_i = ({\alpha}_{\rm max},{\alpha}_{\rm
max})/({\alpha}_i,{\alpha}_i)$. Let the corresponding fundamental
weight be ${\omega}_i$. The irrep of $\gf$ with highest weight
${\Omega}$ may then be extended to an irrep of $Y(\gf)$ using
(\ref{rep}) for \begin{equation}
\begin{array}{rcl}
 & (i) & {\Omega}={\omega}_i \quad {\rm when } \quad
 n_i=k_i\\[0.2in]

 {\rm and} & (ii) &
  {\Omega}=t{\omega}_i\quad {\rm when }
 \quad n_i=1
 \qquad (t\in{\mathbb Z}).\end{array}\label{irreps} \end{equation}
These include all the fundamental irreps of $a_n$ and $c_n$, and
the vector and spinor irreps of $b_n$ and $d_n$. Only for one
algebra, $e_8$, is there no such rep.

A sketch of the proof of this is as follows. First, we need to
know the $\gf$-rep $X$ in which the right-hand side of (\ref{Y3})
acts. Since the $J_a$ form an adjoint representation of $\gf$, it
is clear from (\ref{Y3'}) that the left-hand side of (\ref{Y3}) is
contained in $\Lambda^2(\gf)$, the antisymmetric part of $\gf\otimes
\gf$. Further (and, as we saw above, equivalent to $H^2(\gf)=0$),
\begin{equation} f_{d(ab}u_{c)d}=0 \hspace{0.2in} \Rightarrow \hspace{0.2in}
u_{ab} = f_{abc}v_c \,, \end{equation} so that $X\oplus \gf=\Lambda^2(\gf)$,
and it turns out that $X$ is irreducible for all $\gf$. The image
of the right-hand side of (\ref{Y3}) in $End(V)$ is zero if
$X\otimes V\not\supset V$  by the Wigner-Eckhart theorem. Knowing
$X$, Drinfeld was then able to find the $V$, listed above, for
which this is true.

\subsubsection{The $Y(\gf)$-rep $\gf\oplus{\mathbb C}$}

The more general case, in which the $Y(\gf)$-irrep is
$\gf$-reducible, is much harder. Indeed, the only explicitly known
such rep is again due to Drinfeld, with $V=\gf\oplus{\mathbb C}$
(that is, {\em adjoint}$\oplus${\em singlet}). (Note that there
can be no rep of $Y(\gf)$ based on the adjoint rep of $\gf$ alone,
since $X \subset \gf\otimes \gf$.) The action of $Y(\gf)$ is
\begin{eqnarray}
\nonumber \rho(I_a) x = [I_a,x] &  & \rho(J_a) x = <x,I_a> \\
\label{gplusC}\label{adj} \rho(I_a) \lambda = 0 &  & \rho(J_a)
\lambda = d \lambda I_a
\end{eqnarray}
on $(x,\lambda) \in \gf \oplus {\mathbb C}$, where $<,>$ is an
inner product on $\gf$, and $d \in {\mathbb C}$  is dependent on
$\gf$ and on the choice of inner product. In fact, it is rather
intriguing that, for the exceptional algebras, $d$ depends only on
dim$\,\gf$, and that there is a uniform formula for the whole of
the exceptional series $a_2,g_2,d_4,f_4,e_6,e_7,e_8$, including
its classical elements. This, alongside the appearance of $X$ in
\cite{deligne,cvitanovic} and the unified $\cR$-matrix structure
of \cite{westbury}, suggests that it might be interesting to
investigate the connection between Yangians and the `magic square'
construction of the exceptional $\gf$.

\subsubsection{The tensor product graph}\label{TPG}

There are no explicit constructions of $Y(\gf)$ actions on more
general $\gf$-reducible $Y(\gf)$-irreps. However, we can construct
tensor products of the reps above, and some conclusions may be
drawn. Let us denote $\gf$-reps with upper-case letters
$V,W,\ldots$, and $L_\mu\left(Y(\gf)\right)$-reps ({\em i.e.\
}acted on by the automorphism (\ref{auto}), and thus carrying a
parameter $\mu$) with lower-case letters $v(\mu),w(\mu),\ldots$.
The essential point which will emerge is this: suppose we first
decompose $v(\mu)$ into $\gf$-irreps; of course we then know the
action of the $I_a$ on each. But we also know that the $J_a$ act
in the adjoint rep, and this limits the $\gf$-irreps between which
it may have non-trivial action.

Suppose we wish to construct the tensor product $u(\mu/2)\otimes
u(-\mu/2)$ where $u=U$ is an irrep $\rho$ of the form
(\ref{irreps}). The action on $U$ is \begin{equation}\label{irreps2}
\tilde\rho^{u(\pm \mu/2)}(I_a) = \rho(I_a) \,, \qquad
\tilde\rho^{u(\pm \mu/2)}(J_a) = \pm {\mu\over 2}\rho(I_a) \,, \end{equation}
 so that the action on the tensor product,
constructed using the coproduct (\ref{Y1del},\,\ref{Y2del}), is
\begin{eqnarray} \tilde\rho^{u(\mu/2)\otimes u(-\mu/2)}(I_a) & = &\quad
\rho(I_a)\otimes 1
+ 1 \otimes \rho(I_a)\label{act1}\\[0.1in]
\tilde\rho^{u(\mu/2)\otimes u(-\mu/2)}(J_a) & = & -{\mu\over
2}\left(
 1 \otimes \rho(I_a) - \rho(I_a)\otimes 1\right) + {1\over 4}
\left[\rho\otimes\rho(C),1\otimes \rho(I_a) - \rho(I_a)\otimes
1\right],\qquad\quad\label{act2}\end{eqnarray} where we have used the fact
that \begin{equation} {1\over 2} f_{abc} I_c\otimes I_b = {1\over 4} \left[ C,
1\otimes I_a-I_a\otimes 1\right]\,,\qquad {\rm with} \quad C\equiv
I_d\otimes I_d\,.\end{equation} It is clear from (\ref{act2}) that the $J_a$
act in the adjoint rep of (and which we shall write as) $\gf$, and
that they reverse parity. Further, \begin{equation} C={1\over 2}
\left(\Delta(I_d)^2 - I_d I_d\otimes 1 - 1 \otimes I_d
I_d\right)\,,\end{equation} so that $C$ takes the numerical value ${1\over
2}\left(C_2(W)-2C_2(U)\right)$ on a component $W$ of $U\otimes U$,
where $C_2=I_dI_d$ is the quadratic Casimir operator of $\gf$. Thus
the action of $J_a$ from $W$ to $W'$ may be non-trivial only if
$W'\subset \gf\otimes W$, if $W'$ and $W$ have opposite parity, and
if $\mu\neq {1\over 4} \left(C_2(W')-C_2(W)\right)$ (for if this
last equality is satisfied the right-hand side of (\ref{act2})
vanishes). However, the action from $W'$ to $W$ will not then
vanish -- and the $Y(\gf)$-rep $u(\mu/2)\otimes u(-\mu/2)$,
irreducible for general $\mu$, will be reducible, but not fully
reducible.

Let us look at an example, with $U=\tableau{1}$ the vector rep of
$SO(N)$. Then \begin{equation}\tableau{1}\otimes \tableau{1} = \tableau{2}
\oplus \tableau{1 1} \oplus 1\,,\end{equation} where we have denoted by
$\tableau{2}$ the traceless rank-two symmetric tensor, by
$\tableau{1 1}$ the rank-two antisymmetric tensor (the adjoint),
and by $1$ the one-dimensional representation. The $Y(\gf)$ action
on $u(\mu/2)\otimes u(-\mu/2)$ is most easily described by forming
these components into a graph, \begin{equation} \tableau{2}
\;\stackrel{1}{\longrightarrow}\; \tableau{1 1}
\;\stackrel{N/2-1}{\longrightarrow}\; 1\,,\end{equation} with a directed edge
from $W$ to $W'$ labelled with ${1\over
4}\left(C_2(W)-C_2(W')\right)$. We see that this is general
irreducible, but, at $\mu=\pm 1$, for example, is reducible (and
note also that both these possibilities, $\tableau{2}$ and
$\tableau{1 1}\,\oplus 1$, are irreps described earlier, in
(\ref{irreps}) and (\ref{gplusC}) respectively). This `tensor
product graph' (TPG) method \cite{TPG} is generally applicable
provided $U\otimes U$ (or indeed $U\otimes V$, with $V\neq U$) has
no multiplicities.

Via (\ref{R}), it also enables us to determine $\cR(\mu)$, which,
since it commutes with $\tilde\rho(I_a)$, (\ref{act1}), must be of
the form \begin{equation} \cR(\mu) = \sum_{W \subset U \otimes U} \tau_W(\mu)
P_W \end{equation} (where $P_W$ is the projector onto $W\subset U\otimes U$).
Then its commutation with $\tilde\rho(J_a)$, (\ref{act2}), implies
that, for each pair $W,W'$ of connected nodes of the TPG,
\begin{equation}\label{taus} \tau_{W'} = {\delta-\mu\over \delta+\mu}\,
\tau_W\,,\qquad{\rm where} \quad\delta = {1\over 4}\left(
C_2(W)-C_2(W')\right)\,.\end{equation} For the admissible $U$ of (\ref{rep}),
this system of equations proves to be consistent and determines
$\cR(\mu)$ up to an overall scalar factor; while for all other $U$
the equations are inconsistent. In our $SO(N)$ example above, the
$\cR(\mu)$ is that found in \cite{zams}. We also see that the
$\cR(\mu)$ constructed this way are rational in $\mu$, justifying
to some extent the claim at the end of the last section.

The above technique, applied to solutions of the YBE (the $\cR$),
is known as the `fusion procedure' \cite{KSR}, and may be used
even if the TPG fails. For example: suppose we wish, in the above
example, to calculate the $\cR$ acting in $W\otimes W$ where
$W=\tableau{1 1}\oplus 1$. Now $W\otimes W$ has various
multiplicities, and the TPG fails, but we can use the fact that
$u(1/2)\otimes u(-1/2)$ is reducible --- that $\cR(-1)$ does not
contain $P_{\tableau{2}}$ --- to construct $\cR$ on $W\otimes
W\subset U^{\otimes 4}$; it is \begin{equation}\label{R4} \cR(-1)\otimes
\cR(-1)\,.\, 1\otimes \cR(\mu-1)\otimes 1\,.\, \cR(\mu)\otimes
\cR(\mu)\,.\, 1\otimes \cR(\mu+1)\otimes 1\,.\end{equation} The essential
point is that the YBE allows the $\cR(-1)\otimes\cR(-1)$ term,
similar to the projector onto $(\tableau{1 1}\oplus 1)^{\otimes
2}\subset \tableau{1}^{\otimes 4}$, to be moved through to the
right of this expression, which can therefore consistently be
restricted to act on $W\otimes W$. (Incidentally, the resulting
$\cR$ \cite{mack91} can also be calculated directly \cite{chari91}
from the $Y(\gf)$ action (\ref{gplusC}) on $\gf\oplus{\mathbb C}$.)


\subsection{Yangians and Dorey's rule}

As promised, we finish with an aspect of the representation theory
which is rather nicer for $Y(\gf)$ than for $\gf$.

We denote by $V_i$ the $i$th fundamental $\gf$-irrep, with highest
weight $\omega_i$, and by $v_i(\mu)$ the $i$th fundamental
$L_\mu(Y(\gf))$-irrep, which, we recall, has $V_i$ as its `top'
component\footnote{That is, `top' according to a partial ordering
of highest weights in terms of the fundamental weights.}.

\subsubsection{Tensor products of fundamental $\gf$-irreps}

For representations of $\gf$, the Clebsch-Gordan decomposition
tells us when $V_i\otimes V_j \supset \bar{V}_k$, or, better to
say, when \begin{equation} V_i \otimes V_j \otimes V_k \supset 1\label{CG}\end{equation}
(where we again denote the one-dimensional rep as $1$). The
solutions to this were characterized in terms of weights in a
longstanding conjecture \cite{PRV}, proved relatively recently
\cite{kumar}. When specialized to the fundamental reps this states
that (\ref{CG}) holds if there exists an element $\sigma$ in the
Weyl group (of transformations of the root lattice generated by
the reflections $w_i$ through planes perpendicular to the simple
roots) $W$ of $\gf$ such that the dominant weight\footnote{A
dominant weight is one whose coefficients, when it is expressed in
terms of the simple weights, are all positive. They are also,
therefore, the highest weights.} conjugate to $-\omega_j-\sigma
\omega_k$ is $\omega_i$.


\subsubsection{Dorey's rule}\label{Dorey's rule}

A rule due to Dorey \cite{dorey}, and shown by Braden
\cite{braden} to be a restricted case of the above, was originally
discovered in the context of purely elastic scattering theories
(PESTs, indeed) in 1+1D integrable models, where it described
particle fusings (and thereby $S$-matrix pole structure and
perturbative three-point couplings in affine Toda field theory, at
least when $\gf$ is simply-laced \cite{BCDS}). It states that a
fusing $ijk$ occurs iff there exist integers $r$ and $s$ such that
\begin{equation} \omega_i + c^r \omega_j + c^s \omega_k=0\,,\label{dorey}\end{equation}
where $c=\prod_{i=1}^rw_i$ is a Coxeter element of the Weyl group.
(The ordering of the simple roots does not matter.) The simplest
case in which (\ref{dorey}) is satisfied but (\ref{CG}) is not is
the self-coupling of the second fundamental rep (the rank-two
antisymmetric tensor) of $d_5$.

Coxeter elements have many nice properties which make
(\ref{dorey}) more attractive than (\ref{CG}). Their action
partitions the roots of $\gf$ into $r$ orbits, each of size $h$,
where $h$ is the Coxeter number of $\gf$ (which therefore satisfies
$r(h+1)=$dim$\,\gf$). Further, $c$ acts very simply on $r$ planes
through the root lattice, in each of which it is a rotation by
$s\pi/h$, where $s$ is an exponent of $\gf$. In PESTs this yields a
beautiful\footnote{To be convinced of this one only need look at
the illustrations in \cite{coxeter}.} geometric interpretation of
the conservation in three-point couplings of a set of local
charges, with spins equal to the exponents: instead of $r$
conservation equations, one has a single equation in ${\mathbb
R}^r$, projected onto $r$ planes \cite{dorey}.

\subsubsection{Tensor products of fundamental $Y(\gf)$-irreps}

Various results from integrable models suggested that Dorey's rule
might apply to the fusion of fundamental $Y(\gf)$ reps
\cite{nankai}, and it was proved in 1995 \cite{chari95} that there
exist $\mu,\nu,\lambda$ such that \begin{equation} v_i(\mu)\otimes v_j(\nu)
\otimes v_k(\lambda) \supset 1\,,\label{CGY}\end{equation} precisely when
(\ref{dorey}) holds --- the values of  $\mu,\nu,\lambda$ are
related to the angles of the rotations described in the last
paragraph. (This is strictly true only for simply-laced $\gf$; for
nonsimply-laced $\gf$ there is a correspondence, but it is much
more subtle.) The proof, however, was case-by-case, and it is an
open question whether there is a more natural way to access the
hidden geometry of Yangian representations.



\subsection{Some further reading}

An alternative way to present the Yangian story is to begin with
(\ref{RTT})\footnote{Albeit with `quantum' and `auxiliary' spaces
exchanging roles --- see sect.\ref{qq}.} and an explicit
$R$-matrix and proceed from there to construct
$Y({\mathfrak{gl}}_n)$ and, via an appropriate `quantum
determinant', $Y({\mathfrak{sl}}_n)$ (see \cite{molev} and
references therein). Drinfeld also provided another realization of
$Y(\gf)$ in \cite{drinf87}, analogous to a Cartan-Weyl basis, which
is often used to study $Y(\gf)$ representation theory, and a set of
polynomials in correspondence with (finite-dimensional) $Y(\gf)$
reps which can be used to classify them (allowing one, for
example, to deduce the existence of the fundamental $Y(\gf)$ irreps
discussed in this chapter). For connections between these two
approaches see \cite{crampe}; for that between $Y(\gf)$ and the
Bethe ansatz see \cite{KR,KS}; for that with Hecke algebras see
\cite{cher,drinf86}; for that with separation-of-variables
techniques see \cite{skly}. We noted earlier that $Y(\gf)$ may be
thought of as a deformation of a polynomial algebra, with
parameter $z$: the analogue of the full loop algebra, with powers
of $z^{-1}$ as well, is the `quantum double' of $Y(\gf)$
\cite{Ydouble1,Ydouble2}. Super-Yangians have their origins in
\cite{nazsuper}; some representations were studied in
\cite{RBZhang}. Finally, $Y(\gf)$ can also be obtained in the
$q\rightarrow 1$ limit of a $q$-deformed untwisted affine algebra
$U_q(\hat{g})$ in the `spin gradation' (equivalent for
simply-laced $\gf$ to the principal gradation), as remarked in
\cite{drinf1,CPbook} and detailed for $\gf=a_1$ in \cite{BL2} -- we
give further details in an appendix. Indeed, the structure of the
representation ring of $U_q(\hat{g})$, for $q$ not a root of
unity, is the same as that of $Y(\gf)$.

\section{Y-,T- and Q-systems}
\label{sec:y-t-q}

\section{Kirillov-Reshetikhin formula}
\label{sec:kirill-resh-form}

\section{Graded tensor product decomposition and cluster algebras }
\label{sec:grad-tens-prod}

Klimyk's formula: \url{http://mathoverflow.net/questions/85593/decompose-tensor-product-of-type-g-2-lie-algebras}

\section{AdS/CFT}
\label{sec:adscft}



\end{document}
