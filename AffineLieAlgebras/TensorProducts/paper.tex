\documentclass[12pt]{iopart}

%\usepackage{amsfonts}
\usepackage{pb-diagram}
%%%%%%%%%%%%%%%%%%%%%%%%%%%%%%%%%%%%%%%%%%%%%%%%%%%%%%%%%%%%%%%%%%%%%%%%%%%%%%%%%%%%%%%%%%%%%%%%%%% 
\usepackage{iopams,setstack}
\usepackage{multicol}
\usepackage{color}
\usepackage{hyperref}
\usepackage{graphicx}
\usepackage[utf8x]{inputenc}
\usepackage[english,russian]{babel}

\newtheorem{Def}{Definition}[section]
\newtheorem{theorem}{Theorem}
\newtheorem{statement}{Statement}
\newtheorem{Cnj}[Def]{Conjecture}
\newtheorem{Prop}[Def]{Property}
\newtheorem{example}{Example}[section]


\newcommand{\go}{\stackrel{\circ }{\mathfrak{g}}}
\newcommand{\ao}{\stackrel{\circ }{\mathfrak{a}}}
\newcommand{\co}[1]{\stackrel{\circ }{#1}}
\newcommand{\pia}{\pi_{\mathfrak{a}}}
\newcommand{\piab}{\pi_{\mathfrak{a}_{\bot}}}
\newcommand{\gf}{\mathfrak{g}}
\newcommand{\gfh}{\hat{\mathfrak{g}}}
\newcommand{\af}{\mathfrak{a}}
\newcommand{\afh}{\hat{\mathfrak{a}}}
\newcommand{\bff}{\mathfrak{b}}
\newcommand{\afb}{\mathfrak{a}_{\bot}}
\newcommand{\hf}{\mathfrak{h}}
\newcommand{\hfg}{\hf_{\gf}}
\newcommand{\hfa}{\hf_{\af}}
\newcommand{\hfb}{\mathfrak{h}_{\bot}}
\newcommand{\pf}{\mathfrak{p}}
\newcommand{\aft}{\widetilde{\mathfrak{a}}}
\newcommand{\sfr}{\mathfrak{s}}


\begin{document}

\ftc{Feigin-Loktev fusion product and formulas for graded tensor product multiplicities}

\author{V.D.~Lyakhovsky$^1$, A.A.~Nazarov$^{1,2}$, O.V.~Postnova$^{1}$, A..~Komisartchuk$^{1}$}
\address{ $^1$ Department of High-energy and elementary particle physics,  St Petersburg State University, 198904, Saint-Petersburg, Russia}
\ead{ $^{2}$ e-mail: antonnaz@gmail.com}



\begin{abstract}
  Feigin-Loktev fusion product appeared from the study of operator algebra of conformal field
  theory. It generalizes tensor product of representations of simple Lie algebras by introducing
  grading. Graded product multiplicities are generalized Kostka polynomials. We start with the
  fermionic formulae for these multiplicities and derive closed explicit expressions that do not
  contain summation over integer partitions. We present closed formulas for graded tensor product
  multiplicities for the fundamental representations of simple Lie algebras of small rank. These
  formulas can be regarded as q-analogue of multinomial coefficients. 
\noindent{\it Keywords\/}: Lie algebra, tensor product, multiplicity, cluster algebra, grading
\end{abstract}

\submitto{\jpa}

\section{Introduction}
\label{sec:introduction}

Feigin-Loktev construction of graded tensor product of finite-dimensional representations of simple
Lie algebras was recently interpreted from the point of view of cluster algebras theory. This
construction of graded tensor product is rooted in the fusion product of conformal field theory.
Cluster algebra interpretation produces the generalization of Kirillov-Reshetikhin formula for
tensor product decomposition coefficients to the graded case. Kirillov-Reshetikhin formula is the
fermionic formula and follows from Bethe ansatz equations. On the other hand tensor product
decomposition can be calculated recursively with something like so-called Klimyk rule which relies
on the Weyl character formula. So far there is no such simple method for graded tensor product. In
the present paper we provide a recurrent relations for the graded tensor product decomposition
coefficients and compare recurrent computations with the Kirillov-Reshetikhin formula. Note that
recurrent approach is not limited to any specific class of modules and is more general.

In certain cases fusion modules are Weyl or Demazure modules of the current algebra
\cite{chari2001weyl,chari2006weyl,chari2014modules,chari2015demazure}. Then formulas for graded
tensor product multiplicities are known (see \cite{chari2006weyl,chari2015demazure}). Some of these
cases involve non-symmetric Macdonald polynomials \cite{0803.1146v1}. 

\begin{itemize}
\item Di Francesco-Kedem's paper
\item Feigin-Loktev tensor product.
  
  Let $\gf$ be a semisimple Lie algebra and b $L^{\mu}_{\gf}$ denote its finite-dimensional
  irreducible module with highest weight $\mu$. Feigin and Loktev considered more general cyclic
  modules but we limit ourselves with irreducible modules here. First step is to define a filtration
  on the tensor product of the irreducible modules $V=L^{\mu_{1}}\otimes\dots \otimes L^{\mu_{n}}$.
  Consider the current algebra $\gf[[t]]$ of $\gf$-valued complex polynomials with generators
  $X[n]=X\otimes t^{n}$ where $X\in \gf$. For $z\in\mathbb{C}$ we can define the action of $X[n]\in
  \gf[[t]]$ on $v\in L^{\mu}_{\gf}$ by $X[n]\cdot v = z^{n}X v$. Take complex vector
  $(z_{1},\dots,z_{n})$ then we have the action of $\gf[[t]]$ on $V=L^{\mu_{1}}\otimes\dots \otimes
  L^{\mu_{n}}$ by the rule
  \begin{equation}
    \label{eq:1}
    X[m](v_{1}\otimes \dots\otimes v_{n})=\sum_{i=1}^{n} z_{i}^{m}
    v_{1}\otimes\dots\otimes X v_{i}\otimes\dots\otimes v_{n}.    
  \end{equation}

 In the paper \cite{feigin1999generalized} it was shown that the module $V$ of $\gf[[t]]$ is
 generated by the action of lowering operators on the tensor product of highest weight vectors
 $v_{1}\otimes\dots \otimes v_{n}$. The algebra $\gf[[t]]$ is graded by the degree of polynomial
 and the algebra $U(\gf[[t]])$ is also graded. Then we have the induced filtration on $V$:
 \begin{equation}
   \label{eq:2}
   F^{i}V=U^{\leq i}(\gf[[t]])v_{1}\otimes\dots\otimes v_{n},\quad F^{0}V\subset F^{1}V\subset\dots
 \end{equation}

One can show that the number of graded components $F_{j}=F^{j}V/F^{j-1}V$ is finite. 

 Note that $F^{0}V$ is the irreducible $\gf$-submodule corresponding to the highest weight in the
 tensor product decomposition
 \begin{equation}
   \label{eq:3}
   L^{\mu_{1}}\otimes\dots \otimes L^{\mu_{n}}=L^{\nu_{1}}\oplus\dots\oplus L^{\nu_{k}}
 \end{equation}

Having defined the filtration on $V$ one can introduce ``graded tensor product decomposition
coefficients'' or Hilbert polynomials by the formula
\begin{equation}
  \label{eq:4}
  M^{\mu_{1},\dots \mu_{n}}_{\lambda}(t)=\sum_{k\geq 0}t^{k} \mathrm{dim}\mathrm{Hom}_{\gf} (F_{k},L^{\lambda})
\end{equation}

If we set $t=1$ these coefficients coincide with tensor product decomposition coefficients:
\begin{equation}
  \label{eq:5}
  M^{\mu_{1},\dots,\mu_{n}}_{\lambda}(1)=  M^{\mu_{1},\dots,\mu_{n}}_{\lambda}
\end{equation}

 \begin{example}
   Consider tensor powers of fundamental representation of $sl_{2}$. 
   
 \end{example}

Tensor product as affine Lie algebra module:   

Klimyk rule:
\begin{equation}
  \label{eq:6}
  \mathrm{ch}L^{\mu}\cdot\mathrm{ch}L^{\nu}=\sum_{\xi\in \mbox{weights}L^{\nu}}\mathrm{ch} L^{\mu+\xi}
\end{equation}

\item Fusion product of Virasoro modules, fusion in WZW models and Feigin-Loktev tensor product 

\item Use generalized Kirillov-Reshetikhin formula:\\
    Tensor product of $n_{a,j}$ modules with highest weight $j\omega_{a}$
    \begin{equation}
      \label{eq:3}
      M^{(\bf{n})}_{\lambda}(t)=\sum_{\{m_{a,j}\}}t^{\sum_{a,b,i,j}m_{a,j}C_{\gf}^{ab}\mathrm{min}(i,j)m_{b,j}}\prod_{a,j}\left[\begin{array}{c}p_{a,j}+m_{a,j}\\ m_{a,j}\end{array}\right]_{t},
      \end{equation}
      where $C_{\gf}$ -- Cartan matrix,  the sum is restricted:
      \begin{equation}
        \label{eq:6}
        \sum_{a,j}j\omega_{a}(n_{a,j}-\sum_{b}C^{ab}m_{b,j})=\lambda
      \end{equation}
      \begin{equation}
        \label{eq:7}
        p_{a,j}=\sum_{j}\mathrm{min}(i,j) n_{a,j}-\sum_{b}\mathrm{min}(i,j)m_{b,j}
      \end{equation}
     $\left[\begin{array}{c}n\\ k\end{array}\right]_{t}=\frac{(n!)_{t}}{(k!)_{t}((n-k)!)_{t}}$, $(n)_{t}=1+t+\dots t^{n-1}$.

\item Weyl character formula and formulas for tensor product
\item Weyl character formula for graded tensor product
\item Computations: examples, performance
\item Cluster algebra perspective
\end{itemize}

\section{Simple example}
\label{sec:simple-example}

Let's remind the properties of $q$-binomial coefficients
(http://mathworld.wolfram.com/q-BinomialCoefficient.html). Recurrent relation for binomial coefficients
\begin{equation}
  \label{eq:7}
  \left(\begin{array}{c}n\\ k\end{array}\right)=\left(\begin{array}{c}n-1\\
      k-1\end{array}\right)+\left(\begin{array}{c}n-1\\ k\end{array}\right) 
\end{equation}
is easy to prove:
\begin{equation}
  \label{eq:8}
  \frac{n!}{k!(n-k)!}=\frac{(n-1)!}{(k-1)!(n-k)!}+\frac{(n-1)!}{k!(n-k-1)!}
\end{equation}
Multiply by $k!(n-k)!/(n-1)!$:
\begin{equation}
  \label{eq:9}
  n=k+(n-k)
\end{equation}

Similarly for $q$-binomials we should get
\begin{equation}
  \label{eq:10}
  (n)_{q}=(k)_{q}+(n-k)_{q},
\end{equation}
but this is not the case, so we need to multiply $(n-k)_{q}$ by $q^{k}$. The resulting recurrent
relation is
\begin{equation}
  \label{eq:11}
  \left[\begin{array}{c}n\\ k\end{array}\right]_{q}=\left[\begin{array}{c}n-1\\ k-1\end{array}\right]_{q}+q^{k}\left[\begin{array}{c}n-1\\ k\end{array}\right]_{q}
\end{equation}

For $sl_{2}$ tensor product decomposition coefficients we have
\begin{equation}
  \label{eq:12}
  M^{n}_{k}(t)=\left[\begin{array}{c}n\\ k\end{array}\right]_{t}-\left[\begin{array}{c}n\\ k-1\end{array}\right]_{t}
\end{equation}
\begin{equation}
  \label{eq:13}
  M^{n}_{k}(t)=M^{n-1}_{k-1}(t)+t^{k-1}M^{n-1}_{k}+(t-1)t^{k-1}\left[\begin{array}{c}n-1\\ k\end{array}\right]_{t}
\end{equation}

$sl_2$ case here
Формула Кириллова-Решетихина дает кратность неприводимого представления со старшим весом $\lambda$ в
тензорном произведении  $n_{a,j}$ представлений со старшими весами $j\omega_{a}$, где $\omega_{a}$
-- фундаментальный вес с номером $a=1\dots \mathrm{rank}\gf$. 
\begin{equation}
  \label{eq:3}
  M^{(\bf{n})}_{\lambda}(t)=\sum_{\{m_{a,j}\}}t^{\sum_{a,b,i,j}m_{a,j}C_{\gf}^{ab}\mathrm{min}(i,j)m_{b,j}}\prod_{a,j}\left[\begin{array}{c}p_{a,j}+m_{a,j}\\ m_{a,j}\end{array}\right]_{t},
\end{equation}
здесь $C_{\gf}$ -- матрица Картана, а сумма ограничена условиями
\begin{equation}
  \label{eq:6}
  \sum_{a,j}j\omega_{a}(n_{a,j}-\sum_{b}C^{ab}m_{b,j})=\lambda
\end{equation}
\begin{equation}
  \label{eq:7}
  p_{a,i}=\sum_{j}\mathrm{min}(i,j) \left(n_{a,j}-\sum_{b}C^{ab}m_{b,j}\right)
\end{equation}
$\left[\begin{array}{c}n\\ k\end{array}\right]_{t}=\frac{(n!)_{t}}{(k!)_{t}((n-k)!)_{t}}$, $(n)_{t}=1+t+\dots t^{n-1}$.

Сперва рассмотрим простейший пример, $\gf=sl_{2}$, $t=1$. Формула Кириллова-Решетихина будет иметь вид
\begin{equation}
  \label{eq:1}
  M_{\lambda}^{(n_{1},n_{2},\dots)}=\sum_{m_{j}}\prod_{j}\left(
    \begin{array}{c}
      p_{j}+m_{j}\\
      m_{j}
    \end{array}\right),
\end{equation}
\begin{equation}
  \label{eq:2}
  \sum_{j}j\omega (n_{j}-2m_{j})=\lambda
\end{equation}
\begin{equation}
  \label{eq:4}
  p_{j}=\sum_{i}\min(i,j)(n_{i}-2m_{i})\geq 0
\end{equation}
Попробуем вычислить что-нибудь по этой формуле. Например, кратность представления со старшим весом
$1$ (индекс Дынкина) в произведении пяти фундаментальных представлений. $n_{1}=5, n_{2}=n_{3}=\dots=0$
\begin{equation}
  \label{eq:5}
  (n_{1}-2m_{1})+2(-2m_{2})+3(-m_{3})+\dots=1
\end{equation}
\begin{equation}
  \label{eq:9}
  m_{j}\geq 0
\end{equation}
\begin{equation}
  \label{eq:8}
    (5-2m_{1})+2(-2m_{2})+3(-m_{3})+\dots=1
\end{equation}
Решения: $m_{1}=2, m_{j}=0$, $m_{1}=0, m_{2}=1$. 
Для $p_{j}$ имеем в первом случае $p_{1}=1$, во втором $p_{1}=5-2=3$,$p_{2}= 5+2\cdot (-2)=1$ в итоге кратность
\begin{equation}
  \label{eq:10}
   M_{1}^{(5)}=\left(
    \begin{array}{c}
      1+2\\
      2
    \end{array}\right)+
\left(\begin{array}{c}
      1+1\\
      1
    \end{array}\right)=5
\end{equation}


\section{Proof}
\label{sec:proof}
$sl_{3}, so_{5}$ here
  \begin{itemize}
  \item $t$-deformation for the tensor product formula for $\mathfrak{sl}_{2}$:
    \begin{equation}
      \label{eq:8}
      m^{\otimes n}_{l}(t)=\left[\begin{array}{c}n-l\\n\end{array}\right]_{t}-      \left[\begin{array}{c}n-l-1\\n\end{array}\right]_{t}
    \end{equation}
  \item  $t$-deformation for the tensor product formula for $\mathfrak{sl}_{3}$:
    \begin{equation}
      \label{eq:9}
      \begin{array}{l}
      M^{n}_{p_{1},p_{2}}(t)=(n,p_{1},p_{2})_{t}-(n,p_{1}-1,p_{2})_{t}+(n,p_{1}-2,p_{2}-1)_{t}\\-(n,p_{1},p_{2}-1)_{t}+(n,p_{1}-1,p_{2}-2)_{t}-(n,p_{1}-2,p_{2}-2)_{t},        
      \end{array}
    \end{equation}
    where $(n,p_{1},p_{2})_{t}=\frac{(n!)_{t}}{(p_{2}!)_{t}((n-p_{1})!)_{t}((p_{1}-p_{2})!)_{t}}$
  \end{itemize}

Compare to Kirillov-Reshetikhin formula:

Рассмотрим теперь алгебру $sl_{3}$. Матрица Картана
\begin{equation}
  \label{eq:11}
  C=\left(\begin{array}{cc}
      2 & -1\\
      -1 & 2
    \end{array}\right)
\end{equation}
Вычислим кратность представления со старшим весом с индексами Дынкина $(1,2)$ в произведении пяти
фундаментальных представлений $L^{\omega_{1}}$. $n_{1,1}=5, n_{ij}=0$. Уравнения на $m_{aj}$:
\begin{eqnarray}
  \label{eq:12}
  (5-2m_{1,1}+m_{2,1})+2(-2m_{12}+m_{22})+3(-2m_{13}+m_{23})+4(-2m_{14}+m_{24})+\dots=1\\
  (m_{1,1}-2m_{2,1})+2(m_{12}-2m_{22})+3(m_{13}-2m_{23})+4(m_{14}-2m_{24})+\dots=2
\end{eqnarray}
сложим:
\begin{equation}
  \label{eq:13}
  (5-m_{1,1}-m_{2,1})-2(m_{12}+m_{22})-3(m_{13}+m_{23})-4(m_{14}+m_{24})+\dots=3
\end{equation}
или
\begin{equation}
  \label{eq:14}
    -(m_{1,1}+m_{2,1})-2(m_{12}+m_{22})-3(m_{13}+m_{23})-4(m_{14}+m_{24})+\dots=-2
\end{equation}
решения:
$m_{11}=2, m_{ij}=0$, $m_{12}=1, m_{ij}=0$
Вычислим $p_{ij}$. 
\begin{eqnarray}
  \label{eq:15}
  p_{11}=(n_{11}-2m_{11}+m_{21})+(-2m_{12}+m_{22})+\dots\\
  p_{21}=(m_{11}-2m_{21})+(m_{12}-2m_{22})+\dots\\
  p_{12}=(n_{11}-2m_{11}+m_{21})+2(-2m_{12}+m_{22})+\dots\\
  p_{22}=(m_{11}-2m_{21})+2(m_{12}-2m_{22})+\dots
\end{eqnarray}
В первом случае $p_{11}=1$, а остальные не нужны. Во втором $p_{11}=3, p_{21}=1, p_{12}=1,
p_{22}=2$, из которых нужно только $p_{12}$. Подставляем в формулу Кириллова-Решетихина:
\begin{equation}
  \label{eq:16}
    M_{(2,1)}^{(5)}=\left(
    \begin{array}{c}
      1+2\\
      2
    \end{array}\right)+
\left(\begin{array}{c}
      1+1\\
      1
    \end{array}\right)=5
\end{equation}
Что совпадает с разложением тензорного произведения другими способами. 

Заметим, что здесь, как и в случае $sl_{2}$, при поиске решений надо учитывать те $m_{ij}$, для
которых $n_{ij}=0$. 


\section*{Conlusion}
\label{sec:conlusion}


\begin{equation}
  \label{eq:17}
  \sum_{\nu=j}^{q-1} t^{(q-\nu)(q-\nu-1)} C_{q-j}^{\nu-j} C_{l-q-1}^{q-\nu-1} = 
  \sum_{\nu=j}^{q-1} t^{(q-\nu)(q-\nu-1)} C_{q-j}^{q-\nu} C_{l-q-1}^{q-\nu-1} 
\end{equation}
обозначим $q-1-\nu$ через $k$, тогда $\nu=q-1-k$, суммирование идет по $k=0\dots q-j-1$ и мы имеем
\begin{equation}
  \label{eq:18}
  \sum_{k=0}^{q-j-1} t^{(k+1)k} C_{q-j}^{k+1} C_{l-q-1}^{k} =   \sum_{k=0}^{q-j-1} t^{(k+1)k} C_{q-j}^{q-j-1-k} C_{l-q-1}^{k} = C_{l-j-1}^{q-j-1},
\end{equation}
где в последнем равенстве использовано тождество Ва
\cite{di2013quantum} 

Recurrent relation for fusion modules characters is here \cite{di2015difference}
\bibliography{cluster}{} 
\bibliographystyle{iopart-num}

\end{document}
