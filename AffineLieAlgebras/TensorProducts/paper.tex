\documentclass[12pt]{iopart}

%\usepackage{amsfonts}
\usepackage{pb-diagram}
%%%%%%%%%%%%%%%%%%%%%%%%%%%%%%%%%%%%%%%%%%%%%%%%%%%%%%%%%%%%%%%%%%%%%%%%%%%%%%%%%%%%%%%%%%%%%%%%%%% 
\usepackage{iopams,setstack}
\usepackage{multicol}
\usepackage{color}
\usepackage{hyperref}
\usepackage{graphicx}
\usepackage[utf8x]{inputenc}
\usepackage[english]{babel}

\newtheorem{Def}{Definition}[section]
\newtheorem{theorem}{Theorem}
\newtheorem{statement}{Statement}
\newtheorem{Cnj}[Def]{Conjecture}
\newtheorem{Prop}[Def]{Property}
\newtheorem{example}{Example}[section]


\newcommand{\go}{\stackrel{\circ }{\mathfrak{g}}}
\newcommand{\ao}{\stackrel{\circ }{\mathfrak{a}}}
\newcommand{\co}[1]{\stackrel{\circ }{#1}}
\newcommand{\pia}{\pi_{\mathfrak{a}}}
\newcommand{\piab}{\pi_{\mathfrak{a}_{\bot}}}
\newcommand{\gf}{\mathfrak{g}}
\newcommand{\gfh}{\hat{\mathfrak{g}}}
\newcommand{\af}{\mathfrak{a}}
\newcommand{\afh}{\hat{\mathfrak{a}}}
\newcommand{\bff}{\mathfrak{b}}
\newcommand{\afb}{\mathfrak{a}_{\bot}}
\newcommand{\hf}{\mathfrak{h}}
\newcommand{\hfg}{\hf_{\gf}}
\newcommand{\hfa}{\hf_{\af}}
\newcommand{\hfb}{\mathfrak{h}_{\bot}}
\newcommand{\pf}{\mathfrak{p}}
\newcommand{\aft}{\widetilde{\mathfrak{a}}}
\newcommand{\sfr}{\mathfrak{s}}


\begin{document}

\ftc{Recurrent formulae for graded tensor product multiplicities and cluster algebras}

\author{V.D.~Lyakhovsky$^1$, A.A.~Nazarov$^{1,2}$, O.V.~Postnova$^{1}$, A..~Komisartchuk$^{1}$}
\address{ $^1$ Department of High-energy and elementary particle physics,  St Petersburg State University, 198904, Saint-Petersburg, Russia}
\ead{ $^{2}$ e-mail: antonnaz@gmail.com}



\begin{abstract}

We show the connection of formulae for tensor product multiplicities with cluster algebras. 
\noindent{\it Keywords\/}: Lie algebra, tensor product, multiplicity, cluster algebra, grading
\end{abstract}

\submitto{\jpa}

\section{Introduction}
\label{sec:introduction}

Feigin-Loktev construction of graded tensor product of finite-dimensional representations of simple
Lie algebras was recently interpreted from the point of view of cluster algebras theory. This
construction of graded tensor product is rooted in the fusion product of conformal field theory.
Cluster algebra interpretation produces the generalization of Kirillov-Reshetikhin formula for
tensor product decomposition coefficients to the graded case. Kirillov-Reshetikhin formula is the
fermionic formula and follows from Bethe ansatz equations. On the other hand tensor product
decomposition can be calculated recursively with something like so-called Klimyk rule which relies
on the Weyl character formula. So far there is no such simple method for graded tensor product. In
the present paper we provide a recurrent relations for the graded tensor product decomposition
coefficients and compare recurrent computations with the Kirillov-Reshetikhin formula. Note that
recurrent approach is not limited to any specific class of modules and is more general. 

\begin{itemize}
\item Di Francesco-Kedem's paper
\item Feigin-Loktev tensor product.
  
  Let $\gf$ be a semisimple Lie algebra and b $L^{\mu}_{\gf}$ denote its finite-dimensional
  irreducible module with highest weight $\mu$. Feigin and Loktev considered more general cyclic
  modules but we limit ourselves with irreducible modules here. First step is to define a filtration
  on the tensor product of the irreducible modules $V=L^{\mu_{1}}\otimes\dots \otimes L^{\mu_{n}}$.
  Consider the current algebra $\gf[[t]]$ of $\gf$-valued complex polynomials with generators
  $X[n]=X\otimes t^{n}$ where $X\in \gf$. For $z\in\mathbb{C}$ we can define the action of $X[n]\in
  \gf[[t]]$ on $v\in L^{\mu}_{\gf}$ by $X[n]\cdot v = z^{n}X v$. Take complex vector
  $(z_{1},\dots,z_{n})$ then we have the action of $\gf[[t]]$ on $V=L^{\mu_{1}}\otimes\dots \otimes
  L^{\mu_{n}}$ by the rule
  \begin{equation}
    \label{eq:1}
    X[m](v_{1}\otimes \dots\otimes v_{n})=\sum_{i=1}^{n} z_{i}^{m}
    v_{1}\otimes\dots\otimes X v_{i}\otimes\dots\otimes v_{n}.    
  \end{equation}

 In the paper \cite{feigin1999generalized} it was shown that the module $V$ of $\gf[[t]]$ is
 generated by the action of lowering operators on the tensor product of highest weight vectors
 $v_{1}\otimes\dots \otimes v_{n}$. The algebra $\gf[[t]]$ is graded by the degree of polynomial
 and the algebra $U(\gf[[t]])$ is also graded. Then we have the induced filtration on $V$:
 \begin{equation}
   \label{eq:2}
   F^{i}V=U^{\leq i}(\gf[[t]])v_{1}\otimes\dots\otimes v_{n}.
 \end{equation}
 Note that $F^{0}V$ is the irreducible $\gf$-submodule corresponding to the highest weight in the
 tensor product decomposition
 \begin{equation}
   \label{eq:3}
   L^{\mu_{1}}\otimes\dots \otimes L^{\mu_{n}}=L^{\nu_{1}}\oplus\dots\oplus L^{\nu_{k}}
 \end{equation}



\item Fusion product of Virasoro modules, fusion in WZW models and Feigin-Loktev tensor product 
\item KR-formula
\item Weyl character formula and formulas for tensor product
\item Weyl character formula for graded tensor product
\item Computations: examples, performance
\item Cluster algebra perspective
\end{itemize}

\cite{di2013quantum} 
\bibliography{cluster}{} 
\bibliographystyle{iopart-num}

\end{document}
