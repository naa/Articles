\documentclass{article}

\addtolength{\textheight}{-10mm}
\addtolength{\topmargin}{-5mm}

\pagestyle{plain}

\def\title#1{\begin{center}#1\end{center}}

\def\author#1{\centerline{\small{\textsc{#1}}}}

\def\address#1#2{\begin{center}\small\emph{#1}\\E-mail: \texttt{#2}\end{center}}

\def\refname{{\small References}}

\begin{document}
\title{\textsc{Reduction of representations of affine Lie algebra to affine sub-algebra and applications to CFT}}
%\footnote{This work is supported by etc, etc.}}

\author{A.~Nazarov}

\address{Department of High Energy and Elementary Particles Physics, \\ Saint-Petersburg State University\\
Ulyanovskaya str. 1, Petrodvorets., \\
198504 St-Petersburg, Russia}{antonnaz@gmail.com}

\bigskip

\small

   It was demonstrated \cite{ilyin812pbc} that decompositions of integrable highest weight
modules of a simple Lie algebra with respect to its reductive subalgebra
obey the set of algebraic relations leading to the recursive properties for
the corresponding branching coefficients. These properties can be used to formulate an algorithm for explicit computation of branching coefficients.

The algorithm for reduction of representation of affine Lie algebra to representation of affine sub-algebra is described. The use of Weyl symmetry allows to compute branching coefficients only in closure of main Weyk chamber. It gives the opportunity of efficient computation of branching coefficients.

Reduction of representations has several applications in two-dimensional conformal field theory. We discuss the examples of the applications of reduction of representations to construction of modular invariants of Wess-Zumino-Novikov-Witten models.

\begin{thebibliography}{5}\footnotesize


\bibitem{ilyin812pbc}
M.~Ilyin, P.~Kulish, and V.~Lyakhovsky, ``{On a property of branching
  coefficients for affine Lie algebras},'' {\em Algebra i Analiz} {\bf 21:2} 

\bibitem{kulish4sfa}
P.~Kulish and V.~Lyakhovsky, ``{String Functions for Affine Lie Algebras
  Integrable Modules},'' {\em Symmetry, Integrability and Geometry: Methods and
  Applications} {\bf 4}

\bibitem{Nazarov2009}
A.~Nazarov, ``Computational tools for representation theory of affine Lie algebras,'' in {\em Workshop on Advanced Computer Simulation Methods for Junior Scientists}, Saint-Petersburg 2009.

\end{thebibliography}

\end{document}

