\documentclass[a4paper,12pt]{article}
\usepackage{ucs}
\usepackage[unicode,verbose]{hyperref}
\usepackage{amsmath,amssymb,amsthm}
\usepackage{pb-diagram}
\usepackage{multicol}
%\usepackage[utf8x]{inputenc}
%\usepackage[russian]{babel}
\usepackage{cmap}
\usepackage{color}
\usepackage[pdftex]{graphicx}
\pagestyle{plain}

%\usepackage{verbatim} 
\newenvironment{comment}
{\par\noindent{\bf TODO}\\}
{\\\hfill$\scriptstyle\blacksquare$\par}

\newtheorem{statement}{Statement}
\theoremstyle{definition} \newtheorem{Def}{Definition}
\newcommand{\go}{\overset{\circ }{\frak{g}}}
\newcommand{\ao}{\overset{\circ }{\frak{a}}}
\newcommand{\co}[1]{\overset{\circ }{#1}}

\begin{document}
\title{Modular properties of string, branching and dual functions}
\author{Vladimir Lyakhovsky \thanks{ Supported by
 RFFI grant N 09-01-00504 and the National Project RNP.2.1.1./1575 }\\
Theoretical Department, SPb State University,\\
198904, Sankt-Petersburg, Russia \\
e-mail:lyakh1507@nm.ru \\
[5mm] Anton Nazarov \thanks{ Supported by
the National Project RNP.2.1.1./1575 }\\
Theoretical Department, SPb State University,\\
198904, Sankt-Petersburg, Russia \\
e-mail:antonnaz@gmail.com
}
\maketitle

\begin{abstract}
  We review modular forms and discuss modular properties of string, branching and dual functions.
\end{abstract}


\end{document}
