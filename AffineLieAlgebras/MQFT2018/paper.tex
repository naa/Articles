

\documentclass{article}
\usepackage{hyperref}
\usepackage{graphicx}
\usepackage[utf8x]{inputenc}
\usepackage[T1,T2A]{fontenc}
\usepackage[russian,english]{babel}

\usepackage{amsmath,amssymb,amsthm,amsfonts}



\newtheorem{Def}{Definition}[section]
\newtheorem{theorem}{Theorem}
\newtheorem{statement}{Statement}
\newtheorem{Cnj}[Def]{Conjecture}
\newtheorem{Prop}[Def]{Property}
\newtheorem{example}{Example}[section]

\newcommand{\Li}{\mathrm{Li}}
\newcommand{\dx}{\mathrm{dx}~}
\newcommand{\dy}{\mathrm{dy}~}
\newcommand{\dz}{\mathrm{dz}}
\newcommand{\go}{\stackrel{\circ }{\mathfrak{g}}}
\newcommand{\ao}{\stackrel{\circ }{\mathfrak{a}}}
\newcommand{\co}[1]{\stackrel{\circ }{#1}}
\newcommand{\pia}{\pi_{\mathfrak{a}}}
\newcommand{\piab}{\pi_{\mathfrak{a}_{\bot}}}
\newcommand{\gf}{\mathfrak{g}}
\newcommand{\nf}{\mathfrak{g}}
\newcommand{\gfh}{\hat{\mathfrak{g}}}
\newcommand{\af}{\mathfrak{a}}
\newcommand{\afh}{\hat{\mathfrak{a}}}
\newcommand{\bff}{\mathfrak{b}}
\newcommand{\slnn}{\mathfrak{sl}_{n}}
\newcommand{\sln}[1]{\mathfrak{sl}_{#1}}
\newcommand{\afb}{\mathfrak{a}_{\bot}}
\newcommand{\hf}{\mathfrak{h}}
\newcommand{\hfg}{\hf_{\gf}}
\newcommand{\hfa}{\hf_{\af}}
\newcommand{\hfb}{\mathfrak{h}_{\bot}}
\newcommand{\pf}{\mathfrak{p}}
\newcommand{\aft}{\widetilde{\mathfrak{a}}}
\newcommand{\sfr}{\mathfrak{s}}
%\newcommand{\dim}{\mathrm{dim} }

\begin{document}
\title{Finite-size scaling of free energy in dimer model on hexagonal domain}

\author{A.A.~Nazarov$^{1}$\\
{\small
  $^{1}$Department of Physics, St. Petersburg State University,} \\
{\small  Ulyanovskaya 1, 198504 St.~Petersburg, Russia}\\
%\small{$^{2}$}\\
%\small{$^{3}$}\\
\small{email:antonnaz@gmail.com}
}
\date{}
\maketitle

\begin{abstract}
    We consider dimer model on a hexagonal lattice. This model can be seen as a ``pile of cubes in the
  corner''. The partition function is given by the determinant of Kasteleyn matrix. We use the
  expression for the partition function to derive the scaling behavior of free energy in the limit
  of lattice mesh tending to zero and temperature tending to infinity. We discuss the connection of
  the expansion coefficients to the geometry of the domain.
\end{abstract}


\section*{Introduction}
\label{sec:introduction}
The dimer model appeared as a very simplified model of solutions \cite{Fowler-1937}. The molecules
are represented by the rigid tiles on a lattice. The partition function was computed on various
lattices and with different boundary conditions \cite{doi:10.1080/14786436108243366,
  P.W-1961,kenyon2009lectures}.

Dimer models are the integrable lattice models of statistical physics that are under an active
theoretical~\cite{zj2000,ferrari} and numerical
investigation~\cite{ks2018}. Particular dimer model appears for certain choice
of parameters in the well-known six-vertex model  \cite{pronko2017thesis}. Dimer models are
closely related to the theory of alternating matrices \cite{elkies1992alternating1,elkies1992alternating2}

Specifically, the well-known limit shape phenomenon~\cite{vershik1977kerov} was discovered for dimer
model~\cite{kenyon2006dimers} and a connection with the theory of random matrices was
established~\cite{johansson2002non}.


Study of dimer models on various lattices and domains leads to interesting connections with the
geometry of curved manifolds and with spectra of discrete and continuous Dirac and Laplace operators
\cite{kenyon2002laplacian,kenyon2000asymptotic}. Scaling limit of dimer model is proven to be
described by a Gaussian free-field theory \cite{kenyon2001dominos}, but finite-size corrections were
not considered previously. These corrections are important to close the gap between numerical
simulations and theoretical results.

We consider a particular case of the dimer model on a hexagonal domain of hexagonal lattice, that
can be seen as a ``pile of cubes in the corner''. The energy of configuration is the total number of
cubes. For this particular case we use an exact combinatorial formula for the partition function to
derive the expressions for scaling limit of free energy and first three terms of the finite-size
corrections. We show that the first term is identically zero. The second term has a universal part
that is given by the Euler constant and a geometry-dependent part that is written explicitly in
elementary functions. The third term which encapsulates logarithmic dependence on the mesh size is
also shown to be universal.

We than compare these results with our previous numeric simulations \cite{belov2018finite}. 
\section{Model definition}
\label{sec:model-definition}
The configurations of the dimer model are perfect matchings (sets of non-touching edges, covering
all the vertices) on some graph ${\cal G}$ with some choice of weights $\omega(e)$ on the edges. The
model is solvable on the bipartite graphs, i.e. the partition function can be computed if the
weights are introduced in such a way that each face with 0 mod 4 edges there is an odd number of
negative edge weights and each face with 2 mod 4 edges has an even number of of negative edge
weights. Then the signs of the edge weights form a so-called ``Kasteleyn orientation''on graph, the
weighting is called ``Kasteleyn weighting'' \cite{kenyon2001dominos,kenyon2009lectures}.

For a bipartite graph ${\cal G}$, color the vertices black and white in such a way that all the
vertices adjacent to the black one are white. Denote by $W$,$B$ the sets of white and black
vertices and by $w,b$ the elements of these sets. 

The weights can be encoded as the ``Kasteleyn matrix'' -- weighted, signed adjacency matrix ${\cal K}$ with
the matrix elements ${\cal K}(w,b)$ equal to the weight of the edge $w\to b$: ${\cal K}(w,b)=\omega(w\to b)$.

Then the partition function is equal to the absolute value of the determinant of the Kasteleyn
matrix\cite{P.W-1961,doi:10.1080/14786436108243366}: 
\begin{equation}
  \label{eq:15}
  Z=\sum_{\mathrm{conf}}\prod_{e\in \mathrm{conf}}w(e)=|\det {\cal K}|
\end{equation}

Kasteleyn matrix defines a discrete Dirac operator $D$, in vertex $v$ we have:
\begin{equation}
  \label{eq:16}
  (Df)(v)=\sum_{u}{\cal K}(v,u) f(u)
\end{equation}

Kenyon \cite{kenyon2002laplacian,kenyon2000asymptotic} and others \cite{sridhar2015asymptotic}
considered asymptotics of the determinants of the discrete Dirac and Laplace operators, the problem
that, as can be seen from the above, is very close to the scaling of the free energy. But the finite
size corrections to the free energy scaling were not computed. 
  
We consider coverings of the hexagonal domain on the hexagonal lattice consisting of the subsets of
lattice edges such that every vertex is the endpoint of exactly one edge.

We can draw a rhombus on a dual lattice around each edge in the configuration. The picture of
``cubes in the corner'' presented in the Fig.~\ref{dhf} is obtained. Let us write on the top of each
uppermost cube the height of its column of cubes. Looking at this picture from the top, we obtain a
height function defined on the rectangular domain of the square lattice. 

\begin{figure}[htbp]
\center{\scalebox{0.4}{\includegraphics{loz.eps}}}
\caption{\label{dhf}A configuration of dimers on the hexagonal lattice and a corresponding picture
  of ``cubes in the corner''.}
\end{figure}

Let us define the sizes $M$, $N$, and $K$ of the sides of the hexagon.
The above description can be formalized by
%set of configurations of the model can be defined all possible ways to write
setting the non-negative numbers up to $K$ in the boxes of the rectangular $M\times N$ table so that a value in
each box is not greater than values in the adjacent upper and left boxes
\begin{equation}
  \label{eq:1}
  h_{ij}\leq h_{i-1,j},\quad h_{ij}\leq h_{i,j-1}.
\end{equation}

The weight of a particular configuration is given by the exponent of the volume of all cubes or by a
sum of the height function values:
\begin{equation*}
  \label{eq:10}
  E[conf]=\sum_{i,j} h_{ij}=\mathrm{Volume}
\end{equation*}
We set Boltzman constant equal to 1, then the partition function is 
\begin{equation*}
  \label{eq:14}
  Z=\sum_{conf} e^{-\frac{E[conf]}{T}}=\sum_{conf}q^{\mathrm{Volume}[conf]}, 
\end{equation*}
where $q=\exp\left(-1/T\right)$ .

For this particular case, the partition function is given by the classical Macmahon combinatorial
formula~\cite{vuletic2009generalization}
\begin{equation}
  \label{eq:12}
   Z[M,N,K,q]=\prod_{i=1}^{M}\prod_{j=1}^{N}\prod_{k=1}^{K}\frac{1-q^{i+j+k-1}}{1-q^{i+j+k-2}}
\end{equation}


MacMahon formula is obtained for the following definition of the Kasteleyn matrix. Embed the
hexagonal lattice in the complex plane $\mathbb{C}$ in such a way that some edges are parallel to
the real line. Then take
\begin{eqnarray}
  \label{eq:18}
  {\cal K}(w,b)=q^{\Re w+\Im w} \quad \mathrm{if}\quad \Im w=\Im b\\
  {\cal K}(w,b)=1 \quad \mathrm{if}\quad \Im w\neq\Im b
\end{eqnarray}


%% The scaling limit is achieved on the infinite lattice as $T\to\infty$. To study the scaling behavior
%% using Monte-Carlo simulations on the finite lattices one can consider temperatures proportional to
%% the lattice size and, then, extrapolate them to infinity.
%%   
%  \frametitle{Kasteleyn matrix and Dirac operator}
%%  Consider infinite-dimensional graded algebra, e.g. $\gf[[t]]$ or $\hat\gf$
  
%  \frametitle{Free energy scaling}
The   free energy per site is defined as$^{*}$
\footnote{For convenience we omit the factor $\frac{1}{T}$ in the usual definition of the free energy.}
\begin{equation*}
  \label{eq:17}
  % \frac{f}{T}
  f=-\frac{1}{V}\ln Z(M,N,K,q)
\end{equation*}
Here $V$ is the number of vertices, it is twice the number of dimers and twice the number of  cube faces:
\begin{equation*}
  \label{eq:19}
  V=2(MN+NK+MK)
\end{equation*}

%!!!! Kenyon uses $V=\#Dimers$.
  
We are interested in the scaling limit, combined with the thermodynamic limit, when
$ T\to \infty, \quad M,N,K\to \infty$, such that ratios
$\frac{M}{T}=a,\quad \frac{N}{T}=b, \quad \frac{K}{T}=c$ remain fixed. In what follows we use
$\varepsilon=\frac{1}{T}$, which can be seen as the scale of the model, e.g. mesh size.

In the next section we compute the asymptotic expansion of the free energy $f$ in $\varepsilon$ and
derive $\varepsilon$-independent closed expressions for the first several coefficients in this
expansion.
  
\section{The computation of the free energy asymptotic expansion}
\label{sec:free-energy-scaling}
First we substitute MacMahon formula \eqref{eq:12} into the free energy definition \eqref{eq:17} and
obtain
\begin{equation}
  \label{eq:20}
    %-\frac{f}{T}
    f=-\frac{1}{V}\ln Z =- \sum_{i=1}^{M} \sum_{j=1}^{N} \sum_{k=1}^{K} \frac{1}{V}
  \ln\left(\frac{1-e^{-\varepsilon (i+j+k)} e^{\varepsilon}}{1-e^{-\varepsilon (i+j+k)} e^{2\varepsilon}}\right)
\end{equation}
Expanding the exponents $e^{\varepsilon}, e^{2\varepsilon}$ and logarithms in powers of $\varepsilon$ we obtain

%%series and split the
%%logarithm of product into the sum of logarithms. Thus we get:
%%\begin{multline}
%%  \label{eq:2}
%%   -\frac{f}{T}= \sum_{i=1}^{M} \sum_{j=1}^{N} \sum_{k=1}^{K} \frac{1}{V}\left\{
%%    \ln\left(1-\frac{1}{e^{\varepsilon(i+j+k)}-1}\sum_{l=1}^{\infty} \frac{\varepsilon^{l}}{l!}\right)-\right.\\
%%  -\left.\ln\left(1-\frac{1}{e^{\varepsilon(i+j+k)}-1}\sum_{l=1}^{\infty} \frac{2^{l}\varepsilon^{l}}{l!}\right)\right\}
%%\end{multline}
%%Then expand the logarithms up to $\varepsilon^{3}$:
\begin{multline}
  \label{eq:4}
  %\frac{f}{T}
  f=- \sum_{i=1}^{M} \sum_{j=1}^{N} \sum_{k=1}^{K} \frac{1}{V}\left\{
    g(i\varepsilon,j\varepsilon,k\varepsilon)+
    \frac{3}{2}\varepsilon\left[g(i\varepsilon,j\varepsilon,k\varepsilon)+g(i\varepsilon,j\varepsilon,k\varepsilon)^{2}\right]\right.\\
  \left.+\frac{7}{6}\varepsilon^{2}\left[g(i\varepsilon,j\varepsilon,k\varepsilon)+3 g(i\varepsilon,j\varepsilon,k\varepsilon)^{2}+
      2 g(i\varepsilon,j\varepsilon,k\varepsilon)^{3}\right]\right\}  + \mathcal{O}(\varepsilon^{3})
\end{multline}

Here and below we will use the notation:
%Denote by $g$ the expression that will appear multiple times:
\begin{equation}
  \label{eq:5}
  g(x,y,z)=\frac{1}{e^{x+y+z}-1}
\end{equation}

This expression looks like a combination of Riemann sums for some integrals. Note that these sums
are finite, but some of the corresponding integrals are divergent.

%%But we need to obtain
%%corrections in $\varepsilon$ to the integrals as $\varepsilon\to 0$.
%%
%%We can use the following approximation formula for integral of some function $\tilde{f}$ to take
%%care of the corrections:

Let $G$ be an analytic function in the cube $[0,a]\times[0,b]\times[0,c]$, then
\begin{multline}
  \label{eq:21}
  \int_{0}^{a} \int_{0}^{b}\int_{0}^{c}G(x,y,z) \dx\; \dy\;
  \dz\approx\sum_{i=1}^{M}\sum_{j=1}^{N}\sum_{k=1}^{K}\left\{\varepsilon^{3}G\left(i\varepsilon,j\varepsilon,k\varepsilon\right)-\right.\\
  \left.-\left[\frac{\varepsilon^{4}}{2}\left((\partial_{x}+\partial_{y}+\partial_{z})G\right)(i\varepsilon,j\varepsilon,k\varepsilon)\right]+\right.\\
  \left.\varepsilon^{5}\left(\frac{1}{6}(\partial_{x}^{2}+\partial_{y}^{2}+\partial_{z}^{2})+\frac{1}{4}(\partial_{x}\partial_{y}+\partial_{x}\partial_{z}+\partial_{y}\partial_{z})\right)G\left(i\varepsilon,j\varepsilon,k\varepsilon\right)
  \right\}+\mathcal{O}(\varepsilon^{6})
\end{multline}
This formula can be easily derived by dividing the volume of integration into cubes with side
$\varepsilon$ and substituting Taylor series for $G$ into integrals.

We can use this formula to approximate sums by integrals. Similar formula can be written for a
function of two variables and double sums. We will not need higher order terms, but will use the
approximation:
\begin{equation}
\label{eq:8}
  \sum_{j=1}^{N}\sum_{k=1}^{K}\varepsilon^{2}G\left(i\varepsilon,j\varepsilon,k\varepsilon\right)\approx \int_{0}^{b}\int_{0}^{c}G(i\varepsilon,y,z) \dy\; \dz+\mathcal{O}(\varepsilon^{3}))
\end{equation}

%Now we can apply the integral representation.

Note that
\begin{equation}
  \label{eq:6}
  \begin{array}{c}
    \partial_{x}g(x,y,z)=-g(x,y,z)-g(x,y,z)^{2}\\
    \partial_{x}^{2}g(x,y,z)=g(x,y,z)+3g(x,y,z)^{2}+2g(x,y,z)^{3}
  \end{array}
\end{equation}

Taking this into account and using the relation \eqref{eq:21} to express the triple sum
$ \sum_{i=1}^{M} \sum_{j=1}^{N} \sum_{k=1}^{K} \frac{1}{V}    g(i\varepsilon,j\varepsilon,k\varepsilon)$
in \eqref{eq:4} as the integral plus corrections, we obtain
\begin{multline}
\label{eq:23}
 f=-\frac{1}{2(ab+bc+ca)}\left\{\int_{0}^{a} \int_{0}^{b}\int_{0}^{c}\frac{\dx \dy \dz}{e^{x+y+z}-1}+\right.\\
  \left.+\sum_{i=1}^{M}\sum_{j=1}^{N}\sum_{k=1}^{K}\frac{\varepsilon^{5}}{12}\left[g(i\varepsilon,j\varepsilon,k\varepsilon)+3
      g(i\varepsilon,j\varepsilon,k\varepsilon)^{2}+2
      g(i\varepsilon,j\varepsilon,k\varepsilon)^{3}\right]\right\}
\end{multline}
Due to the equations \eqref{eq:6}, the corrections are of the same form as the higher order terms in
$\varepsilon$. The linear in $\varepsilon$ term in \eqref{eq:4} is cancelled by three partial
derivatives in \eqref{eq:21}. The coefficient of the quadratic term is changed by the addition of
second derivatives from \eqref{eq:21} to $\frac{1}{12}$.


There is no singularity in the sum, but
$\int_{0}^{a} \int_{0}^{b}\int_{0}^{c}\frac{\dx \dy \dz}{(\exp(x+y+z)-1)^{3}}$ diverges
logarithmically. So we can not just apply the integral representation~\eqref{eq:21} directly.

We apply the formula \eqref{eq:8} to the internal sums in the triple sum, since there is no
singularity in $\frac{1}{\left((\exp(i\varepsilon+j\varepsilon+k\varepsilon)-1\right)^{3}}$ for
$i\neq 0$:

\begin{equation}
  \label{eq:9}
\sum_{i=1}^{M}\left(\sum_{j=1}^{N}\sum_{k=1}^{K}\frac{\varepsilon^{5}}{12}
      \frac{e^{2(i\varepsilon+j\varepsilon+k\varepsilon)}+e^{i\varepsilon+j\varepsilon+k\varepsilon}}{\left(e^{i\varepsilon+j\varepsilon+k\varepsilon}-1\right)^{3}}  \right)\approx
    \frac{\varepsilon^{3}}{12}\sum_{i=1}^{M} \int_{0}^{b}\int_{0}^{c}
      \left(\frac{e^{2(i\varepsilon+y+z)}+e^{i\varepsilon+y+z}}{\left(e^{i\varepsilon+y+z}-1\right)^{3}}\right) \dy \dz+\mathcal{O}(\varepsilon^{3})
\end{equation}

The double integral can be taken explicitly, so we obtain
\begin{equation}
  \label{eq:11}
  \frac{\varepsilon^{3}}{12}\sum_{i=1}^{M}\left(     \frac{1}{e^{b+c+i\varepsilon}-1}+
  \frac{1}{1-e^{b+i\varepsilon}}+\frac{1}{1-e^{c+i\varepsilon}}+\frac{1}{e^{i\varepsilon}-1}\right)
\end{equation}

The function $\frac{1}{e^{x}-1}$ has a divergence, so we can not apply integral approximation directly. We first need to 
subtract and add $\frac{1}{i\varepsilon}$. The sum over $i$ of this term can be approximated using Euler formula:
\begin{equation}
    \label{eq:24}
    \sum_{i=1}^{M}\frac{1}{i}=\frac{1}{M}\sum_{i=1}^{M}\frac{1}{i/M}=\int_{\frac{1}{M}}^{1}\frac{\dx}{x}+\gamma+O(\varepsilon)=-\ln\varepsilon+\ln a+\gamma+O(\varepsilon)
  \end{equation}
Thus we get the integral
\begin{multline}
  \label{eq:13}
    \frac{\varepsilon^{3}}{12}\left[\sum_{i=1}^{M}\left(     \frac{1}{e^{b+c+i\varepsilon}-1}+
      \frac{1}{1-e^{b+i\varepsilon}}+\frac{1}{1-e^{c+i\varepsilon}}+\frac{1}{e^{i\varepsilon}-1}-\frac{1}{i\varepsilon}\right)-\ln\varepsilon+\ln a+\gamma+\mathcal{O}(\varepsilon^{3})\right]\approx\\
   \frac{\varepsilon^{2}}{12}\left[\int_{0}^{a}\left(     \frac{1}{e^{b+c+x}-1}+
      \frac{1}{1-e^{b+x}}+\frac{1}{1-e^{c+x}}+\frac{1}{e^{x}-1}-\frac{1}{x}\right)\dx-\ln\varepsilon+\ln a+\gamma\right]+\mathcal{O}(\varepsilon^{3})    
\end{multline}
The integral is now taken explicitly, it is equal to
\begin{multline}
  \label{eq:22}
     \frac{\varepsilon^{2}}{12}\left[\int_{0}^{a}\left(     \frac{1}{e^{b+c+x}-1}+
         \frac{1}{1-e^{b+x}}+\frac{1}{1-e^{c+x}}+\frac{1}{e^{x}-1}-\frac{1}{x}\right)\dx-\ln\varepsilon+\ln a+\gamma\right]=\\
     \frac{\varepsilon^{2}}{12}\left[\ln \left(\frac{(e^{a}-1)(e^{b}-1)(e^{c}-1)(e^{a+b+c}-1)}{a (e^{a+b}-1)(e^{b+c}-1)(e^{a+c}-1)}\right)-\ln\varepsilon+\ln a+\gamma\right]
\end{multline}

  
%%Substituting this approximation and using integral representation for the quadratic
%%  term in $\varepsilon$, we obtain for the free energy:
%%  \begin{multline*}
%%    \frac{f}{T}=-\frac{1}{ab+bc+ca}\left\{\int_{0}^{a} \int_{0}^{b}\int_{0}^{c}\frac{\dx \dy \dz}{e^{x+y+z}-1}+\right.\\
%%    \left.+\frac{\varepsilon^{2}}{12}\int_{0}^{a} \int_{0}^{b}\int_{0}^{c}
%%      \left(\frac{e^{2(x+y+z)}+e^{x+y+z}}{\left(e^{x+y+z}-1\right)^{3}}-\frac{1}{bc}\frac{1}{x}\right)\dx \dy \dz+\right.\\
%%    \left.+\frac{\varepsilon^{2}}{12}(\ln a-\gamma)+\frac{\varepsilon^{2}}{12}\ln \varepsilon\right\}    
%%  \end{multline*}
%%The integral can be taken explicitly, it is equal to
%%\begin{multline}
%%  \label{eq:7}
%%  \int_{0}^{a} \int_{0}^{b}\int_{0}^{c}
%%  \left(\frac{e^{2(x+y+z)}+e^{x+y+z}}{\left(e^{x+y+z}-1\right)^{3}}-\frac{1}{bc}\frac{1}{x}\right)\dx~\dy~\dz=\\
%%  =\ln \left(\frac{(e^{a}-1)(e^{b}-1)(e^{c}-1)(e^{a+b+c}-1)}{a (e^{a+b}-1)(e^{b+c}-1)(e^{a+c}-1)}\right)=\\
%%  =\ln \left(\frac{(e^{a}-1)(e^{b}-1)(e^{c}-1)(e^{a+b+c}-1)}{(e^{a+b}-1)(e^{b+c}-1)(e^{a+c}-1)}\right) -\ln a.
%%\end{multline}

The term $\ln a$ is cancelled by the  and symmetry between $a,b,c$ is restored.
Substituting this approximation into the expression \eqref{eq:23} we obtain 
 the final expression:
\begin{multline}
  \label{eq:3}
 f=-\frac{1}{2(ab+bc+ca)}\left\{\int_{0}^{a} \int_{0}^{b}\int_{0}^{c}\frac{\dx \dy \dz}{e^{x+y+z}-1}+\right.\\
  \left.+\frac{\varepsilon^{2}}{12}\left[\ln\left(\frac{(e^{a}-1)(e^{b}-1)(e^{c}-1)(e^{a+b+c}-1)}{(e^{a+b}-1)(e^{b+c}-1)(e^{a+c}-1)}\right)-
      \gamma\right]+\frac{\varepsilon^{2}}{12}\ln \varepsilon\right\} +\mathcal{O}(\varepsilon^{3})
\end{multline}

\section{Physical meaning of the expansion coefficients}
\label{sec:accur-expans-phys}

The expansion  (\ref{eq:3}) can be written as
\begin{equation}
  \label{eq:26}
  f=f_{0}+f_{1}\varepsilon + f_{2}\varepsilon^{2} +f_{3}\varepsilon^{2}\ln\varepsilon,
\end{equation}
where the coefficients are
\begin{equation}
  \label{eq:27}
  f_{0}=-\frac{1}{2(ab+bc+ca)}\int_{0}^{a} \int_{0}^{b}\int_{0}^{c}\frac{\dx \dy \dz}{e^{x+y+z}-1},
\end{equation}
\begin{equation}
  \label{eq:28}
  f_{1}=0,
\end{equation}
\begin{equation}
  \label{eq:29}
  f_{2}=-\frac{1}{2(ab+bc+ca)}\frac{1}{12}\left[\ln\left(\frac{(e^{a}-1)(e^{b}-1)(e^{c}-1)(e^{a+b+c}-1)}{(e^{a+b}-1)(e^{b+c}-1)(e^{a+c}-1)}\right)-
    \gamma\right],
\end{equation}
and
\begin{equation}
  \label{eq:30}
  f_{3}=-\frac{1}{2(ab+bc+ca)}\frac{1}{12}
\end{equation}
In the scaling limit $\varepsilon\to 0$ so called ``limit shape phenomenon''~\cite{kenyon2006dimers}
appears in the dimer model. The areas around the corners of the domain are ``frozen'' with height
function values being fixed. An analytical ``Arctic curve'' delimits frozen regions from the region
where the behavior is described by the effective free field theory
\cite{kenyon2009lectures,kenyon2008height,kenyon2006dimers}.

The behavior (\ref{eq:26}) of the logarithm of the partition function is generic in two-dimensional
models \cite{cardy1988finite}.

First two terms $f_{0}$ and $f_{1}$ are interpreted as a bulk and
boundary free energies in the corresponding field theory. Since we have $f_{1}=0$, we can conclude
that boundary tension is zero.
The value of the first term $f_{0}$ can be rewritten using polylogarithm functions as
\begin{multline}
  \label{eq:34}
  f_{0}=\frac{1}{2(ab+bc+ca)}\left[abc + \Li_{3}(e^{a})+\Li_{3}(e^{b})+\Li_{3}(e^{c})-
    \Li_{3}(e^{a+b})\right.\\
  \left.-\Li_{3}(e^{b+c})-    \Li_{3}(e^{a+c})+    \Li_{3}(e^{a+b+c})-\zeta(3)\right]
\end{multline}


The third term $f_{2}$ depends only on the shape of the domain through $a,b,c$ with a universal
contribution that is equal to the Euler constant $\gamma$. We conjecture that Euler constant will
appear even for a different choice of Kasteleyn weighting. In the future work we will consider
coordinate dependent values of $q$, as was done in the paper \cite{okounkov2007random}. 

The term proportional to the logarithm of the scale $\varepsilon$ is also universal
\cite{cardy1988finite} and should appear in all two-dimensional theories with boundary. In the paper
\cite{cardy1988finite} it was argued that on a manifold of a characteristic length $L$ with a smooth
boundary such a term must have the following form:
\begin{equation}
  \label{eq:31}
  \delta F = -\frac{1}{6}{\bf c} \chi \ln L,
\end{equation}
where ${\bf c}$ is central charge of the effective field theory and $\chi$ is the Euler characteristic of
the manifold
\begin{equation}
  \label{eq:32}
  \chi=2-2 h-b,
\end{equation}
where $h$ is the number of handles and $b$ is the number of boundaries. 

Since the non-frozen domain is delimited by a smooth boundary, we have $\chi=1$ and
\begin{equation}
  \label{eq:33}
  f_{3}=\frac{1}{2(ab+bc+ca)} \frac{{\bf c}}{6},
\end{equation}
which suggests the value of central charge ${\bf c}=\frac{1}{2}$. This value is in agreement with the
identification of the dimer model with free fermions in the paper \cite{dijkgraaf2009dimer}. 

%% In the table we present the results of computations of free energy for the case $a=b=c=1$ and
%% various values of $\varepsilon$ using an exact formula and expansion (\ref{eq:3}). In the third column
%% difference is presented. We see that the difference decays faster than $\varepsilon^{3}$. The expansion
%% (\ref{eq:3}) has good precision for comparison with Monte-Carlo simulation results such as presented
%% in
%% \cite{belov2018finite}. 
%% 
%% \begin{table}[htpb]
%% \begin{center}
%% \begin{tabular}{c|ccc}
%%   $\varepsilon$&Exact value of $f$&$f$ from expansion (\ref{eq:3})&Difference\\
%%   \hline
%%   0.2 & 0.06734815 & 0.06559065 & 0.00175751\\
%%   0.1 & 0.06689788 & 0.06681834 & 0.00007954\\
%%   0.05 & 0.06721738 & 0.06719747 & 0.00001992\\
%%   0.02 & 0.06732867 & 0.06732549 & $3.18791\cdot 10^{-6}$\\
%%   0.01 & 0.06734815 & 0.06734736 & $7.97026 \cdot 10^{-7}$
%%   0.005 & 0.06735375 & 0.06735355 & $1.99259 \cdot 10^{-7}$
%% \end{tabular}
%% \caption{Values of the free energy for $a=b=c=1$ computed by the exact formula (\ref{eq:20}) and an
%%   approximate formula (\ref{eq:3}). The difference decays faster than $\varepsilon^{3}$.
%% }
%% \label{tttt1}
%% \end{center}
%% \end{table}
%% 
%% 

%%   \begin{figure}[h!tb]
%% 
%%     \includegraphics{exact-vs-expansion.pdf}
%%     \caption{Plot of the expression \eqref{eq:3} for $\varepsilon=\frac{1}{15}\dots\frac{1}{2}$, $a=3,b=2,c=1$
%%       exact values. Note that difference between exact and approximate values decreases with
%%       $\varepsilon$.}
%%   \end{figure}
%% 
%%\section{Scaling in the non-uniform model}
%%\label{sec:scaling-non-uniform}
%%
%%Consider a skew 3D partition with the partition function
%%\begin{equation}
%%  \label{eq:25}
%%  Z({q_{t}},\lambda)=\sum_{\lambda(t)}\prod_{t\in\mathbb{Z}}q_{t}^{|\lambda(t)|}
%%\end{equation}
%%
\section*{Conclusion and outlook}
\label{sec:conclusion}

In the present paper we computed the asymptotic expansion of the free energy in the dimer model on a
hexagonal domain of the hexagonal lattice. We've discussed the physical meaning of the expansion
coefficients and argued that our results support the identification of the scaling behavior of the
dimer model with the free-fermion field theory.

In further work we will show the connection of the expansion coefficients with the spectral
properties of Dirac operator on the non-frozen domain and study the universality of the presented
expressions by considering the model with the non-uniform Kasteleyn weighting with $q$ depending on
the coordinate. 

\section*{Acknowledgments}
\label{sec:acknowledgements}
I am grateful to professor Nikolai Reshetikhin for his guidance in this work. I thank Pavel Belov
for useful discussions and general support.

I thank the organizers and participants of the conference MQFT-2018 for the opportunity
to present our results and useful discussions.

This research is supported by RFBR grant No. 18-01-00916.
%The calculations were carried out using the facilities of the SPbU Resource Center ``Computational Center of SPbU''.

\bibliographystyle{utphys}
\bibliography{listing,bibliography,dimers}{} 
\end{document}
