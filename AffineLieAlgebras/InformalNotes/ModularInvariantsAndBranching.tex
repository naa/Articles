\documentclass[a4paper,12pt]{article}
\usepackage[unicode,verbose]{hyperref}
\usepackage{amsmath,amssymb,amsthm} \usepackage{pb-diagram}
\usepackage{ucs}
\usepackage[utf8x]{inputenc}
\usepackage[russian]{babel}
\usepackage{cmap}
\usepackage{graphicx}
\pagestyle{plain}
\theoremstyle{definition} \newtheorem{Def}{Definition}
\begin{document}
\tableofcontents

\section{Модулярная инвариантность}
\label{sec:modular-invariance}

Если мы изучаем конформную теорию поля на плоскости или на сфере, то мы можем рассматривать голоморфный и антиголоморфный сектора независимо. Если говорить о WZW-моделях, то примарные поля принадлежат тензорному произведению неприводимых представлений аффинной алгебры.

Если мы говорим о применении конформной теории для описания поведения струн, то теория должна быть определена на римановых поверхностях большего рода ($h>0$), чтобы можно было описывать взаимодействия струн. Считается, что для этого необходимо (и, возможно, достаточно \cite{gaberdiel2000icf}) чтобы теория была определена на торе. 

В теории критического поведения конформная инвариантность имеет место только в критической точке, где голоморфный и антиголоморфный сектора расцеплены. Но вблизи критической точки эти сектора должны быть связаны, и так как мы предполагаем плавный переход к критической точке в пространстве параметров, то эта связь должна сохраняться и в критической точке. Физический спектр теории должен плавно меняться, когда мы покидаем критическую точку, и связь голоморфного и антиголоморфного сектора вдали от критической точки должна приводить к ограничениям на набор состояний в критической точке. Этого можно достичь через геометрию, то есть накладывая граничные условия на состояния \cite{difrancesco1997cft}. Здесь естественно рассматривать периодические граничные условия, которые эквивалентны рассмотрению теории на торе.

Если мы наложили периодические граничные условия с периодами $\omega_1, \omega_2,\; \tau=\omega_2/\omega_1$, то статсумма записывается в виде
\begin{equation}
  \label{eq:1}
  Z(\tau)=Tr \exp 2\pi i (\tau (L_0-c/24)-\bar{\tau} (\bar{L}_0-c/24))
\end{equation}
Или, если ввести $q=\exp 2\pi i \tau$
\begin{equation}
  \label{eq:2}
  Z(\tau)=Tr \left (q^{L_0-c/24}\bar{q}^{\bar{L}_0-c/24}\right)
\end{equation}
Двумерный тор представляет собой фактор пространство $\mathbb{R}^2\approx \mathbb{C}$ по отношениям эквивалентности $z\sim z+w_1$ and $z\sim z+w_2$, где $w_1$ и $w_2$ не параллельны. 

Разные параметризации тора связаны модулярными преобразованиями, таким образом возникает требование модулярной инвариантности статсуммы.

Комплексная структура такого тора конформно эквивалентна тору, для которого соотношения эквивалентности записываются в виде $z\sim z+1$ и $z\sim z+\tau$, где $\tau$ в верхней полуплоскости $\mathbb{C}$.

Легко видеть, что $\tau$, $T(\tau)=\tau+1$ и $S(\tau)=-\frac{1}{\tau}$ описывают конформно-эквивалентные торы. Отображения $T$ и $S$ порождают группу  $SL(2,\mathbb{Z})/\mathbb{Z}_2$, состоящую из матриц вида
\begin{equation}
  \label{eq:99}
  A=
  \begin{pmatrix}
    a & b\\
    c & d 
  \end{pmatrix}
  \quad\mbox{где}\; a,b,c,d\in\mathbb{Z},\quad ad-bc=1,
\end{equation}
и матрицы $A$ и $-A$ действуют одинаково на $\tau$
\begin{equation}
  \label{eq:100}
  \tau\to A\tau=\frac{a\tau+b}{c\tau+d}
\end{equation}
 $\tau$ называется модулярным параметром, а группа $SL(2,\mathbb{Z})/\mathbb{Z}_2$ --- модулярной группой.

Конформная теория поля задаётся примарными полями $\Phi_a$ с конформными размерностями $\Delta_a$:
\begin{equation}
  \label{eq:3}
  \begin{split}
    \Phi_{a}(z)\underset{z\to w(z)}{\longrightarrow} \left(\frac{dw}{dz}\right)^{\Delta_a}\Phi_{a}(w(z))\\
    L_n \Phi_a=0,\quad n>0\\
    L_0 \Phi_a=\Delta_a \Phi_a\\
  \end{split}
\end{equation}

Примарные поля живут в пространствах $\mathcal{H}_{(i,j)}$, которые представляют собой тензорные произведения неприводимого представления  $\mathcal{H}_j$ киральной алгебры и неприводимого представления $\bar{\mathcal{H}}_{\bar{j}}$ антикиральной алгебры. Тогда статсуммы на торе (\ref{eq:2}) может быть переписана в виде  
\begin{equation}
  \label{eq:4}
  \sum_{(j,\bar j)}\chi_j(q)\bar \chi_{\bar j}(\bar q)
\end{equation}
где $\chi_j$ --- характер пердставления $\mathcal{H}_j$,
\begin{equation}
  \label{eq:5}
  \chi_j(\tau)=Tr_{\mathcal{H}_j}(q^{L_0-\frac{c}{24}})\quad \mbox{где}\; q=e^{2\pi i \tau}
\end{equation}
Характеры переходят друг в друга при модулярных преобразованиях:
\begin{equation}
  \label{eq:107}
  \chi_j\left(-\frac{1}{\tau}\right)=\sum_k S_{jk}\chi_k(\tau)\quad \mbox{и}\quad \chi_j(\tau+1)=\sum_kT_{jk}\chi_k(\tau),
\end{equation}
где $S$ и $T$ --- постоянные матрицы. Это верно для большого класса конформных теорий поля \cite{gaberdiel2000icf}. 

Для WZW-моделей представления определяются старшими весами $\hat \lambda, \hat \xi$. Тогда
\begin{equation}
  \label{eq:6}
  \mathcal{H}=\bigoplus_{\hat \lambda,\hat \xi\in P^{(k)}_{+}}M_{\hat \lambda,\hat \xi} L_{\hat \lambda}\otimes L_{\hat \xi}
\end{equation}
Статсумма даётся выражением
\begin{equation}
  \label{eq:7}
  Z(\tau)=\sum_{\hat \lambda,\hat \xi\in P^{(k)}_{+}} \chi_{\hat \lambda}(\tau)M_{\hat \lambda\hat\xi}\bar \chi_{\hat \xi}(\bar \tau)
\end{equation}
Элементы так называемой матрицы масс $M_{\hat \lambda\hat\xi}$ можно рассматривать как кратности примарных полей с весами $\hat\lambda,\hat \xi$. У них есть следующие свойства: $M_{\hat \lambda\hat\xi}\in \mathbb{Z}_+$, модулярная инвариантность
\begin{equation}
  \label{eq:8}
  \begin{aligned}
    T^{\dagger}MT=S^{\dagger}MS=M,\\
    [M,S]=[M,T]=0,
  \end{aligned}
\end{equation}
и $M_{00}=1$ для единственности вакуума.

Простейший случай диагональной матрицы $M$ соответствует равным голоморфным и антиголоморфным конформным размерностям. Есть несколько способов построения недиагональных модулярных инвариантов из диагональных \cite{difrancesco1997cft}:
\begin{itemize}
\item Метод внешних автоморфизмов
\item Конформное вложение в большую теорию
\item Перестановки Галуа
\end{itemize}
Мы будем обсуждать только конформные вложения.

\section{Конформные вложения}
\label{sec:conformal-embeddings}
Conformal embeddings should preserve conformal invariance, so Sugawara central charge should be the same for enveloping and embedded theory. !!! Provide an explanation !!!

The states for the theory that corresponds to the algebra $\mathfrak{g}$
\begin{equation}
  \label{eq:109}
  J^{a_1}_{-n_1}J^{a_2}_{-n_2}\dots\left|\lambda\right>\quad n_1\geq n_2\geq\dots>0
\end{equation}
For sub-algebra $\mathfrak{p}\subset\mathfrak{g}$
\begin{equation}
  \label{eq:110}
  \tilde{J}^{a'_1}_{-n_1}\tilde{J}^{a'_2}_{-n_2}\dots\left|\mathcal{P}\lambda\right>
\end{equation}
Here $\tilde{J}^{a'_j}_{-n_j}$ are the generators of $\mathfrak{p}$ and $\mathcal{P}$ is the projection of $\mathfrak{g}$ to $\mathfrak{p}$. $\mathfrak{g}$-invariance of vacuum entails its $\mathfrak{p}$-invariance, but it is not the case for energy-momentum tensor. So energy-momentum tensor of bigger theory should consist only of generators of $\hat{\mathfrak{p}}$. Then $T_{\hat{\mathfrak{g}}}=T_{\hat{\mathfrak{p}}}\Rightarrow c(\hat{\mathfrak{g}})=c(\hat{\mathfrak{p}})$. This leads to equation
\begin{equation}
  \label{eq:111}
  \frac{k\;dim\mathfrak{g}}{k+g}=\frac{x_e k\; dim\mathfrak{p}}{x_ek+p}
\end{equation}
Here $x_e$ is the embedding index and $g$, $p$ are dual Coxeter numbers of corresponding algebras. !!! Prove and expand !!!

It can be shown that solutions of equation (\ref{eq:111}) exist only for level 1 $k=1$.

There exist only finite number of such embeddings.

{\bf Examples}
\begin{itemize}
\item $su(2)\subset su(3),\; x_e=4$
\item $\hat{su}(2)_{10}\subset\hat{sp}(4)_1$
\item $\hat{su}(2)_{28}\subset(\hat{G_2})_1$
\item $\hat{su}(2)_{16}\oplus\hat{su}(3)\subset (\hat{E_8})_1$
\end{itemize}

Branching $\hat{\lambda}\to \bigoplus_{\hat{\mu}}b_{\hat{\lambda}\hat{\mu}}\hat{\mu}$
\begin{equation}
  \label{eq:112}
  h_{\hat{\lambda}}+n=h_{\hat{\mu}},\quad \frac{(\lambda,\lambda+2\rho)}{2(1+g)}+n=\frac{(\mu,\mu+2\rho)}{2(x_e+p)}
\end{equation}
The simple way is to calculate the dimensions of all the representations of simple finite-dimensional Lie algebras $\mathfrak{g},\mathfrak{p}$ and find all the triples $(\lambda,\mu,n)$, satisfying (\ref{eq:112}). Then investigate the representation $L_{\hat{\lambda}}$ of level $n$ as the sum of irreducible representations of $\mathfrak{g}$ and write out all the branching rules of these representations in representations of subalgebra $\mathfrak{p}$. The number of times the representation with highest weight $\mu$ appears in this listing is the required branching coefficient $b_{\hat{\lambda},\hat{\mu}}$.

  Then non-diagonal modular invariants are obtained by substitution of character relations. For example, having calculated branching coefficients for $\hat{su}(2)_{28}\subset (\hat{G_2})_1$
  \begin{equation}
    \label{eq:113}
    \begin{aligned}
      & [1,0,0]\to [28,0]\oplus [18,10]\oplus [10,18]\oplus [0,28]\\
      & [0,0,1]\to [22,6]\oplus [16,12]\oplus [12,16]\oplus [6,22]\\
    \end{aligned}
  \end{equation}
we get following modular-invariant partition function
\begin{equation}
  \label{eq:114}
  Z=\left|\chi_{[28,0]}+\chi_{[18,10]}+\chi_{[10,18]}+\chi_{[0,28]}\right|^2+\left|\chi_{[22,6]}+\chi_{[16,12]}+\chi_{[12,16]}+\chi_{[6,22]}\right|^2
\end{equation}

\bibliography{CFTNotes}{}
\bibliographystyle{utphys}

\end{document}
