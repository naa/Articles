\documentclass[12pt]{article}
\usepackage{amsmath,amssymb,amsthm,amsfonts}
\usepackage{multicol}
\usepackage{color}
\usepackage{hyperref}
\usepackage{graphicx}
\usepackage{ esint }
\newtheorem{theorem}{Theorem}
\newtheorem{definition}{Definition}

\newcommand{\co}[1]{\stackrel{\circ }{#1}}
\newcommand{\gf}{\mathfrak{g}}
\newcommand{\nfp}{\mathfrak{n}^{+}}
\newcommand{\nfm}{\mathfrak{n}^{-}}
\newcommand{\af}{\mathfrak{a}}
\newcommand{\uf}{\mathfrak{u}}
\newcommand{\sfr}{\mathfrak{s}}
\newcommand{\aft}{\widetilde{\mathfrak{a}}}
\newcommand{\afb}{\mathfrak{a}_{\bot}}
\newcommand{\hf}{\mathfrak{h}}
\newcommand{\hfb}{\mathfrak{h}_{\bot}}
\newcommand{\pf}{\mathfrak{p}}

\newcommand{\gfh}{\hat{\mathfrak{g}}}
\newcommand{\afh}{\hat{\mathfrak{a}}}
\newcommand{\sfh}{\hat{\mathfrak{s}}}
\newcommand{\bff}{\mathfrak{b}}
\newcommand{\hfg}{\hf_{\gf}}

\begin{document}
\title{Notes on non-abelian generalization of Benjamin-Ono equation}
%\author{Anton Nazarov}%% $^{1,2}$}

\maketitle

\begin{abstract}
Benjamin-Ono equation describes waves in deep ocean. This integrable equation is actively studied.
Connection with conformal field theory and quantum Hall liquid was discovered in these studies. 
We introduce non-abelian generalization of Benjamin-Ono equation and demonstrate the connection with
Wess-Zumino-Novikov-Witten and coset models of conformal field theory.
\end{abstract}

\section{Introduction: Benjamin-Ono equation}
\label{sec:introduction}

There are several 1+1-dimensional partial differential equations describing waves. First consider
Riemann equation on function $u(x,t)$:
\begin{equation}
  \label{eq:1}
  u_{t}+ u u_{x} =0
\end{equation}

This equation can be linearized by setting
\begin{equation}
  \label{eq:2}
  u=\bar u+u_{1} (x,t)
\end{equation}
and assuming $u_{1}$ to be small. Then we get chiral wave equation
\begin{equation}
  \label{eq:3}
  u_{t}+\bar u u_{x}+\dots=0
\end{equation}
Solution of Cauchy problem is given by a retarded function
\begin{equation}
  \label{eq:4}
  u(t,x)=u_{0}(x-\bar u t),
\end{equation}
where
\begin{equation}
  \label{eq:5}
  u(x,t=0)=u_{0}(x)
\end{equation}

Solution to Riemann equation can be formally written as
\begin{equation}
  \label{eq:6}
  u(t,x)=u_{0}(x-u(t,x)\cdot t)
\end{equation}
For bulged initial data this solution is unstable, shock waves are formed, then solution is multivalued and hydrodynamic
instability occurs. Concave initial data on the other hand is getting smoother over time. 

Corrections to Riemann equation (\ref{eq:1}) dramatically change behavior of solutions. For example
dispersion term kills infinities. Equation with dispersion term $u_{xx}$
\begin{equation}
  \label{eq:7}
  u_{t}+u u_{x}+\nu u_{xx}=0
\end{equation}
is called Burgers equation. Here $\nu$ is liquid viscosity. 
The Hamiltonian
\begin{equation}
  \label{eq:8}
  H=\frac{1}{2} \int u^{2} dx
\end{equation}
is not conserved, so the dynamic is non-Hamiltonian. In the limit $\nu\to 0$ bulged initial data
tends over time to function with a step. It is so called ``weak solution'' of Riemann equation. 

One way to obtain equation with conserved energy is to add third derivative:
\begin{equation}
  \label{eq:9}
  u_{t}+u u_{x}+u_{xxx}=0.
\end{equation}
This is famous KdV (Korteweg–de Vries) equation. 

Now the Hamiltonian is
\begin{equation}
  \label{eq:10}
  H=\frac{1}{3}\int u^{2}+u u_{x} dx
\end{equation}
Soliton wave is given by the formula
\begin{equation}
  \label{eq:11}
  u_{s}(x-ct)=\frac{1}{\cosh^{2}(x-ct)}
\end{equation}
These waves exist in shallow water when depth is less then wave length. 

Appearance of solitons is a distinct property of integrable equations, which are always Hamiltonian
since they have infinitely many conserved quantities
\begin{equation}
  \label{eq:12}
  \left\{ H_{n},H\right\} =0\quad \Leftrightarrow \quad\partial_{t} H_{n}=0
\end{equation}

One of these conserved quantities for KdV equation is soliton mass (Check this expression)
\begin{equation}
  \label{eq:13}
  I_{1}=\int \frac{1}{\cosh^{2}(x-ct)} dx
\end{equation}
It can take arbitrary values. 

Solitons are separable, they are very stable and generate large time asymptotics in hydrodynamics. 

Equations with even number of derivatives are non-Hamiltonian but we can amend it by making these
terms non-local. E.g.
\begin{equation}
  \label{eq:14}
  u_{t}+u u_{x}+u^{H}_{xx}=0,
\end{equation}
where $u^{H}$ is Hilbert transform
\begin{equation}
  \label{eq:15}
  u^{H}(x)=\frac{1}{\pi} \fint \frac{u(x')}{x-x'}dx' 
\end{equation}
This is Benjamin-Ono equation which describes waves on boundary between two liquids of different
densities in a deep ocean. Solitons for this equation have the form
\begin{equation}
  \label{eq:16}
  u(x,t)=f(x-ct)=\frac{1}{\pi} \frac{A}{(x-A t)^{2}+\frac{1}{A^{2}}}
\end{equation}
Here $A$ is wave velocity and $\frac{1}{A^{2}}$ is width of the wave. Very peculiarly in this
equation mass is quantized.
\begin{equation}
  \label{eq:17}
  I_{1}=\int u(x,t) dx =1\quad \partial_{t} I_{1}=0
\end{equation}
Many-soliton solution have integer mass. For equation with some coefficient $\beta$
\begin{equation}
  \label{eq:18}
   u_{t}+u u_{x}+\beta u^{H}_{xx}=0
\end{equation}
mass is $N\beta$, where $N\in \mathbb{Z}$. So we have quantization on classical level. 

If the mass of initial data is not quantized, it decays into several solitons of integer mass moving
in one direction and non-integer mass noise which moves in the opposite direction. 


The equation can be quantized. Mass is then changed to $\beta-\hbar$, where $\beta=\beta' \hbar$ and
$\beta-\hbar=(\beta'-1)\hbar$. (?)

The Hamiltonian is
\begin{equation}
  \label{eq:19}
  H=\frac{1}{2}\int \left(u^{2}+\frac{1}{3} u^{2} u^{H}_{x}\right) dx
\end{equation}
Now to quantize the equation we assume that function $u(x,t)$ is defined on the circle or radius
$\frac{L}{2\pi}$. Then we can expand $u$ in modes
\begin{equation}
  \label{eq:20}
  u(x)=\sum_{k} e^{i k x}a_{k}
\end{equation}
The coefficients $a_{k}$ satisfy commutation relation
\begin{equation}
  \label{eq:21}
  \{a_{m},a_{n}\}=n\delta_{nm}
\end{equation}

Equation can be written in Hamiltonian form
\begin{equation}
  \label{eq:22}
  u_{t}=\{H,u\},
\end{equation}
and
\begin{equation}
  \label{eq:23}
  H=\sum_{k>0} a_{-k} L_{k},
\end{equation}
where
\begin{equation}
  \label{eq:24}
  L_{k}=\sum_{n>0} a_{n+k} a_{n}+\beta\sum_{n>0} n a_{n+k}
\end{equation}
The functions $L_{k}$ satisfy commutation relations w.r.t. Poisson bracket of Virasoro algebra
\begin{equation}
  \label{eq:25}
  \{L_{k},L_{l}\}=(k-l)L_{k+l}+\frac{c}{12} k(k^{2}-1) \delta_{k+l,0},
\end{equation}
where
\begin{equation}
  \label{eq:26}
  c=-6\beta^{2}
\end{equation}
So this is Poisson-Lie realization of Virasoro algebra (classical Virasoro algebra). 

Quantization is straightforward:
\begin{equation}
  \label{eq:27}
  [a_{m}, a_{n}]=\hbar n \delta_{nm}
\end{equation}
\begin{equation}
  \label{eq:28}
  H=\sum_{k>0}:a_{-k}L_{k}:
\end{equation}
An we obtain commutation relations of Virasoro algebra
\begin{equation}
  \label{eq:29}
   [L_{k},L_{l}]=(k-l)L_{k+l}+\frac{c}{12} k(k^{2}-1) \delta_{k+l,0},
\end{equation}
but now central charge is different
\begin{equation}
  \label{eq:30}
  c=1-6\left( \sqrt{\beta} -\frac{1}{\sqrt{\beta}}\right)^{2}
\end{equation}

Benjamin-Ono equation is also connected to random matrices. Steady state equation $u_{t}=0$ is
equivalent to loop equation for Hermitian matrices. 

Quantization of solitons is somehow connected with conformal field theory. In fractional quantum
Hall waves on the boundary are described by Benjamin-Ono equation \cite{?Wiegmann}. 

In conformal field theory Virasoro symmetry can be extended to Lie groups. For example, if
$u=\partial_{x} \varphi$ and $e^{i\varphi}$ is position of a particle on a ring, then $u$ is
particle displacement. Since
\begin{equation}
  \label{eq:31}
  e^{i\varphi_{1}}e^{i\varphi_{2}}=e^{i(\varphi_{1}+\varphi_{2})}
\end{equation}
we deal with $U(1)$ group. 

The Benjamin-Ono equation for $\varphi$ is
\begin{equation}
  \label{eq:35}
  \partial_{t} \varphi +\frac{1}{2} (\partial_{x}\varphi)^{2}+\partial_{x}^{2}\varphi^{H}=0
\end{equation}

Now consider $\varphi\in su(2)$. Then $g=e^{i\varphi}$ is Lie group element and
\begin{equation}
  \label{eq:32}
  \hat u=-ig^{-1} \partial_{x} g\in su(2)
\end{equation}
is connection in line bundle. It is $su_{k}(2)$-current $\hat J$.
\begin{equation}
  \label{eq:33}
  J(x)=\sum \hat J_{n} e^{inx}
\end{equation}
Elements $J_{n}$ generate Kac-Moody algebra
\begin{equation}
  \label{eq:34}
  [J_{n},J_{m}]=(n-m)J_{n+m} + k n \delta_{n+m,0}
\end{equation}

The problem is to construct non-abelian extension of Benjamin-Ono equation. 

If $g\leftrightarrow e^{i\varphi}$, then $\partial_{x}\varphi\leftrightarrow g^{-1}\partial_{x}g=J$.
Using (\ref{eq:35}) we can conjecture form of the generalized equation
\begin{equation}
  \label{eq:36}
  \partial_{t}J+\partial_{x}[J\cdot J]+\dots=0
\end{equation}
Such equation is supposed to have solutions with quantized mass. 

Note also that Benjamin-Ono equation is a certain limit of intermediate long-wave equation (ILW).
The term $u^{H}=\fint \frac{u}{x-x'}dx'$ is the limit $L\to\infty$ of
\begin{equation}
  \label{eq:37}
  u^{H}=\fint u \coth \frac{x-x'}{L} dx'
\end{equation}
for $2L$-periodic ocean. Hilbert transform connects real and imaginary parts of complex function
$(\Re f)^{H}\to \Im f$. For complex functions the equation should be like
\begin{equation}
  \label{eq:38}
  f_{t}=f f_{z}+if_{zz}, \quad ix=z
\end{equation}
For $\varphi=\sum_{k}\frac{a_{k}}{k} e^{ikx}$ we have stress-energy tensor
\begin{equation}
  \label{eq:39}
  T=\sum_{k}L_{k}z^{-k-2}=T_{xx}+T_{yy}-2i T_{xy}=(\partial \varphi)^{2} +i\alpha\partial^{2}\varphi,
\end{equation}
and
\begin{equation}
  \label{eq:40}
  [a_{k},a_{n}]=\delta_{n+k,0},
\end{equation}
\begin{equation}
  \label{eq:41}
  L_{n}=\sum_{k}:a_{k}a_{n-k}:
\end{equation}
\begin{equation}
  \label{eq:42}
  H=\sum_{k}:a_{-k}L_{k}:
\end{equation}

Benjamin-Ono equation is equivalent to
\begin{equation}
  \label{eq:43}
  \partial_{t}u=[H,u]
\end{equation}
But we have boundary condition $a^{\dagger}_{k}=a_{-k}$. For $su(2)$-generalization we have
$a_{k}\to J_{k}, J=g \partial g$, but on the boundary we have coset $su(2)/u(1)\approx S^{2}$. And
$J=J_{x}\sigma_{x}+J_{y}\sigma_{y}+J_{z}\sigma_{z}$ should be changed to $\sigma_{z}\cdot J$. So if
we want to get Benjamin-Ono equation for spin waves we need to rewrite $H=\sum_{k}:J_{-k} L_{k}:$ on
the boundary.  
\end{document}
