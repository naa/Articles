\documentclass[pdftex]{beamer} 
% \usepackage[pdftex]{graphicx}
\usepackage{amsmath,amssymb,amsthm} 
\usepackage{pb-diagram}
% \usepackage[authoryear]{natbib}
% \bibliographystyle{plainnat}
% \setcitestyle{square,aysep={}}
\usepackage{pb-diagram}
\usepackage{ucs}
\usepackage[utf8x]{inputenc}
% \usepackage[russian]{babel}
\usepackage{epstopdf}
\usepackage{multicol}
\usepackage{cancel}

\usepackage{amsfonts}

%%%%%%%%%%%%%%%%%%%%%%%%%%%%%%%%%%%%%%%%%%%%%%%%%%%%%%%%%%%%%%%%%%%%%%%%%%%%%%%%%%%%%%%%%%%%%%%%%%% 

% \newtheorem{theorem}{Theorem}
%% \newtheorem{acknowledgement}[theorem]{Acknowledgement}
%% \newtheorem{algorithm}[theorem]{Algorithm}
%% \newtheorem{axiom}[theorem]{Axiom}
%% \newtheorem{case}[theorem]{Case}
%% \newtheorem{claim}[theorem]{Claim}
%% \newtheorem{conclusion}[theorem]{Conclusion}
%% \newtheorem{condition}[theorem]{Condition}
%% \newtheorem{conjecture}[theorem]{Conjecture}
%% \newtheorem{mycorollary}[theorem]{Corollary}
%% \newtheorem{mycriterion}[theorem]{Criterion}
%% \newtheorem{mydefinition}[theorem]{Definition}
%% \newtheorem{myexample}[theorem]{Example}
%% \newtheorem{myexercise}[theorem]{Exercise}
%% \newtheorem{mylemma}[theorem]{Lemma}
%% \newtheorem{mynotation}[theorem]{Notation}
%% \newtheorem{myproblem}[theorem]{Problem}
%% \newtheorem{myproposition}[theorem]{Proposition}
%% \newtheorem{myremark}[theorem]{Remark}
%% \newtheorem{mysolution}[theorem]{Solution}
%% \newtheorem{mysummary}[theorem]{Summary}
%% \newenvironment{myproof}[1][Proof]{\textbf{#1.} }{\ \rule{0.5em}{0.5em}}


\newcommand{\go}{\stackrel{\circ }{\mathfrak{g}}}
\newcommand{\ao}{\stackrel{\circ }{\mathfrak{a}}}
\newcommand{\co}[1]{\stackrel{\circ }{#1}}
\newcommand{\pia}{\pi_{\mathfrak{a}}}
\newcommand{\piab}{\pi_{\mathfrak{a}_{\bot}}}
\newcommand{\gf}{\mathfrak{g}}
\newcommand{\gfh}{\hat{\mathfrak{g}}}
\newcommand{\af}{\mathfrak{a}}
\newcommand{\afh}{\hat{\mathfrak{a}}}
\newcommand{\bff}{\mathfrak{b}}
\newcommand{\afb}{\mathfrak{a}_{\bot}}
\newcommand{\hf}{\mathfrak{h}}
\newcommand{\hfg}{\hf_{\gf}}
\newcommand{\hfb}{\mathfrak{h}_{\bot}}
\newcommand{\pf}{\mathfrak{p}}
\newcommand{\aft}{\widetilde{\mathfrak{a}}}

% \pagestyle{plain}

\theoremstyle{definition} \newtheorem{Def}{Definition}
\setbeamertemplate{caption}[empty]
\newcommand{\tr}{\hat\triangleright} \newcommand{\trc}{\triangleright}
\newcommand{\adk}{a^{\dagger}_{\kappa}} \newcommand{\ak}{a_{\kappa}}
\def\bF{\mbox{$\overline{\cal F}$}} \def\F{\mbox{$\cal F$}}

\usetheme{default}
% \usetheme{Warsaw}
\title[Tricritical Ising model]{Integrability and s-holomorphicity in tricritical Ising model}
\author[Anton Nazarov]{Anton Nazarov}

\institute[SPbSU]{
  Department of high-energy physics,\\
  Faculty of physics,\\ 
  Chebyshev laboratory,\\
  Faculty of mathematics and mechanics,\\
  Saint-Petersburg State University,\\
  198904, Saint-Petersburg, Russia\\
  e-mail: anton.nazarov@hep.phys.spbu.ru
}

\date[SQS'2013] % (optional, should be abbreviation of conference name)
{Supersymmetries and Quantum Symmetries 2013,\\ Joint Institute for Nuclear Research, Bogoliubov Laboratory of Theoretical Physics, August, 2013}

\begin{document}
\maketitle
\section{Introduction}
\begin{frame}
  \frametitle{Plan of the talk}
  \begin{itemize}
  \item Tricritical Ising model
  \item RSOS formulation and integrability
  \item CFT description and SLE 
  \item S-holomorphic observable
  \item Conclusion
  \end{itemize}
\end{frame}
\begin{frame}
  \frametitle{ Tricritical Ising model}
  Hamiltonian of the model
  \begin{equation}
    \label{eq:1}
    H = -\beta \sum_{<i,j>}\sigma_i\sigma_j - \mu \sum_{i}(\sigma_i)^2  
  \end{equation}
  Partition function
  \begin{equation}
    \label{eq:2}
    Z=\sum_{\mathrm{conf}} W[\mathrm{conf}]=\sum_{\mathrm{conf}} e^{-\frac{H[\mathrm{conf}]}{kT}}
  \end{equation}

\end{frame}
\begin{frame}
  \frametitle{$A_4$  restricted solid-on-solid model}
  Dynkin diagram, nodes: $1,\dots,4$, adjacent nodes -- adjacent values
  \begin{equation}
    \label{eq:4}
    A_4:\quad
    \begin{diagram}
      \node{1}\arrow{e,-}\node{2}\arrow{e,-}\node{3}\arrow{e,-}\node{4}
    \end{diagram}
  \end{equation}

  \begin{equation}
    \label{eq:3}
    W[\mathrm{conf}]=\prod_{\mathrm{faces}} W(\mathrm{face})
  \end{equation}

  \begin{equation}
    \label{eq:6}
    W(\mathrm{face}) = W\left(\left.\begin{array}{cc}d & c\\a &b\end{array} \right| \lambda, u\right)
  \end{equation}
  $\lambda$ -- highest eigenvalue of adjacency matrix, $u$ -- spectral parameter.
  \begin{equation}
    \label{eq:7}
    u=\lambda/2\quad \mbox{--\quad conformal point}
  \end{equation}
  \begin{equation}
    \label{eq:8}
    \lambda = \frac{\pi}{5}
  \end{equation}
\end{frame}
\begin{frame}
\frametitle{$A_4$ RSOS model weights at conformal point}
After proper normalization we have
\begin{equation}
     \label{eq:12}
     \begin{array}{l}
       W\left(                                          
         \begin{array}{cc}
           2 & 1 \\
           1 & 2
         \end{array}
       \right)=
       W\left(
         \begin{array}{cc}
           3 & 4 \\
           4 & 3
         \end{array}
       \right)=   W\left(                                         
         \begin{array}{cc}
           1 & 2 \\
           2 & 1
         \end{array}
       \right)= 
       W\left(                                                  
         \begin{array}{cc}
           4 & 3 \\
           3 & 4
         \end{array}
       \right)=\\\quad=\sqrt{\frac{2}{1+\sqrt{5}}}+\sqrt{\frac{1}{2}\left(1+\sqrt{5}\right)}\\
 W\left(                                      
           \begin{array}{cc}
             2 & 3 \\
             1 & 2
           \end{array}
         \right)=
         W\left(                                         
           \begin{array}{cc}
             3 & 2 \\
             4 & 3
           \end{array}
         \right)=    W\left(                                         
           \begin{array}{cc}
             3 & 4 \\
             2 & 3
           \end{array}
         \right)=    W\left(                                         
           \begin{array}{cc}
             2 & 1 \\
             3 & 2
           \end{array}
         \right)=\\
         W\left(                                       
           \begin{array}{cc}
             3 & 2 \\
             2 & 1
           \end{array}
         \right)=
         W\left(                                                       
           \begin{array}{cc}
             1 & 2 \\
             2 & 3
           \end{array}
         \right)=    W\left(                                                      
           \begin{array}{cc}
             4 & 3 \\
             3 & 2
           \end{array}
         \right)=
         W\left(                                                        
           \begin{array}{cc}
             2 & 3 \\
             3 & 4
           \end{array}
         \right)=\\
         \quad=\sqrt[4]{\frac{2}{1+\sqrt{5}}}\\
         W\left(
           \begin{array}{cc}
             3 & 2 \\
             2 & 3
           \end{array}
         \right)=    W\left(
           \begin{array}{cc}
             2 & 3 \\
             3 & 2
           \end{array}
         \right)=2 
       \end{array}
     \end{equation}
   \end{frame}
   \begin{frame}
     \frametitle{RSOS formulation and integrability}
     Consider model on a cylinder $N\times 2M$. 

     Double row transfer matrix:
     \begin{multline}
       \label{eq:5}
       D_N(u,\mbox{left boundary},\mbox{right boundary})=\sum_{c_0,\dots,c_N} B_{\mathrm{left}}(c_0,b_0,d_0| u)\cdot\\
       \cdot\prod_{j=0}^{N-1} W \left(\left.\begin{array}{cc} c_j & c_{j+1} \\ b_j & b_{j+1} \end{array}\right| u\right) 
       W \left(\left.\begin{array}{cc} d_j & d_{j+1} \\ c_j & c_{j+1} \end{array}\right| u\right) B_{\mathrm{right}} (c_N, b_N, d_N|u)
     \end{multline}
     Partition function is a sum of powers of eigenvalues
     \begin{equation}
       \label{eq:9}
       Z_{NM}(u)=\mathrm{tr} (D_N(u))^M=\sum_k \Lambda_N(u)_k ^M
     \end{equation}
     At conformal point $D_N$ is decomposed into a product of one-row transfer matrices. 

   \end{frame}
  \begin{frame}
    \frametitle{ CFT description and SLE }
    Take limit and obtain
     \begin{equation}
       \label{eq:10}
       Z_{r_1,a_1|r_2,a_2}(q) = \sum_{r=1}^{g-2} \sum_{s=1}^{g-1} F(A_{g-2})_{r_1,r_2}^r F({\cal G})_{a_1,a_2}^s \chi_{(r,s)}(q)
     \end{equation}
     \begin{equation}
       \label{eq:11}
       c=1-\frac{6}{g(g-1)}
     \end{equation}
     \begin{equation}
       \label{eq:13}
       \Delta_{(r,s)}=\frac{(rg-s(g-1))^2-1}{4g(g-1)}
     \end{equation}

     $\Delta_{(r,s)}$, b.c.c., SLE, $\kappa$. 
  \end{frame}
  \begin{frame}
    \frametitle{ S-holomorphic observable}
    Local conservation laws, discrete holomorphicity, continuation operator and transfer matrix
  \end{frame}
  \begin{frame}
    \frametitle{ Conclusion}
  \end{frame}
\bibliography{bibliography}{} 
  \bibliographystyle{apalike}
\end{document}
