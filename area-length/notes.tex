\documentclass[12pt]{article}
\usepackage{amsmath,amssymb,amsthm,amsfonts}
\usepackage{multicol}
\usepackage{color}
\usepackage{hyperref}
\usepackage{graphicx}

\newtheorem{theorem}{Theorem}
\newtheorem{definition}{Definition}

\newcommand{\co}[1]{\stackrel{\circ }{#1}}
\newcommand{\gf}{\mathfrak{g}}
\newcommand{\nfp}{\mathfrak{n}^{+}}
\newcommand{\nfm}{\mathfrak{n}^{-}}
\newcommand{\af}{\mathfrak{a}}
\newcommand{\uf}{\mathfrak{u}}
\newcommand{\sfr}{\mathfrak{s}}
\newcommand{\aft}{\widetilde{\mathfrak{a}}}
\newcommand{\afb}{\mathfrak{a}_{\bot}}
\newcommand{\hf}{\mathfrak{h}}
\newcommand{\hfb}{\mathfrak{h}_{\bot}}
\newcommand{\pf}{\mathfrak{p}}

\newcommand{\gfh}{\hat{\mathfrak{g}}}
\newcommand{\afh}{\hat{\mathfrak{a}}}
\newcommand{\sfh}{\hat{\mathfrak{s}}}
\newcommand{\bff}{\mathfrak{b}}
\newcommand{\hfg}{\hf_{\gf}}

\begin{document}
\title{Notes on area-length problem for CLE}
%\author{Anton Nazarov}%% $^{1,2}$}

\maketitle

\begin{abstract}
%  Here I document staff related to my work on area-length problem.

  We consider conformal loop ensemble. The ratio
  of loop area to certain power of loop length exhibit universal
  scaling behavior. Scaling function is proportional to Bessel's
  function. We show it using boundary Liouville field theory fusion rules for degenerate field.
\end{abstract}

\section{Introduction}
\label{sec:introduction}


One can ask simple question: how many loops is there?

To make this question more precise consider conformal loop ensemble \cite{sheffield2010conformal}. It is critical limit of
lattice different models for different values of $\kappa$. Loops can be interpreted as a critical
limits of domain walls in lattice models. 
We are interested in distribution of loops by length and area. 

Though several publications exist
\cite{cardy2003exact,cardy2003crossover,cardy2001exact,cardy1994geometrical}
on this problem and the answer is
obtained from computer simulation \cite{ richard2001scaling}, good understanding was not
obtained. 


Some results exist in literature for particular case of self-avoiding walks ($\kappa=\frac{8}{3}$). We start
with the description of these results \ref{sec:self-avoid-polyg} and show that general problem
leads to Liouville field theory and obtain our result using Liouville
field theory fusion rules \ref{sec:liouville-theory}. 




v\subsection{Self-avoiding polygons}
\label{sec:self-avoid-polyg}

The model of self-avoiding polygons was numerically studied in the paper \cite{richard2001scaling}.
The simplest case is the generating function for polygons of given length. It can be introduced as
\begin{equation}
  \label{eq:86}
  G(x)=\sum_l p_l x^l,
\end{equation}
where  $p_l$ is the number of different polygons on square lattice of length $l$. Different are
polygons that can not be superposed by a lattice translation.
If we know generating function we can obtain number of polygons of given length differentiating
w.r.t. $x$. 

We can also introduce generating function for polygons of given length and area:
\begin{equation}
  \label{eq:75}
  G(x,q)=\sum_{m,n} p_{m,n} x^{m} q^{n},
\end{equation}
where $p_{m,n}$ is the number of polygons with perimeter $m$, enclosing an area $n$ passing through
a given edge (or node). 
Cardy \cite{cardy2001exact} use following definition for generating function of rooted loops, using
$g=-\ln q$:
\begin{equation}
  \label{eq:76}
  G^{(r)}(x,g)=\sum_{l,A} p_{l,A}^{(r)} x^{l} e^{-gA}
\end{equation}
Generating function for rooted polygons $G^{(r)}(x,g)$ is related to $G(x,g)$ by
\begin{equation}
  \label{eq:81}
  G^{(r)}(x,g)=x \frac{d}{dx} G(x,g),
\end{equation}
because root can be chosen in $l$ ways. 


This model has critical point $(x_{c},0)$ and generating functions displays scaling behavior
\begin{equation}
  \label{eq:77}
  G(x,g)\sim G(x,g)_{\mathrm{reg}}+g^{\theta} F\left((x_{c}-x)g^{-\phi}\right)
\end{equation}
Critical indices $\theta,\phi$ are related to conventional exponents $\nu,\alpha$:
\begin{eqnarray}
  \label{eq:78}
  \phi=\frac{1}{2}\nu\\
  \theta=\frac{1}{2}(1-\alpha) \nu
\end{eqnarray}
Hyperscaling hypothesis adds another relation
\begin{equation}
  \label{eq:79}
  \alpha=2-2\nu
\end{equation}
Numerical results predict
\begin{eqnarray}
  \label{eq:80}
  \nu=\frac{3}{4}\\
  \theta=\frac{1}{3}\\
  \phi=\frac{2}{3}
\end{eqnarray}
We want to find scaling function $F$.

Using their numeric results, authors of the paper \cite{richard2001scaling} were able to come with
the following scaling for rooted generating function:
\begin{equation}
  \label{eq:112}
  G^{(r)}(\mu,\mu_B)\sim \mu^{\frac{1}{3}} \frac{\mathrm{Ai}'(s)}{\mathrm{Ai}(s)},\quad \mbox{where}\; s=\left(\frac{\mu_{B}^{2}}{\mu}\right)^{\phi}
\end{equation}
We want to get physical and geometrical understanding of this result and its generalization to
different measures on closed curves.

Let us reformulate this problem a bit so that it could be generalized. Consider self-avoiding
polygons of length not greater than $n$ on a square lattice. This implies uniform measure on
polygons. The scaling limit of self-avoiding walk when lattice mesh size $a\to 0$ and length $l\to
\infty$ is conjectured to be described by Schramm-Loewner evolution with parameter
$\kappa=\frac{8}{3}$. Since we consider closed loops the limit is loops of conformal loop ensemble
with $\kappa=\frac{8}{3}$.
The natural question is how to generalize the definition of functions $G$ and $F$ to CLE with
arbitrary value of $\kappa$?

If we consider self-avoiding polygons of length up to $n$ the generating function $G_{n}(x)$ can be
introduced as non-local observable. Denote by $N_{n}$ total number of polygons of length up to $n$.
Since we have uniform measure, the probability for polygon to have length $l$ is
$\frac{p_{l}}{N_{n}}$. Then the generating function $G_{n}(x)$ is the expectation value of $x^{l}=e^{-\mu_{B}l}$
\begin{equation}
  \label{eq:102}
  G_{n}(x)=\sum_{l<n} p_{l} x^{l} = N_{n} \sum_{l<n} \frac{p_{l}}{N_{n}} x^{l} = N_{n}\mathbb{E}[x^{l}]
\end{equation}
Generating function for polygons of given area can be written in similar way as
\begin{equation}
  \label{eq:103}
   G_{n}(x,\mu)=\sum_{l<n} p_{l,A} x^{l} e^{-\mu A} = N_{n} \sum_{l<n} \frac{p_{l,A}}{N_{n}} x^{l} =
   N_{n}\mathbb{E}[x^{l} e^{-\mu A}]
\end{equation}
Note that the  observables $\mathbb{E}[e^{-\mu_{B}l}]$ ,$\mathbb{E}[e^{-\mu_{B}l -\mu A}]$ are
non-local. So while the generalization from self-avoiding walk to arbitrary SLE is natural it is not
clear how one can compute such observables.

\subsection{CFT and local SLE geometry}
\label{sec:cft-local-sle}

Local observables w.r.t. SLE can be described by correlation functions of boundary conformal field
theory. So we need a theory which reproduces CFT in the limit $\mu,\mu_{B}\to 0$ and has non-local
observables such as area and length. Note also that the parameter $\kappa$ is connected with the
central charge of conformal field theory by the relation
\begin{equation}
  \label{eq:113}
  c=\frac{(8-3\kappa)(\kappa-6)}{2\kappa}
\end{equation}


Liouville theory of two-dimensional gravity is a natural choice, but first we need to discuss
how CFT is used to study local geometry of SLE curves. 

Parameter $\kappa$ of SLE and CLE is connected with Hausdorff dimension of the curve. 


Define Hausdorff dimension as follows.
\begin{definition}
  {\it Hausdorff content} of the set $S$ is
  \begin{equation}
    \label{eq:3}
    C^d_H = \mathrm{inf} \sum_i r_i^d,
  \end{equation}
  where $\{r_i\}$ are such that there exists cover of $S$ by balls of
  radii $r_i>0$. 

  {\it Hausdorff dimension}
  \begin{equation}
    \label{eq:4}
    d_H\equiv\mathrm{dim}_H(S) = \mathrm{inf}\{ d\geq 0: C^d_H(S)=0\}
  \end{equation}
\end{definition}


Consider loop of length $L$. The radius of the loop is connected with
the length as $L\sim R^{d_H}$. The area is related to radius $A\sim
R^2$, so $A\sim L^{\frac{2}{d_H}}$.
(It is proved or almost proved in papers by Lawler).

For CLE the connection of Hausdorff dimension and parameter $\kappa$
is known:
\begin{equation}
  \label{eq:5}
  d_H=1+\frac{\kappa}{8}
\end{equation}

($\nu^{-1}=d_H$)

Note that number of rooted loops of given area scales as the area to
the power minus one. 

Hausdorff dimension in CFT is calculated by the insertion of certain primary field called ``pinning
operator'' \cite{bettelheim2005harmonic}. Expanded presentation can be found in the paper
\cite{gruzberg2006stochastic}. The argument is as follows. 

Consider a closed curve. Harmonic measure of some part of the curve can be understood as a charge of
that part if the curve is made of conducting material and its total charge is equal to one. 
Closed curve can be covered by small circles of radius $r$ centered at points
$z_{i}$. Consider one such circle. The curve divides the circle into two parts, inner and outer
w.r.t. to curve. Outer part can be conformally mapped to the unit disk. Denote this map by $w(z)$.
With the normalization such that one point is fixed and derivative at infinity is equal to one the
moments of harmonic measure in small circle of radius $r$ scale as
\begin{equation}
  \label{eq:116}
  \mu(r)\sim |w(r)|\sim r |w'(r)|
\end{equation}
This scaling follows from electrostatic interpretation of harmonic measure, since $\mu(r)$ is charge
inside the circle and by Gauss theorem it is equal to the flux of electric field through the
boundary. Typical absolute value of electric field on the circle is $|w'(r)|$ and boundary length is
$2\pi r$.

So scaling of the moments of $\mu(r)$ is determined by scaling of $|w'(r)|$:
\begin{equation}
  \label{eq:117}
  \prec |w'(r)|^{h}\succ \sim r^{\Delta(h)}, \quad r\to 0
\end{equation}

The moments of harmonic measure for the whole curve can be written as a sum over small circles:
\begin{equation}
  \label{eq:120}
  M_{h}=\sum_{i=1}^{N} \mu(z_{i},r)^{h}
\end{equation}

In the limit $r\to 0$ number of circles grows and moments scale as
\begin{equation}
  \label{eq:121}
  M_{h}\sim r^{\tau(h)}
\end{equation}
The function $\tau(h)$ is called multifractal spectrum. For simple curve $\tau(h)=h-1$, so anomalous
exponents $\delta(h)$ are defined as $\delta(h)=\tau(h)-h+1$. 

Note that $M_{0}$ is a total number of circles and $\tau(0)=-d_{H}$ by definition of Hausdorff
dimension. For $h=1$ the sum is equal to the total charge of the curve so it does not scale and
$\tau(1)=0$. 

To calculate multifractal spectrum we need to average over the values at points $z_{i}$. One can
assume the ergodicity and rewrite $M_{h}$ as
\begin{equation}
  \label{eq:122}
  M_{h}=N \prec \mu(z,r)^{h}\succ
\end{equation}
for some point $z$. The total number of circles $N$ is equal to zeroth momentum $M_{0}$, so
$M_{h}\sim r^{\tau(0)} \prec \mu(z,r)^{h}\succ$. Local multifractal spectrum $\tilde\tau(h)$ is
defined as $ \prec \mu(z,r)^{h}\succ\sim r^{\tilde\tau(h)}$ and local anomalous exponents are
$\Delta_{bulk}(h)=\tilde\tau(h) -h$.

Following relations hold for multifractal spectrum and anomalous exponents:
$\tilde\tau(h)=\tau(h)-\tau(0)$, $\Delta_{bulk}(h)=\delta(h)-\delta(0)$. The most important for us
here is the connection of Hausdorff dimension and local anomalous exponent 
\begin{equation}
  \label{eq:150}
  d_{H}=1+\Delta_{bulk}(1)
\end{equation}

Multifractal spectrum and anomalous exponents were first obtained by Duplantier in paper
\cite{duplantier2000conformally} using quantum gravity. The result is
\begin{equation}
  \label{eq:123}
  \Delta_{bulk}(h)=-\frac{h}{2}+\left(\frac{1}{16}+\frac{1}{4\kappa}\right)
  (\kappa-4+\sqrt{(\kappa-4)^{2}+16\kappa h})
\end{equation}
\begin{equation}
  \label{eq:124}
  \tau(h)=\frac{1}{2}(h-1) +\frac{25-c}{24}\left(\sqrt{\frac{24 h +1-c}{25-c}}-1\right), 
\end{equation}
where $c$ and $\kappa$ are connected by \eqref{eq:113}.

So we have following result for Hausdorff dimension (eq \eqref{eq:5}):
\begin{equation}
  \label{eq:125}
  d_{H}=1+\frac{\kappa}{8}
\end{equation}



The stochastic averages for (local) moments of harmonic measure are represented by different CFT
correlation functions depending upon the position of point of interest. If we are interested in a
bulk point lying on a curve, we need to put at this point a special operator. It is $\Psi_{10}$
which has an interpretation as an operator that creates two lines coming from the point in the bulk.

Identification of this operator is based upon the model of compactified boson field. Level lines of
such a field are critical curves described by Schramm-Loewner evolution. Vertex operators of the
model are either ``electric'' or ``magnetic''. Magnetic operators are interpreted as discontinuities
in field value. 

This interpretation is based on the following analogy. Consider $O(n)$ model on the domain
$\mathcal{D}$ on the honeycomb lattice. It's configurations can be seen as the configurations of
oriented loops with arbitrary orientation. Then each site has weight $e^{\pm i e_{0}\pi/6}$ if loop
is turning right or left at the site. The parameter $e_{0}$ is connected with $n$ by $n=2\cos \pi
e_{0}$. Then we need to introduce height function that will give us field in critical limit. Choose
an arbitrary site of dual lattice as a reference point $H=0$ and increase or decrease $H$ by $\pi$
when crossing loop depending upon its orientation.

Since the choice of loop orientations is arbitrary field $H$ should be compactified $H\simeq
H+2\pi$.
 In critical limit this gives us free boson theory:
\begin{equation}
  \label{eq:126}
  S=\frac{g}{4\pi} \int_{\mathcal{D}} d^{2}x (\nabla H)^{2} +\frac{i e_{0}}{2\pi}
  \int_{\partial\mathcal{D}} dl KH +\frac{i e_{0}}{8\pi} \int_{\mathcal{D}} d^{2}x
  RH+\lambda\int_{\mathcal{D}} d^{2}x V(H)
\end{equation}
Boundary loops has different weights. To take this into account we need to add second term to the
action. Third term comes from the curvature of surface which can be realized microscopically by
insertion of lattice faces with number of edges not equal to 6. The last term is locking potential
which is needed for field compactification. In the limit $\lambda\to\infty$ field $H=k\pi$, so the
potential is
\begin{equation}
  \label{eq:127}
  V(H)=\sum_{k\in\mathbb{Z}\setminus \{0\}}v_{k}e^{2ikH}
\end{equation}
Each term in this expansion is vertex operator with scaling dimension $x_{k}=\frac{2}{g}k(k-e_{0})$,
so for $e_{0}\in [0,1]$ the operator with $k=1$ is most relevant and it is strictly marginal if
$x_{k}=2$, which is required for conformal invariance. This fixes us the parameter $g$:
\begin{equation}
  \label{eq:128}
  e_{0}=1-g\quad n=-2\cos \pi g
\end{equation}
Dense phase is for $e_{0}\in [0,1], g\in [0,1]$ and dilute phase of $O(n)$ model corresponds to
$e_{0}\in [-1,0], g\in [1,2]$.
Now critical curves can be identified with the level lines of field $H$ and $g=\frac{4}{\kappa}$. 

Following notation is more conventional:
$\varphi=\sqrt{2g}H$ with new coupling constant in action equal to $\frac{1}{2}$. Compactification
radius for this choice is $\mathcal{R}=\sqrt{\frac{8}{\kappa}}$ and action reads
\begin{equation}
  \label{eq:129}
  S[\varphi]=\frac{1}{8\pi}\int_{\mathcal{D}}d^{2}x ((\nabla \varphi)^{2} + i2\sqrt{2} \alpha_{0}
  R\varphi)+i\frac{\sqrt{2}\alpha_{0}}{2\pi} \int_{\partial\mathcal{D}} dl
  K\varphi+\int_{\mathcal{D}}d^{2}x e^{i\sqrt{2}\alpha_{+}\varphi}
\end{equation}
where $2\alpha_{0}=\frac{\sqrt{\kappa}}{2}e_{0}=\frac{\sqrt{\kappa}}{2}-\frac{2}{\sqrt{\kappa}}$,
$\alpha_{\pm}=\alpha_{0}\pm\sqrt{\alpha_{0}^{2}+1}$, $\alpha_{+}=\frac{\sqrt{\kappa}}{2}$,
$\alpha_{-}=-\frac{2}{\sqrt{\kappa}}$. 

Central charge of this theory is
\begin{equation}
  \label{eq:130}
  c=1-24\alpha_{0}^{2}=1-3\frac{(\kappa-4)^{2}}{2\kappa}
\end{equation}
In computations of correlation functions the last term in action is treated perturbatively. Due to
neutrality condition at most first order terms survive. 

We now describe theory on a full plane which has the following primary fields and local operators. 

By choosing coordinates we can make space locally flat with the curvature at infinity. Then instead
of term with the curvature $R$ we get vertex operator $V_{-2\alpha_{0},-2\alpha_{0}}(\infty)$.

We can decompose field $\varphi(z,\bar z)$ into holomorphic and antiholomorphic part and introduce a
dual field $\tilde\varphi$ as
\begin{equation}
  \label{eq:131}
  \varphi(z,\bar z)=\phi(z)+\bar\phi(\bar z),\quad \tilde\varphi=-i\phi(z)+i\bar\phi(z)
\end{equation}

Holomorphic and antiholomorphic vertex operators are
\begin{equation}
  \label{eq:132}
  V_{\alpha}(z)=e^{i\sqrt{2}\alpha \phi(z)},\quad \bar V_{\bar\alpha}(\bar z)=e^{i\sqrt{2}\bar\alpha
    \bar \phi(\bar z)}
\end{equation}

Correlation functions in Coulomb gas picture are easily calculated:
\begin{equation}
  \label{eq:157}
  \left< V_{\alpha_{1}}(z_{1})\dots
    V_{\alpha_{n}}(z_{n})\right>=\prod_{i<j}(z_{i}-z_{j})^{2\alpha_{i}\alpha_{j}}, \quad \mbox{if}\quad \sum_{i}\alpha_{i}=2\alpha_{0}
\end{equation}

Full vertex operator is a product of them
\begin{equation}
  \label{eq:133}
  V_{\alpha,\bar\alpha}(z,\bar z)=V_{\alpha}(z)\bar V_{\bar \alpha}(\bar z)
\end{equation}
There is another possible decomposition with physical interpretation of electric and magnetic operators:
\begin{equation}
  \label{eq:134}
  \alpha=e+m\quad \bar\alpha=e-m
\end{equation}
\begin{equation}
  \label{eq:135}
  V_{e,0}(z,\bar z)=e^{i\sqrt{2} e\varphi(z,\bar z)},\quad   V_{0,m}(z,\bar z)=e^{-\sqrt{2} m\tilde\varphi(z,\bar z)}
\end{equation}

Conformal weights of these operators are 
\begin{equation}
  \label{eq:136}
  h(\alpha)=\alpha(\alpha-2\alpha_{0})=(e+m)(e+m-2\alpha_{0})
\end{equation}
\begin{equation}
  \label{eq:137}
  \bar h(\bar\alpha)=\bar\alpha(\bar\alpha-2\alpha_{0})=(e-m)(e-m-2\alpha_{0})
\end{equation}

Spinless operators with $h=\bar h$ are possible when $\alpha=\bar\alpha$ or
$\bar\alpha=2\alpha_{0}-\alpha$. These two cases correspond to pure electric operators with $m=0$ or
magnetic operators with arbitrary $m$ and electric charge $e=\alpha_{0}$. 

The following notation is used in \cite{gruzberg2006stochastic} for spinless vertex operators:
\begin{equation}
  \label{eq:138}
  V^{(\alpha)}=V_{\alpha,2\alpha_{0}-\alpha}=V_{\alpha}\bar V_{2\alpha_{0}-\alpha}
\end{equation}

Charges and conformal weights of degenerate fields from Kac table are
\begin{equation}
  \label{eq:139}
  \alpha_{r,s}=\frac{1}{2}(1-r)\alpha_{+}+\frac{1}{2}(1-s)\alpha_{-},\quad h_{r,s}=\frac{(r\kappa-4s)^{2}-(\kappa-4)^{2}}{16\kappa}
\end{equation}
So holomorphic and full primary fields are denoted by $\psi_{r,s}(z)\sim V_{\alpha_{r,s}}(z)$ and
$\psi_{r,s}(z,\bar z)\sim V^{(\alpha_{r,s})}(z,\bar z)$

Theory on a domain with boundary is different, because we need to take into account boundary condition
\begin{equation}
  \label{eq:140}
  \partial_{l}\varphi|_{\partial \mathcal{D}}=0.
\end{equation}
But we can map the domain $\mathcal{D}$ to upper half-plane $\mathbb{H}$, then we can take into
account boundary conditions by analytic continuation to the whole plane and duplication of fields:
\begin{equation}
  \label{eq:141}
  \bar\phi(\bar z)\to -\phi(z^{*})
\end{equation}
\begin{eqnarray}
  \label{eq:142}
  V_{e,0}\to e^{i\sqrt{2}e\phi(z)} e^{-i\sqrt{2} e \phi(z^{*})}\\
  V_{0,m}\to e^{i\sqrt{2}m\phi(z)} e^{-i\sqrt{2} m \phi(z^{*})}
\end{eqnarray}

On the boundary vertex operators fuse with their images:
\begin{equation}
  \label{eq:144}
    V_{e,m}(z,\bar z)\to V^{(2m)}(x) e^{i 2\sqrt{2}m \phi(x)}
\end{equation}

Critical curves are level lines of field $\varphi$, so by imposing boundary condition $\varphi=0$
on one half of the boundary and $\varphi=\pi n R$ on another half we get $n$ curves growing from the
origin. We now argue that curves are produced by a magnetic operator $V_{0,m}$ with properly chosen
$m$. 

\begin{equation}
  \label{eq:145}
  V_{e,0}(z,\bar z)V_{0,m}(z',\bar z')=\left(\frac{z-z'}{\bar z-\bar z'}\right)^{2em}
  V_{e+m}(z')\bar V_{e-m}(\bar z')+\dots=e^{4i em \arg(z-z')} V_{e+m}(z')\bar V_{e-m}(\bar z')+\dots
\end{equation}
So when electric operator goes around magnetic operator we get a field jump of $4\sqrt{2}\pi m$.
Magnetic operator is vortex operator. So
\begin{equation}
  \label{eq:146}
  m=\frac{\sqrt{2}}{8}nR = \frac{n}{2\sqrt{\kappa}}=-\frac{n}{4} \alpha_{-}
\end{equation}
Since we want curve-creating operator to be spinless we have $e=\alpha_{0}$ and
$\alpha=\alpha_{0}-\frac{n}{4}\alpha_{-}=\alpha_{0,\frac{n}{2}}$. So our curve-creating operator is
$\psi_{0,n/2}(z,\bar z)$ with conformal weight
\begin{equation}
  \label{eq:147}
  h_{0,n/2}=\frac{4n^{2}-(\kappa-4)^{2}}{16\kappa}
\end{equation}

Pinning operator creates one curve passing through the point in the bulk, so it is $\psi_{0,1}$. Its
conformal weight is
\begin{equation}
  \label{eq:148}
  h_{0,1}=\frac{8-\kappa}{16},
\end{equation}
We want to show that Hausdorff dimension is
\begin{equation}
  \label{eq:149}
  d_{H}=2-2h_{0,1}
\end{equation}

To do so we need to connect local anomalous scaling exponents with conformal weights of vertex
operators. Local moments of harmonic measure $\mu(z,r)^{h}$ are connected with conformal maps.
Consider conformal map $w$ of the exterior of critical curve $\gamma$ to upper half-plane or unit
circle. Normalize this map such that $w(z)=z$ and $w(z'\to\infty)=z'+o(z')$. As we have seen the
scaling of $w$ near $z$ is related to harmonic measure.

The partition function $Z(0,L)$ of configurations that contain a closed curve passing through the
points $0$ and $L$ is given by the correlator of two pinning operators $\psi_{0,1}$:
\begin{equation}
  \label{eq:151}
  \frac{Z(0,L)}{Z}=\left<\psi_{0,1}(0)\psi_{0,1}(L)\right>,
\end{equation}
where $Z$ is partition function of all configurations. 

The partition function $Z(0,L)$ can be calculated as the product of partition functions inside and
outside of the curve averaged over all possible curves:
\begin{equation}
  \label{eq:152}
  Z(0,L)=\prec Z_{\gamma}^{\mathrm{int}} Z_{\gamma}^{\mathrm{ext}}\succ
\end{equation}
To get the moments of harmonic measure we need to insert some operator $\mathcal{O}(r)$ (product of
operators), for charge neutrality we need to add another one at infinity $\mathcal{O}(\infty)$.
Since we are interested in small $r<<L$ we can fuse $\mathcal{O}(\infty)$ with $\psi_{0,1}(L)$ into
$\Psi(\infty)$. So we need to compute correlation function $\left<\psi_{0,1}(0) \mathcal{O}(r)
  \Psi(\infty)\right>$. It can be done in two ways, directly and averaging
$\left<\mathcal{O}(r)\mathcal{O}(\infty)\right>_{\gamma}$ over all curves:
\begin{equation}
  \label{eq:153}
  \left<\psi_{0,1}(0) \mathcal{O}(r)\Psi(\infty)\right>= \prec
  \left<\mathcal{O}(r)\mathcal{O}(\infty)\right>_{\gamma}^{\mathrm{ext}} Z_{\gamma}^{\mathrm{ext}} Z_{\gamma}^{\mathrm{int}}\succ
\end{equation}
The correlation function of operators $\mathcal{O}$ is statistically independent from other factors
and has all the dependence on $r$. Apply to this correlation function conformal map $w$:
\begin{equation}
  \label{eq:154}
  \left<\mathcal{O}(r)\mathcal{O}(\infty)\right>_{\gamma}^{\mathrm{ext}}=|w'(r)|^{h}\left<\mathcal{O}(w(r))\mathcal{O}(\infty)\right>,
\end{equation}
where $h$ is conformal weight of $\mathcal{O}$. So
\begin{equation}
  \label{eq:155}
  \prec|w'(r)|^{h}\succ\sim\frac{ \left<\psi_{0,1}(0) \mathcal{O}(r)\Psi(\infty)\right>}{\left<\mathcal{O}(w(r))\mathcal{O}(\infty)\right>}
\end{equation}

To get Hausdorff dimension $d_{H}=1+\Delta(1)$ \eqref{eq:150} we need to insert operator
$V_{\alpha(1)/2,\alpha(1)/2}$, since we consider system with boundary. Then it fuses with its image
and produce vertex operator with conformal weight $h=1$. 
\begin{equation}
  \label{eq:156}
  \prec|w'(r)|^{2h(\alpha(1)/2)}\succ\sim\frac{\left< V_{\alpha(1)/2}(z) V_{\alpha(1)/2}(z^{*}) \psi_{0,1}(0) \Psi(\infty)\right>}{\left<V_{\alpha(1)/2}(w(z)) V_{\alpha(1)/2}(w^{*}(z))\right>}
\end{equation}

Use the formula \eqref{eq:157} to calculate correlation functions on the right-hand side. The
numerator gives us $r^{2\alpha_{0,1}\alpha(1)}$ for $z,z^{*}\to r,\; z-z^{*}\to 0$, the denominator
$(w(z)-w^{*}(z))^{\alpha(1)^{2}/2}\sim (|z| |w'(z)|)^{\alpha(1)^{2}/2}$. So total power of $|w'(r)|$
is 
$2h(\alpha(1)/2)+\alpha(1)^{2}/2=\alpha(1)
(\alpha(1)/2-2\alpha_{0})+\alpha(1)^{2}/2=\alpha(1)(\alpha(1)-2\alpha_{0})=1$, and total power of
$r$ is $2\alpha_{0,1}\alpha(1)-\alpha(1)^{2}/2$
\begin{equation}
  \label{eq:158}
  \prec |w'(r)|\succ \sim r^{\Delta(1)}=r^{2\alpha_{0,1}\alpha(1)-\alpha(1)^{2}/2}
\end{equation}

If we substitute $h_{0,1}=\frac{1}{2}-\frac{\kappa}{16}$ and
$2\alpha_{0}=\frac{\sqrt{\kappa}}{2}-\frac{2}{\sqrt{\kappa}}$ we obtain $\Delta(1)=\frac{\kappa}{8}$.

%%  To see it one can consider circle of small radius $\varepsilon$ centered at $z$. Put some conformal
%%  boundary condition on one half of the circle and another condition on the rest of the circle. Then
%%  critical curve would join the points of change in boundary conditions. The expectation values of
%%  observables in this geometry are given by CFT correlation functions with two boundary condition
%%  changing operators inserted:
%%  \begin{equation}
%%    \label{eq:119}
%%    \left< \Psi_{12}(z')\Psi_{12}(z'')\right>
%%  \end{equation}
%%  where $|z'-z|\leq \varepsilon$ and $|z''-z|\leq \varepsilon$. The boundary condition changing
%%  operators in Coulomb gas picture can be presented as the vertex operators
%%  $V_{\alpha_{12}}=e^{i\sqrt{2}\alpha_{12}\varphi(z)}$. When $\varepsilon\to 0$ we can fuse this two
%%  operators to single operator $V_{2\alpha_{12}}$. Note that $\alpha_{12}=-\frac{1}{2}\alpha_{-}$
%%  \eqref{eq:55}, so $V_{2\alpha_{12}}\sim V_{-\alpha_{-}}\sim V_{\alpha_{+}}$ (?)
%%  
%%  
%%  The moments of harmonic measure are connected with the CFT correlation functions by
%%  \begin{equation}
%%    \label{eq:118}
%%    \left<O_{h}(r) \Psi_{10}(0)\Psi(\infty)\right>\sim r^{h'-2h}\prec |w'(r)|^{h}\succ
%%  \end{equation}
%%  
%%  
%%  
%%  
%%  


%%  Let $N^r(L)$ be number of rooted loops with a given length. By $N^r(L,A)$ we denote the number of
%%  loops of given length and area. $N^r(L)=\int_0^{\infty} N^r(L,A) dA$. Area weighted number of loops
%%  is denoted by $G^r(L,\mu)$. (Here $\mu$ is bulk coupling constant ).
%%  \begin{equation}
%%    \label{eq:2}
%%    G^r(L,\mu)=\int e^{\mu A} N^r(L,A) dA
%%  \end{equation}
%%  
%%  \begin{equation}
%%    \label{eq:1}
%%    G^r (\mu,x) = \sum_{\mathrm{clusters}} N^r (L,A) x^L e^{-\mu A}
%%  \end{equation}
%%  
%%  At $x\to x_c$
%%  \begin{equation}
%%    \label{eq:6}
%%    G^r(\mu,x)= \mu^{\theta} F\left(\frac{x-x_c}{\mu^{\phi}}\right)
%%  \end{equation}
%%  


%% 
%% \subsection{$O(n)$ model}
%% \label{sec:introduction-1}
%% 
%% Self-avoiding polygons can be thought as the limit $n\to 0$ of loops in $O(n)$ model.
%% 
%% Partition function for $O(n)$ model is
%% \begin{equation}
%%   \label{eq:9}
%%   Z_{O(n)} = \sum_{\mbox{loop conf}} x^{\mbox{total length}}
%%   n^{\mbox{number of loops}}
%% \end{equation}
%% Critical point is $x=\frac{1}{\sqrt{2+\sqrt{2-n}}}$, the continuum
%% limit in this point is described by minimal model $(p,p')$ with
%% $n=-2\cos\left(\frac{\pi p}{p'}\right)$. $c=1-6\frac{(p-p')^2}{pp'}$, $b=\sqrt{\frac{3}{2}}$.
%% 
%% Self-avoiding walk is obtained in the limit $n\to 0$. Then $c=0$ and
%% we have logarithmic conformal field theory. 
%% 
%% Logarithmic CFT $\sim$ Liouville theory ???
%% 
%% \section{Scaling of random loops}
%% \label{sec:scaling-random-loops}
%% 
%% Cardy \cite{cardy2001exact} argues that generating function $G^{(r)}$ for rooted self-avoiding loops is given by a derivative w.r.t. $x$
%% of expectation value of $e^{-gA}$ (eq.6):
%% \begin{equation}
%%   \label{eq:82}
%%   Z=\left<e^{-gA}\right>_{\mathrm{SAL}}=\left<e^{-\sqrt{g}\int J_{\lambda}A_{\lambda}d^{2}x}\right>_{\mathrm{SAL}.A}
%% \end{equation}
%% But reasoning is quite strange and unsound. Due to the need to use $g$ for metric in later sections
%% we will denote area coupling constant by $\mu$ and loop length coupling constant by $\mu_{B}=-\ln x$
%% (boundary). 
%% 
%% 
%% In this section we introduce scaling function for random loops. 
%% 
%% Consider rooted loops in $O(n)$ model.
%% 
%% The idea is to write $G$ as a correlation function in Liouville
%% theory.
%% 
%% 
%% Partition function is product over loops of different length
%% \begin{equation}
%%   \label{eq:10}
%%   Z_{O(n)} = \sum_{\mbox{loop conf}} \prod_L x^{L N(L)} n^{N(L)} = \sum_{\mbox{loop conf}} \prod_L\prod_A x^{L N(L,A)} n^{N(L,A)}
%% \end{equation}
%% 
%% 
%% We can rewrite $G^r(\mu,x)$ as the derivative of generating function
%% \begin{equation}
%%   \label{eq:7}
%%   G^r(\mu,x)=\mu\partial_x \log Z(\mu,\mu_B)
%% \end{equation}
%% 
%% Here
%% \begin{equation}
%%   \label{eq:8}
%%   Z(\mu,\mu_B) = \left< e^{-\mu A} \right> = \frac{\sum_{\mbox{loop conf}}
%%   \prod_L\prod_A x^{L N(L,A)} n^{N(L,A)} e^{-\mu A}}{\sum_{\mbox{loop conf}} \prod_L\prod_A x^{L N(L,A)} n^{N(L,A)}}
%% \end{equation}
%% 
%% 

\subsection{Quantum gravity and global geometry of critical curves}
\label{sec:quant-grav-glob}

Theory of quantum gravity in two dimensions, implemented as a matter CFT coupled to Liouville field
describes the behavior of statistical models in fluctuating geometry. 
While usually configurations of  Liouville field are interpreted as random curved Riemann surfaces,
we need to consider random flat shapes. So functional integral in our case is over different space
of functions. The matter is implemented as a minimal model in Coulomb gas picture with matter field
$\chi$.

Full action of two-dimensional quantum gravity on world-sheet manifold $\mathcal{M}$ with boundary
$\partial \mathcal{M}$ in background metric $\hat g$ is given by the formula \cite{kostov2004boundary}:
\begin{multline}
  \label{eq:160}
  S[\chi,\varphi]=\int_{\mathcal{M}}\left(\frac{1}{4\pi}[(\nabla \varphi)^{2}+(\nabla
    \chi)^{2}+(Q\varphi-ie_{0}\chi)\hat R]+\mu e^{2b\varphi}\right)d^{2}z +\\
  +  \int_{\partial
    \mathcal{M}}\left(\frac{1}{2\pi}(Q\varphi-ie_{0}\chi)\hat K + \mu_{B}e^{b\varphi}\right)dx+\mbox{ghosts}
\end{multline}
where $\hat R$ is curvature of background metric in the bulk, $\hat K$ -- curvature of background
metric on the boundary, 
\begin{equation}
  \label{eq:161}
  Q=\frac{1}{b}+b,\quad e_{0}=\frac{1}{b}-b
\end{equation}
and ghosts central charge is $-26$, so total central charge vanishes:
\begin{equation}
  \label{eq:162}
  c_{tot}=c_{\varphi}+c_{\chi}+c_{\mathrm{ghosts}}=(1+6Q^{2})+(1-6e_{0}^{2})-26=0
\end{equation}
We can use conformal map to rewrite the action on the upper half-plane $\mathbb{H}$:
\begin{equation}
  \label{eq:159}
  S[\varphi,\chi]=\int_{\mathbb{H}} d^{2}z \left(\frac{1}{4\pi}[(\nabla \varphi)^{2}+(\nabla
    \chi)^{2}]+\mu e^{2b\varphi}\right) +\int_{-\infty}^{\infty} dx \mu_{B}e^{b\varphi} + \mbox{ghosts}
\end{equation}
where charges are now hidden in fields asymptotic at the infinity:
\begin{equation}
  \label{eq:163}
  \varphi\sim Q\log|z|^{2},\quad \chi\sim-e_{0}\log|z|^{2}
\end{equation}

The partition function is defined as a functional integral over matter and Liouville field:
\begin{equation}
  \label{eq:164}
  Z(\mu,\mu_{B})=\int \mathcal{D}\chi \mathcal{D}\varphi e^{-S[\chi,\varphi]}
\end{equation}
To define the measure one needs to fix the boundary conditions. Kostov et. al choose non-homogeneous
Neumann boundary condition for the Liouville field:
\begin{equation}
  \label{eq:165}
  i(\partial -\bar\partial)\varphi=4\pi\mu_{B}e^{b\varphi}
\end{equation}

There is a duality $b\to \frac{1}{b}$ in this theory. In particular, it is possible to choose
boundary term as $e^{\frac{\varphi}{b}}$. This choice may be relevant to us since it corresponds to
dual boundary cosmological constant $\tilde\mu_{B}$ which has scaling $\tilde\mu_{B}\sim
\mu^{\frac{1}{2b^{2}}}$ and is related to fractal boundary. 

We want to consider configurations with fractal boundaries but flat bulk. So the difference is in
space of functions $\varphi$ we are integrating over in \eqref{eq:164}. 

Note that self-avoiding walks is the model with $\kappa=\frac{8}{3}$, $c=0$,
$e_{0}=\frac{1}{\sqrt{6}}$, $b=\sqrt{\frac{3}{2}}$ or $b=\sqrt{\frac{2}{3}}$. Since in paper
\cite{kostov2004boundary} $b$ is assumed to be less than $1$, we will use
\begin{equation}
  \label{eq:166}
  b=\sqrt{\frac{2}{3}}, \quad 0<b<1
\end{equation}
Matter theory is just a free boson, so primary fields are vertex electric and magnetic operators
\eqref{eq:135} (with the change of notations $\alpha\to \sqrt{2} e$, $2\alpha_{0}\to  e_{0}$). In
quantum gravity vertex operators are dressed by Liouville field
\begin{equation}
  \label{eq:167}
  V_{e,\alpha}(z,\bar z)=e^{2ie\chi(z,\bar z)} e^{2\alpha\varphi(z,\bar z)}
\end{equation}
in such a way that conformal dimension $\Delta=1$:
\begin{equation}
  \label{eq:168}
  e(e-e_{0})+\alpha(Q-\alpha)=1
\end{equation}
Similarly for magnetic (vortex) operators we have
\begin{equation}
  \label{eq:169}
  \chi(z,\bar z)=\chi(z)+\bar\chi(\bar z),\quad \tilde\chi(z,\bar z)=\chi(z)-\bar\chi(\bar z)
\end{equation}
\begin{equation}
  \label{eq:170}
  V_{m,\alpha}(z,\bar z)=e^{2im\tilde\chi(z,\bar z)} e^{2\alpha\varphi(z,\bar z)}
\end{equation}
Since left and right conformal dimensions
\begin{equation}
  \label{eq:171}
  \Delta_{m}=m(m-e_{0}), \quad \bar\Delta_{m}=m(m+e_{0})
\end{equation}
are not equal the field is not spinless. So we need to consider mixed operator $\mathcal{O}_{e,m}$
with dimensions \eqref{eq:134}
\begin{equation}
  \label{eq:172}
  \Delta_{e,m}=(e+m)(e+m-e_{0}), \quad \bar \Delta_{e,m}=(e-m)(e-m-e_{0})
\end{equation}
The spin $\Delta-\bar\Delta$ vanishes if $e=\frac{e_{0}}{2}$.
\begin{equation}
  \label{eq:173}
  \Delta_{\frac{e_{0}}{2},m}=m^{2}-\frac{e_{0}^{2}}{4}
\end{equation}

Primary fields from Kac table with dimensions
\begin{equation}
  \label{eq:174}
  \Delta_{rs}=\frac{1}{4}\left[\left(\frac{r}{b}-sb\right)^{2}-e_{0}^{2}\right]
\end{equation}
can be constructed as electric operators with charges
\begin{equation}
  \label{eq:175}
  e_{rs}=\frac{1}{2}\left(e_{0}-\frac{r}{b}+sb\right)
\end{equation}
or as magnetic operators with electric charge $\frac{e_{0}}{2}$ and magnetic charge
\begin{equation}
  \label{eq:176}
  m_{rs}=\frac{1}{2}\left(-\frac{r}{b}+sb\right)
\end{equation}



\section{Liouville theory}
\label{sec:liouville-theory}


%%  Liouville theory, flat metric, asymptotic for field, Coulomb gas formalism.
%%  \cite{nakayama2004liouville,ponsot2002boundary,teschner2001liouville,fateev2000boundary,teschner2000remarks,zamolodchikov1996conformal} 
%%  
%%  Liouville theory describes internal metric of surface. It is possible to obtain length of boundary
%%  and area of surface with such a metric. To get the answer for flat loops we need to impose some
%%  condition on metric in the bulk. 
%%  

In last 30 years great progress was made in study of quantum Liouville theory made in papers
\cite{fateev2000boundary,zamolodchikov1996conformal,knizhnik1988fractal} and in study of matter on
random discrete surfaces
\cite{kazakov1986ising,duplantier1990geometrical,gaudin1989n,duplantier1988conformal}. These two
approaches to two-dimensional quantum gravity are in agreement
\cite{kostov2004boundary,kostov2003boundary}. In recent years strict mathematical definitions and proofs for some
field-theoretic results were obtained \cite{duplantier2011liouville,garban2012quantum}. All these
solid developments allow us to use Liouville theory as a theory of random surfaces and apply its
methods to the geometry of random loops. Our approach is similar to the application of Liouville
theory to harmonic measure problems \cite{duplantier2000conformally}. 

In present text we do not try to be mathematically strict and do not use the definition of Liouville
measure from \cite{duplantier2011liouville} but we suppose that our result can be proven using
methods of that paper.

Before moving to quantum Liouville theory and scaling limit of generating functions of self-avoiding
polygons we need to discuss the geometry of Riemann surfaces and polygons on complex plane and its
connection with Liouville equation. 

\subsection{Uniformization of Riemann surfaces and Liouville equation}
\label{sec:unif-riem-surf}

The problem of uniformization of Riemann surfaces was of central importance in mathematics at the
end of 19th -- beginning of 20th century. Great mathematicians such as Riemann, Klein and Poincare
proposed different approaches to its solution and the uniformization theorem was finally proven by
Koebe in 1907. 

\begin{definition}
  A Riemann surface is a connected Hausdorff space $M$ (distinct points have disjoint neighbourhoods) together with a collection
  of charts ${U_{\alpha} , z_{\alpha} }$ with the following properties:
  \begin{itemize}
  \item The $U_{\alpha}$ form an open covering of $M$ .
  \item Each $z_{\alpha}$ is a homeomorphic mapping ($\exists$ continuous inverse) of $U_{\alpha}$ onto an open subset of the complex plane $\mathbb{C}$.
  \item If $U_{\alpha}\cap  U_{\beta} \neq \emptyset$, then $f_{\alpha\beta} = z_{\beta}\circ  z_{\alpha}^{-1}$ is complex analytic on $z_{\alpha} (U_{\alpha} \cap U_{\beta} )$.
  \end{itemize}  
\end{definition}

The statement of the theorem is 
\begin{theorem}
  Every simply connected Riemann surface is conformally equivalent to one of the three domains: the
  open unit disk, the complex plane, or the Riemann sphere. In particular it admits a Riemannian
  metric of constant curvature.
\end{theorem} 

This classifies Riemannian surfaces as elliptic (positively curved -- rather, admitting a constant
positively curved metric), parabolic (flat), and hyperbolic (negatively curved) according to their
universal cover.

Proof is based upon existence and properties of Green function for the surface $M$.

 We can
 compose each $z_{\alpha}$ with a Mobius transformation of the disk onto itself so that we can
assume that for each coordinate disk $U_{\alpha}$ and each $p_{0}\in  U_{\alpha}$ there is a coordinate function
$z$ with $z (p_{0} ) = 0$.

     Suppose $M$ is a Riemann surface. Fix $p_{0}\in  M$ and let $z : U\to  \mathbb{D}$ be a coordinate
function such that $z(p_{0} ) = 0$. Let $\mathcal{F}_{p_{0}}$ be the collection of subharmonic
functions $v$ on $M \setminus p_{0}$ satisfying
\begin{itemize}
\item $v = 0$ on $M \setminus K$, for some compact $K\subset M$ with $K \neq M$, and
\item $\limsup_{p\to p_{0}} (v(p) + log |z(p)|) < \infty$
\end{itemize}

Set
\begin{equation}
  \label{eq:104}
  g_{M} (p, p_{0} ) = \sup\{v(p) : v\in \mathcal{F}_{p_{0}} \}
\end{equation}
  
Second condition does not depend on the choice of the coordinate function $z_{\alpha}$ , provided
$z_{\alpha} (p_{0} ) = 0$. The collection $\mathcal{F}_{p_{0}}$ is a Perron family, so by Harnack's
Theorem      $g_M (p, p_{0} )$ is either harmonic in  $M \setminus \{p_{0} \}$, or  is equal to $+\infty$  for
all $p\in M \setminus \{p_{0} \}$.
In the first case, $g_{M} (p, p_{0} )$ is called Green's function with pole (or logarithmic singularity)
at $p_{0}$ . In the second case we say that Green's function does not exist.

The Uniformization Theorem can be reformulated as follows (Koebe[1907]):

\begin{theorem}
  Suppose $M$ is a simply connected Riemann surface. If Green's function exists for $M$ , then there
  is a one-to-one analytic map of $M$ onto $\mathbb{D}$. If $M$ is compact, then there is a
  one-to-one analytic map of $M$ onto $\mathbb{C}^{*}$ . If $M$ is not compact and if Green's
  function does not exist for $M$ , then there is a one-to-one analytic map of $M$ onto
  $\mathbb{C}$.
\end{theorem}

Note that uniformization theorem does not give a construction of uniformization map. 

More modern formulation of uniformization principle operates with quotient of universal cover of
Riemann surface by the action of some Fuchsian group. This group is monodromy group of Liouville
equation (?)

Let $D$ denote $\mathbb{C}$, $\mathbb{C}^{*}$ or $\mathbb{D}$.
\begin{theorem}
Every Riemann surface $M$ is conformally equivalent to the quotient $D/\Gamma$ with $\Gamma$  a freely acting
discontinuous group of fractional transformations preserving $D$.
\end{theorem}

Let us consider the case of Riemann surfaces with universal covering $\mathbb{H}\sim \mathbb{D}$ and
denote by $J_{H}$ the complex analytic covering $J_{H} : \mathbb{H}\to M$ . In this case $\Gamma$
(the automorphism group of $J_{H}$ ) is a finitely generated Fuchsian 
group $\Gamma\subset PSL(2,
\mathbb{R}) = SL(2, \mathbb{R})/{I, -I}$ acting on $\mathbb{H}$ by linear fractional transformations
\begin{equation}
  \label{eq:106}
  w\in \mathbb{H}\quad \gamma\cdot w=\frac{aw+b}{cw+d}\in\mathbb{H},\quad
  \gamma=\begin{pmatrix} a & b\\ c & d\end{pmatrix} \in\Gamma\subset PSL(2,\mathbb{R})
\end{equation}

By the fixed point equation
\begin{equation}
  \label{eq:107}
  w_{\pm}=\frac{a-d\pm\sqrt{(a+d)^{2}-4}}{2c}
\end{equation}
it follows that $\gamma\neq I$ can be classified according to the value of $|\mathrm{tr} \gamma|$:
\begin{enumerate}
\item Elliptic case: $|\mathrm{tr} \gamma|<2$, $\gamma$ has one fixed point on $\mathbb{H}$:
  $w_{-}=\bar w_{+}\not\in\mathbb{R}$ and $M$ has branch point
\item Parabolic case: $|\mathrm{tr}\gamma|=2$, $w_{-}=w_{+}\in\mathbb{R}$ and Riemann surface $M$
  has a puncture.
\item Hyperbolic case: $|\mathrm{tr}\gamma|>2$, Riemann surface has handles. 
\end{enumerate}

   A Riemann surface $M$ isomorphic to the quotient $\mathbb{H}/\Gamma$ has the Poincar\'e metric $\hat g$ as the
unique metric with scalar curvature $R_{\hat g} = -1$ compatible with its complex structure. This
implies the uniqueness of the solution of the Liouville equation on $M$. The Poincar\'e metric
on $\mathbb{H}$ is
\begin{equation}
  \label{eq:108}
  ds^{2}=\frac{|dw|^{2}}{(\Im w)^{2}}
\end{equation}
Note that $P SL(2, \mathbb{R})$ transformations are isometries of $\mathbb{H}$ endowed with the Poincar\'e metric.
   An important property of $\Gamma$ is that it is isomorphic to the fundamental group $\pi_{1}(M)$.

Substitute $w=J_{H}^{-1}(z)$ in \eqref{eq:108} and get Poincar\'e metric on $M$:
\begin{equation}
  \label{eq:109}
  ds^{2}=2\hat g_{z\bar z}|dz|^{2}=e^{\varphi(z,\bar z)} |dz|^{2}=\frac{|J_{H}^{-1}(z)'|^{2}}{(\Im
    J_{H}^{-1}(z))^{2}} |dz|^{2}
\end{equation}

The curvature in complex coordinates is given by
\begin{equation}
  \label{eq:110}
  R_{\hat g}=-\hat g^{z\bar z}\partial \bar \partial \ln \hat g_{z\bar z}, \quad \hat g^{z\bar z}=2e^{-\varphi}
\end{equation}
The condition $R_{\hat g}=-1$ is equivalent to Liouville equation
\begin{equation}
  \label{eq:111}
  \partial \bar \partial \varphi(z,\bar z)=\frac{1}{2} e^{\varphi(z,\bar z)}
\end{equation}
The field $\tilde\varphi=\varphi+\ln \mu, \; \mu>0$ defines a metric of constant curvature $-\mu$. 

\subsection{Classical theory}
\label{sec:classical-theory}

The action for Liouville theory on a surface $\Gamma$ in arbitrary background metric $g$ is
\begin{equation}
  \label{eq:13}
  S=\frac{1}{4\pi} \int_{\Gamma} \sqrt{g} \left( g^{ab} \partial_{a}\varphi \partial_{b}\varphi + Q R \varphi +4\pi \mu e^{2b\varphi}\right) d^{2}x
\end{equation}

This theory gives internal metric $\tilde g^{ab}=e^{\varphi(x)}g^{ab}(x)$


Classical equations of motion are
\begin{equation}
  \label{eq:16}
  \Delta \varphi(x) = \frac{1}{2} \left(b+\frac{1}{b}\right) R (x) + 4\pi \mu e^{2 b \varphi}
\end{equation}
Here $\Delta$ is Laplace-Beltrami operator
\begin{equation}
  \label{eq:17}
  \Delta=\frac{1}{\sqrt{g}}\partial_{\mu} \left(\sqrt{g} g^{\mu\nu} \partial_{\nu}\right)
\end{equation}

Since
\begin{equation}
  \label{eq:93}
  \sqrt{g}(R-\Delta \varphi)=\sqrt{\tilde g} \tilde R, 
\end{equation}
for metric $\tilde g= e^{2b\varphi} g$, equation of motion can be written as
\begin{equation}
  \label{eq:94}
  \tilde R(x) + 4\pi \mu=0
\end{equation}
So classical configurations have constant curvature $-4\pi\mu$.

In most cases it is possible to use flat background metric, but one needs to be careful with
singularities. For example, consider a manifold with the topology of a sphere. We can use flat
background metric but we need to introduce singularity at infinity \cite{zamolodchikovlectures}. 

In flat background metric $g: ds^{2}=dz d\bar z$ equations of motions are
\begin{equation}
  \label{eq:90}
  4\partial\bar \partial \varphi = 4\pi\mu e^{2b\varphi}
\end{equation}
To solve it we consider auxiliary equation:
\begin{equation}
  \label{eq:91}
  4\partial^{2} \psi + t\psi=0
\end{equation}
Function $t$ here is $zz$ component of energy-momentum tensor. Denote by $\psi$  vector
$\psi=(\psi_{1},\psi_{2})$ of linearly independent solutions. Then solution of equation
\eqref{eq:90} is given by the expression
\begin{equation}
  \label{eq:92}
  \varphi=-\frac{1}{b} \log \left(\bar \psi(\bar z) \lambda \psi(z)\right),
\end{equation}
where $\lambda$ is some matrix that is invariant w.r.t. to the action of monodromy group around the
singularity. 

\begin{equation}
  \label{eq:89}
  g=
\end{equation}

**** More on classical solutions here *****

\subsection{Functional integral formulation}
\label{sec:funct-integr-form}

To have quantum theory of gravity we need to define somehow functional integral. It can symbolically
written as
\begin{equation}
  \label{eq:101}
  \int [\mathcal{D}\varphi] e^{-S[\varphi]}
\end{equation}
Actual definition is tricky, because we need to quantize metric $e^{2b\varphi}g^{\mu\nu}$ and this
metric takes part not only in action for field $\varphi$, but also in action for matter field and
even functional integral measure \cite{david1988conformal,distler1989conformal,polyakov1987gauge}. 

In theory with the action
$$S=\frac{1}{4\pi} \int \sqrt{g}\left( \frac{1}{2} g^{\mu\nu} \partial_{\mu} \varphi \partial_{\nu}
  \varphi + Q R\varphi + 4\pi\mu e^{2b\varphi}\right) d^{2}x,$$ 
where $g$ is a background metric, we need to assume some regularization procedure to define a
functional integral, because we want to integrate over measure that does not depend upon metric
\begin{equation}
  \label{eq:114}
  ||\delta\varphi||^{2}=\int \sqrt{g} e^{\varphi(x)}(\delta\varphi)^{2} d^{2}x
\end{equation}
This measure is non-linear and non translation-invariant and we need to substitute it with the
measure
\begin{equation}
  \label{eq:115}
    ||\delta\varphi||^{2}=\int \sqrt{g}(\delta\varphi)^{2} d^{2}x,
\end{equation}
that is  w.r.t. to change $\varphi(x)\to \varphi(x) +C(x)$ for arbitrary fixed function $C(x)$. 

There is for now no way to do it, so 
the idea is that the form of the action after regularization is the same but parameters are changed.
The parameters are then obtained from consistency condition, which is background-independence of the
theory. 

So now $\mu$ is not a classical cosmological constant and $\mu,b,Q$ are connected through
consistency requirements for the theory. 
The consistency requirement is background-independence. The action should not change when we choose
another background metric and shift field $\varphi$ accordingly. 
Let $\tilde g = e^{\sigma} g$ and redefine $\varphi$ as $\tilde \varphi=\varphi +
\frac{Q}{2}\sigma$. 

Consider first the case $\mu=0$.
then all the terms with $\sigma$ are cancelled if we take into account contributions from matter
partition function. But it is possible only if 
$$Q=\frac{25-c_{M}}{6}$$. 
To make exponential term background-independent we need to have $Q=\frac{1}{b}+b$. 

If we consider correlation functions of fields $V_{\alpha}=e^{2\alpha\varphi}$ and impose
background-independence conditions we obtain relation on conformal weights
$$\Delta_{\alpha}=\alpha(Q-\alpha)$$
So for theory to be background-independent we need to have these relations. 

Conformal transformation $z\to w(z)$ can be presented as a change in background metric, so 
$$\varphi(w,\bar w) = \varphi(z,\bar z) -\frac{Q}{2}\log \left|\frac{dw}{dz}\right|^{2}$$
If we consider flat background metric, we need to have asymptotic $\varphi(z,\bar z)\sim -Q \log
|z|^{2}$ at infinity to have smooth metric $e^{2b\varphi}$. 


Now if we consider flat theory with $\mu=0$ everything is good, but when we introduce the observable
$e^{-\mu A}=e^{-\mu \int d^{2}x e^{2b\varphi}}$ we need to regularize exponential field, impose
consistency conditions and we get ordinary Liouville theory as the result! 
So Liouville theory seems to be unique in this way. 

My understanding is that averaging over flat configurations weighted by $e^{-\mu A}$ is equivalent
to averaging over configurations with cosmological constant $\mu$ which introduces non-zero
curvature of singularity at infinity in flat background. 

By the way, it is also written that they consider unnormalized correlation functions (without the
division by $Z$) because there are difficulties with the definition of functional integral $\int
D\varphi e^{-S[\varphi]}$. 
So we need to impose some normalization on their results. 

We can select locally flat metric $g^{ab}=\delta^{ab}$ on some sheet, then the action takes the form
\begin{equation}
  \label{eq:14}
  S= \int d^{2}x \frac{1}{4\pi} (\partial_{a} \varphi)^{2}+\mu e^{2b\varphi}
\end{equation}
Since we rewrote the action in this form we need to take into account the curvature. We do it by
supposing the existence of conical singularity at infinity.
\begin{equation}
  \label{eq:15}
  \varphi(z)=-Q \log (z\bar z)+O(1) \quad z\to\infty
\end{equation}
The equations of motion take the from
\begin{equation}
  \label{eq:18}
  \Delta \varphi = 4\pi b e^{2b\varphi}
\end{equation}
The background charge is connected with the parameter $b$ by equation
\begin{equation}
  \label{eq:20}
  Q=b+\frac{1}{b}
\end{equation}

Liouville field theory is a conformal field theory with central charge
\begin{equation}
  \label{eq:19}
  c=1+6\left(b+\frac{1}{b}\right)^{2}
\end{equation}
and primary fields are given by vertex operators $V_{\alpha}=e^{2\alpha\varphi}$. But primary fields
$V_{\alpha}$ and $V_{Q-\alpha}$ are to be identified. 
Reflection amplitudes are used for this identification:
\begin{equation}
  \label{eq:87}
  V_{\alpha}(z,\bar z)=S(\alpha) V_{Q-\alpha}(z,\bar z),
\end{equation}
where
\begin{equation}
  \label{eq:88}
  S(\alpha)=\frac{\left(\pi\mu\gamma(b^{2})\right)^{\frac{Q-2\alpha}{b}}}{b^{2}}
  \frac{\gamma(2\alpha b -b^{2})}{\gamma\left(2-\frac{2\alpha}{b}+\frac{1}{b^{2}}\right)}
\end{equation}


For $\alpha=\frac{Q}{2}+i P, P\in \mathbb{R}$
fields $V_{\alpha}$ are non-local. Conformal dimension of primary field $V_{\alpha}$ is
\begin{equation}
  \label{eq:32}
  \Delta_{\alpha}=\alpha(Q-\alpha)
\end{equation}

Fields $V_{-\frac{nb}{2}}, \; n\in \mathbb{N}$ are degenerate, correlation functions of these fields
satisfy differential equations of order $n+1$. 

\subsubsection{KPZ-scaling}
\label{sec:kpz-scaling}

Knizhnik-Polyakov-Zamolodchikov (KPZ \cite{knizhnik1988fractal}) scaling relation gives behavior of
Liouville correlation functions w.r.t. to change in coupling constants $\mu, \mu_{B}$. This relation
is easy to obtain. Consider reparametrisation $\varphi=\tilde\varphi-\frac{1}{2b}\ln\mu$. The action
$S[\mu,\mu_{B}]$ is then
\begin{equation}
  \label{eq:84}
  S[\mu,\mu_{B}]=-\frac{1}{4\pi}\int d^{2}x \frac{1}{2b}QR\ln\mu+S\left[1,\frac{\mu_{B}}{\sqrt{\mu}}\right]
\end{equation}
Since
\begin{equation}
  \label{eq:85}
  \int d^{2}x R =4\pi\chi,
\end{equation}
where $\chi=2-2\#handles-\#boundaries$ is Euler characteristic of the surface, we get
\begin{equation}
  \label{eq:99}
    S[\mu,\mu_{B}]=-\frac{1}{2b}Q\chi\ln\mu+S\left[1,\frac{\mu_{B}}{\sqrt{\mu}}\right]
\end{equation}
Substitute it into functional integral for correlation function of arbitrary bulk and boundary
vertex operators:
\begin{multline}
  \label{eq:100}
  \langle V_{\alpha_{1}}(z_{1})\dots V_{\alpha_{n}}(z_{n}) B_{\beta_{1}}(x_{1})\dots
  B_{\beta_{m}}(x_{m})\rangle=\\
  \int \mathcal{D}\tilde\varphi e^{2\alpha_{1}\tilde\varphi(z_{1}}\dots
  e^{2\alpha_{n}\tilde\varphi(z_{n})} \mu^{-\frac{1}{b} \sum_{i}\alpha_{i}} 
  e^{\beta_{1}\tilde\varphi(x_{1}}\dots
  e^{\beta_{m}\tilde\varphi(x_{m})} \mu^{-\frac{1}{2b} \sum_{i}\beta_{i}} e^{-S[\tilde\varphi]}
  \mu^{\frac{Q\chi}{2b}}\\
  \sim \mu^{\frac{Q\chi-2\sum_{i}\alpha_{i}-\sum_{j}\beta_{j}}{2b}}
\end{multline}



Computation of correlation functions in Liouville field theory relies on screening operators and
recurrent relations that appear from OPE for auxiliary correlation functions.
\subsection{Coulomb gas representation and screening operators}
\label{sec:coul-gas-repr}

Calculation of correlation functions in Liouville field theory can be done as follows. 
Consider correlation some function with operator $V_{-\frac{b}{2}}$, it satisfies second order
differential equation:
\begin{equation}
  \label{eq:33}
  \left(\frac{1}{b^{2}}\partial^{2}+T(z)\right)\left<V_{-\frac{b}{2}}(z) \dots \right>=0
\end{equation}
Now to find some correlation function $\left<V_{\alpha_{1}}(z_{1})\dots
  V_{\alpha_{k}}(z_{k})\right>$ we consider auxiliary correlation function with additional
operator $V_{-\frac{b}{2}}(z)$, find a solution to differential equation and fix the integration
constants using fusion rules for field $V_{-\frac{b}{2}}$.

It can be shown (add reference here? \cite{nakayama2004liouville},\cite{teschner2001liouville}) that 
\begin{equation}
  \label{eq:34}
  V_{-\frac{b}{2}}\times V_{\alpha}= [V_{\alpha-\frac{b}{2}}]+[V_{\alpha+\frac{b}{2}}]
\end{equation}


Full form of the operator product expansion is
\begin{equation}
  \label{eq:39}
  A(z)B(w)\underset{z\to w}{\longrightarrow} \sum_{C} \frac{f_{AB}^{C}}{(z-w)^{h_{A}+h_{B}-h_{C}}} C(w)
\end{equation}

We use auxiliary correlation function that can be computed using Coulomb gas picture. The
computation relies on following ideas.


\subsubsection{Coulomb gas picture. Free boson theory. Minimal models}
\label{sec:free-boson-minimal-models}


\begin{equation}
  \label{eq:40}
  S=\int d^{2}z b (\partial \varphi)^{2},\quad \left<\varphi(z,\bar z)\varphi(w,\bar
    w)\right>=-\frac{1}{2b}\left(\ln (z-w) + \ln (\bar z - \bar w)\right)+const
\end{equation}

\begin{equation}
  \label{eq:42}
  T(z)=-\frac{1}{2}:\partial \varphi \partial \varphi:, \quad c=1
\end{equation}
In holomorphic sector primary fields are $\partial \varphi$ with $h=1$ and
$V_{\alpha}(z)=e^{i \alpha \varphi(z)}$ with dimension $h_{\alpha}=\frac{\alpha^{2}}{2}$.

Field $\varphi(z)$ is not purely holomorphic, it contains zero mode
\begin{equation}
  \label{eq:43}
  \varphi(z)=\varphi_{0}-ia_{0}\ln z+\sum_{n\neq 0}\frac{1}{n}a_{n} z^{-n},\quad
  [a_{n},a_{m}]=n\delta_{n+m,0}, \quad [\varphi_{0},a_{0}]=i
\end{equation}
But we can limit ourselves to chiral operators.

If $\alpha_{1}+\dots+\alpha_{n}=0$ then 
\begin{equation}
  \label{eq:41}
  \left<e^{\alpha_{1}\varphi}(z_{1})\dots e^{\alpha_{n}\varphi}(z_{n})\right> = \prod_{i<j}(z_{i}-z_{j})^{-\alpha_{i}\alpha_{j}}
\end{equation}
Since $\left< :e^{A_{1}}::e^{A_{2}}:\dots :A^{A_{n}}:\right> =
\exp\sum_{i<j}\left<A_{i}A_{j}\right>$, we can take $A_{i}=\alpha_{i}\varphi(z_{i})$ and use the
fact that $\left<A_{i}A_{j}\right>=\ln |z_{i}-z_{j}|^{-\alpha_{i}\alpha_{j}}$. 

Neutrality condition can be seen from the invariance of the action w.r.t. $\varphi\to\varphi+a$,
then the correlator obtains factor $e^{\sum \alpha_{i}}$ which is 1 if $\sum \alpha_{i}=0$.

Vacuum expectation value of operator with non-zero charge $\alpha$ should be zero. 

To obtain models with $c\neq 1$ one needs to introduce the background charge:
\begin{equation}
  \label{eq:44}
  S=\frac{1}{8\pi} \int d^{2}x \sqrt{g} (\partial_{\mu}\varphi\partial^{\mu}\varphi+2\gamma\varphi R)
\end{equation}
Here $R$ is scalar curvature, $\gamma$ - constant. 

(Gauss-Bonnet theorem: $\int d^{2}x \sqrt{g} R = 8\pi(1-h)$, $h$-number of handles).

After introducing interaction with the curvature Ward identities are changed and now neutrality
condition is $\sum \alpha_{i}=2\alpha_{0}$, where $i\sqrt{2}\alpha_{0}=\gamma$. 
Stress-energy tensor now takes the form
\begin{equation}
  \label{eq:45}
  T_{\mu\nu}=T^{(0)}_{\mu\nu}-\frac{\gamma}{2\pi}\left(\partial_{\mu}\partial_{\nu}\varphi-\frac{1}{2}\eta_{\mu\nu}\partial^{\rho}\partial_{\rho}\varphi\right)
\end{equation}
and
\begin{equation}
  \label{eq:46}
  T(z)=-\frac{1}{2}:\partial \varphi\partial\varphi:+i\sqrt{2}\alpha_{0}\partial^{2}\varphi
\end{equation}
If we calculate OPE of $T$ we see that $\partial\varphi$ is not primary any more \cite{difrancesco1997cft}:
\begin{equation}
  \label{eq:47}
  T(z)\partial\varphi(w)\sim
  \frac{i2\sqrt{2}\alpha_{0}}{(z-w)^{3}}+\frac{\partial\varphi(w)}{(z-w)^{2}} +\frac{\partial^{2}\varphi(w)}{z-w}
\end{equation}
Vertex operators are still primary, but conformal dimensions are now
\begin{equation}
  \label{eq:48}
  h_{\alpha}=\alpha^{2}-2\alpha_{0}\alpha
\end{equation}
and central charge is
\begin{equation}
  \label{eq:49}
  c=1-24\alpha_{0}^{2}
\end{equation}
The correlator \eqref{eq:41} is preserved. Now vertex operators $V_{\alpha}$ and
$V_{2\alpha_{0}-\alpha}$ are equivalent, so two-point function
$\left<V_{\alpha}(z)V_{\alpha}(w)\right>$  is not zero, but it should be represented as
$\left<V_{\alpha}(z)V_{2\alpha_{0}-\alpha}(w)\right>=\frac{1}{(z-w)^{h_{\alpha}}}$. 

In general we insert non-local screening operator with zero conformal dimension but nonzero charge.
If $\psi$ is primary field with $h_{\psi}=1$ its integral $  A=\oint dz \psi(z) $ is a nonlocal
operator of conformal dimension zero (invariant under conformal mapping $z\to w$ 
\begin{equation}
  \label{eq:50}
  A\to \oint dz \psi(z) \left(\frac{\partial w}{\partial z}\right)=\oint dw \psi(w)
\end{equation}
Then
\begin{multline}
  \label{eq:51}
  [L_{n},A]=\oint dz [L_{n},\psi(z)]=\\\oint dz  (n+1) z^{n}\psi(z)+z^{n+1}\partial \psi(z)=\\
  \oint dz \partial (z^{n+1}\psi (z))=0
\end{multline}
Screening operators are constructed from vertex operators
\begin{equation}
  \label{eq:52}
  V_{\pm}=V_{\alpha_{\pm}}, \quad \alpha_{\pm}=\alpha_{0}\pm \sqrt{\alpha_{0}^{2}+1}, \quad
  \alpha_{+}\alpha_{-}=-1,\quad \alpha_{+}+\alpha_{-}=2\alpha_{0}
\end{equation}
\begin{equation}
  \label{eq:53}
  Q_{\pm}=\oint dz V_{\pm}(z)=\oint dz e^{i\sqrt{2}\alpha_{\pm}\varphi(z)}
\end{equation}
Now for two-point function we can write
\begin{equation}
  \label{eq:54}
  \left<V_{\alpha}(z)V_{\alpha}(w)Q_{-}^{m}Q_{+}^{n}\right>,
\end{equation}
with neutrality condition $2\alpha+n\alpha_{+}+m\alpha_{-}=2\alpha_{0}$,  so $2\alpha$ is an integer
combination of $\alpha_{-}$ and $\alpha_{+}$.
So we obtain Kac formula for conformal weights of fields in minimal models
\begin{equation}
  \label{eq:55}
  \alpha_{r,s}=\frac{1}{2}(1-r)\alpha_{+}+\frac{1}{2}(1-s)\alpha_{-}\quad 
  h_{r,s}(c)=\frac{1}{4}(r\alpha_{+}+s\alpha_{-})^{2}-\alpha_{0}^{2}
\end{equation}

If we calculate three-point function and require operator algebra to be closed we can obtain
conditions on possible values of central charge or background charge $\alpha_{0}$. 

Computation of three-point function relies on following trick: find four-point function as a
solution of differential equation following from null-vector condition and then recover structure
constants for three-point function from the limit $z\to 0$, where $z$ is cross-ratio. 

Structure constants in three-point function are equal to fusion coefficients for corresponding
fields, for example substitute fusion rule \eqref{eq:39} to three-point function

  \begin{equation}
    \label{eq:56}
    \langle
    \varphi_{1}(x_{1})\varphi_{2}(x_{2})\varphi_{3}(x_{3})\rangle=\frac{C^{abc}_{123}}{x_{12}^{\Delta_{1}+\Delta_{2}-\Delta_{3}}
      x_{23}^{\Delta_{2}+\Delta_{3}-\Delta_{1}}x_{13}^{\Delta_{3}+\Delta_{1}-\Delta_{2}}}.     
  \end{equation}


Assume that we know four-point function
$\left<\varphi_{1}(z_{1})\varphi_{2}(z_{2})\varphi_{3}(z_{3})\varphi_{4}(z_{4})\right>$, then we
take limits $z_{1}\to z_{2}$ and $z_{3}\to z_{4}$ and obtain product of structure constants
$C_{12}^{r}C_{34}^{r}$ as a coefficient before $\frac{1}{z_{24}^{r}}$.

To compute four-point function we use conformal invariance to send $z_{1}\to 0$, $z_{2}\to z$
$z_{3}\to 1$ and $z_{4}\to \infty$:
\begin{equation}
  \label{eq:58}
  \left<\varphi_{1}(z_{1})\varphi_{2}(z_{2})\varphi_{3}(z_{3})\varphi_{4}(z_{4})\right>=\left<
    V_{r_{1},s_{1}}(z_{1}) V_{r_{2},s_{2}}(z_{2}) V_{r_{3},s_{3}}(z_{3}) V_{-r_{4},-s_{4}}(z_{4})
    Q_{+}^{r} Q_{-}^{s}\right>,
\end{equation}
where neutrality condition gives us
\begin{equation}
  \label{eq:59}
  \begin{array}{c}
    r_{4}=r_{1}+r_{2}+r_{3}-2r-2\\
    s_{4}=s_{1}+s_{2}+s_{3}-2s-2
  \end{array}
\end{equation}
then 
\begin{multline}
  \label{eq:58}
  \left<\varphi_{1}(0)\varphi_{2}(z)\varphi_{3}(1)\varphi_{4}(\infty)\right>=\left<
    V_{r_{1},s_{1}}(0) V_{r_{2},s_{2}}(z) V_{r_{3},s_{3}}(1) V_{-r_{4},-s_{4}}(\infty)
    Q_{+}^{r} Q_{-}^{s}\right>=\\
  z^{\frac{1}{3}\sum h_{i}-h_{1}-h_{2}}(1-z)^{\frac{1}{3}\sum h_{i}-h_{2}-h_{3}} G(z)
\end{multline}

The simplest case is when $\varphi_{2}=\varphi_{(2,1)}$, then we need to compute integral
\begin{multline}
  \label{eq:60}
  \left<
    V_{r_{1},s_{1}}(0) V_{2,1}(z) V_{r_{3},s_{3}}(1) V_{-r_{4},-s_{4}}(\infty)
    Q_{+}\right>=\\
  \oint dw \left<     V_{r_{1},s_{1}}(0) V_{2,1}(z) V_{r_{3},s_{3}}(1) V_{-r_{4},-s_{4}}(\infty)
    V_{+}(w)\right>=\\
  z^{2\alpha_{2,1}\alpha_{r_{1},s_{1}}}  (1-z)^{2\alpha_{2,1}\alpha_{r_{3},s_{3}}} G(z)
\end{multline}
Substitute \eqref{eq:41} and in holomorphic sector we obtain
\begin{equation}
  \label{eq:61}
  \oint dw w^{a}(w-1)^{b}(w-z)^{c},
\end{equation}
where
\begin{equation}
  \label{eq:62}
  a=2\alpha_{+}\alpha_{r_{1},s_{1}},\quad b=2\alpha_{+}\alpha_{r_{3},s_{3}}, c=2\alpha_{+}\alpha_{2,1}
\end{equation}

The difficulty is in choice of integration contour, we need to choose two independent contours which
will give us two independent solutions of differential equation \eqref{eq:33}. 

Two choices of contour \cite{difrancesco1997cft} give integrals
\begin{equation}
  \label{eq:63}
  \begin{array}{ll}
    I_{1}(a,b,c; z) &= \int_{1}^{\infty} dw w^{a} (w-1)^{b} (w-z)^{c}\\
    I_{1}(a,b,c; z) &= \int_{0}^{z} dw w^{a} (1-w)^{b} (w-z)^{c}\\
    &=z^{1+a+b+c}\int_{0}^{1}dw  w^{a}(1-w)^{b}(1-zw)^{c}
  \end{array}
\end{equation}
These integrals can be calculated by power series expansion of terms with $z$, use of integral
formula for beta-function 
\begin{equation}
  \label{eq:64}
    B(x,y)=\frac{\Gamma(x)\Gamma(y)}{\Gamma(x+y)} = \int_{0}^{1} t^{x-1} (1-t)^{y-1} dt =
    \int_{0}^{\infty} t^{x-1} (1+t)^{-x-y} dt
  \end{equation}
then convert power series to hypergeometric function definition using $\Gamma(t+1)=t\Gamma(t)$
\begin{equation}  
  F(\lambda,\mu,\nu; z)=\sum_{k=0}^{\infty}\frac{1}{k!}\frac{(\lambda)_{k}(\mu)_{k}}{(\nu)_{k}}
  z^{k}, \;\mbox{where}\; (x)_{k}=x(x-1)\dots (x-k+1)
\end{equation}
Final result for conformal block is
\begin{eqnarray}
  \label{eq:65}
  I_{1}(a,b,c;z)=\frac{\Gamma(-a-b-c-1)\Gamma(b+1)}{\Gamma(-a-c)} F(-c,-a-b-c-1,-a-c; z)\\
  I_{2}(a,b,c;z)=z^{1+a+b+c}\frac{\Gamma(a+1)\Gamma(c+1)}{\Gamma(a+c+2)} F(-b,a+1,a+c+2; z)\\
  G(z,\bar z)=\sum_{i,j=1,2}X_{ij} I_{i}(z) \bar I_{j}(\bar z)
\end{eqnarray}
Calculation of coefficients $X_{ij}$ can be made using monodromy properties of correlation function
around singularities. It is equivalent to operator product expansion of primary fields. Functions
$I_{1}, I_{2}$ are in correspondence with intermediate fields in OPE for pairs of primary fields in
four-point function. Different ways to calculate using OPE correspond to different conformal blocks:
\begin{equation}
  \label{eq:66}
  \mathcal{G}(z,\bar z) =
  \left<\varphi_{1}(z_{1})\varphi_{2}(z_{2})\varphi_{3}(z_{3})\varphi_{4}(z_{4})\right> =
  \sum_{r,s;\bar r, \bar s}\mathcal{F}_{r,s}(z) \bar{\mathcal{F}}_{\bar r,\bar s}(\bar z)
\end{equation}
where $\mathcal{F}_{r,s}$ corresponds to a field $\varphi_{(r,s)}$ in OPE
\begin{equation}
  \label{eq:67}
  \varphi_{(r,s)}\in \varphi_{(r_{1},s_{1})}\times \varphi_{(r_{2},s_{2})}
\end{equation}

Complete results for four-point functions in minimal model with Coulomb gas representation are
presented in paper \cite{dotsenko1985four}. 

\subsubsection{Liouville theory}
\label{sec:liouville-theory-1}

Liouville theory is very similar to Coulomb gas picture for minimal models. Scalar curvature at
infinity is $4\pi b$. 

Stress-energy tensor is
\begin{equation}
  \label{eq:68}
  T(z)=-(\partial \varphi)^{2}+Q \partial^{2}\varphi
\end{equation}
Central charge is related to $Q$ by formula \eqref{eq:19}, similarly to minimal models
\eqref{eq:49}. 

It can be shown (add reference here? \cite{nakayama2004liouville},\cite{teschner2001liouville}) that 
\begin{equation}
  \label{eq:34}
  V_{-\frac{b}{2}}\times V_{\alpha}= [V_{\alpha-\frac{b}{2}}]+[V_{\alpha+\frac{b}{2}}]
\end{equation}

We can introduce structure constants of this operator product expansion \cite{fateev2000boundary}
\begin{equation}
  \label{eq:35}
   V_{-\frac{b}{2}} V_{\alpha}= C_{+}[V_{\alpha-\frac{b}{2}}]+C_{-}[V_{\alpha+\frac{b}{2}}]
\end{equation}

This structure constants are equal to constants in three-point functions and can be computed by
screening integrals. For $C_{+}$ we have neutrality condition
$\alpha-\frac{b}{2}+Q-\alpha+\frac{b}{2}=Q$, so we do not need screening integrals and $C_{+}=1$.
For $C_{-}$ we screen by first order of Liouville interaction $-\mu\int e^{2b\varphi} d^{2}x$ 
(Add notes from Zamolodchikov lectures p.131,141 \cite{zamolodchikovlectures}): 
\begin{multline}
  \label{eq:36}
  C_{-}=-\mu\int d^{2}x \left<V_{\alpha}(0)V_{-\frac{b}{2}}(1) e^{2b\varphi(x)}
    V_{Q-\alpha-\frac{b}{2}}(\infty)\right>=\\
  \int d^{2}x x^{2\alpha b} (x-1)^{-b^{2}}=\\
  -\mu\frac{\pi\gamma(2b\alpha-1-b^{2})}{\gamma(-b^{2})\gamma(2b\alpha)},
\end{multline}
where
\begin{equation}
  \label{eq:37}
  \gamma(x)=\frac{\Gamma(x)}{\Gamma(1-x)}
\end{equation}

To calculate two-point function
$\left<V_{\alpha}(x)V_{\alpha}(0)\right>=\frac{D(\alpha)}{|x|^{2\Delta_{\alpha}}}$ we consider
auxiliary three-point function
\begin{equation}
  \label{eq:38}
  \left<V_{\alpha}(x_{1})V_{\alpha+\frac{b}{2}}(x_{2})V_{-\frac{b}{2}}(z)\right>
\end{equation}
Then we consider two operator product expansions when $z\to x_{1}$ and when $z\to x_{2}$. In the
first case only the term $C_{-} V_{\alpha+\frac{b}{2}}$ survives and we get
$C_{-}D\left(\alpha+\frac{b}{2}\right)$. In the second case we get $C_{+}D(\alpha)$. Since $C_{+}=1$
we get functional equation on $D$:
\begin{equation}
  \label{eq:57}
  \frac{D\left(\alpha+\frac{b}{2}\right)}{D(\alpha)}=\frac{1}{C_{-}(\alpha)}
\end{equation}
The solution of this equation is
\begin{equation}
  \label{eq:69}
  D(\alpha)=(\pi\mu\gamma(b^{2}))^{\frac{Q-2\alpha}{b}} \frac{\gamma(2b\alpha-b^{2})}{b^{2}\gamma\left(2-\frac{2\alpha}{b}+\frac{1}{b^{2}}\right)}
\end{equation}

Result for three-point function can be obtained in the same way by considering auxiliary four-point
function that satisfy second order differential equation due to degeneracy of the field $V_{-\frac{b}{2}}$.

\subsection{Boundary Liouville theory}
\label{sec:bound-liouv-theory}



Boundary Liouville theory:
\begin{equation}
  \label{eq:12}
  S=\oint_{S^{1}} \phi \mathcal{N} \phi |dw| + \oint_{S^{1}} \mu_{B} e^{b\phi} + \mu\int_{D} e^{2b\phi} dw d\bar{w}
\end{equation}

\begin{equation}
  \label{eq:11}
  G^{r}(\mu,\mu_{B}) = \mu \frac{\left<V_{\frac{b}{2}} B_{-\frac{1}{b}}\right>}{\left<V_{\frac{b}{2}}\right>}
\end{equation}

In paper \cite{fateev2000boundary} following expression was obtained for one-point boundary
correlation function
\begin{equation}
  \label{eq:22}
  \left<V_{\alpha}(x)\right>=\frac{U(\alpha|\mu_{B})}{|z-\bar z|^{2\Delta_{\alpha}}}
\end{equation}

\begin{equation}
  \label{eq:21}
  U(\alpha)=\frac{2}{b}\left(\pi\mu\gamma( b^{2})\right)^{\frac{Q-2\alpha}{2b}} \Gamma(2b\alpha-b^{2})
  \Gamma\left(\frac{2\alpha}{b}-\frac{1}{b^{2}}-1\right) \cosh (2\alpha-Q)\pi s
\end{equation}
where $s$ is determined from the equation
\begin{equation}
  \label{eq:23}
  \cosh^{2} \pi b s=\frac{\mu_{B}^{2}}{\mu}\sin \pi b^{2}
\end{equation}
and $\gamma(x)=\frac{\Gamma(x)}{\Gamma(1-x)}$.


This one point function is related to generating function of surfaces with a conic singularity of given length and area in
two-dimensional gravity theory. 
Partition function for  surfaces with a conic singularity of given length and area in
two-dimensional gravity theory is denoted by $Z_{\alpha}(A,l)$. Canonical transformation gives us
\begin{equation}
  \label{eq:24}
  W_{\alpha}(l,\mu)=\int_{0}^{\infty} \frac{dA}{A} e^{-\mu A}Z_{\alpha}(A,l)
\end{equation}

For the partition function of surfaces with given area $A$ and length $l$ and conical singularity
with angle $a(\alpha)$ following expression is obtained:
\begin{equation}
  \label{eq:29}
  Z_{\alpha}(A,l)=\frac{1}{b}\frac{\Gamma(2\alpha b
    -b^{2})}{\Gamma\left(1+\frac{1}{b^{2}}-\frac{2\alpha}{b}\right)} \left(\frac{l\Gamma(b^{2})}{2
      A}\right)^{\frac{(Q-2\alpha)}{b}} \exp\left(-\frac{l^{2}}{4 A \sin \pi b^{2}}\right)
\end{equation}

How is it connected with the equation \eqref{eq:21}?

Note that number of surfaces of given area $A$ and length $l$ is given by a canonical ensemble
(parameters $N$, $E$)
\begin{equation}
  \label{eq:25}
  N(l,A) = \sum_{\mbox{surfaces}} \delta(\mbox{Area}-A)\delta(\mbox{Length}-l)
\end{equation}
\begin{equation}
  \label{eq:26}
  F=\log Z
\end{equation}
Using Legendre transformation ($f^{*}(p)=\sup_{x\in I} (xp - f(x))$ $x\to \mbox{tangent}$) we move
to the description with bulk and boundary interaction constants. 
\begin{equation}
  \label{eq:27}
  \min_{\mu}(F(\mu)-\mu A) = S = \log N(A) \quad \mbox{Enthropy}
\end{equation}
Thus we obtain grand canonical ensemble (number of particles is not fixed, $w_{n}=\frac{1}{Z}
e^{\frac{-(E_{n}-\mu N)}{kT}}$)

$Z(\mu_{B},\mu) = \sum_{\mbox{surfaces}} e^{-\mu_{B}l -\mu A}=\int D\varphi e^{-S[\varphi]}$
\begin{equation}
  \label{eq:28}
  \int D\varphi = \int dA dl \int D\phi \quad \mbox{but} \quad \int D\phi = N(A,l)
\end{equation}
It is useful to note different definition of integration measure in \cite{fateev2000boundary}: $\int
\frac{d A}{A}\frac{d l }{l} N(A,l)$

\begin{multline}
  \label{eq:30}
  \left< 1 \right>_{\alpha}=\int \left< V_{\alpha}(z)\right> dz = \int dz \sum_{\mbox{surfaces with cone
    }\alpha\mbox{ at }z} e^{-S[\mbox{surface}]}=\\
  \int \frac{dl}{l} \int dz  \sum_{\mbox{surfaces with cone
    }\alpha\mbox{ at }z \mbox{ of length }l}e^{-S[\mbox{surface}]}=\\
  \int\frac{dl}{l} W_{a}(l) e^{-\mu_{B}l}
\end{multline}


\begin{multline}
  \label{eq:31}
  Z_{\alpha}(\mu,\mu_{B}) = \int \left< V_{\alpha}(z)\right> dz=\\\int D\varphi\int d^{2}z \exp\left(-\frac{1}{4\pi} \int d^{2}z
    b(\nabla\varphi)^{2}-\mu_{B}\oint e^{b\varphi}dx -\mu\int d^{2}z e^{2b\varphi}\right.\\
  \left.-\alpha\int
  d^{2}w \varphi(w) \delta(z-w)\right)
\end{multline}
So we see that one point boundary correlation function gives us partition function for surfaces of
given length and area. 


\subsection{Computation of one-point function}
\label{sec:comp-one-point}

To actually compute bulk one-point function \eqref{eq:22} in boundary Liouville field theory  we
consider auxiliary bulk two-point function
\begin{equation}
  \label{eq:70}
  G_{\alpha,-\frac{b}{2}}=\left<V_{\alpha}(x)V_{-\frac{b}{2}}(z)\right>
\end{equation}
Since we consider theory on upper half-plane, this two-point function satisfy the same equation as
four-point function $\left<V_{\alpha}(x)V_{-\frac{b}{2}}(z)V_{\alpha}(\bar x)V_{-\frac{b}{2}}(\bar
  z)\right>$ in the whole plane. (Zero mode?) 
(Note: $\left<\phi(z,\bar z)\right>_{b}\sim\left<\phi(z)\phi(\bar z)\right>=\frac{1}{(z-\bar
  z)^{2h}}$, then $\left<\phi(y)\right>\sim \frac{1}{y^{2h}}$)

Since our four-point function satisfy second order ODE, it can be presented as a linear combination
of two independent solutions -- conformal blocks. Change of variables allows to rewrite the equation
as a hypergeometric equation, so conformal blocks are given by hypergeometric functions:
\begin{equation}
  \label{eq:71}
  \mathcal{G}_{\pm}(x,z)=\frac{|x-\bar x|^{2\Delta_{\alpha}-2\Delta_{-\frac{b}{2}}}}{|z-\bar
    x|^{4\Delta_{\alpha}}} \mathcal{F}_{\pm}(\eta),
\end{equation}
where $\Delta_{-\frac{b}{2}}=-\frac{1}{2}-\frac{3b^{2}}{4}$ and 
\begin{equation}
  \label{eq:72}
  \eta=\frac{(z-x)(\bar z-\bar x)}{(z-\bar x)(\bar z - x)}
\end{equation}
\begin{eqnarray}
  \label{eq:73}
  F_{+}(\eta)=\eta^{\alpha b}(1-\eta)^{-\frac{b^{2}}{2}} F(2\alpha b -1-2b^{2},-b^{2},2\alpha
  b-b^{2};\eta)\\
  F_{-}(\eta)=\eta^{1+b^{2}-\alpha b}(1-\eta)^{-\frac{b^{2}}{2}} F(-b^{2},1-2\alpha b,2+b^{2}-2\alpha
  b;\eta)
\end{eqnarray}
Coefficients before independent solutions $\mathcal{G}_{\pm}$ are fixed by substituting OPE
\eqref{eq:35} (substitute it to equation?):
\begin{equation}
  \label{eq:74}
  G_{\alpha,-\frac{b}{2}}=C_{+}(\alpha) U\left(\alpha-\frac{b}{2}\right) \mathcal{G}_{+}(x,z)+
  C_{-}(\alpha) U\left(\alpha+\frac{b}{2}\right) \mathcal{G}_{-}(x,z)
\end{equation}
Next we need to consider OPE when both operators $V_{\alpha}(x), V_{-\frac{b}{2}}(z)$ approach the
boundary. 

Bulk operator $V_{-\frac{b}{2}}$ near the boundary gives rise to two boundary families $B_{0}$ and
$B_{-b}$. The fusion of $B_{0}=I$ to $V_{\alpha}$ is $\left<V_{\alpha}(x)
  B_{0}(0)\right>=\frac{R(\alpha,0|\mu_{B})}{|z-\bar z|^{2\Delta_{\alpha}}}$, and the fusion of
$V_{-\frac{b}{2}}$ 

\section{Quantum gravity from random graphs}
\label{sec:quantum-gravity-from}

There is another approach to quantum gravity. Instead of field theory on a surface one can consider
random discrete surfaces and statistical models on them. Random triangulations or quadrangulations
of a surface can be represented by dual random graphs \cite{kazakov1986ising}.  The matter is
introduced as a lattice statistical model on such random graph. Then computation of partition
function (or correlation functions) for matter on random surface is equivalent to summation over
random planar Feynman diagrams for certain matrix model. 

Using this approach Ivan Kostov was able to obtain disk partition function
\cite{kostov2004boundary,kostov2003boundary}. His results are as follows. 

\begin{equation}
  \label{eq:95}
  W(\tau) = -\frac{M^{g}}{2\sin \pi g} \cosh g\tau,
\end{equation}
where parameters $M$, $\tau$ are defined by
\begin{equation}
  \label{eq:96}
  \mu=\frac{1}{\pi} \frac{\Gamma\left(1-\frac{1}{g}\right)}{\Gamma\left(\frac{1}{g}\right)} M^{2}
\end{equation}
\begin{equation}
  \label{eq:97}
  \mu_{B}=\frac{\Gamma\left(1-\frac{1}{g}\right)}{\pi} M\cosh \tau
\end{equation}

Here $g=\frac{1}{b^{2}}$. The result \eqref{eq:95} is supposed to coincide with the results of
\cite{fateev2000boundary} \eqref{eq:21} in the following sense:
\begin{equation}
  \label{eq:98}
  \partial_{\mu} W = \partial_{\mu_{B}} U
\end{equation}
for $\alpha=b$ (?)

Substitute $  M^{2}=\mu\frac{\pi}{1}
\frac{\Gamma\left(\frac{1}{g}\right)}{\Gamma\left(1-\frac{1}{g}\right)}$ to \eqref{eq:95}:
\begin{equation}
  \label{eq:83}
  W(\tau) = -\frac{\left(\mu\frac{\pi}{1}
\frac{\Gamma\left(\frac{1}{g}\right)}{\Gamma\left(1-\frac{1}{g}\right)}\right)^{g/2}}{2\sin \pi g} \cosh g\tau,
\end{equation}

and compare with \eqref{eq:21} for $\alpha=b$: 

\begin{equation}
  U(b)=\frac{2}{b} \left(\pi\mu\gamma( b^{2})\right)^{\frac{-b+1/b}{2b}} \Gamma(b^{2})
  \Gamma\left(1-\frac{1}{b^{2}}\right) \cosh (b-1/b)\pi s
\end{equation}
where $s$ is determined from the equation
\begin{equation}
  \cosh^{2} \pi b s=\frac{\mu_{B}^{2}}{\mu}\sin \pi b^{2}
\end{equation}


\section{Scaling limit and CFT on a cone}
\label{sec:scaling-limit-cft}

Here we argue that scaling limit is connected with correlation
functions of CFT on a cone. Then we introduce recurrence relations on
the angle of the cone. We show that this recurrent relation is
equivalent to fusion rules of degenerate field in CFT. 


\section{CFT fusion coefficients and Bessel's functions}
\label{sec:cft-fusi-coeff}

In this section we solve recurrent relation using fusion rules and
show that scaling function is equal to Bessel's function. 

\section{Scaling of generating function for polygons}
\label{sec:scal-gener-funct}

The solution is
\begin{equation}
  \label{eq:105}
  \left< V_{\alpha_{01}} B_{\beta_{10}}\right>
\end{equation}

\section{Conclusion}
\label{sec:conclusion}
Some concluding remarks are here. 

\bibliography{bibliography}{} 
\bibliographystyle{utphys}

\end{document}
