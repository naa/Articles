\documentclass[12pt]{article}
\usepackage{amsmath,amssymb,amsthm,amsfonts}
\usepackage{multicol}
\usepackage{color}
\usepackage{hyperref}
\usepackage{graphicx}

\newtheorem{theorem}{Theorem}
\newtheorem{definition}{Definition}

\newcommand{\co}[1]{\stackrel{\circ }{#1}}
\newcommand{\gf}{\mathfrak{g}}
\newcommand{\nfp}{\mathfrak{n}^{+}}
\newcommand{\nfm}{\mathfrak{n}^{-}}
\newcommand{\af}{\mathfrak{a}}
\newcommand{\uf}{\mathfrak{u}}
\newcommand{\sfr}{\mathfrak{s}}
\newcommand{\aft}{\widetilde{\mathfrak{a}}}
\newcommand{\afb}{\mathfrak{a}_{\bot}}
\newcommand{\hf}{\mathfrak{h}}
\newcommand{\hfb}{\mathfrak{h}_{\bot}}
\newcommand{\pf}{\mathfrak{p}}

\newcommand{\gfh}{\hat{\mathfrak{g}}}
\newcommand{\afh}{\hat{\mathfrak{a}}}
\newcommand{\sfh}{\hat{\mathfrak{s}}}
\newcommand{\bff}{\mathfrak{b}}
\newcommand{\hfg}{\hf_{\gf}}

\begin{document}
\title{Notes on area-length problem for CLE}
\author{Anton Nazarov}%% $^{1,2}$}

\maketitle

\begin{abstract}
  Here I document staff related to my work on area-length problem.

  We consider the scaling limit of conformal loop ensemble. The ratio
  of loop area to certain power of loop length exhibit universal
  scaling behavior. Scaling function is proportional to Bessel's
  function. We show it using fusion rules for degenerate field in
  conformal field theory on a cone. 
\end{abstract}

\section{Introduction}
\label{sec:introduction}

Here we discuss the history of the problem and state our answer to it. 

Though several publications exist
\cite{cardy2003exact,cardy2003crossover,cardy2001exact,cardy1994geometrical}
on this problem and the answer is
obtained from computer simulation \cite{richard2001scaling}, good understanding was not
obtained. 

\subsectio{$O(n)$ model}
\label{sec:introduction-1}

Partition function for $O(n)$ model is
\begin{equation}
  \label{eq:9}
  Z_{O(n)} = \sum_{\mbox{loop conf}} x^{\mbox{total length}}
  n^{\mbox{number of loops}}
\end{equation}


\section{Scaling of random loops}
\label{sec:scaling-random-loops}

In this section we introduce scaling function for random loops. 

Consider rooted loops in $O(n)$ model. Let $N^r(L)$ be number of
rooted loops with a given length.  By $N^r(L,A)$ we denote the number
of loops of given length and area. $N^r(L)=\int_0^{\infty} N^r(L,A)
dA$. Area weighted number of loops is denoted by $G^r(L,\mu)$. (Here
$\mu$ is bulk coupling constant ?).
\begin{equation}
  \label{eq:2}
  G^r(L,\mu)=\int e^{\mu A} N^r(L,A) dA
\end{equation}

\begin{equation}
  \label{eq:1}
  G^r (\mu,x) = \sum_{\mathrm{clusters}} N^r (L,A) x^L e^{-\mu A}
\end{equation}

The idea is to write $G$ as a correlation function in Liouville
theory.

Note that number of rooted loops of given area scales as the area to
the power minus one. 

Define Hausdorff dimension as follows.
\begin{definition}
  {\it Hausdorff content} of the set $S$ is
  \begin{equation}
    \label{eq:3}
    C^d_H = \mathrm{inf} \sum_i r_i^d,
  \end{equation}
  where $\{r_i\}$ are such that there exists cover of $S$ by balls of
  radii $r_i>0$. 

  {\it Hausdorff dimension}
  \begin{equation}
    \label{eq:4}
    d_H\equiv\mathrm{dim}_H(S) = \mathrm{inf}\{ d\geq 0: C^d_H(S)=0\}
  \end{equation}
\end{definition}

Consider loop of length $L$. The radius of the loop is connected with
the length as $L\sim R^{d_H}$. The area is related to radius $A\sim
R^2$, so $A\sim L^{\frac{2}{d_H}$.
(It is proved or almost proved in papers by Lawler).

For CLE the connection of Hausdorff dimension and parameter $\kappa$
is known:
\begin{equation}
  \label{eq:5}
  d_H=1+\frac{\kappa}{8}
\end{equation}

($\nu^{-1}=d_H$ ????)


At $x\to x_c$
\begin{equation}
  \label{eq:6}
  G^r(\mu,x)= \mu^{\theta} F\left(\frac{x-x_c}{\mu^{\phi}}\right)
\end{equation}

Partition function is product over loops of different length
\begin{equation}
  \label{eq:10}
  Z_{O(n)} = \sum_{\mbox{loop conf}} \prod_L x^{L N(L)} n^{N(L)} = \sum_{\mbox{loop conf}} \prod_L\prod_A x^{L N(L,A)} n^{N(L,A)}
\end{equation}


We can rewrite $G^r(\mu,x)$ as the derivative of generating function
\begin{equation}
  \label{eq:7}
  G^r(\mu,x)=\mu\partial_x \log Z(\mu,\mu_B)
\end{equation}

Here
\begin{equation}
  \label{eq:8}
  Z(\mu,\mu_B) = \left< e^{-\mu A} \right> = \frac{\sum_{\mbox{loop conf}}
  \prod_L\prod_A x^{L N(L,A)} n^{N(L,A)} e^{-\mu A}}{\sum_{\mbox{loop conf}} \prod_L\prod_A x^{L N(L,A)} n^{N(L,A)}}
\end{equation}



\section{Scaling limit and CFT on a cone}
\label{sec:scaling-limit-cft}

Here we argue that scaling limit is connected with correlation
functions of CFT on a cone. Then we introduce recurrence relations on
the angle of the cone. We show that this recurrent relation is
equivalent to fusion rules of degenerate field in CFT. 

\section{CFT fusion coefficients and Bessel's functions}
\label{sec:cft-fusi-coeff}

In this section we solve recurrent relation using fusion rules and
show that scaling function is equal to Bessel's function. 

\section{Conclusion}
\label{sec:conclusion}
Some concluding remarks are here. 

\bibliography{bibliography}{} 
\bibliographystyle{utphys}

\end{document}
