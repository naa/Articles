\documentclass[12pt]{article}
\usepackage{amsmath,amssymb,amsthm,amsfonts}
\usepackage{multicol}
\usepackage{color}
\usepackage{hyperref}
\usepackage{graphicx}

\newtheorem{theorem}{Theorem}
\newtheorem{definition}{Definition}

\newcommand{\co}[1]{\stackrel{\circ }{#1}}
\newcommand{\gf}{\mathfrak{g}}
\newcommand{\nfp}{\mathfrak{n}^{+}}
\newcommand{\nfm}{\mathfrak{n}^{-}}
\newcommand{\af}{\mathfrak{a}}
\newcommand{\uf}{\mathfrak{u}}
\newcommand{\sfr}{\mathfrak{s}}
\newcommand{\aft}{\widetilde{\mathfrak{a}}}
\newcommand{\afb}{\mathfrak{a}_{\bot}}
\newcommand{\hf}{\mathfrak{h}}
\newcommand{\hfb}{\mathfrak{h}_{\bot}}
\newcommand{\pf}{\mathfrak{p}}

\newcommand{\gfh}{\hat{\mathfrak{g}}}
\newcommand{\afh}{\hat{\mathfrak{a}}}
\newcommand{\sfh}{\hat{\mathfrak{s}}}
\newcommand{\bff}{\mathfrak{b}}
\newcommand{\hfg}{\hf_{\gf}}

\begin{document}
\title{Notes on area-length problem for CLE}
\author{Anton Nazarov}%% $^{1,2}$}

\maketitle

\begin{abstract}
  Here I document staff related to my work on area-length problem.

  We consider the scaling limit of conformal loop ensemble. The ratio
  of loop area to certain power of loop length exhibit universal
  scaling behavior. Scaling function is proportional to Bessel's
  function. We show it using fusion rules for degenerate field in
  conformal field theory on a cone. 
\end{abstract}

\section{Introduction}
\label{sec:introduction}

Here we discuss the history of the problem and state our answer to it. 

Though several publications exist
\cite{cardy2003exact,cardy2003crossover,cardy2001exact,cardy1994geometrical}
on this problem and the answer is
obtained from computer simulation \cite{richard2001scaling}, good understanding was not
obtained. 

\subsection{$O(n)$ model}
\label{sec:introduction-1}

Partition function for $O(n)$ model is
\begin{equation}
  \label{eq:9}
  Z_{O(n)} = \sum_{\mbox{loop conf}} x^{\mbox{total length}}
  n^{\mbox{number of loops}}
\end{equation}
Critical point is $x=\frac{1}{\sqrt{2+\sqrt{2-n}}}$, the continuum
limit in this point is described by minimal model $(p,p')$ with
$n=-2\cos\left(\frac{\pi p}{p'}\right)$. $c=1-6\frac{(p-p')^2}{pp'}$, $b=$.

Self-avoiding walk is obtained in the limit $n\to 0$. Then $c=0$ and
we have logarithmic conformal field theory. 

\section{Scaling of random loops}
\label{sec:scaling-random-loops}

In this section we introduce scaling function for random loops. 

Consider rooted loops in $O(n)$ model. Let $N^r(L)$ be number of
rooted loops with a given length.  By $N^r(L,A)$ we denote the number
of loops of given length and area. $N^r(L)=\int_0^{\infty} N^r(L,A)
dA$. Area weighted number of loops is denoted by $G^r(L,\mu)$. (Here
$\mu$ is bulk coupling constant ?).
\begin{equation}
  \label{eq:2}
  G^r(L,\mu)=\int e^{\mu A} N^r(L,A) dA
\end{equation}

\begin{equation}
  \label{eq:1}
  G^r (\mu,x) = \sum_{\mathrm{clusters}} N^r (L,A) x^L e^{-\mu A}
\end{equation}

The idea is to write $G$ as a correlation function in Liouville
theory.

Note that number of rooted loops of given area scales as the area to
the power minus one. 

Define Hausdorff dimension as follows.
\begin{definition}
  {\it Hausdorff content} of the set $S$ is
  \begin{equation}
    \label{eq:3}
    C^d_H = \mathrm{inf} \sum_i r_i^d,
  \end{equation}
  where $\{r_i\}$ are such that there exists cover of $S$ by balls of
  radii $r_i>0$. 

  {\it Hausdorff dimension}
  \begin{equation}
    \label{eq:4}
    d_H\equiv\mathrm{dim}_H(S) = \mathrm{inf}\{ d\geq 0: C^d_H(S)=0\}
  \end{equation}
\end{definition}

Consider loop of length $L$. The radius of the loop is connected with
the length as $L\sim R^{d_H}$. The area is related to radius $A\sim
R^2$, so $A\sim L^{\frac{2}{d_H}}$.
(It is proved or almost proved in papers by Lawler).

For CLE the connection of Hausdorff dimension and parameter $\kappa$
is known:
\begin{equation}
  \label{eq:5}
  d_H=1+\frac{\kappa}{8}
\end{equation}

($\nu^{-1}=d_H$ ????)


At $x\to x_c$
\begin{equation}
  \label{eq:6}
  G^r(\mu,x)= \mu^{\theta} F\left(\frac{x-x_c}{\mu^{\phi}}\right)
\end{equation}

Partition function is product over loops of different length
\begin{equation}
  \label{eq:10}
  Z_{O(n)} = \sum_{\mbox{loop conf}} \prod_L x^{L N(L)} n^{N(L)} = \sum_{\mbox{loop conf}} \prod_L\prod_A x^{L N(L,A)} n^{N(L,A)}
\end{equation}


We can rewrite $G^r(\mu,x)$ as the derivative of generating function
\begin{equation}
  \label{eq:7}
  G^r(\mu,x)=\mu\partial_x \log Z(\mu,\mu_B)
\end{equation}

Here
\begin{equation}
  \label{eq:8}
  Z(\mu,\mu_B) = \left< e^{-\mu A} \right> = \frac{\sum_{\mbox{loop conf}}
  \prod_L\prod_A x^{L N(L,A)} n^{N(L,A)} e^{-\mu A}}{\sum_{\mbox{loop conf}} \prod_L\prod_A x^{L N(L,A)} n^{N(L,A)}}
\end{equation}


\section{Liouville theory}
\label{sec:liouville-theory}

Liouville theory, flat metric, asymptotic for field, Coulomb gas formalism. \cite{nakayama2004liouville,ponsot2002boundary,teschner2001liouville,fateev2000boundary,teschner2000remarks,zamolodchikov1996conformal}

The action for Liouville theory on a surface $\Gamma$ in arbitrary world-sheet metric $g$ is
\begin{equation}
  \label{eq:13}
  S=\frac{1}{4\pi} \int_{\Gamma} \sqrt{g} \left( g^{ab} \partial_{a}\varphi \partial_{b}\varphi + Q R \varphi +4\pi \mu e^{2b\varphi}\right) d^{2}x
\end{equation}
Classical equations of motion are
\begin{equation}
  \label{eq:16}
  \Delta \varphi(x) = \frac{1}{2} \left(b+\frac{1}{b}\right) R (x) + 4\pi b e^{2 b \varphi}
\end{equation}
Here $\Delta$ is Laplace-Beltrami operator
\begin{equation}
  \label{eq:17}
  \Delta=\frac{1}{\sqrt{g}}\partial_{\mu} \left(\sqrt{g} g^{\mu\nu} \partial_{\nu}\right)
\end{equation}

We can select locally flat metric $g^{ab}=\delta^{ab}$ on some sheet, then the action takes the form
\begin{equation}
  \label{eq:14}
  S= \int d^{2}x \frac{1}{4\pi} (\partial_{a} \varphi)^{2}+\mu e^{2b\varphi}
\end{equation}
Since we rewrote the action in this form we need to take into account the curvature. We do it by
supposing the existence of conical singularity at infinity.
\begin{equation}
  \label{eq:15}
  \varphi(z)=-Q \log (z\bar z)+O(1) \quad z\to\infty
\end{equation}
The equations of motion take the from
\begin{equation}
  \label{eq:18}
  \Delta \varphi = 4\pi b e^{2b\varphi}
\end{equation}
The background charge is connected with the parameter $b$ by equation
\begin{equation}
  \label{eq:20}
  Q=b+\frac{1}{b}
\end{equation}

Liouville field theory is a conformal field theory with central charge
\begin{equation}
  \label{eq:19}
  c=1+6\left(b+\frac{1}{b}\right)^{2}
\end{equation}
and primary fields are given by vertex operators $V_{\alpha}=e^{\alpha\varphi}$. But primary fields
$V_{\alpha}$ and $V_{Q-\alpha}$ are to be identified. For $\alpha=\frac{Q}{2}+i P, P\in \mathbb{R}$
fields $V_{\alpha}$ are non-local. Conformal dimension of primary field $V_{\alpha}$ is
\begin{equation}
  \label{eq:32}
  \Delta_{\alpha}=\alpha(Q-\alpha)
\end{equation}

Fields $V_{-\frac{nb}{2}}, \; n\in \mathbb{N}$ are degenerate, correlation functions of these fields
satisfy differential equations of order $n+1$. 

Computation of correlation functions in Liouville field theory relies on screening operators and
recurrent relations that appear from OPE for auxiliary correlation functions.
\subsection{Coulomb gas representation and screening operators}
\label{sec:coul-gas-repr}

Calculation of correlation functions in Liouville field theory can be done as follows. 
Consider correlation some function with operator $V_{-\frac{b}{2}}$, it satisfies second order
differential equation:
\begin{equation}
  \label{eq:33}
  \left(\frac{1}{b^{2}}\partial^{2}+T(z)\right)\left<V_{-\frac{b}{2}}(z) \dots \right>=0
\end{equation}
Now to find some correlation function $\left<V_{\alpha_{1}}(z_{1})\dots
  V_{\alpha_{k}}(z_{k})\right>$ we consider auxiliary correlation function with additional
operator $V_{-\frac{b}{2}}(z)$, find a solution to differential equation and fix the integration
constants using fusion rules for field $V_{-\frac{b}{2}}$.

It can be shown (add reference here? \cite{nakayama2004liouville},\cite{teschner2001liouville}) that 
\begin{equation}
  \label{eq:34}
  V_{-\frac{b}{2}}\times V_{\alpha}= [V_{\alpha-\frac{b}{2}}]+[V_{\alpha+\frac{b}{2}}]
\end{equation}
We can introduce structure constants of this operator product expansion \cite{fateev2000boundary}
\begin{equation}
  \label{eq:35}
   V_{-\frac{b}{2}} V_{\alpha}= C_{+}[V_{\alpha-\frac{b}{2}}]+C_{-}[V_{\alpha+\frac{b}{2}}]
\end{equation}

Why? $C_{+}=1$,
\begin{multline}
  \label{eq:36}
  C_{-}=-\mu\int d^{2}x \left<V_{\alpha}(0)V_{-\frac{b}{2}}(1) e^{2b\varphi(x)}
    V_{Q-\alpha-\frac{b}{2}}(\infty)\right>=\\
  -\mu\frac{\pi\gamma(2b\alpha-1-b^{2})}{\gamma(-b^{2})\gamma(2b\alpha)},
\end{multline}
where
\begin{equation}
  \label{eq:37}
  \gamma(x)=\frac{\Gamma(x)}{\Gamma(1-x)}
\end{equation}

To calculate two-point function
$\left<V_{\alpha}(x)V_{\alpha}(0)\right>=\frac{D(\alpha)}{|x|^{2\Delta_{\alpha}}}$ we consider
auxiliary three-point function
\begin{equation}
  \label{eq:38}
  \left<V_{\alpha}(x_{1})V_{\alpha+\frac{b}{2}}(x_{2})V_{-\frac{b}{2}}(z)\right>
\end{equation}



\subsection{Boundary Liouville theory}
\label{sec:bound-liouv-theory}



Boundary Liouville theory:
\begin{equation}
  \label{eq:12}
  S=\oint_{S^{1}} \phi \mathcal{N} \phi |dw| + \oint_{S^{1}} \mu_{B} e^{b\phi} + \mu\int_{D} e^{2b\phi} dw d\bar{w}
\end{equation}

\begin{equation}
  \label{eq:11}
  G^{r}(\mu,\mu_{B}) = \mu \frac{\left<V_{\frac{b}{2}} B_{-\frac{1}{b}}\right>}{\left<V_{\frac{b}{2}}\right>}
\end{equation}

In paper \cite{fateev2000boundary} following expression was obtained for one-point boundary
correlation function
\begin{equation}
  \label{eq:22}
  \left<V_{\alpha}(x)\right>=\frac{U(\alpha|\mu_{B})}{|z-\bar z|^{2\Delta_{\alpha}}}
\end{equation}

\begin{equation}
  \label{eq:21}
  U(\alpha)=\frac{2}{b}\left(\pi\mu\gamma( b^{2})\right)^{\frac{Q-2\alpha}{2b}} \Gamma(2b\alpha-b^{2})
  \Gamma\left(\frac{2\alpha}{b}-\frac{1}{b^{2}}-1\right) \cosh (2\alpha-Q)\pi s
\end{equation}
where $s$ is determined from the equation
\begin{equation}
  \label{eq:23}
  \cosh^{2} \pi b s=\frac{\mu_{B}^{2}}{\mu}\sin \pi b^{2}
\end{equation}
and $\gamma(x)=\frac{\Gamma(x)}{\Gamma(1-x)}$.


This one point function is related to generating function of surfaces with a conic singularity of given length and area in
two-dimensional gravity theory. 
Partition function for  surfaces with a conic singularity of given length and area in
two-dimensional gravity theory is denoted by $Z_{\alpha}(A,l)$. Canonical transformation gives us
\begin{equation}
  \label{eq:24}
  W_{\alpha}(l,\mu)=\int_{0}^{\infty} \frac{dA}{A} e^{-\mu A}Z_{\alpha}(A,l)
\end{equation}

For the partition function of surfaces with given area $A$ and length $l$ and conical singularity
with angle $a(\alpha)$ following expression is obtained:
\begin{equation}
  \label{eq:29}
  Z_{\alpha}(A,l)=\frac{1}{b}\frac{\Gamma(2\alpha b
    -b^{2})}{\Gamma\left(1+\frac{1}{b^{2}}-\frac{2\alpha}{b}\right)} \left(\frac{l\Gamma(b^{2})}{2
      A}\right)^{\frac{(Q-2\alpha)}{b}} \exp\left(-\frac{l^{2}}{4 A \sin \pi b^{2}}\right)
\end{equation}

How is it connected with the equation \eqref{eq:21}?

Note that number of surfaces of given area $A$ and length $l$ is given by a canonical ensemble
(parameters $N$, $E$)
\begin{equation}
  \label{eq:25}
  N(l,A) = \sum_{\mbox{surfaces}} \delta(\mbox{Area}-A)\delta(\mbox{Length}-l)
\end{equation}
\begin{equation}
  \label{eq:26}
  F=\log Z
\end{equation}
Using Legendre transformation ($f^{*}(p)=\sup_{x\in I} (xp - f(x))$ $x\to \mbox{tangent}$) we move
to the description with bulk and boundary interaction constants. 
\begin{equation}
  \label{eq:27}
  \min_{\mu}(F(\mu)-\mu A) = S = \log N(A) \quad \mbox{Enthropy}
\end{equation}
Thus we obtain grand canonical ensemble (number of particles is not fixed, $w_{n}=\frac{1}{Z}
e^{\frac{-(E_{n}-\mu N)}{kT}}$)

$Z(\mu_{B},\mu) = \sum_{\mbox{surfaces}} e^{-\mu_{B}l -\mu A}=\int D\varphi e^{-S[\varphi]}$
\begin{equation}
  \label{eq:28}
  \int D\varphi = \int dA dl \int D\phi \quad \mbox{but} \quad \int D\phi = N(A,l)
\end{equation}
It is useful to note different definition of integration measure in \cite{fateev2000boundary}: $\int
\frac{d A}{A}\frac{d l }{l} N(A,l)$

\begin{multline}
  \label{eq:30}
  \left< 1 \right>_{\alpha}=\int \left< V_{\alpha}(z)\right> dz = \int dz \sum_{\mbox{surfaces with cone
    }\alpha\mbox{ at }z} e^{-S[\mbox{surface}]}=\\
  \int \frac{dl}{l} \int dz  \sum_{\mbox{surfaces with cone
    }\alpha\mbox{ at }z \mbox{ of length }l}e^{-S[\mbox{surface}]}=\\
  \int\frac{dl}{l} W_{a}(l) e^{-\mu_{B}l}
\end{multline}


\begin{multline}
  \label{eq:31}
  Z_{\alpha}(\mu,\mu_{B}) = \int \left< V_{\alpha}(z)\right> dz=\\\int D\varphi\int d^{2}z \exp\left(-\frac{1}{4\pi} \int d^{2}z
    b(\nabla\varphi)^{2}-\mu_{B}\oint e^{b\varphi}dx -\mu\int d^{2}z e^{2b\varphi}\right.\\
  \left.-\alpha\int
  d^{2}w \varphi(w) \delta(z-w)\right)
\end{multline}
So we see that one point boundary correlation function gives us partition function for surfaces of
given length and area. 

\section{Scaling limit and CFT on a cone}
\label{sec:scaling-limit-cft}

Here we argue that scaling limit is connected with correlation
functions of CFT on a cone. Then we introduce recurrence relations on
the angle of the cone. We show that this recurrent relation is
equivalent to fusion rules of degenerate field in CFT. 


\section{CFT fusion coefficients and Bessel's functions}
\label{sec:cft-fusi-coeff}

In this section we solve recurrent relation using fusion rules and
show that scaling function is equal to Bessel's function. 

\section{Conclusion}
\label{sec:conclusion}
Some concluding remarks are here. 

\bibliography{bibliography}{} 
\bibliographystyle{utphys}

\end{document}
