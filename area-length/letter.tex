\documentclass[12pt]{article}
\usepackage[russian]{babel}
\usepackage[utf8x]{inputenc}
\begin{document}
Здравствуйте!
У меня тут есть одна идея, но я ее еще не совсем додумал. В общих чертах
она такая. Пусть мы ищем производящую функцию для многоугольников
заданной длины на квадратной решетке. Для начала возьмем решетку
конечного размера $L\times L$ с тороидальными граничными условиями. Тогда мы
хотим найти 
$$G(x)=\sum_l p_l x^l$$,
здесь $p_l$ - число различных многоугольников длины $l$. Различные - это
те, которые нельзя совместить трансляцией. (Вроде бы у Гутмана и Ричарда
именно так). 
Если на той же решетке рассмотреть конфигурации $O(n)$-модели, то
статсумма будет
$$Z_{O(n)}(x,n)=\sum_{\mathrm{conf}} x^l n^{\#\mathrm{loops}}$$
В пределе $n\to 0$ вклад дают только конфигурации из одной петли. Число
таких конфигураций длины $l$ - это $p_l$ умножить на число трансляций
$L^2$, если игнорировать конфигурации с петлями, намотанными на тор
(нелокальные). 

То есть 
$$G(x)=\lim_{n\to 0} \frac{1}{L^2} Z_{O(n)}(x,n)$$. 
Производящая функция для петель с корнем дается производной по $x$, так
как корень у петли длины $l$ можно выбрать $l$ способами:
$$G^{(r)}(x)=\partial_x G(x) = \lim_{n\to 0} \frac{1}{L^2} \partial_x
Z_{O(n)} (x,n)$$.

Теперь перейдем к производящим функциям для петель с длиной и площадью:
$$G(x,\mu)=\sum_{l,A} p_{l,A} x^l e^{-\mu A}$$
$$G^{(r)}(x,\mu)=\partial_x G(x,\mu)$$
В $O(n)$-модели можно рассмотреть аналогичную наблюдаемую 
$$\langle e^{-\mu A} \rangle_{O(n)} = \frac{1}{Z_{O(n)} (x,n)}
\sum_{\mathrm{conf}} e^{-\mu A} x^l  n^{\#\mathrm{loops}}$$
Тогда интересующая нас производящая функция дается выражением
$$G(x,\mu)=\lim_{n\to 0} \frac{1}{L^2} Z_{O(n)}(x,n) \langle<e^{-\mu
A}\rangle_{O(n)}$$
Видимо в пределе $L\to \infty$ в окрестности критической точки $x_c$
надо как-то перенормировать бесконечный множитель
$\frac{Z_{O(n)}(x,n)}{L^2}$. 
Для производящей функции петель с корнем получается 
$$\partial_x \left( Z_{O(n)}(x,n) \langle<e^{-\mu
A}\rangle_{O(n)}\right) = \sum_{\mathrm{conf}} l x^l e^{-\mu A}
n^{\#\mathrm{loops}} = $$
$$Z_{O(n)}(x,n) \frac{1}{Z_{O(n)}(x,n)}
\sum_{\mathrm{conf}} l x^l e^{-\mu A} n^{\#\mathrm{loops}} =
Z_{O(n)}(x,n) \langle<e^{-\mu A}\rangle_{O(n)},$$
то есть 
$$G^{(r)}(x,\mu)=\lim_{n\to 0} \frac{1}{L^2} Z_{O(n)}(x,n) \langle<l
e^{-\mu A}\rangle_{O(n)}$$
Таким образом нам надо вычислять $\langle l e^{-\mu A}\rangle_{O(n)}$ в
окрестности критической точки. 

В критической точке $O(n)$ модель описывается конформной теорией поля.
Для этой конформной теории поля можно использовать представление
кулоновского газа, чтобы записать действие. 
$$S^{*} [\varphi] = \in d^2 x (\partial \varphi)^2 + QR \varphi$$
Предел $n\to 0$ должен как-то получаться переходом к пределу по фоновому
заряду. Тут я пока не очень понимаю, что может происходить.
Нас интересует окрестность критической точки. Там теорию можно
представить как деформацию некоторым примарным полем $\Phi$:
$$S=S^{*} + \lambda \int d^2 x \Phi(x)$$
Но примарные поля в представлении кулоновского газа даются вершинными
операторами $e^{\alpha \varphi}$. То есть действие деформированной
теории принимает вид 
$$S=[\varphi] = \in d^2 x (\partial \varphi)^2 + QR \varphi+\lambda
e^{\alpha \varphi}$$,
практически действие теории Лиувилля (с точностью до связи $\lambda,
\alpha$ с зарядом $Q$). Но я пока не понимаю, какое именно примарное
поле соответствует деформации по $x$. Вроде бы если вес деформирующего
поля единица, то есть деформация маргинальная, то деформированная теория
хорошо перенормируется. Это соответствует $\alpha=2b$ при
$b=\sqrt{\frac{3}{2}}$, а при $\alpha=b$ выходит вес $-1$. Я не очень
хорошо знаю про деформации конформной теории поля и про то, какие
примарные поля соответствуют отходу от критической точки по каким
параметрам. Буду признателен за ссылку, у меня под рукой есть только
книжка Mussardo и оригинальная работа Замолодчикова про модель Изинга. 

Наблюдаемые в деформированной теории записываются так:
$$\langle X \rangle_{\lambda} = \frac{\langle X \exp{\lambda \int d^2 x
e^{\alpha \varphi}}\rangle_0}{Z_{\lambda}}$$,
где 
$$Z_{\lambda}=\langle \exp{\lambda \int d^2 x e^{\alpha
\varphi}}\rangle_0$$
То есть в нашем случае получается что-то вроде 
$$\langle e^{-\mu A}\rangle_{\lambda} = \frac{\int D\varphi
e^{S^{*}[\varphi]} e^{-\mu A} e^{\lambda \int d^2 x e^{\alpha
\varphi}}}{\int D\varphi e^{S^{*}[\varphi]} e^{\lambda \int d^2 x
e^{\alpha \varphi}}}$$.
Не очень понятно, будет ли даваться площадь той же формулой $A=\int d^2
x e^{2b\varphi}$, что и в теории Лиувилля, но если так, то получается,
что интересующая нас величина дается отношением корреляторов некоторых
вершинных операторов. 
Аналогично для $$\langle l e^{-\mu A}\rangle_{\lambda}$$. 
То есть вроде бы понятно, откуда может получаться отношение функций Бесселя. 

Остаются вопросы: какая именно деформация?
Правда ли, что получается теория Лиувилля?


\newpage
Aнтон, привет,

Я путешествую с севера на юг. Сейчас посередине в городе Hue, не знаю как написать по-русски. Завтра
буду в городе Hoi, тоже не знаю как написать.

Вот несколько комментариев по поводу Вашего письма.

1) Рассмотрим $О(n)$ модель. Она описывает CFT c $c<1$, где центральный заряд функция n. Случай $n=0$
соответствует SAW $c=0, \kappa=16/6$.
В этом и только в этом случае можно получить информацию об одной петле из ансамбля многих петель из
предела $n\to 0$.

Eсли же $c\neq 0$, этот трюк не помогает (по крайней мере, я не знаю как его обобщить, и думаю что
никак).
\\
2) FZZ предлагает другой способ. Рассматривать сразу только одну петлю (поверхность в их случае). Их
статсумма считает только одну поверхность, однако,
это, не вложенная (embedded) в 3-мерное или какое другое пространство). Нам нужна вложенная
поверхность в 2-мерное пространство. Разница, я думаю, только в переучете
эквивалентных поверхностей получающихся отображением полу-плоскости в себя.
\\
В этом способе мне не понятно как получается функция Бесселя в знаменателе. 
\\
3 ) Подход FZZ апеллирует к флуцтуируещей метрике $\sqrt g= e^{2b\phi}$. Напротив, $О(n)$ модель и CFT
описывается флуктуируещим Гауссовым полем $\varphi$. 
Разница между ними довольно тонкая. Мы можем обсудить её подробнее в январе. Пока что отмечу одну
важную деталь.
\\
Оба поля Гауссовы. Рассмотрим CFT и Лиувилль в на кривой поверхности (background metric). 
\\
CFT: $Action= \int 1/2 g^{mu\nu}\partial_\mu \varphi\partial_\nu\phi) dV+ i\alpha_0 \int R\varphi
dV$\\

Liouville: $Action= \int 1/2 g^{mu\nu}\partial_\mu \phi\partial_\nu\phi) dV+ \alpha_0 \int R\varphi
dV$\\

where $dV=\sqrt g d^2z$-- элемент объема.\\

Отметьте отсутствие $i$ во второй формуле. В CFT, второй член - фаза, в Лиувилле он действительный. 
Это не случайно. Поля имеют разный физ смысл.\\

Наверняка, Вы это хорошо понимаете.\\

Всего доброго,
П.

\end{document}
