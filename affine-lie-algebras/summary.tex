\documentclass[a4paper,12pt]{article}
\usepackage[unicode,verbose]{hyperref}
\usepackage{amsmath,amssymb,amsthm} \usepackage{pb-diagram}
\usepackage{ucs} \usepackage[utf8x]{inputenc}
\usepackage[russian]{babel}
\usepackage{cmap}
\pagestyle{plain}

\theoremstyle{definition} \newtheorem{Def}{Определение}

\newcommand{\tr}{\hat\triangleright}
\newcommand{\trc}{\triangleright}
\newcommand{\adk}{a^{\dagger}_{\kappa}} 
\newcommand{\ak}{a_{\kappa}}
\def\bF{\mbox{$\overline{\cal F}$}} 
\def\F{\mbox{$\cal F$}}
\begin{document}

\section{Начальные сведения}
\subsection{Про алгебры Ли}
\label{sec:Lie}

\begin{Def} {\it Внутренний автоморфизм} - который можно записать в виде $Ad(e^x),\ x\in g$
\end{Def}
\begin{Def} {\it Ранг} $g$ - размерность подалгебры Картана
\end{Def}

Присоединенное действие подалгебры Картана диагонализуемо.

Алгебра раскладывается в прямую сумму собственных подпространств относительно действия подалгебры Картана: $g=h\bigoplus_{\alpha}g_{\alpha},\ g_{\alpha}=\{X:[H.X]=\alpha(H)X\}$

$\alpha\in h^*$ - корень, $g_{\alpha}$ - корневое пространство, $X\in g_{\alpha}$ - корневой вектор
$\Delta$ - набор всех корней, $\Delta_0$ - с нулем

\begin{Def}
  $ B(X,Y)=tr_g (ad(X)ad(Y))$
\end{Def}

$h$ и $h^*$ изоморфны посредством $\left.B\right|_{h\times h}$, $H_{\alpha}\leftrightarrow \alpha:\ B(H_{\alpha},H)=\alpha(H)$

$B(X_{\alpha},X_{-\alpha})=1$
$s_{\alpha}=\mathbb{C}H_{\alpha}\oplus $
 
Регулярный элемент $H\in h:\ \alpha(H)\neq 0$ для всех $\alpha$.

Возьмем регулярный элемент $H_0$, тогда $\alpha(H_0)$ положительно или отрицательно, соответственно положителен или отрицателен корень $\alpha$.

$\alpha\neq \beta+\gamma \Rightarrow \alpha$ - фундаментальный или простой корень

Число фундаментальных корней $\alpha_i$ равно размерности $h$, то есть рангу $g$
$B(\alpha_i,\alpha_j)\leq 0$

\begin{Def}
$a_{ij}=\alpha_j(H_i)=\frac{2\alpha_j(H_{\alpha_i})}{\alpha_i(H_{\alpha_i})}=\frac{2B(\alpha_i,\alpha_j)}{B(\alpha_i,\alpha_i)}$ - матрица Картана, она обладает следующими свойствами:
\begin{itemize}
\item $a_{ij}$ - целые
\item $a_{ii}=2$
\item $i\neq j \Rightarrow a_{ij}\leq 0$
\item $a_{ij}=0 \Leftrightarrow a_{ji}=0$
\end{itemize}
\end{Def}

$B := B(\alpha_i,\alpha_j)$- положительно определенная симметричная матрица
$D=\frac{2\delta_{ij}}{B(\alpha_i,\alpha_i)}$ - положительная диагональная матрица
Тогда $A=DB$

В этом случае матрица Картана называется картановской матрицей конечного типа.

\begin{Def}
{\it Диаграммы Дынкина}
\begin{itemize}
\item $\alpha_1\dots \alpha_l \leftrightarrow \bigcirc$
\item для каждой пары $i,j$ соединяем $i$-ую и $j$-ую вершины $max(|\alpha_{ij}|,|\alpha_{ji}|)$ линиями
\item если $i\neq j$ и $\alpha_{ij}\leq -2$,то надо нарисовать стрелку от $j$ к $i$ 
\end{itemize}
\end{Def}

\[
a=
\left(
\begin{array}{cc}
2 & -2 \\
-1 & 2 \\
\end{array}
\right)
\]
\[
\begin{diagram}
  \node{\bigcirc} 
  \node{\bigcirc} \arrow{w,t,}{}
\end{diagram}
\]

Здесь надо нарисовать диаграммы для всех типов алгебр Ли

Что такое аффинные алгебры
\subsubsection{Матрица Картана}

\begin{Def}
  {\bf Матрица Картана} $nxn$, ранг $l$, $a_{ij}: a_{ii}=2, a_{ij}\leq 0, a_{ij}=0\Rightarrow a_{ji}=0$
\end{Def}
\begin{Def}
  {\bf Реализация матрицы $A$}: $h,\Pi,\Pi^{v}$, где $h$ -
  n-мерное комплексное векторное пространство,
  $\Pi=\{\alpha_1,\dots,\alpha_n\in h^{*}\}$,
  $\Pi^{v}=\{\alpha_1^v,\dots \alpha_n^v \in h\}$, $\Pi,\Pi^v$ -
  линейно независимы, $<\alpha_i^v,\alpha_j>=a_{ij}$, $dim\ h - n=n-l$
\end{Def}
\section{Конформная теория поля и аффинные алгебры Ли}

\subsection{Конформная теория поля}
\label{sec:--CFT-Intro}

Общий вид преобразований 
\begin{equation}
  \label{eq:1}
  g_{\mu\nu}(x)\to g'_{\mu\nu}(x')=\Omega(x)g_{\mu\nu}(x)
\end{equation}
Бесконечно малое преобразование
\begin{eqnarray}
  \label{eq:2}
  x'_{\mu}=x_{\mu}+\epsilon_{\mu}\\
  \partial_{\mu}\epsilon_{\nu}+\partial_{\nu} \epsilon_{\mu}=\frac{2}{d}(\partial\cdot \epsilon)\eta_{\mu\nu}
\end{eqnarray}
В плоском двумерном пространстве $\partial_1\epsilon_1=\partial_2\epsilon_2\;;\; \partial_2\epsilon_1=-\partial_1\epsilon_2$, поэтому можно всё переписать так: $z,\bar{z}=x_1\pm ix_2\; ; \;\epsilon(z)=\epsilon_1+i\epsilon_2\;;\; \epsilon(\bar{z})=\epsilon_1-i\epsilon_2$
Преобразования запишутся как
\begin{eqnarray}
  \label{eq:4}
  z\to f(z)\\
  \bar{z}\to f(\bar{z})\\
  ds^2=dz d\bar{z}\to \left|\frac{\partial f}{\partial z}\right|^2 dz d\bar{z}\\
  \Omega=\left|\frac{\partial f}{\partial z}\right|^2
\end{eqnarray}
Генераторы преобразований 
\begin{eqnarray}
  \label{eq:5}
  l_n=-z^{n+1}\partial_z\\
  \bar{l_n}=-\bar{z}^{n+1}\partial_{\bar{z}}\\
  \left[l_m,l_n\right]=(m-n)l_{m+n}\\
  \left[\bar{l}_m,\bar{l}_n\right]=(m-n)\bar{l}_{m+n}\\
  \left[l_m,\bar{l}_n\right]=0
\end{eqnarray}


          % \begin{diagram}
          %   \node{H} \arrow{e,t}{\epsilon} \node{k} \node{k}
          %   \arrow{e,r}{\eta} \arrow{se,r}{\eta\otimes\eta}
          %   \node{H}\\
          %   \node{H\otimes H} \arrow{n,r}{\mu}
          %   \arrow{ne,r}{\epsilon\otimes\epsilon} \node[3]{H\otimes
          %     H} \arrow{n,r}{\Delta}
          % \end{diagram}
          % \]
\begin{displaymath}
\begin{diagram}
  \node{\mbox{Топология}} \arrow{e,t}{} \node{\mbox{Действие WZW}} \\
  \node{} \arrow[2]{e,t}{\mbox{Токи, конструкция Сугавары}} \node[3]{\mbox{Алгебры Вирасоро на аффинную алгебру}}
\end{diagram}
\end{displaymath}
\begin{list}{Questions}{}
\item Связь со струнами?
\item Другие топологии?
\item В каких случаях возникают более общие алгебры Каца-Муди?
\end{list}
\subsection{Топология WZW}
Действие WZW записывается в терминах отображения $\gamma : \mathbb{R}^2\to G$, где $G$ - компактная простая группа Ли.
$G$ обладает следующим свойством: $\pi_1(G)=\pi_2(G)=0$, $\pi_3(G)=\mathbb(Z)$
\begin{Def}
  {\it Гомотопическая группа} $\pi_n(X)$ -  группа классов эквивалентности отображений $S^n\to X$. Два отображения гомотопически эквивалентны, если одно переводится в другое непрерывной деформацией, 
т.е. для $f_1,f_2, \exists \phi:S^n\times [0,1]\to G : \phi(x,0)=f_1(x), \phi(x,1)=f_2(x)$

The concept of composition that we want for the n-th homotopy group is the same, except that now the domains
that we stick together are cubes, and we must glue them along a face. We therefore define the sum of maps
 $f,g : [0,1]^n → X$ by the formula $(f + g)(t[1], t[2], ... t[n]) = f(2t[1], t[2], ... t[n])$ for t[1] in [0,1/2]
and $(f + g)(t[1], t[2], ... t[n]) = g(2t[1]-1, t[2], ... t[n]) $for t[1] in [1/2,1]. For the corresponding
definition in terms of spheres, define the sum $f + g$ of maps $f, g : S^n \to X$ to be $\psi$ composed with $h$, where $\psi$
is the map from $S^n$ to the wedge sum of two n-spheres that collapses the equator and h is the map from the
wedge sum of two n-spheres to X that is defined to be f on the first sphere and g on the second.

\end{Def}
\begin{Def}
  {\it Группа гомологий}
\end{Def}
\begin{list}{Homotopy-Homology}{}
\item По теореме Гуревича (Hurewicz) гомотопические группы связаны с гомологиями: а именно, если $\pi_k(X)$ нулевые при k меньших n, то $\pi_n(X)=H_n(X)$, либо, если $n=1$, то $\pi_1(X)/[\pi_1(X),\pi_1(X)]=H_1(X)$. 
\end{list}

   2. Далее, гомологии связаны с когомологиями -- двойственность Пуанкаре, если угодно -- в статье речь     
*     идет, кажется, о группе Ли, она ориентируема, в этом контексте двойственность Пуанкаре выглядит       
      особенно просто.                                                                                      
   3. Ранги целочисленных когомологий совпадают с размерностями вещественных когомологий.                   
   4. Последние могут быть вычислены с помощью, например, комплекса де Рама (теорема де Рама).              
   5. А по теории Ходжа (Hodge) в каждом классе когомологий де Рама есть ровно одна гармоническая форма (в  
      статье, кстати, направильное определение, правильное такое: это форма, которая зануляется оператором  
      Лапласа).                                                                                             


\end{document}