\documentclass[aps,prl,reprint]{revtex4-1}
\usepackage{amsmath,amssymb,amsthm,amsfonts}
%\usepackage{multicol}
\usepackage{color}
\usepackage{hyperref}
\usepackage{graphicx}2
\usepackage[russian]{babel}
\usepackage[utf8]{inputenc}
\newtheorem{theorem}{Theorem}
\newtheorem{definition}{Definition}

\newcommand{\co}[1]{\stackrel{\circ }{#1}}
\newcommand{\gf}{\mathfrak{g}}
\newcommand{\nfp}{\mathfrak{n}^{+}}
\newcommand{\nfm}{\mathfrak{n}^{-}}
\newcommand{\af}{\mathfrak{a}}
\newcommand{\uf}{\mathfrak{u}}
\newcommand{\sfr}{\mathfrak{s}}
\newcommand{\aft}{\widetilde{\mathfrak{a}}}
\newcommand{\afb}{\mathfrak{a}_{\bot}}
\newcommand{\hf}{\mathfrak{h}}
\newcommand{\hfb}{\mathfrak{h}_{\bot}}
\newcommand{\pf}{\mathfrak{p}}

\newcommand{\gfh}{\hat{\mathfrak{g}}}
\newcommand{\afh}{\hat{\mathfrak{a}}}
\newcommand{\sfh}{\hat{\mathfrak{s}}}
\newcommand{\bff}{\mathfrak{b}}
\newcommand{\hfg}{\hf_{\gf}}

\begin{document}
\title{Determination of CFT central charge from computer simulations using Wang-Landau algorithm}
\author{Pavel Belov}
\affiliation{Saint-Petersburg state university}
\author{Anton Nazarov}
\affiliation{Saint-Petersburg state university}
\author{Alexander Sorokin}
\affiliation{Saint-Petersburg Nuclear Physics Institute}



\begin{abstract}
  We propose a new way to determine a central charge of conformal field theory models corresponding
  to the 2d lattice models. The idea is to use Wang-Landau algorithm to determine free energy of the
  lattice model on the torus and then calculate central charge from free energy scaling w.r.t. torus
  radii. In this way central charge is directly obtained with a good precision while fit of data is
  relatively easy. We illustrate the method using Ising, tricritical Ising (Blume-Capel), Potts and
  site-diluted Ising models.
\end{abstract}

\maketitle

\section{Introduction}
\label{sec:introduction}

Quantum field theory is an excellent framework to understand the influence of geometry on the
systems behavior in solid state physics. This is especially true in case of critical behavior of
two-dimensional systems, where conformal invariance plays crucial role. 

Conformal field theory in two dimensions is especially powerful since it possess an
infinite-dimensional Virasoro symmetry. The Hilbert space is a union of Virasoro algebra modules. If
the number of these modules is finite the theory is called a minimal model of CFT. Conformal field
theories have small number of parameters -- central charge and conformal weights of primary fields.
It is possible to compute analytically or numerically a lot of observables such as multi-point
correlation functions. Also it is possible to study a lot of different boundary conditions, behavior
in different geometries and various perturbations of conformal field theories. All these properties
make CFT very useful for the description of various experiments and numerical simulations. But the
problem is that often there are no suitable counterparts to the CFT fields in the lattice models or
experiments. So it is apriori not known which CFT describes the critical behavior of a given lattice
model. The most important task is to find value of CFT central charge. Several methods were proposed
by different authors \cite{feiguin2007interacting}, \cite{bastiaansen1998monte},
\cite{lauwers1991estimation} (?). First of these methods is only applicable to the one-dimensional
integrable models and the second requires careful simulations and precise multi-parametric numerical
fitting of conformal weights. We propose more direct approach based on the newer Wang-Landau
simulation algorithm \cite{wang2001efficient}.

\section{Theory}
\label{sec:theory}

Wang-Landau algorithm simulates the energy distribution $\rho(E)=e^{g(E)}$. Energy range is split
in some number of intervals (which can coincide with the number of discrete energies). The algorithm starts with random
lattice configuration, empty array of logarithms of energy densities $g(E_{1}),\dots,g(E_{n})$,
empty visitation histogram $h(E_{1}),\dots,h(E_{n})$ and some initial value (usually
1) of constant $a$. Then one lattice site and new value of
lattice variable at this site are chosen at
random. New state with the changed value at site is accepted with the probability
$e^{g(E_{new})-g(E_{old})}$, visitation number $h(E)$ is increased by 1 and $g(E)$ is increased by
$a$. This procedure is repeated until the visitation histogram is relatively flat. Then the value of
$a$ is divided by 2, the histogram is emptied and next step of algorithm begins. Usually about 20-30
such steps are done in simulations. Having the energy distribution $\rho(E)$ the partition function
is given by
\begin{equation}
  \label{eq:8}
  Z=\sum_{E_{i}} \rho(E_{i}) e^{-\frac{E_{i}}{T}}
\end{equation}

Availability of the partition function in simulations is the crucial difference between Wang-Landau and
other Monte-Carlo methods, such as Metropolis or Wolf algorithms. Having the partition function we
can consider free energy density
\begin{equation}
  \label{eq:10}
  f(T)=-\frac{T}{Volume}\log Z(T)
\end{equation}
In conformal field theory there is a well-known relation connecting free energy on an infinite
cylinder of circumference $N$ with the free energy on a plane (see, for example
\cite{difrancesco1997cft}, Chapter 5):
\begin{equation}
  \label{eq:11}
  f_{cyl}(N)=f_{plane}-\frac{\pi c}{6 N^{2}}.
\end{equation}
Central charge appears in this relation so it could be used to extract central charge value from the
simulation data. In order to do so we simulate a model on a torus of circumferences $N$ and $M$
to obtain free energy density $f_{tor}(N,M)$. Then we extrapolate free energy value when $M\to\infty$
to get free energy on a cylinder $f_{cyl}(N)=\lim_{M\to\infty} f_{tor}(N,M)$. Central charge can be
obtained by fitting $f_{cyl}(N)$ data into the equation~\eqref{eq:11}. 

For this program to work we need to study the behavior  of $f_{tor}(N,M)$ when $M\to\infty$.
To do so we consider CFT partition function on the torus of circumferences $M$ and $N$.
  Bastiaansen and Knops \cite{bastiaansen1998monte} had to consider a skewed torus since at
that time there was no Monte-Carlo method for the free energy computation, but for us straight torus is
enough. Modular parameter of the torus is given by the quotient of two periods:
\begin{equation}
  \label{eq:2}
  \tau=i\frac{M}{N}
\end{equation}
In the usual CFT quantization one needs to chose time direction. We take it to be along the
period $M$ of the torus. The Hamiltonian is the generator of time translation and is given by a sum
of Virasoro generators $L_{0}, \bar L_{0}$:
\begin{equation}
  \label{eq:3}
  H=\frac{2\pi}{N} (L_{0}+\bar L_{0})-\frac{\pi c}{6 N},
\end{equation}
the additional term with the central charge $c$ appears from the conformal mapping from plane. 

We can consider the exponent of the Hamiltonian as a row-to-row transfer matrix. Then translating
from row to row along the time direction $M$ times we get a partition function  on the torus
\begin{equation}
  \label{eq:1}
  Z=\sum_{j} \left<j\right|e^{-M H} \left|j\right>
\end{equation}
The sum here is over the states in Hilbert space. 
The Hilbert space of a conformal field theory is a direct sum of Virasoro algebra modules $V_{i}$ generated
by primary fields $\varphi_{i}$ and parametrised by conformal weights $h_{i}, \bar h_{i}$. Conformal
field theory always contains the identity operator with the conformal weight $h_{0}=0$. The
conformal weights of primary fields are between zero and one:  $0\leq h_{i},\bar h_{\bar i}\leq 1$. We can
choose the basis of Virasoro eigenstates $\left|h_{i}+m_{i},\bar h_{i}+\bar m_{i}\right>$, where
$m_{i},\bar m_{i}$ are positive integers:
\begin{equation}
  \label{eq:4}
  \begin{array}{l}
    L_{0}\left|h_{i}+m_{i},\bar h_{i}+\bar m_{i}\right>=(h_{i}+m_{i})\left|h_{i}+m_{i},\bar h_{i}+\bar
    m_{i}\right>\\
  \bar L_{0}\left|h_{i}+m_{i},\bar h_{i}+\bar m_{i}\right>=(\bar h_{i}+\bar m_{i})\left|h_{i}+m_{i},\bar h_{i}+\bar
    m_{i}\right>    
  \end{array}
\end{equation}

It is customary to use the parameter $q=\exp(2\pi i \tau)$.  In our case of straight torus the parameter $q$ is real, $q=\bar q=e^{-\frac{2\pi M}{N}}$.
Then for the partition function we get
\begin{multline}
  \label{eq:5}
  \frac{Z(q)}{Z_{0}}=q^{-\frac{c}{24}} \bar q^{-\frac{c}{24}}\sum_{j} n_{j} q^{h_{j}+m_{j}}\bar q^{\bar h_{j}+\bar
    m_{j}}=\\=\sum_{i,\bar i} {\cal M}_{i,\bar i}\chi_{i}(q) \bar\chi_{\bar i}(\bar q),
\end{multline}
where $j$ runs over all states and $i, \bar i$ run over primary states,
$\chi_{i}(q)=q^{h_{i}-\frac{c}{24}}\sum_{n\geq 0}d_{i}(n)q^{n}$ is the character of the Virasoro algebra module, $n_{j}$ the multiplicity of
the secondary state and ${\cal M}_{i,\bar i}$ is the multiplicity of the representation $V_{i}\otimes\bar
V_{\bar i}$. The partition function is defined up to normalization $Z_{0}$ that is interpreted as a
partition function on a plane. The multiplicities $\mathcal{M}_{i,\bar i}$ are constrained by  the
modular invariance of the partition function.  We do not need to know them and we also can obtain
conformal weights $h_{i}$ from the simulation data due to the following considerations. First it is important to note that CFT always contains the
identity field with $h_{0}=\bar h_{0}=0$ with the multiplicity $\mathcal{M}_{0,0}=1$.

The partition function in conformal field theory does not depend on temperature since CFT is
applicable only in thermodynamic limit in critical point. To connect CFT computation with the
simulation data we need to insert critical temperature in the expression for the free energy
$  f(N,M)=-\frac{T_{c}}{MN}\log Z(q)$. Let's denote by $f_{0}$ the limit
$\lim_{N\to\infty,M\to\infty}f(N,M)$, this term corresponds to the partition function on the plane.
Then we have
\begin{multline}
  \label{eq:13}
  f(N,M)=f_{0} - \frac{T_{c}\pi c}{6
    N^{2}}\\-\frac{T_{c}}{MN}\log\left[1+\sum_{i\neq 0,\bar i\neq 0} \mathcal{M}_{i,\bar i} q^{h_{i}+\bar h_{\bar
        i}} \left(\sum_{n\geq 0} d_{i}(n) q^{n}\right)\left(\sum_{m\geq 0} d_{\bar i}(m)
      q^{m}\right)\right.\\ +\left.\sum_{n\geq 1}d_{0,0}(n)q^{n}\right]
\end{multline}
One in the last logarithm is due to the identity field and we've moved the  contributions from the  secondary states of
the identity field $\sum_{n\geq 1}d_{0,0}(n)q^{n}$ to the right. Note that if $M\geq N$ the
parameter $q=e^{-\frac{2\pi M}{N}}$ is very small, so we can expand the logarithm holding only
leading contributions with $q^{h_{i}}$:
\begin{multline}
  \label{eq:14}
  f(N,M)=f_{0} - \frac{T_{c}\pi c}{6
    N^{2}}-\frac{T_{c}}{MN}\left[\sum_{i\neq 0,\bar i\neq 0} \mathcal{M}_{i,\bar i} q^{h_{i}+\bar h_{\bar i}} \right].
\end{multline}
If $M>N$ the third term is much smaller than the second one, so in principle we can use measures for
$f(N,M)$ for $M>>N$ (e.g. $M=10 N$) as values of $f(N)$. As the measure of inaccuracy we can take
the difference between $f(N, M)$ and $f(N, M-N)$. 

From  the exact value of CFT partition
function for the Ising model on the torus we see that this gives reasonable precision.
CFT partition function for the critical Ising model on the torus is given by the formula
\begin{equation}
  \label{eq:15}
  Z_{Ising}(q)=\frac{1}{2}\left(\left|\frac{\theta_{2}(0|\tau)}{\eta(\tau)}\right|+\left|\frac{\theta_{3}(0|\tau)}{\eta(\tau)}\right|+\left|\frac{\theta_{4}(0|\tau)}{\eta(\tau)}\right|\right),
\end{equation}
where $\theta_{2},\theta_{3},\theta_{4}$ are Jacobi theta functions. Computing $f(N,M)$ numerically from this
expression it is easy to see that there is no difference in 8 digits after dot between $f(N,10 N)$
and $f(N)$. 

On other hand fitting with the just the smallest conformal weight $h+\bar h$ also gives very good
precision. One can even extract conformal weights $h_{i}$ from such a fit. 

If it is not known what conformal field theory describes critical behavior of the model
under consideration one can also take mean value between fits with $h_{min}=0$ and $h_{min}=1$ . We
show the difference in precision for all three methods in the following table. The value for $f(10)$
is obtained in three ways from the same simulation data for $f(10,n\cdot 10)$, where we took
$n=1\dots 10$ and used 25-steps Wang-Landau algorithm with flatness level $0.8$. 
\begin{equation}
  \label{eq:16}
  \begin{array}{c||c|c|c}
    \mbox{method} & f(10,100) & \mbox{fit}\; h+\bar h=\frac{1}{8} & \mbox{mean of fits} \;0,1\\
    \hline
    f(10) & 2.11568 & 2.11577 & 2.115 \\
    \mbox{inaccuracy} & 0.00009 & 0.00009 & 0.002
  \end{array}
\end{equation}
Taking into account the measure of inaccuracy for all three methods we obtain following values for
central charge. We used the following set of $N$-values:
$N=5,6,7,8,9,10,11,12,13,14,15,16,17,18,19,24,30,40$. 

\begin{equation}
  \label{eq:17}
  \begin{array}{c||c|c|c}
    \mbox{method} & f(10,100) & \mbox{fit}\; h+\bar h=\frac{1}{8} & \mbox{mean of fits} \;0,1\\
    \hline
    c &  0.507\pm 0.001 & 0.506\pm 0.001 & 0.46\pm 0.06 
  \end{array}
\end{equation}
Note that we have not estimated the imprecision from the finite lattice size. If we use only bigger
lattices with $N\geq 15$  from the same data set,  we get a bit different results for central
charge: $c=0.509\pm 0.002, c=0.505\pm 0.002, c=0.46\pm 0.13$. We see that the inaccuracy from using
the finite lattices is of the same order as the difference between first and second methods. 

In the next section we present detailed results of our simulations of different lattice models. 

\section{Simulation}
\label{sec:simulation}

We did the
simulation as follows. First we simulated the system on the lattices with periodic boundary
conditions for different values of $M/N$ and different values of circumference $N$. The partition
function and free energy are available in simulations subject to proper normalization. We used the
fact that for lattice models vacuum degeneracy is known to normalize the partition function. Then we
fit the dependence on $M/N$ of our data for free energy with the decreasing exponent $-b e^{-a
  \frac{M}{N}}$ to account for the third and other terms in formula \eqref{eq:7}. In the limit
$\frac{M}{N}\to\infty$ only first two terms survive, so with this fit we have obtained limit values
of $f(N)$. Now from the fit with $\frac{1}{N^{2}}$ we determine the value of central charge $c$. 

Let us illustrate our method on the simplest example of the Ising model. The vacuum is twice
degenerate so we normalize the obtained energy distribution  ... 

Pics here

q-state Potts

Tricritical Ising

Site-diluted Ising model

In recent study \cite{najafi2016monte} of the domain wall geometry of site-dilute Ising model it was
suggested that the central charge of the theory depends on dilution parameter $p$. We see in our
simulations that to the contrary the central charge is universal. 


\section*{Conclusion}
\label{sec:conclusion}


\bibliography{listing}{} 
\bibliographystyle{apsrev4-1}

\end{document}
