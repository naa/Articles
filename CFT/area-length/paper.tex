\documentclass[12pt]{article}
\usepackage{amsmath,amssymb,amsthm,amsfonts}
\usepackage{multicol}
\usepackage{color}
\usepackage{hyperref}
\usepackage{graphicx}

\newtheorem{theorem}{Theorem}
\newtheorem{definition}{Definition}

\newcommand{\co}[1]{\stackrel{\circ }{#1}}
\newcommand{\gf}{\mathfrak{g}}
\newcommand{\nfp}{\mathfrak{n}^{+}}
\newcommand{\nfm}{\mathfrak{n}^{-}}
\newcommand{\af}{\mathfrak{a}}
\newcommand{\uf}{\mathfrak{u}}
\newcommand{\sfr}{\mathfrak{s}}
\newcommand{\aft}{\widetilde{\mathfrak{a}}}
\newcommand{\afb}{\mathfrak{a}_{\bot}}
\newcommand{\hf}{\mathfrak{h}}
\newcommand{\hfb}{\mathfrak{h}_{\bot}}
\newcommand{\pf}{\mathfrak{p}}

\newcommand{\gfh}{\hat{\mathfrak{g}}}
\newcommand{\afh}{\hat{\mathfrak{a}}}
\newcommand{\sfh}{\hat{\mathfrak{s}}}
\newcommand{\bff}{\mathfrak{b}}
\newcommand{\hfg}{\hf_{\gf}}

\begin{document}
\title{Generating function for self-avoiding polygons and Liouville quantum gravity}
\author{Anton Nazarov, Pavel Weigmann}%% $^{1,2}$}

\maketitle

\begin{abstract}
  We show that scaling behaviour of generating function for self-avoiding polygons on conical
  surface is described by correlation function of Liouville quantum gravity
\end{abstract}

\section{Introduction}
\label{sec:introduction}

Self-avoiding polygons are popular objects for numerical simulations. Use of advanced algorithms
allows to obtain very precise numeric results. 

Guttman and coauthors \cite{richard2001scaling} were able to get numeric answer to the following
question: ``what is the distribution with respect to length and area of self-avoiding polygons?''
Moreover, they were able to fit numeric results with Airy function. There were several attempts to
explain this peculiar development theoretically (see
\cite{cardy2003crossover,cardy2003exact,cardy2001exact}), but they are not satisfactory from our
point of view.

In present paper we generalize the construction of self-avoiding polygons to arbitrary conformal
measure on closed curves. Then we show that distribution in area and length is given by certain
correlation function in two-dimensional Liouville theory of gravity. The paper is organized as
follows. In this section we describe the results of Guttman \cite{richard2001scaling}. Next we
introduce general conformal measure on closed discrete curves (sec. \ref{sec:meas-curv-renorm}).
Then we argue that scaling limit is described by Liouville field theory (sec.
\ref{sec:liouv-field-theory}) and conclude the paper with possible approach to strict proof of
present results (sec. \ref{sec:conclusion}).

Self-avoiding polygons in paper \cite{richard2001scaling} were generated using pivot algorithm.
Maximal length on square lattice was 66. Generating function for polygons of given length and area
is introduced as
\begin{equation}
  \label{eq:75}
  G(x,q)=\sum_{m,n} p_{m,n} x^{m} q^{n},
\end{equation}
where $p_{m,n}$ is the number of polygons with perimeter $m$, enclosing an area $n$ passing through
a given edge (or node). 

Cardy \cite{cardy2001exact} use the following definition for generating function of rooted loops,
with $g=-\ln q$:
\begin{equation}
  \label{eq:76}
  G^{(r)}(x,g)=\sum_{l,A} p_{l,A}^{(r)} x^{l} e^{-gA}
\end{equation}
Generating function for rooted polygons $G^{(r)}(x,g)$ is related to $G(x,g)$ by
\begin{equation}
  \label{eq:81}
  G^{(r)}(x,g)=x \frac{d}{dx} G(x,g),
\end{equation}
because root can be chosen in $l$ ways. 

This model has critical point $(x_{c},0)$ and generating functions displays scaling behavior
\begin{equation}
  \label{eq:77}
  G(x,g)\sim G(x,g)_{\mathrm{reg}}+g^{\theta} F\left((x_{c}-x)g^{-\phi}\right)
\end{equation}
Critical indices $\theta,\phi$ are related to conventional exponents $\nu,\alpha$:
\begin{eqnarray}
  \label{eq:78}
  \phi=\frac{1}{2}\nu\\
  \theta=\frac{1}{2}(1-\alpha) \nu
\end{eqnarray}
Hyperscaling hypothesis adds another relation
\begin{equation}
  \label{eq:79}
  \alpha=2-2\nu
\end{equation}
Numerical results predict
\begin{eqnarray}
  \label{eq:80}
  \nu=\frac{3}{4}\\
  \theta=\frac{1}{3}\\
  \phi=\frac{2}{3}
\end{eqnarray}
We want to find scaling function $F$.

For an arbitrary polygon $P$ on a square lattice $L_{a}$ with lattice spacing $a$ we can use
Schwarz–Christoffel mapping conformal map $z(w)$ from unit disk ${\cal D}$ to the interior of $P$.
Self-avoiding polygons can be seen as a probability measure on polygons and on some subset ${\cal
  Z}_{a}$ of conformal maps $z(w):{\cal D}\to \mathbb{C}$ from unit disk to regions on the complex
plane. Precisely we define ${\cal Z}_{a}$ as ${\cal Z}_{a}=\{z(w):{\cal D}\to \mathbb{C}\; |\;
z(\partial{\cal D}) - \mbox{polygon on }L_{a}\}$ and denote the measure on ${\cal Z}_{a}$ by $\mu_{SAP,a}$.

Length of polygon boundary is then given by 
\begin{equation}
  \label{eq:2}
  l(z)=\oint_{\partial{\cal D}} |z'(w)| dw
\end{equation}
and area by an integral
\begin{equation}
  \label{eq:3}
  A(z)=\int_{{\cal D}} |z'(w)| d^{2}w
\end{equation}

Generating function for rooted polygons of given length and area can be seen as an observable given by the formula
\begin{equation}
  \label{eq:1}
  G^{(r)}(\mu_{B},\mu)=\sum_{z\in{\cal Z}_{a}} e^{-\mu A(z)} e^{-\mu_{B} l(z)} d\mu_{SAP,a}(z)
\end{equation}
In the limit $a\to 0$ the set ${\cal Z}_{a}$ is dense in the space of conformal maps (analytic
functions on unit disk). 

In experiment we have sum over polygons of limited length and finite lattice spacing. If length is
still finite but $a\to 0$ the subspace ${\cal Z}_{a}$ is dense in the Hardy space $H^{\infty}$. 

Every simply-connected region on complex plane can be approximated by a polygon for sufficiently
small $a$.

We want to rewrite generating function as a functional integral over space of analytic functions on
unit disk:
\begin{equation}
  \label{eq:4}
   G^{(r)}(\mu_{B},\mu)=\int_{H^{\infty}({\cal D})} e^{-\mu A(z)} e^{-\mu_{B} l(z)} d\mu_{SAP}(z)
\end{equation}

In order to compute such an integral we need to write the measure explicitly. There are two
approaches to this measure. First we can use conformal map $z(w)$ and introduce weight (Lagrangian).
Second we can consider loop as an SLE trace and use that measure. Big problem that we do not address
here is to prove the equivalence of these approaches. 

\section{Measure on curves and renormalization}
\label{sec:meas-curv-renorm}

We can use Bose field action to introduce measure. Field $\chi(w)=\ln |z'(w)|$
\begin{equation}
  \label{eq:5}
  S=\frac{g}{4\pi} \int_{{\cal D}}|\nabla \chi|^{2} +i\frac{e_{0}}{2\pi} \int_{\partial {\cal D}}
  R\chi
\end{equation}
To get rid of curvature  term one can rewrite this matter action on upper half-plane:
\begin{equation}
  \label{eq:7}
  S=\frac{g}{4\pi} \int_{\Im z\geq 0}(\nabla \chi)^{2} 
\end{equation}
We need to impose boundary condition instead of the curvature term: $\chi=-e_{0}\log (z\bar z)$ for
$z\to \infty$.

SLE measure on the curves is represented by conformal field theory with this action. Local
observables are given by correlation functions of boundary and bulk vertex operators built from the
field $\chi$. To produce the curve connecting two points on the boundary one needs to insert two
boundary condition changing operators. Then use conformal map to make boundary a circle of radius
$r$:
\begin{equation}
  \label{eq:9}
  z=i r \frac{w-i}{w+i}
\end{equation}
and take a limit $r\to 0$. In this limit two boundary condition changing operators fuse to one
``pinning'' operator \cite{bettelheim2005harmonic} and curve becomes closed. 

While matter field $\chi$ is used to introduce measure on critical curves, we need another field
$\varphi$ to take into account area and length of closed curve. 

To take into account weights of area and length we should add to the action two terms
\begin{equation}
  \label{eq:8}
  \tilde S=S + \mu \int_{\Im z\geq 0} e^{2\chi} + \mu_{B}\int_{-\infty}^{+\infty} e^{\chi}
\end{equation}

\begin{equation}
  \label{eq:6}
  A[\chi,\varphi]= \int_{\Im z\geq 0} d^{2}z \left(\frac{1}{4\pi} \left[(\nabla \varphi)^{2} +
      (\nabla \chi)^{2}\right] +\mu e^{2b\varphi}\right) +\int_{-\infty}^{+\infty}dx \mu_{B}e^{b\varphi}
\end{equation}


\section{Liouville field theory}
\label{sec:liouv-field-theory}

\section{Conclusion}
\label{sec:conclusion}

\bibliography{bibliography}{} 
\bibliographystyle{utphys}

\end{document}
